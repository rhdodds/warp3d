%
\documentclass[11pt]{report}
\usepackage{geometry} 
\geometry{letterpaper}

%---------------------------------------------
\setlength{\textheight}{630pt}
\setlength{\textwidth}{450pt}
\setlength{\oddsidemargin}{14pt}
\setlength{\parskip}{1ex plus 0.5ex minus 0.2ex}


%----------------------------------------
\usepackage{amsmath}
\usepackage{layout}
\usepackage{color}
\usepackage{hyphenat}

%----------------------------------------------
\usepackage{fancyhdr} \pagestyle{fancy}
\setlength\headheight{15pt}
\lhead{User's Guide - \textit{WARP3D}}
\rhead{\textit{Element Blocking}}
\fancyfoot[L] {\textit{Chapter {\thechapter}\ (Updated 5-11-2013)}}
\fancyfoot[C] {\thesection-\thepage}
\fancyfoot[R] {\textit{Model Definition}}

%---------------------------------------------------
\usepackage{graphicx}
\usepackage[labelformat=empty]{caption}
\numberwithin{equation}{section}

%---------------------------------------------
%     --- make section headers in helvetica ---
%
\frenchspacing
\usepackage{sectsty} 
\usepackage{xspace}
\allsectionsfont{\sffamily} 
\sectionfont{\large}
\usepackage[small,compact]{titlesec} % reduce white space around sections
%
%----------------------------------------------

%---------  local commands ---------------------

\newcommand{\tb} {\textbf}
\newcommand{\df} {\dotfill}
\newcommand{\nin} {\noindent}
\newcommand{\bmf } {\boldsymbol }  %bold math symbol
\newcommand{\bsf } [1]{\textrm{\textit{#1}}\xspace}
\newcommand{\ul} {\underline}
\newcommand{\hv} {\mathsf}   %helvetica text inside an equation
\newcommand{\eg}{\emph{e.g.},\xspace}
\newcommand{\ti}{\emph}
\newcommand{\ol}{\overline}
\newcommand{\veps}{\varepsilon}
\newenvironment{offsetpar}[1]
{\begin{list}{}%
         {\setlength{\leftmargin}{#1}}%
         \item[]%
}
{\end{list}}


%-------------------------------------
\newcounter{sectrefs}
\setcounter{sectrefs}{0}
\setcounter{chapter}{2}
\setcounter{section}{5}

%--------------------------------------
%--------------------------------------
%---------------------------------------

\begin{document}

\section{Element Blocking}
%\layout
\subsection{Concept}
\noindent All element level 
computations in WARP3D proceed on a block-by-block 
basis to facilitate thread-based (shared-memory) parallel execution. Elements in the
model must be assigned to a block, and elements within a block must be
numbered sequentially. The explicit blocking of elements
for computation represents a 
key feature of the WARP3D software architecture. When executing in
parallel, a processor core (thread) is assigned to perform all required 
computations for
elements within a block, for example, computing new tangent stiffness
matrices. If WARP3D is assigned 16 threads at the start of execution,
computations are performed simultaneously on 16 blocks of 
elements.
The processing of elements by blocks in this approach for parallel
execution greatly reduces the run-time overhead to create and delete the
threads for the program given the comparatively large amount of real
work performed within a thread. Further, all element blocks have 
independent data structures and may be processed in any order --- which 
increases parallel efficiency by allowing threads to process blocks
without waiting for other threads to finish their blocks. Testing reveals essentially
ideal, linear speedups for processing of element blocks.

The maximum number 
of elements permitted in a block is set by a compile-time variable in the WARP3D 
source code and is now generally 128\footnote
{Larger block sizes may cause a runtime violation of the system limit
on the stack size per thread -- often 8 MB. In a bash shell, we recommend using the 
command ``\ti{ulimit -s unlimited}" before starting WARP3D to remove the per thread limit
on stack size}. The block sizes affect
the efficiency of execution. Small block sizes increase the number
of blocks to be processed and shorten the length of innermost do loops,
\it{i.e.}\rm, more overhead work to setup threads-loops for blocks and less work
per block.
With larger sizes, the temporary data structures for a block may overflow the local
cache memory on a processor leading to a performance penalty.  

Each computer
system has a a range of block sizes that yield better performance, which
can be found by running a large
model with varying block sizes. However, such knowledge is not crucial for successful,
efficient executions. As an example, a model with 64,000 20-node brick elements
having the large displacement formulation was executed for a range of block sizes. 
The cpu times required for generation of the element tangent
stiffness matrices were recorded for execution using a single thread. The computer
was a top-end Intel processor. The solution with a block size
of 128 yielded the lowest cpu time. Let that time be a relative 1.0. Relative
times for other block sizes were: 32 (1.07); 64 (1.01); 256 (1.68), and
512 (2.04).

With current hardware designs, we recommend maximum 
block sizes of 128 for all models and have
fixed the current limit on block size in the code at 128. 


\subsection{Simplified, Automatic Blocking: \rm{\ti{Available starting in 17.3.2}}}

\noindent The most frequent analyses performed 
by  WARP3D users are
completed with the thread-based, parallel version (OpenMP, no MPI) and with the
Pardiso sparse solver -- direct or iterative implementations
[see also Section 2.10]. This form of
parallel execution is available on Windows, Linux and Mac OS X computers.
MPI execution is available as an option only on Linux computers.

WARP3D provides automatic assignment of elements to blocks for this
very common situation. Nodes and elements are \ti{not} renumbered in this process.
The command has the form
\begin{align*}
&\hv{\ul{blocking}\ \ \ul{automatic}\  (\ul{size}(=)<integr>)\ (\ul{display})} 
\end{align*}
\noindent Examples of the command are
\small
\begin{verbatim}
  blocking automatic
  blocking automatic size 64 display  
\end{verbatim} 
\normalsize
\noindent The blocking command must be given after input of the element incidences
and before the first \ti{compute} command. The default value of \ti{size} is 128.
The optional \ti{display} requests a tabular listing of the blocking assignments.

WARP3D assigns elements sequentially to blocks starting with element 1 and
opening new blocks as
needed. Elements in a block must: (1) be the same type, (2) be associated
with  the same
user-defined material in the input, (3) have same linear or nonlinear
displacement formulation, (4) have the same integration order and (5) have or
not have the $\bf{\bar B}$ formulation for 8-node brick elements.

The following subsections describe less common situations for analyses that use the 
element-by-element (EBE) iterative solver and/or use the MPI + threads version for
parallel execution on Linux. In these situations, users must provide all the
blocking information which is most often produced by a separate
pre-processor program.


\subsection{Command and Requirements}

The assignment of elements to blocks is most conveniently handled by 
the pre-processor software employed to create the finite element model. 
The \it{patwarp}\ \rm  program, for example, converts a Patran neutral file
(a text file) into a 
WARP3D input file and performs the element-to-block assignments (see Appendix C). 

\noindent The block assignment commands have the form
\begin{align*}
&\hv{\ul{blocking}}\\
&\quad  \hv{<blk:integer>\ <blk\ size:integer>\ <first\ elem\ in\ blk:integer>}
\
\end{align*}
\normalsize   
Blocking data must be input after input of the element incidences 
to enable internal checks on the  correctness of blocking
assignments.

The following example input describes the blocking assignments 
for a model having 25,970 elements and a maximum block size of 128. This is
a typical blocking command when all elements are the same type, the same
material model, etc.

\small
\begin{verbatim}
  blocking   
       1     128       1
       2     128     129
       3     128     257
       4     128     385
       5     128     513
       6     128     641
             .
             .
             .
     201     128   25601
     202     128   25729
     203     114   25857
\end{verbatim} 
\normalsize

The following rules define the proper assignment of elements to blocks. 
All elements in a block:
\small
\begin{itemize}
\item must be sequentially numbered within the block
\item must be the same type; \it{e.g.}\rm, l3disop
\item must have the same kinematic formulation (linear or nonlinear)
\item must have the same material computational model, \it{e.g.}\rm, gurson (see note below 
about segmental curves)
\item must have the same integration order (\it{e.g.}\rm\  $2 \times 2 \times 2$)
\end{itemize}

\normalsize

\noindent All elements in a block must have the same computational material model from the 
WARP3D material model library described in the next chapter. 
Users declare ``materials" in the input (\it{e.g.}\rm, a36\_steel, al\_2024) and then 
associate the 
materials with elements. 

For these material models: \ti{deformation}, \ti{bilinear}, \ti{mises}, 
\ti{gurson} and \ti{cohesive},
elements within a block can be associated with 
different materials but all of the materials must use the same computational 
material model, \it{e.g.}\rm, 
\ti{mises}, \ti{gurson}, $\dots$. This enables elements in a block to 
all follow a mises plasticity model, 
for example, with power-law hardening where Young's modulus, Poisson's 
ratio, yield stress, hardening exponent could vary between elements in the 
same block through the use of different materials.

For these material models: \ti{cyclic}, \ti{mises\_hydrogen} and \ti{umat}, all elements
in a block must refer to the \ul{same material} defined previously in the
input. This is a more restrictive limitation
on element assignment to blocks fort these material models
and will be removed in a future release.

When the properties of a �material� are defined by a segmental stress-strain 
curve or a set of stress-strain curves, the materials associated with elements 
in a block must all reference the same stress-strain curve or set of curves.

In the blocking example above, it is quite straightforward to assign elements to blocks of the
same size, except for the last block. In more complex cases where the elements
are different types, where there are multiple types of material models, etc., the blocks
can have widely varying sizes as needed to comply the assignment ``rules"
bulleted above.

The above discussion applies to executions of WARP3D on a single processor/core
or on any number of processors/cores with the standard version
of WARP3D that uses only threads (shared-memory) during execution
[see Chapter 7 for a full discussion on the various methods-procedures
to execute WARP3D in parallel].

\subsection{Blocking with Domain Decomposition}
For extremely large models, WARP3D provides another level of parallel
execution through the ``hybrid" version of the code (MPI + threads). The model is 
partitioned (decomposed) into domains, with each domain assigned to an
MPI rank (process). Each domain is further partitioned into blocks of elements
as described in the previous section.
The assignment of elements to blocks now becomes more 
complex. Chapter 7 describes the details and commands
for parallel execution using the hybrid version of WARP3D. The brief
discussion here describes the necessary concepts. 

Elements are first assigned to the domains of
the model. Storage and computations for a domain often are mapped onto
a specified computer hardware cluster or onto a specified group of processors
through operating system commands external to WARP3D. The user determines
the best number of domains for the analysis
based on the available hardware configuration and the
size/characteristics of the finite element model. 

The domain number for each element block is simply appended as a new integer value
in the commands shown in the above section. Domain numbers start at 0 (zero).
For example, suppose element block 1300 belongs to domain 84 and has 128 elements
starting with element 324,840. The input line in the blocking data would be

\small
\begin{verbatim}
  blocking   
             .
             .
             .
     1300 128  324840  84
             .
             .
             .
   \end{verbatim} 
\normalsize


The element-to-domain 
assignment proves key for execution efficiency during the 
stiffness assembly and equation solving processes. Good 
domains provide the smallest
number of connecting (shared) model nodes over the boundary
surfaces with adjacent
domains -- achieving this goal
this is a topological optimization problem. The overhead during
execution to communicate solution data across 
domains grows significantly with the number of
nodes on the boundary surfaces of the domains. 
Once all elements in the model are assigned to a domain,
the usual blocking process described above,  with the same
rules, is performed separately for
each domain, where all the elements in a block must reside in the same domain.

Given these additional
complexities for solutions with domain partitioning, we very strongly suggest using the
interactive \it{patwarp}\rm\ program (Appendix C) to perform the
element-to-domain assignment and the blocking assignments. This
standalone program asks the user a series of questions about partitioning
(number of domains, etc.).
The \it{patwarp}\rm\ program runs a topological optimization first to assign
elements to domain, then assigns elements to blocks while insuring
elements in a block conform to all
the above rules, sets the ordering of internal vs. external
blocks in a domain, etc. -- and finally writes
the WARP3D input file. 

The domain and blocking construction
invariably requires that \it{patwarp}\rm\ re-number elements in the
model (but not the nodes) to achieve a mesh that optimally conforms
to the domain and blocking rules. At the user's request, 
\it{patwarp}\rm\ will write a new Patran neutral file for the entire
model that reflects the revised element numbering. It also provides
a special Patran compatible, element results file that has the domain and block
number for each element in the model. Users can invoke the Patran 
results processing capabilities together with this file to produce color
images of the model displaying the domains and blocks.



%
%In earlier versions of WARP3D for Cray (vector computers), 
%an additional requirement on the blocking assignment came into
%effect when the pre-conditioned conjugate gradient, iterative equation solver 
%was used (Section 2.10). Elements within
%a block were not permitted to share a common node. This enabled a highly
%efficient, parallel scheme for the ``gather" process during the 
%conjugate gradient iterations using the vector hardware units (the only route
%to parallel execution). This last requirement nearly always 
%necessitated a re-numbering of elements in the model to eliminate
%node conflicts within blocks. The 
%\it{patwarp}\rm\ pre-processor, for example, employs a simple red-black 
%strategy to re-number elements before constructing the WARP3D input file.
%With this approach, the ending blocks typically have a decreasing number of elements
%as it becomes more difficult to find elements not sharing a node. This example
%for a 520 element model illustrates this effect on the blocking 
%\begin{verbatim}
%  blocking
%     1   120     1
%     2   112   121
%     3   109   233
%     4   100   342
%     5    42   442
%     6    24   484
%     7    11   508
%     8     2   519
%\end{verbatim}
%\normalsize
%Although WARP3D no longer operates on these obsolete ``vector" type computers,
%there are indications that vector processing units may return in evolving
%computer designs through dedicated graphics processing units (GPUs) [late 2009].
%\it{We retain this option in the blocking, but it is not required at
%present for threaded (shared-memory) parallel or MPI-based
%parallel execution.}\rm


\end{document}

 

