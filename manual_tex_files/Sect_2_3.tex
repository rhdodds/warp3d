%!TEX TS-program=pdflatexmk
\documentclass[11pt]{report}
\usepackage{geometry} 
\geometry{letterpaper}

%---------------------------------------------
\setlength{\textheight}{630pt}
\setlength{\textwidth}{450pt}
\setlength{\oddsidemargin}{14pt}
\setlength{\parskip}{1ex plus 0.5ex minus 0.2ex}


%----------------------------------------
\usepackage{amsmath}
\usepackage{layout}
\usepackage{color}
\usepackage{array}

%----------------------------------------------
\usepackage{fancyhdr} \pagestyle{fancy}
\setlength\headheight{15pt}
\lhead{\small{User's Guide - \ti{WARP3D}}}
\rhead{\small{\ti{Element Properties}}}
\fancyfoot[L] {\small{\ti{Chapter {\thechapter}}\ \   (Updated: 12-15-2011)}}
\fancyfoot[C] {\small{\thesection-\thepage}}
\fancyfoot[R] {\small{\ti{Model Definition}}}

%---------------------------------------------------
\usepackage{graphicx}
\usepackage[labelformat=empty]{caption}
\numberwithin{equation}{section}

%---------------------------------------------
%     --- make section headers in helvetica ---
%
\usepackage{sectsty} 
\usepackage{xspace}
\allsectionsfont{\sffamily} 
\sectionfont{\large}
\usepackage[small,compact]{titlesec} % reduce white space around sections
%---------------------------------------------->
%
%
%   which fonts system for text and equations. with all commented,
%   the default LaTex CM fonts are used
%
%
\frenchspacing
%\usepackage{pxfonts}  % Palatino text 
%\usepackage{mathpazo} % Palatino text
%\usepackage{txfonts}


%---------  local commands ---------------------


\newcommand{\bmf } {\boldsymbol }  %bold math symbol
\newcommand{\bsf } [1]{\textrm{\ti{#1}}\xspace}
\newcommand{\ul} {\underline}
\newcommand{\hv} {\mathsf}   %helvetica text inside an equation
\newcommand{\eg}{\emph{e.g.},\xspace}
\newcommand{\ie}{\emph{i.e.},\xspace}
\newcommand{\vs}{\emph{vs.}\xspace}
\newcommand{\ti}{\emph}
\newcommand{\veps}{\varepsilon}
\newcommand{\ol}{\overline}
\newenvironment{offsetpar}[1]
{\begin{list}{}%
         {\setlength{\leftmargin}{#1}}%
         \item[]%
}
{\end{list}}

%
%
%        optional definition for bullet lists which
%        reduces white space.
%
\newcommand{\squishlist}{
 \begin{list}{$\bullet$}
  { \setlength{\itemsep}{0pt}
     \setlength{\parsep}{3pt}
     \setlength{\topsep}{3pt}
     \setlength{\partopsep}{0pt}
     \setlength{\leftmargin}{1.5em}
     \setlength{\labelwidth}{1em}
     \setlength{\labelsep}{0.5em} } }

\newcommand{\squishlisttwo}{
 \begin{list}{$\bullet$}
  { \setlength{\itemsep}{0pt}
     \setlength{\parsep}{0pt}
    \setlength{\topsep}{0pt}
    \setlength{\partopsep}{0pt}
    \setlength{\leftmargin}{2em}
    \setlength{\labelwidth}{1.5em}
    \setlength{\labelsep}{0.5em} } }

\newcommand{\squishend}{
  \end{list}  }
%


%-------------------------------------
\newcounter{sectrefs}
\setcounter{sectrefs}{0}
\setcounter{figure}{0}
\setcounter{chapter}{2}
\setcounter{section}{2}
\renewcommand{\thefigure}{\thesection.\arabic{figure}}

%
%--------------------------------------
%
%
%
%              start document 
%              ==========
%
%

\begin{document}

%
%------------------------------------------------------------------------------
\section{Element Types and Properties}
%------------------------------------------------------------------------------

\noindent The \ti{types} of finite elements and their properties are specified 
prior to any compute requests. An elements command on a separate line initiates 
the element definition sequence. Any number lines may follow to 
define the types and properties of all elements in the model. 
The definition of an element requires the following information:
\squishlist
\item the ``type" of element (\eg \ti{l3disop}, \ti{ts15isop}, etc.)
\item	the kinematic formulation (small or large displacements)
\item	reference to a previously defined ``material" that defines linear-elastic 
properties, mass density and the nonlinear properties (if required)
\item	a list of element property identifiers and associated values, \eg
the order of numerical integration
\squishend
\noindent The command syntax is
\begin{align*}
&\hv{\ul{elements}}\\
&\quad  \hv{<element\ nos.:list>}\ \hv{\ul{type}}\ \hv{<element\ type:label>}
\begin{Bmatrix}
\hv{\ul{linear}} \\ \hv{\ul{nonlinear}}
\end{Bmatrix}
\hv{(,)}\\
& \qquad \qquad \hv{\ul{material}\ <matl\ id:label>  
\left[ <elem\ prop\ id:label><value> (,)\right]}
\end{align*}
\noindent The logical input line may be continued over multiple 
physical input lines with commas at any point. Subsequent sections in 
Chapter 3 define the ``type" of elements currently available and the 
properties available for each element type. Element properties 
typically have a property keyword followed by a value. 
Some element properties are ``logical" values which take 
on ``true" values by the presence of the keyword.

The keyword \ti{linear} requests a conventional small-displacement, small-strain kinematic 
formulation. This is the default formulation and is adopted if no specification is given. 
The keyword \ti{nonlinear} requests a geometric nonlinear formulation that 
models large rotations and finite strains.

Every element must have an associated material. Materials must 
be specified prior to their use in element specification.

An example of elements specification is
\small \begin{verbatim}
   elements 
     1�40 type l3disop linear material a36 center_output bbar,
             order 2x2x2 
     500�1000, 1200�200 by �2 q3disop nonlinear material al_2024t,
             order 14pt_rule long
\end{verbatim}\normalsize

\noindent Once defined, the specification for an element 
cannot be modified at any further point in the analysis.




\end{document}


