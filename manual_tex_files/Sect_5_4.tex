%
\documentclass[11pt]{report}
\usepackage{geometry} 
\geometry{letterpaper}
%
%
%   margins and inter-paragraph spacing
%
%---------------------------------------------
\setlength{\textheight}{630pt}
\setlength{\textwidth}{450pt}
\setlength{\oddsidemargin}{14pt}
\setlength{\parskip}{1ex plus 0.5ex minus 0.2ex}

%----------------------------------------
%\usepackage{amsmath}
\usepackage{layout}
\usepackage{color}
\usepackage[fleqn,reqno]{amsmath} % aligned left, numbers right

%----------------------------------------------
\usepackage{fancyhdr} \pagestyle{fancy}
\setlength\headheight{15pt}
\lhead{\small{User's Guide - \ti{WARP3D}}}
\rhead{\small{Crack Growth -  \ti{Cohesive}}}
\fancyfoot[L] {\small{\ti{Chapter {\thechapter}\ \   (Updated: 11-26-2010)}}}

\fancyfoot[C] {\small{\thesection-\thepage}}
\fancyfoot[R] {\small{\ti{Crack Growth}}}

%---------------------------------------------------
\usepackage{graphicx}
\usepackage[labelformat=empty]{caption}
\numberwithin{equation}{section}

%---------------------------------------------
%     --- make section headers in helvetica ---
\usepackage{sectsty} 
\usepackage{xspace}
\allsectionsfont{\sffamily} 
\sectionfont{\large}
\usepackage[small,compact]{titlesec} % reduce white space around sections
%
%
%   which fonts system for text and equations. with all commented,
%   the default LaTex CM fonts are used
%
%
\frenchspacing
%\usepackage{pxfonts}  % Palatino text 
%\usepackage{mathpazo} % Palatino text
%\usepackage{txfonts}

%----------------------------------------------

%---------  local commands ---------------------

\newcommand{\bmf } {\boldsymbol }  %bold math symbol
\newcommand{\bsf } [1]{\textrm{\ti{#1}}\xspace}
\newcommand{\ul} {\underline}
\newcommand{\hv} {\mathsf}   %helvetica text inside an equation
\newcommand{\ti}{\emph}
\newcommand{\eg}{\ti{e.g.},\xspace}
\newcommand{\ie}{\ti{i.e.},\xspace}
%
%
%        optional definition for bullet lists which
%        reduces white space.
%
\newcommand{\squishlist}{
 \begin{list}{$\bullet$}
  { \setlength{\itemsep}{0pt}
     \setlength{\parsep}{3pt}
     \setlength{\topsep}{3pt}
     \setlength{\partopsep}{0pt}
     \setlength{\leftmargin}{1.5em}
     \setlength{\labelwidth}{1em}
     \setlength{\labelsep}{0.5em} } }

\newcommand{\squishlisttwo}{
 \begin{list}{$\bullet$}
  { \setlength{\itemsep}{0pt}
     \setlength{\parsep}{0pt}
    \setlength{\topsep}{0pt}
    \setlength{\partopsep}{0pt}
    \setlength{\leftmargin}{2em}
    \setlength{\labelwidth}{1.5em}
    \setlength{\labelsep}{0.5em} } }

\newcommand{\squishend}{
  \end{list}  }
%
%
%-------------------------------------
\newcounter{sectrefs}
\setcounter{sectrefs}{0}
\setcounter{chapter}{5}
\setcounter{section}{3}
%
%
%
%              start document 
%              ==========
%
\begin{document}
%
%
%*****************************************************
\section{Crack Growth Using Interface-Cohesive Elements}
%*****************************************************
This method of crack formation and growth combines
interface finite elements described in Section 3.3 with the nonlinear,
cohesive constitutive model described in Section 3.8.
With this approach, crack extension by the creation of new traction
free crack faces evolves as a direct outcome of the computations. Pre-existing
cracks are not required in the finite element model. The
cohesive constitutive model first increases and then reduces (eventually to zero)
tractions across surfaces of the
interface elements  with
increasing values of the opening and sliding (shear) displacement jumps across 
the surfaces. Interface elements are effectively deleted 
from the solution when the the displacement jumps
reach large values and cohesive tractions approach zero. During 
subsequent load steps, the interface
element stiffness is taken as zero and the nodal forces exerted by the interface
element on adjacent nodes are relaxed to zero over a user-specified number of
load steps. The user actions required to invoke crack growth with
interface-cohesive elements during an analysis are:

\small
\squishlist

\item specify the property keyword \ti{killable} in the definition of a cohesive
material that invokes the \ti{cohesive} material model. 

\item following the procedures for other nonlinear analyses, define the finite
element model, loading, constraints and nonlinear solution parameters.

\item	use the commands described subsequently in this section to define
parameters controlling the crack growth procedures (limiting values of the
cohesive behavior, number of release steps, printing options, etc.). 
These parameters are specified in a manner analogous to specification of the
nonlinear solution parameters. These crack growth parameters can not be altered
during the analysis.

\item use various combinations of \ti{compute} and \ti{output}	
commands to control the
nonlinear solution over load steps. The crack growth procedures are
automatically invoked by solution management routines in WARP3D.

\item	the analysis restart features of WARP3D fully support crack growth
modeling with the interface element and cohesive material
model approach. Restart files contain the values of growth parameters and the solution
state required to continue an analysis with crack growth.

\squishend
\normalsize
\ti{\ul{Note}}: the present capabilities for crack growth using interface-cohesive elements in
WARP3D described here require that all the killable interface elements 
have the identical cohesive
material formulation. At present the two available formulations
are \ti{exp1\_intf} and \ti{ppr} described in Section 3.8. Multiple,
cohesive materials with different properties may be defined for use in an analysis,
but they all must use the same formulation.
%
%
%*****************************************************
\subsection{General Input Commands}
%*****************************************************
The sequence of commands to initiate the definition 
of crack growth parameters is
\begin{align*}
&\hv{\ul{crack}\ (\ul{grow}th)\ (\ul{param}eters)} \\
&\hv{\ul{type}\ (\ul{of})\ (\ul{crack})\ (\ul{grow}th)} 
\begin{Bmatrix}
\hv{\ul{none}} \\ \hv{\ul{cohesive}}
\end{Bmatrix}
\end{align*}
where \ti{none} turns off subsequent element extinction during the analysis. 
These commands typically are given after the \ti{nonlinear solution
parameters} in the input file. The
keyword \ti{cohesive} invokes extinction of interface elements based on attainment of
a user defined, terminal state in the cohesive constitutive model. If prior crack
extension has occurred, use of the \ti{none}
option immediately suppresses any further crack growth processing. If elements
are in the process of being killed, the reduction to zero of remaining element 
(internal) forces will not be completed.

When the terminal state in the cohesive material is reached, the remaining
internal (nodal) forces
for an interface element are imposed on adjacent nodes in the model as nodal
forces. The crack growth code reduces values of these (small) forces 
linearly to zero over a number of
subsequent load steps. The interface-element stiffness is set immediately to zero and
remains zero for all subsequent load steps. Input commands to describe the force
release model are described in a subsection below.

The crack growth processor provides a convenient printing option to monitor the
growth process. The command has the form
\begin{align*}
&\hv{\ul{print}\ (\ul{status})} 
\begin{Bmatrix}
\hv{\ul{on}} \\ \hv{\ul{off}}\end{Bmatrix}\hv{(\ul{order}
\ (\ul{elem}ents) <element list: integerlist>)}
\end{align*}
where the keyword \ti{on} or \ti{off} is required. An optional list 
of elements previously
marked as \ti{killable} in the element properties input
may be specified for printing. If no list is given, all
elements having a cohesive material model, and with the killable property, are
included in the list (in ascending numerical order). When the optional list is
given, information is printed for elements in the order specified in the list.

At the end of each load step when this printing option is on, a tabular
summary of the current status is printed for each interface element in the list. 
The values printed depend on the formulation specified in the
cohesive material, \eg the \ti{ppr} formulation or the \ti{exp1\_intf}
formulation. To prevent excessive amounts of output, information 
is printed only for those interface-cohesive elements having tractions 
that exceed some fraction
of the peak value(s). Additionally, if adaptive load control 
is enabled the table includes the increase in the current growth 
parameter over the last step.
%
%

%*****************************************************
\subsection{Adaptive Load Control}
%*****************************************************
When load steps are too large, the computed response may neglect key features of
the decohesion process and accumulate significant discrepancies with the
expected model behavior. A large load step, for example, may lead to the jump from
a pre-peak traction level to a post-peak traction
level in just one step. When this occurs, the volume elements
incident on the interface-cohesive elements never experience
the peak stress levels that would otherwise be imposed by the
the cohesive traction-separation response with reduced load step
sizes. For volume elements having a plasticity constitutive model, such ``jumping" across
the peak of the traction-separation curve impacts significantly the development of
plastic zones and, as a consequence, the computed crack-growth response.

To help alleviate this problem, WARP3D provides an adaptive load
control feature that automatically decreases (or increases) subsequent load-step
sizes based on the change in a single, ``effective" measure of the combined
normal and sliding displacement jumps ($\bar\delta$) 
of the interface elements (see
below for definitions of $\bar\delta$). The adaptive
algorithm operates as follows:
%
\small
\squishlist

\item At the end of each load step ($n$), the adaptive code locates the
interface-cohesive element that experienced the maximum change of
$\bar\delta$ during the previous step. Let $\Delta= \bar\delta_n
-\bar\delta_{n-1}$
denote this value for convenience. 

\item If $\Delta$ exceeds the allowable change specified by the user
$(\Delta> \alpha \bar\delta_{peak})$, the adaptive code estimates 
and applies a reduction factor to the user-specified (incremental)
loading levels in subsequent load steps such that the projected 
new values of $\Delta$ become $\approx 0.8 \times \alpha \bar\delta_{peak}$ (see below
for definitions of $\bar\delta_{peak}$). The user 
specifies the value of $\alpha$
during input of the crack growth parameters (see command below). 
Typical values for $\alpha$ range from $0.15-0.3$, with 
$\alpha=0.2$ often used as the value for a wide variety of situations. 

\item Similarly, the adaptive code increases the applied 
(incremental) loading in subsequent steps when 
$\Delta$ becomes much smaller than 
$\alpha \bar\delta_{peak}$. Load increases are triggered when 
$\Delta$ is less than $0.5 \times \alpha \bar\delta_{peak}$
for at least 3 consecutive load steps. When this situation occurs, the load
step size increases to make $\Delta \approx \alpha \bar\delta_{peak}$, 
but in no case is the load step size increased immediately by more than a factor of 2.
The adaptive code does not increase load step sizes unless it has
previously reduced load step sizes at some prior point in the analysis.
\squishend
\normalsize
%
As a result of this adaptive algorithm, the user-specified sizes for load steps
are re-defined (down and then possibly up) continually during the analysis 
to maintain a target 
$\Delta \approx \alpha \bar\delta_{peak}$ in each load step of the crack growth analysis.

The command to request this adaptive load control for the cohesive crack growth
process is:
\begin{align*}
&\hv{\ul{adaptive}\ (\ul{load}) \ (\ul{control})}
\begin{Bmatrix}
\hv{\ul{on}} \\ \hv{\ul{off}}\end{Bmatrix}
\left(\hv{(\ul{max}imum)\ \ul{rela}tive\ \ul{displace}ment\ (\ul{change}) <real>}\right)
\end{align*}
where $\hv{<real>}$ denotes the value of $\alpha$. By default, 
$\Delta=0.2 \times \bar\delta_{peak}$ (0.2 as the input value above) 
and the adaptive load control algorithm is \ti{off}.

%
\noindent \ti{Notes:} (1) The adaptive load control 
here is independent of (and in addition to) the
adaptive load control implemented to assist convergence of the (global) Newton
iterations performed at each load step to solve the nonlinear equilibrium
equations; (2) Displacement extrapolation should not be used with adaptive load
control, \ti{i.e.}, set \ti{extrapolate off} in the nonlinear solution parameters.


The definitions of $\bar\delta$ and $\bar\delta_{peak}$ vary with the 
option selected for the formulation of the \ti{cohesive} material model
as defined here.

\noindent{\bf{Exponential option (\ti{exp1\_intf})}}

%
For the ``exponential" formulation (option \ti{exp1\_intf}) of the
\ti{cohesive} material model, we have: 
\begin{equation*}
\bar\delta = \sqrt{\beta^2 \left(\delta^2_{s1} + \delta^2_{s2}\right)
+ \delta^2_n }\ \ \rm{(Section\ 3.8\ uses\ the\ notation}\ \bar \Delta)
\end{equation*}
where $\beta$ is the user-specified mode-mixity weight factor, and $\delta_{s1},\ 
\delta_{s2}$, and $\delta_n$ are the two sliding and normal 
displacement jumps of the interface element. 

To obtain a single value of $\bar\delta$ for an interface element,
the average values of $\delta_{s1},\ 
\delta_{s2}$, and $\delta_n$ are first averaged over the element integration
points. The averaged values are then used in the above equation to
set $\bar\delta$.

The user specifies the value of $\bar\delta_{peak}$ 
as the \ti{cohesive} model
property \ti{delta\_peak}.

%
\noindent{\bf{Park-Paulino-Roesler option (\ti{ppr})}}

For the \ti{ppr} option of the
\ti{cohesive} material model, we compute at each integration point of an
interface element: 
\begin{equation*}
\bar\delta = \sqrt{\delta^2_t + \delta^2_n }\ ;\quad \text{where\ }
\delta^2_t=\delta^2_{s1} + \delta^2_{s2}
\end{equation*}
and $\delta_{s1},\ 
\delta_{s2}$, and $\delta_n$ are the two sliding and normal 
displacements at the integration point. Adaptive 
computations employ the  maximum value of $\bar\delta$
over the integration points for the element value. 
The value of $\bar\delta_{peak}$ with \ti{ppr} for adaptive load
decisions is found from
\begin{equation*}
\bar\delta_{peak} = \sqrt{\delta^2_{n-p} + \delta^2_{t-p}}\ .
\end{equation*}
\noindent Here, $\delta_{n-p}$ and $\delta_{t-p}$ are displacement jumps at
peak values of the normal and shear traction-separation
curves (see Section 3.8 and figures). 

The above definition for $\bar\delta_{peak}$
may become overly large for pure normal or pure shear loading.
In simple normal loading, for example, the user may specify a shear fracture energy and peak 
traction identical to that for the normal loading for convenience. 
Then $\delta_{t-p} = \delta_{n-p}$ above 
with $\bar\delta_{peak}= \sqrt{2} \delta_{n-p}$. The user may compensate for
this effect when present by specifying a smaller value for $\alpha$.
%
%*****************************************************
\subsection {Element Extinction}
%*****************************************************
The extinction of an interface element begins when a certain level of
deformation is reached (normal and/or sliding
displacements) that degrades severely the corresponding normal and/or
cohesive tractions. The conditions for extinction vary with the formulation
selected for the \ti{cohesive} material model associated with the
element. 

\noindent{\bf{Exponential option (\ti{exp1\_intf})}}

Extinction begins when the effective displacement 
$\bar\delta$ (defined in previous section) reaches a multiple ($\lambda_E>1$)
of the effective displacement, $\bar\delta_{peak}$, corresponding to the peak
value of the effective traction, $\bar T_p$. The user specifies both
$\bar\delta_{peak}$ and $\bar T_p$ through the \ti{cohesive} material properties
\ti{delta\_peak} and \ti{sig\_peak}.

The command to specify the value of $\lambda_E>1$ has the form
\begin{align*}
&\hv{ \ul{crit}ical\ (\ul{effe}ctive)\ \ul{cohes}ive\ (\ul{displace}ment)
\ (\ul{mult}iplier) <value> }
\end{align*}
The default value is $\lambda_E=5.74$.
The effective traction corresponding to this level of deformation should be a small fraction
of $\bar T_p$. For this exponential traction-separation model, a few values
of the effective traction ($\bar T$) for varying $\lambda_E$ values are:
(1) $\lambda_E=7.64$, $\bar T / \bar T_{p}=0.01$; 
(2) $\lambda_E=6.83$, $\bar T / \bar T_{p}=0.02$;
(3) $\lambda_E=5.74$, $\bar T / \bar T_{p}=0.05$;  
(4) $\lambda_E=4.9$, $\bar T / \bar T_{p}=0.10$. 


\noindent{\bf{Park-Paulino-Roesler option (\ti{ppr})}}

Extinction begins when the normal, $\delta_n$, and/or 
(resultant) sliding, $\delta_t$, displacement jumps at \ti{every} integration
point of the element exceed a specified fraction ($\lambda_E<1$) of the
limiting displacements, $\delta_n$ and $\delta_t$,
at which the normal and shear tractions degrade to zero
without consideration of mode-interaction effects (see Section
3.8 description and figures). The values of 
$\delta_n$ and $\delta_t$ are computed by the the \ti{cohesive}
model routines from other user-specified properties of the \ti{ppr} option.
The command to specify the value of $\lambda_E<1$ has the form
\begin{align*}
 &\hv{ \ul{ppr}\ (\ul{displace}ment)\ (\ul{fraction})\ (\ul{for})\  
 \ul{extinction} <value> }
\end{align*}
\noindent The default value is $\lambda_E=0.9$.
%
%*****************************************************
\subsection{Extinction Algorithm}
%*****************************************************
At the beginning of each load step $n$ with $n>1$, and each adaptive sub-step, 
the state of each interface-cohesive element is assessed relative to a user-specified
terminal state (see discussions in this section about terminal states for
each available cohesive formulation).  When the element conditions are such to
require extinction, the following actions are taken:
%
\small
\squishlist
\item The element history data are deleted (prior maximum displacements, tractions,
etc.).
\item Remaining element contributions to the global internal force vector are applied as
nodal forces. All subsequent contributions of the element to global equilibrium
are zero. The element internal force vector when extinction begins is gradually
decreased to zero in a linear fashion over subsequent load steps. Because the
element forces are converted into nodal forces and treated thereafter as
ordinary (user-specified) forces, the (global Newton) adaptive step algorithm is
unaffected by crack growth and often proves essential for obtaining converged
solutions following a growth increment.
\item All subsequent computations for the element stiffness (linear or
tangent) resolve to a zero matrix.
\item 	When all elements connected to a node are made extinct, the node has no
stiffness and introduces a singularity into subsequent equation solving efforts.
To prevent this, the element extinction procedures track the number of elements
attached to each node of the model and automatically supply new constraints on
``free" nodes to eliminate the singularity.
\item The blocking requirements dictate that all elements in a block must have
the property \ti{killable}. When a new element is made 
extinct in a block, checks are made to
determine if all elements in the block have been made extinct; computations on
such blocks may be completely skipped in subsequent load step solutions.
\squishend
\normalsize
%
%
%*****************************************************
\subsection{Release of Extincted, Interface-Element Forces}
%*****************************************************
The remaining (small) internal nodal forces for a 
newly extinct interface element are decreased to zero over a fixed
number of load steps. The user sets the number of steps with the commands
\begin{align*}
&\hv{ \ul{force}\ (\ul{rele}ase)\ (\ul{type})\ \ul{step}s} \\
&\hv{\ul{rele}ase\ (\ul{step}s)\ <integer>} 
\end{align*}
This force release model is the default option and can be used to render extinct
any element in the model, whether or not it lies on the crack plane. The default
value is 5 steps. The number of release steps \ti{cannot} be altered once any
elements have been made extinct.
%
%
%*****************************************************
\subsection{Examples}
%*****************************************************
Complete examples of input to set-up crack growth using 
interface elements and the cohesive material model are:
\small
\begin{verbatim}
   crack growth parameters  $ exponential cohesive option
     type of crack growth cohesive
     critical effective cohesive displacement multiplier 5.74
     force release type steps
     release steps 10
     print status on order elements 36000-34001 by -1
     adaptive load control on maximum relative displacement change 0.25
\end{verbatim}
\normalsize
\small
\begin{verbatim}
   crack growth parameters $ ppr cohesive option
     type of crack growth cohesive
     ppr fraction for extinction 0.95
     force release type steps
     release steps 10
     print status on order elements 36000-34001 by -1
     adaptive load control on maximum relative displacement change 0.30
\end{verbatim}
\normalsize


\end{document}