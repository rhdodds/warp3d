%
\documentclass[11pt]{report}
\usepackage{geometry} 
\geometry{letterpaper}

%---------------------------------------------
\setlength{\textheight}{630pt}
\setlength{\textwidth}{450pt}
\setlength{\oddsidemargin}{14pt}
\setlength{\parskip}{1ex plus 0.5ex minus 0.2ex}

%----------------------------------------
\usepackage{amsmath}
\usepackage{layout}
\usepackage{color}

%----------------------------------------------
\usepackage{fancyhdr} \pagestyle{fancy}
\setlength\headheight{15pt}
\lhead{\small{User's Guide - \textit{WARP3D}}}
\rhead{\small{Material: \textit{cyclic}}}
\fancyfoot[L] {\small{\textit{Chapter {\thechapter}}\ \   (Updated: 8-20-2011)}}
\fancyfoot[C] {\small{\thesection-\thepage}}
\fancyfoot[R] {\small{\textit{Elements and Material Models}}}

%---------------------------------------------------
\usepackage{graphicx}
\usepackage[labelformat=empty]{caption}
\numberwithin{equation}{section}
\usepackage{epstopdf}

%---------------------------------------------
%     --- make section headers in helvetica ---
\usepackage{sectsty} 
\usepackage{xspace}
\allsectionsfont{\sffamily} 
\sectionfont{\large}
\usepackage[small,compact]{titlesec} % reduce white space around sections
%----------------------------------------------
%
%
%   which fonts system for text and equations. with all commented,
%   the default LaTex CM fonts are used
%
%
\frenchspacing
%\usepackage{pxfonts}  % Palatino text 
%\usepackage{mathpazo} % Palatino text
%\usepackage{txfonts}


%---------  local commands ---------------------

\newcommand{\bmf } {\boldsymbol }
\newcommand{\bsf } [1]{\textrm{\textit{#1}}\xspace}
\newcommand{\eg}{\emph{e.g.},\xspace}
\newcommand{\ie}{\emph{i.e.},\xspace}
%
%
%        optional definition for bullet lists which
%        reduces white space.
%
\newcommand{\squishlist}{
 \begin{list}{$\bullet$}
  { \setlength{\itemsep}{0pt}
     \setlength{\parsep}{3pt}
     \setlength{\topsep}{3pt}
     \setlength{\partopsep}{0pt}
     \setlength{\leftmargin}{1.5em}
     \setlength{\labelwidth}{1em}
     \setlength{\labelsep}{0.5em} } }

\newcommand{\squishlisttwo}{
 \begin{list}{$\bullet$}
  { \setlength{\itemsep}{0pt}
     \setlength{\parsep}{0pt}
    \setlength{\topsep}{0pt}
    \setlength{\partopsep}{0pt}
    \setlength{\leftmargin}{2em}
    \setlength{\labelwidth}{1.5em}
    \setlength{\labelsep}{0.5em} } }

\newcommand{\squishend}{
  \end{list}  }
%


%-------------------------------------
\newcounter{sectrefs}
\setcounter{sectrefs}{0}
\setcounter{chapter}{3}
\setcounter{section}{9}
\renewcommand{\thefigure}{\thesection.\arabic{figure}}

%--------------------------------------
%--------------------------------------
%---------------------------------------

\begin{document}

\section{Material Model Type: \textit{cyclic}}
%\layout
\noindent This material model provides nonlinear stress-strain response to describe 
the cyclic plasticity behavior in metals. 
Based on Mises elasto-plasticity, the model approximates the Bauschinger effect, 
ratcheting behavior, and stress-relaxation. At present, this 
material model is strain-rate independent. The 
model employs  an additive decomposition of the strain increments 
for stress updating, and accommodates both small-strain and finite strain 
deformation levels through techniques outlined in Chapter 1. 

Two options exist to define the stress-strain response, \textit{nonlinear\_hardening}
and \textit{generalized\_plasticity} with the following characteristics.
\\ \\
\noindent \textit{\textbf{nonlinear\_hardening}}
\small
\begin {itemize}
\item Nonlinear kinematic hardening following the Armstrong and Frederick 
[\ref{R:FA1965}] rule for evolution of the backstress 
\item Nonlinear isotropic hardening via a saturating exponential relationship between 
the radius of the Mises yield cylinder and the accumulated plastic strain
\item Asymptotic, perfectly-plastic behavior at large plastic strains
\item Sharp yield point (discontinuous tangent)
\item Material properties invariant of temperature
\end{itemize}
\normalsize


\noindent \textit{\textbf{generalized\_plasticity}}
\small
\begin {itemize}
\item Linear kinematic hardening
\item Linear isotropic hardening
\item Nonlinear limit function for admissibility provides nonlinear stress-plastic strain response
\item Asymptotically perfectly plastic or linear hardening behavior at large plastic strains
\item Soft yield point (continuous tangent)
\item Model of Lubliner, Taylor, and Auricchio [\ref{R:LTA1993}]
\item Less computational effort for local stress updating than 
\textit{nonlinear\_hardening} option
\item Temperature-dependent material properties and material response 
\end{itemize}
\normalsize

For both options, the discrete nonlinear equations reduce to a single scalar 
equation for the unknown consistency parameter which is solved in closed-form 
for the \textit{generalized\_plasticity } option and via Newton's method for the 
\textit{nonlinear\_hardening} option. For both options, an automatic 
subincrementation procedure 
improves the accuracy of the updated stresses by subdividing the current 
estimate of the total strain increment over the load step, with stress 
updates executed in the material model successively for each subincrement 
of strain. The strain path for the load step at the material point 
remains proportional (straight) but the stress path may become 
curved with proportional segments computed over each subincrement. The 
adopted number of subincrements relies on a truncation error estimating 
procedure and a user specified tolerance. A cap is placed on the 
maximum number of subincrements in the \textit{generalized\_plasticity} model.

The model offers two approaches to describe the production of thermal strain 
increments in response to temperature changes imposed in the analysis: 
isotropic and anisotropic. Isotropic refers to the conventional description 
which uses the same thermal expansion coefficient (alpha) for each 
normal strain component with zero thermal strain increments generated 
for the shear components (this thermal expansion coefficient 
may vary with temperature for the
\textit{generalized\_plasticity} option -- see Section 2.2). The anisotropic model enables 
definition of a unique thermal expansion 
coefficient for each of the 6 strain components; it is intended to 
support modeling of various initial strain-stress fields, for example, residual 
stresses imposed through an eigenstrain approach. The anisotropic coefficients 
do not vary with temperature. The isotropic model for thermal loading can 
be used in small displacement, large displacement and finite strain analyses. 
The anisotropic model for thermal loading should generally not 
be used in large displacement-rotation analyses as the anisotropic 
coefficients are not rotation neutralized (use of anisotropic 
coefficients to model initial residual stresses 
when the initial displacements and rotations remain small is acceptable).

The following sections discuss the model capabilities and features in detail, 
including the needed input parameters. The 
next section describes the numerical implementation. The final 
section reduces the 3D constitutive equations to 1D forms.

%*****************************************************
\subsection{Stress-Strain Curves}
%*****************************************************
Despite the differences noted above and the distinct governing equations, 
both options have qualitatively similar stress-strain response under 
cyclic loading. Both options of the \textit{cyclic} model intentionally reproduce 
strain ratcheting under nonsymmetric stress cycling and stress relaxation 
under nonsymmetric strain cycling, even under uniaxial and 
plane-stress conditions. Figure \thesection.1 shows examples of this behavior.

\subsubsection {The \bsf{nonlinear\_hardening} option}
Before yielding occurs and during inelastic unloading, the material behavior follows 
a linear-elastic response controlled by the elastic modulus and Poisson's ratio 
($E$ and $\nu$). When plastic flow occurs, the model provides key cyclic plasticity 
features through nonlinear kinematic hardening. Figure \thesection.2, based on the 
work of Lemaitre and Chaboche[\ref{R:LC2000}], illustrates the effect of nonlinear 
kinematic hardening on both the three-dimensional behavior in deviatoric 
stress space and the uniaxial tension-compression behavior. In the 
figure, $\bmf \sigma$ denotes the current stress state, the center of the yield 
surface (the backstress) is $\bmf\alpha$, and $k$ is the current radius of the yield surface, 
shown here to be constant. $H$ and $\gamma$ represent material constants 
that govern the kinematic hardening behavior. The nonlinear kinematic 
hardening reflects a yield surface that translates within a limit surface; this 
manifests as exponential saturation in a uniaxial framework. 

%--------------------------
\begin{figure}[htb]
\begin{center}
\includegraphics[scale=0.7,angle=-90]{Figure_1.eps} 
\caption*{\small 
Fig. \thesection.1: Capabilities of this cyclic plasticity model: (a) stress relaxation
for cyclic non-symmetric strain loading, (b) ratcheting caused by cyclic, non-symmetric 
stress loading.\normalsize}
\label{F:cyclic-capabilities}
\end{center}
\end{figure}
%--------------------------

The addition of isotropic hardening may improve the simulation of some 
materials. Isotropic hardening is implemented as a saturating exponential relationship 
between the current radius of the yield surface and the accumulated plastic strain 
(shown in Fig.\;\thesection.3). Here $\sigma_{ys}$ denotes the tensile yield stress; $Q_u$
($+ \sigma_{ys}$) sets the saturation flow stress in 1-D; and $b_u$ governs the rate at which the
saturation stress is reached with the 1-D, accumulated plastic strain
$\bar \epsilon^p_u$.  The 
tensile yield stress and the initial yield surface radius ($k_0$) are related by 
$k_0 =  \sqrt{2/3}\; \sigma_{ys}$. At initial yield, the 1-D exponential
hardening model has a plastic modulus of $b_u \cdot Q_u$, a quantity
often referred to as $H'$.

\subsubsection {The \bsf{generalized\_plasticity} option}
Before yielding occurs and during 
unloading, the material behavior follows a linear-elastic response controlled 
by the elastic modulus and Poisson's ratio ($E$ and $\nu$).  However, 
the elastic domain does not define 
admissibility; an additional surface acts as the limit function. Thus, plastic flow 
may occur immediately upon reloading if reverse loading has not returned 
the state to the elastic domain.  Note that, unlike the limit 
surface depicted for nonlinear kinematic hardening in Fig.\;\thesection.2, 
this limit surface is explicitly specified and directly governs the 
nonlinearity of stress-strain response.

Figure \thesection.4 presents the uniaxial stress-strain response of the generalized 
plasticity model with \emph{kinematic} hardening.  In the figure,  $\sigma_{ys}$ is the 
uniaxial (tensile) yield stress, $\beta_u$ and $\delta_u$ are 1-D specializations of the
3-D parameters ($\beta, \delta$) that describe the 
evolution of the limit surface, and $(1-\tau)H_u$ is the 1-D kinematic hardening modulus.  
Optional isotropic hardening, with 1-D constant 
modulus $\tau H_u$, is not shown in the figure. The 
isotropic hardening is simply a linear relationship between the 
radius of the yield surface and a measure of the accumulated plastic strain.
The material properties 
listed above -- $E$, $\nu$, $\sigma_{ys}$, $H_u$,  $\beta_u$, and $\delta_u$ --
have a piecewise linear, temperature-dependent response in 
the generalized plasticity model. The fraction of hardening modulus
allocated to isotropic hardening $(\tau)$ and to kinematic
hardening $(1-\tau)$ does not vary with temperature.

%--------------------------
\begin{figure}[htb]
\begin{center}
\includegraphics[scale=0.85,angle=-90]{Figure_2.eps} 
\caption*{\small Fig. \thesection.2: Illustration of the \textit{nonlinear\_hardening} option for 
in both three-dimensional space and uniaxial tension-compression,
after Lemaitre and Chaboche [\ref{R:LC2000}]. Note the
resumption of plastic deformation on reloading
from B only after reaching prior maximum stress at A. 
Also the limiting condition can only have a horizontal asymptote. \normalsize}\label{fig10}

\end{center}
\end{figure}
%--------------------------

%*****************************************************
\subsection {Model Properties}
%*****************************************************
Tables \thesection.1 and \thesection.2 list the properties defined for the 
cyclic material model. The material properties are summarized below.
The property keywords \textit{nonlinear\_hardening} and 
\textit{generalized\_plasticity}
have no associated values. One of them must be specified before the other
property values are listed.

The mathematical formulations and 
computational code for both options employ convenient, 3-D
definitions for the various material constants that appear throughout.
When the 3-D tensorial definitions are specialized to 1-D as needed
to use simple tests for calibration of the parameter values, the 1-D
"equivalent" formulation values and the corresponding 3-D
values are generally are scaled by $3/2$, $\sqrt{2/3}$, etc.

Users input values for the 1-D, uniaxial 
specializations of the 3-D material parameters. The code converts these values
to their 3-D definitions for use in computations.
%--------------------------
\begin{figure}[htb]
\begin{center}
\includegraphics[scale=1.0,angle=-90]{Figure_3.eps} 

\caption*{\small Fig. \thesection.3: Saturating exponential, isotropic hardening for the
\textit{nonlinear\_hardening} option (1-D formulation).
\normalsize}

\end{center}
\end{figure}
%--------------------------


\subsubsection{\underline{\bsf{nonlinear\_hardening} option}}

\noindent Elastic and yield parameters:
\begin {itemize}
\item $E$, $\nu$, $\alpha$ (E, nu, alpha): Young's modulus, Poisson's ratio, thermal expansion coefficient
(or anisotropic expansion coefficients). The properties E, nu, alpha maybe temperature 
dependent (see Section 2.2).
\item $\sigma_{ys}$ (yld\_pt): tensile yield stress upon initial loading 
$( k_0= \sqrt{2/3} \; \sigma_{ys})$.
\end{itemize}

\noindent Parameters for isotropic hardening:
\begin {itemize}
\item q\_u (1-D): Sets the maximum (saturation) tensile stress reached as
$\sigma_{ys} +$ q\_u. Set q\_u$= 0$ for 
no isotropic hardening. $q_{\textrm{3D}}= \sqrt{2/3}\;q_u$.
\item b\_u (1-D): Sets the rate at which the saturation tensile stress 
is reached with accumulated (uniaxial) plastic strain. $b_{\textrm{3D}} =
\sqrt{2/3}\;b_u$.
\end{itemize}

\noindent Parameters for kinematic hardening. 
\begin{itemize}
\item h\_u (1-D): Along with $\gamma_u$, sets the radius of the limit surface. Sets 
the kinematic hardening modulus at the onset of plastic flow 
$(\bmf{\alpha} = \bmf{0})$. $h_{\textrm{3D}} = (2/3) h_u$.
\item gamma\_u (1-D): Along with h\_u, sets the radius 
of the limit surface. Controls the nonlinearity of the backstress evolution.
$\gamma_{\textrm{3D}} = \sqrt{2/3}\;\gamma_u$.
\end{itemize}

\setlength{\hangindent}{0.75in}
\textit{Notes}: Set $h_u = \gamma_u = 0$  for no kinematic hardening. Set 
$\gamma_u = 0, h_u > 0$ to model simple bilinear kinematic hardening.
In 1-D uniaxial loading (no isotropic hardening), the limiting tensile stress is
$\sigma_u^{max} = \sigma_{ys} + h_u / \gamma_u$.

%=================================================
\begin{table}[htb]	
\small
	\centering
%		\begin{tabular}{ | l | c |  c | c | }
		\begin{tabular}{ |p{2.5in} | p{1.5in} | p{0.8in} |p{0.7in} | }
		\hline
		Model Property & Keyword & Mode & Default Value \\
		\hline \hline
Request \textit{nonlinear\_hardening} option &nonlinear\_hardening &\  &\  \\ \hline
Young's modulus	& e	& number	& 30,000 \\ \hline
Poisson's ratio	& nu	 & real	& 0.3 \\ \hline
Mass density	& rho & 	number	& 0.0 \\ \hline
Thermal expansion (isotropic)	& alpha & 	number	& 0.0 \\ \hline
Thermal expansion (an-isotropic) & alphax, alphay, alphaz, alphaxy, alphaxz, alphayz & number & 0.0 \\ \hline
Tensile yield stress	& yld\_pt	& number	& 0.0 \\ \hline
Kinematic hardening parameter (1-D)	& h\_u	& number	& 0.0 \\ \hline
Kinematic hardening parameter (1-D)	& gamma\_u	 & number	& 0.0 \\ \hline
Isotropic hardening modulus (1-D) & 	q\_u & 	number	& 0.0 \\ \hline
Isotropic hardening exponent (1-D) & 	b\_u	 & number	& 0.0 \\ \hline
Subincrementation tolerance	& sig\_tol & 	number &	0.001 \\ \hline
		\end{tabular}
  \caption{\small Table \thesection.1 
Properties for \textit{cyclic} material model: \textit{nonlinear\_hardening} option. (property names are 
not case sensitive)
\normalsize}
	\label{table:nonlinear_hardening}
\end{table}
%=====================================================


\subsubsection{\underline{\bsf{generalized\_plasticity} option}}

\noindent Elastic and yield parameters (see later notes about temperature dependence):
\begin {itemize}
\item $E$, $\nu$, $\alpha$ (E, nu, alpha): Young's modulus, Poisson's ratio, thermal expansion coefficient
(or anisotropic expansion coefficients).
\item $\sigma_{ys}$ (yld\_pt): tensile yield stress upon initial loading 
$( k_0= \sqrt{2/3} \; \sigma_{ys})$.
\end{itemize}


\noindent Hardening and limit surface parameters. Users input values for the 1-D, uniaxial 
specialization of the 3-D formulation:
\begin {itemize}
\item gp\_h\_u (1-D): Sets the terminal hardening rate of the generalized plasticity model under
monotonic uniaxial loading. 
\item gp\_tau: Sets the (temperature invariant) fraction of isotropic to kinematic hardening: 
\begin{equation}
h_{iu} = \tau h_u ~,
\end{equation}
\begin{equation}
h_{ku} = \left( 1-\tau \right) h_u ~.
\end{equation}
\item gp\_beta\_u (1-D): Sets the asymptotic yield stress due to the limit 
surface evolution (hardening further changes the yield stress 
but in a constant, non-saturating manner). $\beta_{\textrm{3D}} = \sqrt{2/3} 
\;\beta_u$.
\item gp\_delta\_u (1-D): Along with $\beta$, sets the rate at which 
asymptotic behavior is approached. $\delta_{\textrm{3D}} = (2/3) \delta_{iu}$.
\end{itemize}

The material properties E, nu, alpha, gp\_sigma\_0 (yld\_pt), gp\_beta\_u, gp\_delta\_u, and
 gp\_h\_u may all be temperature dependent (see Section 2.2 for specification of values).
 
\textit{Notes}: To degenerate the generalized 
plasticity model to the bilinear
plasticity model set gp\_beta\_u = gp\_delta\_u = 0. For 
isotropic hardening only,  set gp\_tau=1;
for kinematic hardening only, set gp\_tau = 0; and for uniaxial loading with 
H=0, the limiting stress is $\sigma_u^{max} = \sigma_{ys} + \beta_u$ 
and $\delta_u >0$ sets the rate at which
the limit stress is reached with plastic straining.


%==========================================================
\begin{table}[htb]	
\small
	\centering
%		\begin{tabular}{ | l | c |  c | c | }
		\begin{tabular}{ |p{2.5in} | p{1.5in} | p{0.8in} |p{0.7in} | }
		\hline
		Model Property & Keyword & Mode & Default Value \\
		\hline \hline
Request \textit{generalized\_plasticity} option &generalized\_plasticity &\  &\  \\ \hline
Young's modulus	& e	& number	& 30,000 \\ \hline
Poisson's ratio	& nu	 & real	& 0.3 \\ \hline
Mass density	& rho & 	number	& 0.0 \\ \hline
Thermal expansion (isotropic)	& alpha & 	number	& 0.0 \\ \hline
Thermal expansion (anisotropic) & alphax, alphay, alphaz, alphaxy, alphaxz, alphayz & number & 0.0 \\ \hline
Tensile yield stress (1-D)	& yld\_pt	& number	& 0.0 \\ \hline
Terminal hardening modulus (1-D)	&  gp\_h\_u	& number	& 0.0 \\ \hline
Fraction of $H_u$ for isotropic hardening & gp\_tau & number &  0.0 \\ \hline
Limit function parameter (1-D) & 	gp\_beta\_u & 	number	& 0.0 \\ \hline
Limit function parameter (1-D) & 	gp\_delta\_u	 & number	& 0.0 \\ \hline
Subincrementation tolerance	& sig\_tol & 	number &	0.001 \\ \hline
		\end{tabular}
  \caption{\small Table \thesection.2 
Properties for \textit{cyclic} material model: \textit{generalized\_plasticity} option. (property names are 
not case sensitive)
\normalsize}
	\label{table:generalized_plasticity}
\end{table}
%=========================================================

%*****************************************************
\subsection {Model Output}
%*****************************************************
By default, the material model prints no messages during computations. If 
requested, the material model prints: (1) the element number and strain 
point number whenever the effective stress first exceeds the 
specified yield stress, and (2) the 
number of subincrements required at each Gauss point 
undergoing plastic yielding. An excessive number of 
subincrements may indicate that the global step size is too large. 
This option for detailed printing is requested with the nonlinear solution 
parameter \textit{material messages on} (refer to Section 2.10.8).

The model makes available the accumulated work density, $U_0$, at 
each Gauss point to the element routines for subsequent
output. $U_0$ at step $n+1$ is evaluated using the trapezoidal rule 

\begin{equation}
U^{n+1}_0 = U^{n}_0 + \textstyle \frac{1}{2} 
\displaystyle  \left ( \bmf t^{n+1} + 
\bmf t^n \right )
: \Delta \bmf d
\end{equation}

\noindent where the unrotated Cauchy stresses ($\bmf t $) and unrotated 
strain increments ($\bmf d$) are adopted for the finite-strain
formulation.

The element stress output contains three values for the material model ``state" variables.
These are:
\squishlist
\item
\noindent \textit{mat\_val1}: a measure of the accumulated plastic strain over loading history,
\begin{equation}
\bar e^p = \int^t_0 \sqrt{\dot \epsilon^p_{ij}\dot \epsilon^p_{ij}   } \thinspace dt
\end{equation}
\item
\noindent \textit{mat\_val2}: current radius of the yield surface, $k$,
\begin{alignat}{1}
\text{\textit{nonlinear\_hardening} option:  } k &=\textstyle \sqrt{\frac {2}{3}}
 \displaystyle \thinspace 
 \left [ \sigma_{ys} + Q_u   \left( 1 - exp  
 \left [-b_u \bar \epsilon^p_u \right ] \right) \right ] \\
\text{\textit{generalized\_plasticity} option:  } k &= 
\textstyle \sqrt{\frac{2}{3}}  \displaystyle \thinspace 
 \left ( \sigma_{ys} + H_{iu} \bar\epsilon^p_u \right )
 \end{alignat}
\item
\noindent \textit{mat\_val3}: state flag, =1 if point is experiencing active 
plastic loading over the step, = 3 otherwise.
\squishend
For reference, the corresponding 1-D value of the accumulated plastic strain is given by
\begin{equation}
\bar \epsilon^p_u = \int^t_0 |\dot \epsilon^p_{11}| dt =
 \sqrt{\textstyle{\frac{2}{3}}}\; \bar e^p~,
\end{equation}
and the current Mises equivalent stress is

\begin{equation}
\sigma_{vm} = \sqrt{\textstyle{\frac{3}{2}}}\;k~.
\end{equation}

%--------------------------
\begin{figure}[htb]
\begin{center}
\includegraphics[scale=1.0,angle=-90]{Figure_4.eps} 

\caption*{\small Fig. \thesection.4: Illustration of \textit{generalized\_plasticity} 
model with kinematic hardening in 
uniaxial tension-compression, after Auricchio and Taylor [\ref{R:AT1995}]. Note: (1) the
abscissa is plastic strain; (2) resumption of plastic deformation (C) on 
reloading (from B) prior to attainment
of the previous maximum stress (A). The limit asymptote is $H_{ku}=(1-\tau)H_u$, which
can be set to zero to match the large-strain behavior of the
\textit{nonlinear\_hardening} option.
\normalsize}

\end{center}
\end{figure}
%--------------------------
%*****************************************************
\subsection {Element Blocking}
%*****************************************************
As described in Section 2.6, elements are assigned to blocks
to facilitate parallel processing. For the computational material model
$cyclic$,
all elements in a block must refer to the same material defined
in an analysis model.

Consider an analysis model with 3 materials defined in the input
named $mat\_a$, $mat\_b$ and $mat\_c$. Each of these materials specifies
$cyclic$ as the computational material model. Different values for each of the
$cyclic$ properties  (\eg $yld\_pt$)may be defined in the materials
$mat\_a$, $mat\_b$ and $mat\_c$.  The blocking requirement
described here requires that all elements assigned to a block have the same
associated material, either $mat\_a$, $mat\_b$ and $mat\_c$.


%*****************************************************
\subsection {Examples}
%*****************************************************
With only a small number of parameters needed to characterize the nonlinear
behavior, these models are relatively straightforward to calibrate. However, the material parameters
may not remain generally applicable for loading paths that differ considerably from the data
used for calibration, \textit{e.g.}, significantly larger or smaller strain ranges
than those applied during calibration tests.

\subsubsection{\underline{\bsf{nonlinear\_hardening} option}}

This example defines a material with both nonlinear kinematic and nonlinear isotropic
hardening (note: \textit{material property names are case insensitive, upper-case letters
may be use to improve readability of input files}).

\small
\begin{verbatim}
   structure cct
  c
    material steel_w_mixed_hardening
      properties cyclic type nonlinear_hardening e 30000 nu 0.3 yld_pt 60,
         q_u 80 b_u 0.01 gamma_u 3000 H_u 5000 sig_tol 0.001
 \end{verbatim}
 \normalsize
 
\noindent This next example defines a material with only nonlinear kinematic hardening.
 \small
 \begin{verbatim}
    material steel_w_kinematic_hardening
      properties cyclic type nonlinear_hardening e 30000 nu 0.3 yld_pt 60,
        q_u 0.0 b_u 0.0 gamma_u 6000 H_u 5000 sig_tol 0.001
 \end{verbatim}
 \normalsize
 
\noindent The following example defines a material with only nonlinear isotropic hardening.
 \small
 \begin{verbatim}
    material isotropic_hardening
      properties cyclic type nonlinear_hardening e 30000 nu 0.3 yld_pt 60.0, 
            q_u 80 b_u 0.01 gamma_u 0.0 H_u 0.0 sig_tol 0.001
 \end{verbatim}
 \normalsize



\subsubsection{\underline{\bsf{generalized\_plasticity} option}}

The following example defines a material with nonlinear limit behavior 
and both linear kinematic and linear isotropic hardening.

\small
 \begin{verbatim}
    material mixed_hardening
      properties cyclic type generalized_plasticity e 30000 nu 0.3 yld_pt 60.0,
           gp_beta_u 100.0 gp_delta_u 20000 gp_h_u 400.0 gp_tau 0.25 sig_tol 0.001
\end{verbatim}
 \normalsize
 
\noindent The following example defines a material with only nonlinear limit behavior and linear kinematic hardening.
\small
 \begin{verbatim}
     material kinematic_hardening
       properties cyclic type generalized_plasticity e 30000 nu 0.3 yld_pt 60.0,
           gp_beta_u 100.0 gp_delta_u 20000 gp_h_u 300.0 gp_tau 0.0 sig_tol 0.001
\end{verbatim}
 \normalsize
 
\noindent The following example defines a material with only nonlinear limit behavior and linear isotropic  hardening.
\small
 \begin{verbatim}
    material isotropic_hardening
      properties cyclic type generalized_plasticity e 30000 nu 0.3 yld_pt 60.0,
           gp_beta_u 100.0 gp_delta_u 20000 gp_h_u 100.0 gp_tau 1.0 sig_tol 0.001
\end{verbatim}
 \normalsize
 
\noindent The following example defines a material with only nonlinear limit behavior (no hardening.)
\small
 \begin{verbatim}
    material no_hardening
      properties cyclic type generalized_plasticity e 30000 nu 0.3 yld_pt 60.0,
           gp_beta_u 100.0 gp_delta_u 20000 gp_h_u 0.0 gp_tau 0.5 sig_tol 0.001
\end{verbatim}
\normalsize

\noindent The following example defines a material with temperature-dependent behavior 
using stress-strain curves at three isothermal temperatures (see also Section 2.2)
\small
 \begin{verbatim}
c
 stress-strain curve 1 temperature 100 e 30000 nu 0.3 alpha 0.0001,
   gp_sigma_0 60.0 gp_beta_u 100.0 gp_delta_u 10000.0 gp_h_u 1000.0
c
 stress-strain curve 2 temperature 300 e 25000 nu 0.25 alpha 0.0002,
  gp_sigma_0 50.0 gp_beta_u 80.0 gp_delta_u 8000.0 gp_h_u 700.0
c
 stress-strain curve 3 temperature 500 e 20000 nu 0.2 alpha 0.0005,
  gp_sigma_0 35.0 gp_beta_u 75.0 gp_delta_u 5000.0 gp_h_u 250.0
c
c
 material gp_temp_dep
    properties cyclic type generalized_plasticity curves 1-3,
     rho 7.3e-7 gp_tau 0.2 sig_tol 0.001
 \end{verbatim}
\normalsize


\subsubsection {Example calibration of an Inconnel 718 alloy}
During development of this model, room temperature cyclic loading tests of
an Inconel 718 alloy provided a specific data set for trial calibration of the
parameters. For the \textit{nonlinear\_hardening} option, the parameters 
are: e = 185 GPa, nu = 0.3, yld\_pt = 0.674 GPa, 
h\_u = 412.5 GPa, gamma\_u = 735,
q\_u = 0, and b\_u = 0. For the \textit{generalized\_plasticity} option the
parameters are: e = 185 GPa, nu = 0.3, yld\_pt = 0.8 GPa, gp\_beta\_u = 0.51,
gp\_delta\_u = 38.3, gp\_h\_u=20.56 GPa, and gp\_tau=0. 

%*****************************************************
\subsection {Plasticity Algorithms - Overview}
%*****************************************************
The material model employs a strain driven algorithm for stress updating, 
based on the current increment of strain, $\Delta \bmf \epsilon =
 \bmf \epsilon^i_{n+1}
- \bmf \epsilon_n$, where $n$ is the previously 
converged global state and $i$ is the current global iteration for step $n+1$. 
The material model uses a subincrementation procedure to improve the 
accuracy of the stress update, the first step of which computes the fraction 
of the step, if any, that remains purely linear-elastic. This procedure passes a 
deviatoric strain increment to the actual plasticity algorithm which then 
solves for the discrete consistency parameter, $\lambda \Delta t = \bar \lambda$. 
The stress and history 
variables are then updated from the computed $\bar \lambda$. The material model 
also provides an approximation for the consistent tangent -- the 
subincrementation process introduces some deviation from a true 
consistent tangent. The material models and the discrete algorithms 
are discussed next, independent of the subincrementation procedure. 

%*****************************************************
\subsection {\bsf{nonlinear\_hardening} option}
%*****************************************************
\subsubsection {Yield function and flow rule}
\noindent The yield function is based on the standard, pressure invariant 
von Mises surface. Let $\delta\bmf{\epsilon}$, $\bmf{\sigma}$, 
and $\bmf{\alpha}$ be the strain increment, stress, and the backstress 
respectively, all symmetric, second order tensors. Defining the 
deviator tensors and tensor norms as
\begin{equation}\label{E:deviators_define}
(){'_{ij}} = () - \frac{{{{()}_{kk}}}}{3}{\delta _{ij}}, 
\quad  \parallel ( \ ) \parallel = \sqrt{ ( \ )_{ij}( \ )_{ij}}
\end{equation}
%
the yield surface is described by
\begin{equation}\label{E:ys_1}
f(\bmf{\sigma}, \bmf{\alpha}, \bar e^p)=
\parallel \bmf{\sigma}' -\bmf{\alpha} \parallel - k(\bar e^p).
\end{equation}
%
The scalar variable $\bar e^p$ denotes a measure the accumulated plastic strain, 
based on additive decomposition of the strain tensor into elastic and plastic parts as,
\begin{equation}
\bmf{\epsilon}'=(\bmf{\epsilon}')^e+(\bmf{\epsilon}')^p~.
\end{equation}
%
The accumulated plastic strain is then a monotonically increasing 
function of time, such that 
\begin{equation}\label{E:equiv_pl_strain}
\dot{\bar e}^p=\parallel \left( \dot{\bmf{\epsilon}}' \right) ^p \parallel~;
\quad \bar e^p = \int^t_0 \dot{\bar e}^p \thinspace dt~.
\end{equation}
%
Applying an associative flow rule, the plastic strain rate is written as
\begin{equation}
\left( \dot{\bmf{\epsilon}}' \right) ^p =
\dot{\lambda}\bmf{n}, \quad
\parallel \left( \dot{\bmf{\epsilon}}' \right) ^p \parallel = 
\dot{\lambda}, \quad \dot{\lambda} \geq 0~,
\end{equation}
%
where $\lambda$ defines the unknown consistency parameter 
and $\bmf{n}$ is the unit normal for the yield surface,
\begin{equation}\label{E:n_define}
\bmf{n}=\displaystyle \frac{\bmf{\sigma}'-
\bmf{\alpha}}{\parallel \bmf{\sigma}'-\bmf{\alpha} \parallel}~,
\quad \parallel \bmf{n} \parallel \equiv 1 ~.
\end{equation}
%
The governing equation for the deviatoric stress rate, $\dot{\bmf{\sigma}}'=
2G\left( \dot{\bmf{\epsilon}}'\right)^e$, becomes

\begin{equation}\label{E:dev_stress_rate}
\dot{\bmf{\sigma}}'=2G \left[ \dot{\bmf{\epsilon}}' - 
\left(\dot{\bmf{\epsilon}}' \right) ^p \right] =
2G \left( \dot{\bmf{\epsilon}}' - \dot{\lambda}\bmf{n} \right)~.
\end{equation}
%

\subsubsection {Strain Hardening}

\noindent The \textit{nonlinear\_hardening} option features both 
kinematic and isotropic hardening.
The evolution of the backstress governs the kinematic hardening as
%
\begin{equation}\label{E:alpha_dot}
\dot{\bmf{\alpha}}=
\dot{\lambda}\left({H}\bmf{n}-\gamma\bmf{\alpha}\right).
\end{equation}
%
The parameters are $H$ with units of stress and the unitless $\gamma$. The
second term creates the nonlinearity; setting $\gamma=0$ reproduces linear kinematic 
hardening of the bilinear model. For isotopic hardening, the current radius of the yield
surface is given by
%
\begin{equation}\label{E:kappa_def}
k\left(\bar e^p \right)= k_0 + 
Q \left( 1-exp \left [{-b \bar e^p}\right] \right) ~.
\end{equation}

\noindent Setting $Q=0$ eliminates isotropic hardening. Here, the material
parameters have their values/definitions for the 3-D formulation. The 1-D specialization
is treated later in this section.

\subsubsection {Discretization}
Continuum Eqs.\;\eqref{E:ys_1}, \eqref{E:equiv_pl_strain}, \eqref{E:dev_stress_rate},
\eqref{E:alpha_dot}, \eqref{E:kappa_def} are discretized using the backward
Euler method to produce the following system of nonlinear algebraic equations:
%
\begin{equation}
{\bmf{\sigma }}{'_{s + 1}} = 
{\bmf{\sigma }}{'_s} + 2G\left( {\delta \bmf{\epsilon}'_{s + 1} - 
\bar \lambda {{\bmf{n}}_{s + 1}}} \right) =
 {\bmf{\sigma }}_{s + 1}^{trial} - 2G\bar \lambda {{\bmf{n}}_{s + 1}}
\end{equation}

\begin{equation}
{{\bmf{\alpha }}_{s + 1}} = 
{{\bmf{\alpha }}_s} + \bar \lambda \left( {H{{\bmf{n}}_{s + 1}} - 
\gamma {{\bmf{\alpha }}_{s + 1}}} \right)
\end{equation}

\begin{equation}
\left\| {{\bmf{\sigma }}{'_{s + 1}} - 
{{\bmf{\alpha }}_{s + 1}}} \right\| = k\left( {\bar e_{s + 1}^p} \right)
\end{equation}

\begin{equation}\label{E:define_epbar_update}
\bar e^p_{s + 1} = \bar e_s^p + \bar \lambda 
\end{equation}

\noindent where the subscript $s$ denotes the previously converged state
and $s+1$ is the current state, with $\delta \bmf{\epsilon}'_{s+1} =
\bmf{\epsilon}'_{s+1} - \bmf{\epsilon}'_s$, the strain increment over the
step from $s$ to $s+1$. Note that $s$ is used here to differentiate this 
subincrement step
from the global step number $n$, since the subincrementation process uses
multiple sub-steps for a global step $n \rightarrow n+1$. These equations are
condensed into a single, nonlinear equation in terms of the discrete consistency
parameter, $\bar \lambda$,

%
\begin{equation}\label {E:resid_define}
\left( {1 + \bar \lambda \gamma } \right)\left[ {k\left( {\bar e_s^p + \bar \lambda } 
\right) + 2G\bar \lambda } \right] + \bar \lambda H - \left\| 
{{{\bmf{\xi }}_{s + 1}}} \right\| = r\left( {\bar \lambda } \right) = 0
\end{equation}
%
\noindent where,

%
\begin{equation}
{{\bmf{\xi }}_{s + 1}} = \left( {1 + \bar \lambda \gamma } 
\right){\bmf{\sigma }}_{s + 1}^{trial} - {{\bmf{\alpha }}_s}
\end{equation}
%

\noindent This equation is solved via Newton's method (see next section) with the stress,
backstress, and accumulated plastic strain updated by

\begin{equation}
{{\bmf{\sigma }}_{s + 1}} = {\bmf{\sigma }}_{s + 1}^{trial} 
- \frac{{\kappa \left( {\bar \lambda } \right)}}{{\rho \left( {\bar \lambda } 
\right)}}{{\bmf{\xi }}_{s + 1}}
\end{equation}

\begin{equation}
{{\bmf{\alpha }}_{s + 1}} = \frac{1}{{1 + \bar \lambda 
\gamma }}\left( {{\bmf{\alpha }}_s^{} - \frac{{\kappa \left( 
{\bar \lambda } \right)}}{{\rho \left( {\bar \lambda } \right)}}{{\bmf{\xi }}_{s + 1}}} \right)
\end{equation}

\noindent where 
\begin{equation}
\kappa \left( {\bar \lambda } \right) = \frac{{2G\bar \lambda }}
{{k\left( {{\bar e^p} + \bar \lambda } \right)}}, \quad \beta \left( {\bar \lambda } \right) =
 \frac{{\bar \lambda H}}{{k\left( {{\bar e^p} + \bar \lambda } \right)}}, \quad \rm{and}
\end{equation}

\begin{equation}
\rho \left( {\bar \lambda } \right) = \left( {1 + \bar \lambda \gamma } \right)\left[ 
{1 + \kappa \left( {\bar \lambda } \right)} \right] + \beta \left( {\bar \lambda } \right)
\end{equation}

\noindent The 3-D measure of accumulated plastic strain is updated using 
Eq. \eqref{E:define_epbar_update}.

\subsubsection {Solution for the consistency parameter}
\noindent The nonlinear residual equation, $r (\bar \lambda)=0$, is solved using Newton's 
method. The $(i+1)^{th}$ estimate of $\bar \lambda$ is computed as

\begin{equation}
{\bar \lambda ^{i + 1}} = {\bar \lambda ^i} + \left[ {r\left( {{{\bar \lambda }^i}} 
\right) /r'\left( {{{\bar \lambda }^i}} \right)} \right]
\end{equation}

\noindent where the derivative, $r' \left( \lambda \right) = dr / d \lambda$, is given by

\begin{equation}
r'\left( {\bar \lambda } \right) = \gamma \left[ {k\left. 
{\left( {\bar e_s^p + \bar \lambda } \right.} \right) + 
2G\bar \lambda } \right] + \bar \lambda \gamma 
\left[ {\frac{{dk\left( {\bar e^p + \bar \lambda } \right)}}{{d\bar e^p}} 
+ 2G} \right] +  H - 
\gamma \frac{{\bmf\xi _{s + 1}^i \cdot 
\bmf\sigma _{s + 1}^{trial}}}{{\left\| {\bmf\xi _{s + 1}^i} \right\|}}
\end{equation}

\noindent The iterations begin with $\bar \lambda^{(1)} = 0$ and continue until convergence
as measured by

\begin{equation}
\left| {r\left( {{{\bar \lambda }^{i + 1}}} \right)} \right| < {\sigma _{ys}}\cdot to{l_\lambda }
\end{equation}

\noindent where $tol_y = 10^{-6}$.
Although iterations most often converge quickly due to the quadratic convergence 
inherent in Newton's method, a large (global) strain step may cause Newton's method 
to fail. After 15 iterations without convergence, the solution procedure terminates, and the model 
requests a reduction in global step size if the adaptive solution option 
is allowed, otherwise the program exits. 

Newton's method may occasionally converge to a negative value for $\bar \lambda$, 
although a plastic step requires $\bar \lambda > 0$. Simo and Hughes 
[\ref{R:SH2000}] assert that for a convex yield surface,  $f \left( \bmf\sigma_{trial},
\bmf\alpha_n, \bar e^p \right )$ $>0$  guarantees the existence of  
$\bar \lambda > 0$. Thus, if Newton's method determines a $\bar \lambda < 0$, the model applies 
interval bisection together with Newton's method to find a positive value. The initial interval 
has a lower bound of $\bar \lambda = 0$.
An upper bound, $\bar \lambda_f$, is sought such that 
$r\left(0\right)r\left(\bar \lambda_f\right)<0$, where $r$ is defined in Eq. \eqref{E:resid_define}. 
The procedure examines up to 4 multiples of an initial guess for $\bar \lambda_f$ (the 
absolute value of the converged, negative root). If an acceptable upper bound is not found, 
the model requests a global step size reduction if possible, otherwise the program terminates. 
Within a suitable interval, the procedure employs Newton's method. If the 
computed $\bar \lambda$
falls outside the interval or 10 consecutive Newton iterations occur without convergence, 
the procedure takes an interval bisection step and computes a new 
starting $\bar \lambda$ as the mean of the interval. If the procedure 
executes 10 interval bisection steps without finding a positive root, 
the material model requests a global step size reduction 
if adaptive solutions are allowed, otherwise the program terminates.


\subsubsection {Consistent tangent}
The consistent tangent operator, required for calculating the element stiffness matrices, 
is a fourth order tensor that provides the change in stress with respect 
to change in strain,

\begin{equation}
{\bf{C}} = \frac{{d{{\bmf{\sigma }}_{n + 1}}}}{{d{\bmf{\epsilon}_{n + 1}}}}~.
\end{equation}
\noindent If state $n+1$ is elastic, $\bf C$ is given by

\begin{equation}
C_{ijkl}^E = K{\delta _{ij}}{\delta _{kl}} + 2G\left[ {\frac{1}{2}\left( {{\delta _{ik}}{\delta _{jl}} + {\delta _{il}}{\delta _{jk}}} \right) - \frac{1}{3}{\delta _{ij}}{\delta _{kl}}} \right]~.
\end{equation}

\noindent Otherwise, if state $n+1$ is plastic loading, the tangent operator is
given by

\begin{equation}\label{E:cons_tan_define}
C_{ijkl}^P = K{\delta _{ij}}{\delta _{kl}} + a\left[ {{\delta _{ik}}{\delta _{jl}} -
\frac{1}{3}{\delta _{ij}}{\delta _{kl}}} \right] + b{n_{ij}}{n_{kl}} - 
c\sigma _{ij}^{trial}n{_{kl}}
\end{equation}

\begin{equation}
a = \frac{{2G}}{\rho }\left( {1 + \bar \lambda \gamma  + \beta } \right)
\end{equation}

\begin{equation}
b = \frac{{k\zeta }}{\rho }\left( {\kappa \rho ' - \kappa '\rho } \right)
\end{equation}

\begin{equation}
c = \gamma \frac{{\zeta \kappa }}{\rho }
\end{equation}

\begin{equation}
\zeta  = \frac{{2G\left( {1 + \bar \lambda \gamma } \right)}}{{k' + 2G +  H + 
\gamma \left( {k + 4G\bar \lambda  - {\bmf{n}} 
\cdot {\bmf{\sigma }}_{n + 1}^{trial}} \right)}}
\end{equation}

\noindent where  $k' = dk / d\bar e^p$ and $\bmf n$ is computed as in Eq. \eqref{E:n_define}.

The fourth term of the Eq. \eqref{E:cons_tan_define}, $c \bmf \sigma^{trial} 
\otimes \bmf n$, is
generally not symmetric. Thus, the implemented tangent is taken a symmetric
average

\begin{equation}
{\left( {C_{ijkl}^P} \right)_{sym}} = \frac{1}{2}\left[ {C_{ijkl}^P + C_{klij}^P} \right]
\end{equation}

Also, the model computes the tangent from data for the 
last stress update, \textit{i.e.}, the last subincrement of the total step. 
Thus the tangent is not fully consistent with the entire global step,
$\Delta \bmf \epsilon = 
\bmf\epsilon^i_{n+1} -  \bmf\epsilon_n$. The subincrementation and
symmetrization of the consistent tangent can lead to a lack of global quadratic 
convergence for general load paths. However, the convergence is superlinear in general 
and becomes essentially quadratic for radial load paths and small step sizes. 

%*****************************************************
\subsection {\bsf{generalized\_plasticity} option}
%*****************************************************
\subsubsection {Yield surface, limit function, and flow rule}
Eqns.\;\eqref{E:deviators_define}-\eqref{E:dev_stress_rate}
from the \textit{nonlinear\_hardening} option apply as well to the 
\textit{generalized\_-plasticity} option. The main differences between the two 
options are: (1) the GP may have constant (kinematic)
hardening at large plastic strains,  (2) the GP model may have temperature-dependent 
material properties,  and (3) the GP
model does not use the 
yield surface to define admissibility. Instead, a nonlinear 
limit function, $F$, describes admissible states where 

\begin{equation}
F = h(f)\left[ {\bmf{n}}:{\bmf{\sigma }}' -\Lambda \dot{\theta}  \right] - \dot \lambda 
\end{equation}

\begin{equation}
h(f) = \frac{1}{{\delta \left( {\beta  - f} \right) + \beta H }}
\end{equation}

\begin{equation}
\Lambda = \dfrac{1}{H_{k}}\dfrac{dH_{k}}{d\theta}\boldsymbol{\alpha}:\boldsymbol{n}
 +\frac{dk_{0}}{d\theta}+\frac{dH_{i}}{d\theta}\bar{e}_{p}
\end{equation}

\noindent such that $\dot \lambda \ge 0$, $F \le 0$, and $\dot \lambda F =0$. 
The term $\Lambda$ reflects the yield surface variation with temperature, $\theta$. 
Note that $f$ is the yield function as given in Eq. \eqref{E:ys_1} and
$\beta$, $ \delta$, and $ H$ are material parameters. $\beta$ 
and $\delta$ are related to their 1-D (uniaxial) counterparts used for 
input values and described previously via


\begin{equation}
\beta   = \textstyle  \sqrt {\frac{2}{3}} \displaystyle
 \beta_u  \quad \rm{and} \quad 
\delta  = \textstyle \frac{2}{3} \displaystyle \delta_u~.
\end{equation}

Also note that the total hardening modulus, $H=H_k + H_i$ is also
related to the 1-D moduli through

\begin{equation}
{H_k} = \textstyle \frac{2}{3} \displaystyle {H_{ku}};\quad {H_i} =
\textstyle  \frac{2}{3} \displaystyle {H_{iu}}~.
\end{equation}

The \textit{generalized\_plasticity} model allows an admissible state to be 
outside the yield surface since the yield and limit surfaces are not coincident.  
Thus yielding can occur immediately on re-loading without first traversing 
a linear-elastic region.  Nonlinear stress-plastic strain response is due 
entirely to nonlinearity in the limit function.

\subsubsection{Strain hardening} %  ----- GP Strain Hardening
The \textit{generalized\_plasticity} model features optional 
linear kinematic and linear isotropic hardening. Evolution of the backstress 
governs the kinematic hardening (translation of the yield surface) by 

\begin{equation}
{\dot{\bmf \alpha }} = \dfrac{\dot{H}_{k}}{H_{k}}\boldsymbol{\alpha} + 
{H_k}{\left( \dot {\bmf{\epsilon}}' \right)^p}
\end{equation}

The parameter $H_k$  is the kinematic hardening modulus, with units of stress.  
The isotropic hardening modulus, $H_i$, (units of stress) governs 
expansion of the yield surface as

\begin{equation}
k( \bar e^p ) = k_0 + H_i \bar e^p ~.
\end{equation}


\subsubsection {Discretization}
The \textit{generalized\_plasticity} implementation employs a predictor-corrector 
strategy for which the trial state assumes all response is linear-elastic. If the 
actual state is linear-elastic, it must be within the yield surface or lie in 
an unloading direction 
$\left( \bmf n \cdot \dot {\bmf {\sigma}}'-\Lambda \dot{\theta} <0\right)$ from 
the previous state.  The discrete version of the unloading test is 

\begin{equation}
f_{s+1}^{trial} - f_s < 0
\end{equation}

\noindent where

\begin{equation}
f_{s+1}^{trial} = \left\| \bmf{\xi}_{s+1}^{trial} \right\| - k_{s+1}^{trial} = \left\| \bmf{\sigma}_{s+1}^{trial} - \bmf{\alpha}_{s+1}^{trial} \right\| - k_{s+1}^{trial}~,
\end{equation}

\begin{equation}
\bmf{\sigma}_{s+1}^{trial} = \dfrac{G_{s+1}}{G_s}\bmf{\sigma}'_s + 2 G_{s+1} \delta  \bmf{\epsilon}'_{s + 1}~,
\end{equation}

\begin{equation}
\bmf{\alpha}_{s+1}^{trial} = \dfrac{H_{k,s+1}}{H_{k,s}} \bmf{\alpha}_s~,
\end{equation}

\begin{equation}
k_{s+1}^{trial} = k_{0,s+1} + H_{i,s+1} \bar e^p_{s} ~.
\end{equation}

The subscript $s$ denotes the previously converged state and $s+1$ is current state, 
with $\delta \bmf \epsilon'_{s+1} = \bmf \epsilon'_{s+1}-\bmf \epsilon'_s$, the strain 
increment over the step from $s$ to $s+1$. Note that $s$ is 
used here to differentiate this step from the global step number $n$, since 
the subincrementation process uses multiple sub-steps for each global 
step $n \rightarrow n+1$.
Applying the backward Euler method to the continuum equations 
yields a single quadratic equation for the consistency 
parameter $\left( \bar \lambda = \dot \lambda \Delta t \right)$:

\begin{equation}
a{\bar \lambda ^2} + b\bar \lambda  + c = 0
\end{equation}

\noindent where

\begin{equation}\label{E:gp_a}
a = \left( {2G + H} \right)\left( { \delta  - \left( 2G + \Delta H_i \right) } \right)
\end{equation}

\begin{equation}\label{E:gp_b}
b = \left( {\delta  +  H} \right)\beta 
 - \left( {\delta  - \left( 2G + \Delta H_i \right)} \right) f_{s+1}^{trial}
+ \left( {2G + H} \right)\left( f_{s+1}^{trial} - f_s \right)
\end{equation}

\begin{equation}\label{E:gp_c}
c =  - \left( f_{s+1}^{trial} \right)
\left( f_{s+1}^{trial} - f_s \right)~.
\end{equation}

\noindent The smallest positive root is the correct solution [\ref{R:AT1995}]. In 
Eqns.\;\eqref{E:gp_a}, \eqref{E:gp_b}, and \eqref{E:gp_c}, the material
properties equal the temperature-dependent values at $s+1$. The term $\Delta H_i$ reflects
the change in the isotropic hardening coefficient from $s$ to $s+1$. 

The stress and internal variables are then updated with

\begin{equation}
{\bmf{\sigma }}{'_{s + 1}} = {\bmf{\sigma }}_{s + 1}^{trial} - 2G_{s+1}\bar \lambda {\bmf{n}}
\end{equation}

\begin{equation}
{{\bmf{\alpha }}_{s + 1}} = {{\bmf{\alpha }}_{s+1}^{trial}} + {H_{k,s+1}}\bar \lambda {\bmf{n}}
\end{equation}

\begin{equation}
\bar e_{s + 1}^p = \bar e_s^p + \bar \lambda 
\end{equation}

\noindent where $\bmf n = \bmf \xi^{trial}_{s+1} / \parallel
 \bmf \xi^{trial}_{s+1} \parallel$. 
 
 
\subsubsection {Consistent tangent}

The consistent tangent operator, required for calculating the element stiffness matrices, 
is a fourth order tensor that provides the change in stress with respect to change in strain,
\begin{equation}
{\bf{C}} = \frac{{d{{\bmf{\sigma }}_{n + 1}}}}{{d{\bmf{\epsilon}_{n + 1}}}}~.
\end{equation}
\noindent If state $n+1$ is elastic, $\bf C$ is given by

\begin{equation}
C_{ijkl}^E = K_{n+1}{\delta _{ij}}{\delta _{kl}} + 
2G_{n+1}\left[ {\frac{1}{2}\left( {{\delta _{ik}}
{\delta _{jl}} + {\delta _{il}}{\delta _{jk}}} \right) - \frac{1}{3}{\delta _{ij}}{\delta _{kl}}} \right]~.
\end{equation}

\noindent Otherwise, if state $n+1$ is plastic loading, the tangent operator is
given by

\begin{equation}\label{E:cons_tan_define_gp}
C_{ijkl}^P = K_{n+1}{\delta _{ij}}{\delta _{kl}} + a\left[ {{\delta _{ik}}{\delta _{jl}} -
\frac{1}{3}{\delta _{ij}}{\delta _{kl}}} \right] + b{n_{ij}}{n_{kl}} 
\end{equation}

\noindent where

\begin{equation}
a = 2G_{n+1}\left( {1 - \frac{{2G_{n+1}\bar \lambda }}{{\left\| 
{{\bmf{\xi }}_{n + 1}} \right\| + \left( 2 G_{n+1} + H_{k,n+1} \right) \lambda }}} \right)
\end{equation}

\begin{equation}
b = 2G_{n+1}\left( { \frac{{2G_{n+1}\bar \lambda }}{{\left\| 
{{\bmf{\xi }}_{n + 1}} \right\| + \left( 2 G_{n+1} + H_{k,n+1} \right) \lambda }} - A} \right)
\end{equation}

\begin{equation}
A = \dfrac{2 G_{n+1} B_1}{\left( 2 G_{n+1} + H_{n+1} \right) B_1 + B_2}~,
\end{equation}

\begin{equation}
B_1 = 2 f_{n+1} - f_n + \left( H_{n+1} + \delta_{n+1} - \Delta H_i \right) \lambda~,
\end{equation}

\begin{equation}
B_2 = \left( \delta_{n+1} + H_{n+1} \right) \beta_{n+1} - \left( H_{n+1} + \delta_{n+1} - \Delta H_i \right) f_{n+1}~.
\end{equation}

The above expression, Eq.\;\eqref {E:cons_tan_define_gp} reveals 
that the consistent tangent for the 
\textit{generalized\_plasticity} model remains symmetric.

%*****************************************************
\subsection {Subincrementation for both \textit{cyclic} options}
%*****************************************************

The subincrementation procedure first computes the elastic fraction of the 
global strain increment, $\Delta \bmf \epsilon = 
\bmf \epsilon^i_{n+1} -   \bmf \epsilon_n$. The fraction $0 \le \eta \le 1$
is found such that 
 
\begin{equation}\label{E:ys_eta}
\left\| {{\bmf{\sigma }}{'_n} + 
2G\eta \Delta {\bmf{\epsilon}'} - {{\bmf{\alpha }}_n}} 
\right\| - k\left( {\bar e_n^p} \right) = 0
\end{equation}

\noindent and the state $\bmf \sigma'_n + 2 G \eta \Delta \bmf \epsilon'_{n+1}$ lies on the
yield surface from global step $n$. Solving for $\eta$ leads to
a quadratic equation with solution

\begin{equation}\label{E:eta_define}
\eta  =  - a/b \pm \sqrt {{{\left( {a/b} \right)}^2} + \left( {c/b} \right)} 
\end{equation}

\noindent where

\begin{equation}
a = \left( {{{\bmf{\sigma }}_n} - {{\bmf{\alpha }}_n}} \right) \cdot 2G\Delta {{\bmf{\epsilon}'}_{n + 1}},
\end{equation}

\begin{equation}
b = 2G\Delta {{\bmf{\epsilon}'}_{n + 1}} \cdot 2G\Delta \bmf \epsilon'_{n+1},
\end{equation}

\begin{equation}
c = k{\left( {\bar e_n^p} \right)^2} - \left( {{{\bmf{\sigma }}_n} - {{\bmf{\alpha }}_n}} \right) \cdot \left( {{{\bmf{\sigma }}_n} - {{\bmf{\alpha }}_n}} \right)~.
\end{equation}

\noindent Note that $c$ is a re-statement of the yield surface, \textit{i.e.}, $c = 0  
\Leftrightarrow f= 0$
and $f > 0 \Leftrightarrow c < 0$. Therefore, if $f=0$ (previous step was plastic loading)

\begin{equation}
\eta  = \frac{{ - a}}{b} \pm \left| {\frac{a}{b}} \right|~.
\end{equation}

\noindent Since $b>0$ and $\eta \ge 0$,
\begin{equation}
\eta=0 \quad \rm{if} \quad a\geq 0 \quad \rm{and} \quad \eta=\displaystyle -
2\left(\frac{a}{b}\right) \quad \rm{if} \quad a<0~.
\end{equation}

The requirement that $\eta < 1$ is automatically satisfied, since this 
procedure is used only if the full strain increment causes the trial stress 
to lie outside the current yield surface and thus the intersection 
with the surface must occur before the full increment is applied.

Also, note that $c \ge 0$, since $f >0$ is not allowed, and 
thus Eq. \eqref{E:eta_define}
always has at least one positive solution.

For the generalized plasticity model,  Eqn.\;\eqref{E:ys_eta} has a nonlinear form
due to temperature-dependent material properties. Ridder's method
provides a satisfactory method to solve
for $\eta$ in the generalized plasticity model. This method converges rapidly 
to the solution of Eqn.\;\eqref{E:ys_eta} that incorporates mechanical and thermal 
loading effects on the yield surface. 

If the elastic fraction is non-zero, the first subincrement is 
effectively a step from $\bmf \epsilon_n$ to $\bmf \epsilon_n +
\eta \Delta \bmf \epsilon_{n+1}$. This is a linear-elastic update; the stress 
at the end of this subincrement is $\bmf \sigma'^{(1)}_{n+1} = \bmf \sigma'_n + 2G \eta \Delta \bmf
\epsilon'_{n+1}$ and the backstress is $\bmf \alpha^{(1)}_{n+1} = \bmf \alpha_n$,
where the superscript refers to the subincrement number. 
Also, the accumulated plastic strain is unchanged from step $n$.

The remaining strain increment, $\left ( 1 - \eta \right )
\Delta \bmf \epsilon_{n+1}$, causes plastic yielding. The 
accuracy of the stress state computed at the end of the increment 
depends, in part, on the size of the plastic fraction of the step.  The premise 
of subincrementation strategies lies in dividing the total strain 
increment into several small steps and computing the stress state 
sequentially for each increment produces a more accurate answer than 
a single step procedure but does so more efficiently than a global 
step size reduction. 

The method employed here uses estimates of truncation error to compute the 
number of subincrements required to satisfy a user supplied 
tolerance on the stress accuracy. First, the stress is computed for 
both a single step $( \bmf \sigma^{(1)})$ and a 2-step 
procedure $( \bmf \sigma^{(2)})$; \textit{i.e.}, a step size 
of $h$ and $h/2$, respectively. Since the numerical integration of the rate 
equations is first order accurate, the error can be estimated from 
the two solutions.  The error is computed as

\begin{equation}
{E_\sigma } = \mathop {\max }\limits_{i,j = 1,3} 
\left( {\sigma _{ij}^{(2)} - \sigma _{ij}^{(1)}} \right)~.
\end{equation}

If the error is greater than the tolerance, the number of subincrements 
needed to satisfy the tolerance, $s$, is estimated as

\begin{equation}
s = {\rm{ceil}}\left( {1.1\frac{{2{E_\sigma }}}{{to{l_\sigma }}}} \right)
\end{equation}

\noindent where $tol_{\sigma} = \sigma_y * sig\_tol$ ($sig\_tol$ is the user specified tolerance)
and the 1.1 factor insures the tolerance is satisfied. The balance,
$2 E_{\sigma} / tol_{\sigma} $, artises from an estimate of the step
size, $h^*$, such that $E \left( h^* \right) = tol_{\sigma}$.

Since at least 2 solutions are generated via this procedure (1-step, 2-step, and 
possibly $s$-step), extrapolation is simple and can produce an additional 
order of magnitude in accuracy.  Extrapolation is carried out on 
a component-wise basis. Let $x$ be any single component of the solution 
(\textit{e.g.}, $x=\sigma'_{11}$ or $x=\alpha_{23}$) and $n_a$ and $n_b$ be 
two different numbers of subincrements for which the solution already exists 
such that $n_a > n_b$.  Then the extrapolated solution is

\begin{equation}
{x_{a,b}} = a{x_a} - b{x_b}
\end{equation}

\noindent where

\begin{equation}
a = \frac{{{n_a}}}{{{n_a} - {n_b}}}~,
\end{equation}

\noindent and

\begin{equation}
b = \frac{{{n_b}}}{{{n_a} - {n_b}}}~.
\end{equation}

\noindent For example, extrapolating 1-step and 2-step solutions
produces

\begin{equation}
{x_{2,1}} = \frac{2}{{2 - 1}}{x_2} - \frac{1}{{2 - 1}}{x_1} = 2{x_2} - {x_1}~.
\end{equation}

The tangent modulus used at the global level is computed according to
Eqs.\;\eqref{E:cons_tan_define} 
and \eqref{E:cons_tan_define_gp} considering only the last subincrement.  It does not 
account for the influence of all the subincrements on the solution or 
extrapolation so it is not completely consistent with the algorithm.  
Nevertheless, there is typically only a small reduction in global convergence rate.

%-----------------------------------------------------
\subsection{Details on 1-D specialization of 3-D plasticity formulation}
%-------------------------------------------------------

\subsubsection{Stresses}
The 3-D version of Mises plasticity is specialized to 1-D for identification of material 
properties from a uniaxial test (tension and/or compression). Define the
stress tensor and the deviator of the stress tensor

\begin{equation}
\bmf \sigma =
 \begin{pmatrix}
     \sigma_{11} & \sigma_{12} & \sigma_{13} \\
     \sigma_{21} & \sigma_{22} & \sigma_{23} \\
     \sigma_{31} & \sigma_{32} & \sigma_{33} \\
 \end{pmatrix}~,
\end{equation}
     
\noindent and

\begin{equation}
\sigma_m = \frac{\sigma_{11} + \sigma_{22} + \sigma_{33}}{3}.
\end{equation}

\noindent The deviator of the stress tensor is then

\begin{equation}
\bmf \sigma' =
 \begin{pmatrix}
     \sigma_{11}-\sigma_m & \sigma_{12} & \sigma_{13} \\
     \sigma_{21} & \sigma_{22}-\sigma_m & \sigma_{23} \\
     \sigma_{31} & \sigma_{32} & \sigma_{33}-\sigma_m \\
 \end{pmatrix}~.
\end{equation}

\noindent For 1-D loading: $\sigma_{11} \ne 0$ and all other components are zero,

\begin{equation} \label{E:sigmadev_tensor_1D}
\bmf \sigma'_u =
 \begin{pmatrix}
     \sigma_{11}-\frac{\sigma_{11}}{3} & 0 & 0 \\
      0 & -\frac{\sigma_{11}}{3} & 0 \\
     0 & 0 & -\frac{\sigma_{11}}{3}\\
 \end{pmatrix}~.
\end{equation}

The Mises yield function can be written in the form

\begin{equation}\label{E:vm_define_1}
\parallel \bmf \sigma' \parallel - k = 0
\end{equation}

\noindent where $k$ is a material constant with units of 
stress and is the radius of the Mises yield cylinder. The norm,
$\parallel \cdot \parallel$, is given by

\begin{equation}
\parallel \bmf \sigma' \parallel = \sqrt{ \sigma'_{ij} \sigma'_{ij} }~.
\end{equation}

\noindent Then for 1-D, we have

\begin{equation}
\parallel \bmf \sigma'_u \parallel = \sqrt{ \textstyle 
{\frac{2}{3}} }\; |\sigma_{11}|~,
\end{equation}

\noindent and $k$ becomes identified as

\begin{equation}\label{E:k_sigma_11_define}
 k = \sqrt{ \textstyle {\frac{2}{3}} }\; |\sigma_{11}|~.
\end{equation}

\noindent In terms of the uniaxial tensile yield stress, $\sigma_{ys}$, $k_0$ at 
the onset of yielding becomes 
$k_0 = \sqrt{2/3} \; \sigma_{ys}$. For simple shear
loading, all stress components but $\tau = |\sigma_{12}| = |\sigma_{21}|$
are zero with

\begin{equation}
\parallel \bmf \sigma'_u \parallel = \sqrt{2} \; |\sigma_{12}| = \sqrt{2} \; \tau~,
\end {equation}

\noindent with $k_0$ then also becoming identified as 

\begin{equation}\label{E:k_shear_define}
k_0 = \sqrt{2}\;  \tau_{ys}~.
\end{equation}

\noindent Equating both definitions of $k_0$, we have

\begin{equation}
\sqrt{2} \; \tau_{ys} = \sqrt{ \textstyle {\frac{2}{3}} } \; \sigma_{ys}~,
\end{equation}
\begin{equation}
\tau_{ys} = \frac{\sigma_{ys}}{ \sqrt{3}}~.
\end{equation}

The Mises yield criterion of Eq. \eqref{E:vm_define_1} may be written in
a variety of other forms by including various constants in both 
$\parallel \bmf \sigma' \parallel$ and $k$.

In Eq. \eqref{E:vm_define_1}, with $k$ in terms of the shear stress,
we can also write

\begin{equation}
\frac{1}{\sqrt{2}} \parallel \bmf \sigma' \parallel -\tau_{ys} = 0~,
\end{equation}
\noindent where the first term is identified now with second
invariant of the deviator stress tensor, $J_2$, giving

\begin{equation}
\sqrt{J_2} - \tau_{ys} = 0~.
\end{equation}

Alternatively, Eq. \eqref{E:k_sigma_11_define} can be used in Eq. \eqref{E:vm_define_1}
in a similar way to write

\begin{equation}
\parallel \bmf \sigma' \parallel - \sqrt{\textstyle{\frac{2}{3}}} \sigma_{ys}= 0~.
\end{equation}

\noindent Now multiply through by $\sqrt{3/2}$ to define

\begin{equation}
\sqrt{\textstyle{\frac{3}{2}}}\parallel \bmf \sigma' \parallel -  \sigma_{ys}= 0~,
\end{equation}

\noindent where again $J_2$ can be introduced to write the yield function in
another quite common form

\begin{equation}
\sqrt{3 J_2} -  \sigma_{ys}= 0~.
\end{equation}

%---------------------------------------------------
\subsubsection{Plastic strains}
%--------------------------------------------------------
The strains and strain rates (wrt loads or time) are treated with additive decomposition

\begin{equation}
\bmf \epsilon = \bmf \epsilon^e + \bmf \epsilon^p
\end {equation}

\noindent and


\begin{equation}
\bmf  {\dot \epsilon} = \bmf {\dot \epsilon}^e + \bmf {\dot\epsilon}^p
\end {equation}

\noindent where the plastic strains are deviatoric and aligned with
the deviatoric stresses, \textit{i.e.}, $\dot {\bmf {\epsilon}}^p
 \propto \parallel \bmf \sigma' \parallel $.

For 3-D plastic deformation, we have


\begin{equation}\label{E:epsplas_tensor}
\bmf \epsilon^p =
 \begin{pmatrix}
     \epsilon^p_{11} & \epsilon^p_{12} & \epsilon^p_{13} \\
     \epsilon^p_{12} & \epsilon^p_{22} & \epsilon^p_{23} \\
     \epsilon^p_{13} & \epsilon^p_{23} & \epsilon^p_{33} \\
 \end{pmatrix}~,
\end{equation}

\noindent where the $trace \left( \bmf \epsilon^p \right )$ 
$= \epsilon^p_{11} + \epsilon^p_{22}+\epsilon^p_{33}\equiv 0$ to reflect 
the incompressibility of plastic strains. The norm of plastic strains is given by

\begin{equation}\label{E:norm_epsilon_p_defined}
\parallel \bmf \epsilon^p \parallel = \sqrt{ \epsilon^p_{ij} \epsilon^p_{ij} }~.
\end{equation}

\noindent The corresponding rate definition is denoted

\begin{equation}\label{E:e_bar_p_rate_define}
\dot {\bar e}^p = \parallel \dot{\bmf \epsilon}^p \parallel ~,
\end{equation}

\noindent where $\parallel \bmf\epsilon^p \parallel$ and
 $\dot{\bar e}^p$ may have increasing/decreasing (instantaneous) values $\ge 0$ over the
loading history. Here $\dot {\bar e}^p$ is a measure of the instantaneous rate $\ge 0$ of
plastic strain accumulation, with the \textit{total} plastic strain accumulation over the
loading history given by

\begin{equation}\label{E:e_bar_p_define}
\bar e^p = \int^t_0 \dot {\bar e}^p\;dt~,
\end{equation}

\noindent with $t$ denoting time or a loading measure.

Specialization to the 1-D, uniaxial case with loading only along the 1-direction yields

\begin{equation}\label{E:epsplas_tensor_1D}
\bmf \epsilon^p_u =
 \begin{pmatrix}
     \epsilon^p_{11} & 0 & 0 \\
     0 & -0.5\epsilon^p_{11}  & 0 \\
     0 & 0 &  -0.5\epsilon^p_{11} \\
 \end{pmatrix}~,
\end{equation}

\noindent where incompressibility dictates the values for  $\epsilon^p_{22}$ and
$\epsilon^p_{33}$. The subscript $u$ distinguishes
specialization to the 1-D (uniaxial) case.  The norm for this 1-D form becomes

\begin{equation}
 \parallel \bmf \epsilon^p_u \parallel = 
\sqrt{\textstyle{\frac{3}{2}}} \; | \epsilon^p_{11}| ~.
\end{equation}

Similarly, the 1-D rate form for the accumulation rate of plastic strain and the
accumulated plastic strain become

\begin{equation}\label{E:dot_e_bar_p_u_define}
\dot{\bar e}^p_u = \parallel \dot{\bmf \epsilon}^p_u \parallel =
\sqrt{\textstyle{\frac{3}{2}}} \; | \dot \epsilon^p_{11}|~,
\end{equation}

\noindent and

\begin{equation}\label{E:e_bar_p_u_define}
\bar e^p_u = \int^t_0  \dot{\bar e}^p_u \; dt = 
\sqrt{\textstyle{\frac{3}{2}}} \; \bar \epsilon^p_u\; \quad \text{where   }
\bar \epsilon^p_u =\int^t_0  | \dot \epsilon^p_{11}| \; dt~.
\end{equation}




%%--------------------------------------------------------------------------------
%\subsubsection{Coupling 1-D and 3-D plastic constitutive models}
%%--------------------------------------------------------------------------------
%The 1-D equivalents of the 3-D formulations expressed in Eqs.\;\eqref{E:k_sigma_11_define},   
%\eqref{E:k_shear_define}, and \eqref{E:e_bar_p_u_define}
%take on a special significance in defining the constitutive relationship for the
%3-D plasticity model in terms of the 1-D, uniaxial stress-strain curve. Most often,
%the lab test result produces an expression of the form

%
%\begin{equation}
%\sigma_{11} = \sigma_{11} ( \epsilon_{11} )~.
%\end{equation}

%The 3-D plasticity constitutive models generally require specification
%of the tensile part of the uniaxial curve $(\sigma_{11}, \epsilon_{11} >0)$
%with the result written in the form

%\begin{equation}\label{E:bar_sigma_define}
%\bar \sigma_u = \bar \sigma_u (  \bar e^p_u  )\;; \quad  \bar e^p_u \ge 0
%\end{equation}

%\noindent where $\bar \sigma_u$ is always $> 0$ and in rate form as

%\begin{equation}
%d \bar \sigma_u = H'_u ( \bar e^p_u ) \;d \bar e^p_u
%\;; \quad  \bar e^p_u , {d\bar e}^p_u\ge 0
%\end{equation}

%\noindent The code user most often inputs the function $H'_u ( \bar e^p_u )$ to the
%finite element code, \textit{i.e.}, Abaqus. Alternatively, some codes (WARP3D) require that the user
%input the full uniaxial stress-strain curve as measured in the test, $\sigma_{11} =
%\sigma_{11}( \epsilon_{11})$, and the code computes $H'_u ( \bar e^p_u )$ as needed
%in the analysis using the familiar expression

%\begin{equation}
%H'_u = \frac{E - E_T(\epsilon_{11})}{E \cdot E_T(\epsilon_{11})}~.
%\end {equation}

%\noindent Here, $E_T$ is simply $d\sigma_{11} / d\epsilon_{11}$.

%

%--------------------------------------------------------------------------------
\subsubsection{Reducing 3-D constitutive equations to 1-D: Generalized Plasticity}
%--------------------------------------------------------------------------------
The continuum 3-D equations for the Generalized Plasticity (GP) model 
(with temperature invariant material properties)  are summarized below.

\begin{equation} \label{E:yield_surface}
f = \parallel \bmf{\sigma}' - \bmf{\alpha} \parallel - k(\bar e^p) 
\end{equation}

\begin{equation} \label{E:gp_isotropic}
k(\bar e^p) = k_0+H_i \bar e^p
\end{equation}

\begin{equation} \label{E:limit_surface}
F = h(f)[\bmf{n}:\dot{\bmf{\sigma}}']-\dot{\lambda}
\end{equation}

or alternatively,
\begin{equation} \label{E:limit_surface}
F = h(f) \left [\frac{d}{dt} ||\bmf{\sigma}'-\bmf{\alpha}||+ \bmf{n}:\dot{\bmf{\alpha}}\right]-\dot{\lambda}
\end{equation}


\begin{equation}\label{E:h_define}
h(f) = \frac{f}{\delta(\beta-f)+H\beta}
\end{equation}

\begin{equation}\label{E:n_define2}
\bmf n =  \frac {df/d\bmf{\sigma}'} {\parallel df/d\bmf{\sigma}' \parallel} =
\frac{\bmf \sigma' - \bmf \alpha} {\parallel \bmf \sigma' - \bmf \alpha \parallel }\;;
\quad \parallel \bmf n \parallel \equiv 1
\end{equation}

\begin{equation}\label{E:normal_flow}
\dot{\bmf{\epsilon}}^p=\dot{\lambda}\bmf{n}
\end{equation}

\begin{equation} \label{E:gp_kinematic}
\dot{\bmf{\alpha}}=H_k\dot{\lambda}\bmf{n}=H_k\dot{\bmf{\epsilon}}^p
\end{equation}

\begin{equation}
H = H_i + H_k
\end{equation}

\noindent where the norm of $\bmf \epsilon^p$, the norm of its rate, 
$\dot{\bar e}^p$, and the
accumulated (monotonically increasing) plastic strain
measure, $\bar e^p$, are defined as in Eqs.\;\eqref{E:norm_epsilon_p_defined},
\eqref{E:e_bar_p_rate_define}, and \eqref{E:e_bar_p_define};
$k_0$ is the radius of the Mises yield cylinder 
at the onset of plastic flow; $H_i$ and $H_k$ provide constant rates of isotropic and
kinematic hardening with increasing plastic flow; $\beta$ and $\delta$ are constant
material parameters. All values above are cast in terms 
of the 3-D formulation (uniaxial forms are introduced later in this section). 

To identify the hardening parameter $H_i$, Eq.\;\eqref{E:gp_isotropic} is specialized 
to the 1-D case using Eq.\;\eqref{E:k_sigma_11_define} and Eq.\;\eqref{E:e_bar_p_u_define}

\begin{equation}
k(\bar e^p_u) = k_0 + H_i  \bar e^p_u~,
\end{equation}

\noindent which can be written in the form (using Eqs.\;\eqref{E:k_sigma_11_define},
\eqref{E:e_bar_p_u_define})

\begin{equation}
k(\bar \epsilon^p_u) = \sqrt{\textstyle {\frac{2}{3}}} \sigma_{ys} + 
 H_i  \left[ \sqrt{\textstyle{\frac{3}{2}}} \bar \epsilon^p_u \right]~
\end{equation}

\noindent Now introduce the uniaxial version of the isotropic (constant) hardening parameter 
as $H_{iu}\equiv(3/2)H_i$, then the isotropic hardening function further reduces to 

\begin{equation}\label{E:H_iu_define}
k\left(\bar \epsilon^p_u\right) = 
\sqrt{ \textstyle {\frac{2}{3}} } 
\left( \sigma_{ys}+
H_{iu} \bar \epsilon^p_u\right)~,
\end{equation}

\noindent where the 1-D form $H_{iu}$ is input by the user to define model properties. 

Following Eq.\;(\ref{E:sigmadev_tensor_1D}), define a direction tensor $\bmf{m}$ which relates the uniaxial stress-based variables to their 3-D tensorial counterparts;

\begin{equation}\label{E:m_define}
\bmf{m}= \frac{1}{3}
 \begin{pmatrix}
     2 & 0 & 0 \\
      0 & -1 & 0 \\
     0 & 0 & -1\\
 \end{pmatrix}~.
\end{equation}

\noindent The 1-D form of the deviator stress tensor 
can be written

\begin{equation}\label{E:sigma_prime_1D_define}
 \bmf{\sigma}_u'=\sigma_{11}\bmf{m}\;,
 \end{equation}

\noindent  where the sense of the deviator takes the sign of
$\sigma_{11}$. Here, $\rm{Tr}(\bmf{\sigma}_u' )\equiv 0$
and $||\bmf{\sigma}_u'||=\sqrt{2/3}\; |\sigma_{11}|$
Similarly, the 1-D form of the backstress may be written most
conveniently as

\begin{equation}\label{E:alpha_u_defined}
 \bmf{\alpha}_u=\alpha_{11}\bmf{m}\,.
\end{equation}
 
\noindent where the $\rm{Tr}(\bmf{m})\equiv 0$ and $||\bmf{\alpha}_u||=\sqrt{2/3}\; |\alpha_{11}|$.
The direct correspondence between $\sigma_{11}$ and $\alpha_{11}$ simplifies
subsequent interpretation of (uniaxial) material properties.

Continuing, the 1-D form of the plastic strain tensor can be written

\begin{equation}\label{E:epsilon_p_equiv_define}
\bmf{\epsilon}^p_u=\textstyle {\frac{3}{2}}\epsilon^p_{11} \bmf{m}\;,
\end{equation}

\noindent where the sense of $\bmf {\epsilon}^p_u$ takes on the sign of $\epsilon^p_{11}$ 
[refer to Eq. (\ref{E:epsplas_tensor_1D})]. 

With these 1-D definitions, the yield function in Equation (\ref{E:yield_surface}) becomes

\begin{equation}\label{E:f_define_1D}
f=|\sigma_{11}-\alpha_{11}|\parallel \bmf{m} \parallel - k(\bar \epsilon^p_u) =  
\sqrt{\textstyle{\frac{2}{3}}}\left[|\sigma_{11}-\alpha_{11}|-\left(\sigma_{ys}+
H_{iu} \bar\epsilon^p_u \right) \right],
\end{equation}


\noindent in which all the components have direct one-dimensional significance.

Now consider evolution of the backstress in 1-D.  Using the rate form of
Eq.\;\eqref{E:epsilon_p_equiv_define}, Eq.\;(\ref{E:gp_kinematic}) reduces to

\begin{equation}
\dot\alpha_{11}\bmf{m}=\textstyle{\frac{3}{2}} 
H_k \dot \epsilon^p_{11} \bmf{m}~,
\end{equation}

\noindent which further simplifies to 

\begin{equation}
\dot\alpha_{11}=H_{ku}\dot\epsilon^p_{11}~,
\end{equation}

\noindent since $\bmf m$ is a constant tensor and with the introduction 
of $H_{ku} \equiv (3/2)H_k$, the uniaxial version of the kinematic 
hardening parameter, as done previously for the isotropic hardening parameter
in Eq.\;\eqref{E:H_iu_define}. The quantity $H_{ku}$ is input by the user to define the
uniaxial equivalent kinematic hardening rate. 

The last step is to solve for $\sigma_{11}$ in terms of $\epsilon^p_{11}$. Using 
Eq.\;\eqref{E:n_define2} and the definitions above for the 1-D forms of $\bmf \sigma'$
and $\bmf \alpha$, we can write


\begin{equation}\label{E:n_u_define}
\bmf n_u = \frac{\bmf \sigma'_u - \bmf \alpha_u} 
{\parallel \bmf \sigma'_u - \bmf \alpha_u \parallel }
=
\frac {\sigma_{11} \bmf m - \alpha_{11} \bmf m }{\parallel \sigma_{11} 
\bmf m - \alpha_{11} \bmf m 
\parallel }
= \sqrt{\textstyle{\frac{3}{2}}}\;
\text{sgn} \left(\sigma_{11} -\alpha_{11} \right) \bmf m~.
\end{equation}

\noindent Equation\;\eqref{E:normal_flow} becomes 
(using Eq.\;\eqref{E:epsilon_p_equiv_define})
\begin{equation}
\textstyle{\frac{3}{2}}\dot\epsilon^p_{11}\bmf{m}=
\dot{\lambda}\sqrt{\frac{3}{2}}\;\bmf{m}\;
 \text{sgn} \left(\sigma_{11} -\alpha_{11} \right) ~,
\end{equation}

\noindent where the sign of $\dot \epsilon^p_{11}$ sets the sign of
$\dot \lambda$. Again the scalar terms on either side of the above 
equation must be equal. Rearranging the scalar equality yields

\begin{equation}\label{E:dot_lambda_u_defined}
\dot{\lambda}=\sqrt{\textstyle{\frac{3}{2}}}\;\dot\epsilon^p_{11}
\text{sgn} \left(\sigma_{11} -\alpha_{11} \right) ~.
\end{equation}

\noindent Also, using the rate form of Eq.\;\eqref{E:sigma_prime_1D_define} 
and the above form for $\bmf n_u$, we have

\begin{equation}
\bmf n_u:\dot{\bmf \sigma}_u'=\sqrt{\textstyle{\frac{2}{3}}}\;
\text{sgn}(\sigma_{11} - \alpha_{11})\;\dot\sigma_{11}.
\end{equation}


Substituting the preceding equations into the limit surface,
Eq. (\ref{E:limit_surface}), generates the relationship between the 
uniaxial stress and uniaxial strain: 

\begin{equation}
h(f)\dot\sigma_{11}=\textstyle{\frac{3}{2}}\dot\epsilon^p_{11}~.
\end{equation}

\noindent The straightforward path to determine the 1-D (uniaxial) equivalents 
of the material parameters $\beta$ and $\delta$ is to let $H_i=H_k=0$, which
makes $\bmf \alpha \equiv \bmf 0$  Then from Eqs.\;\eqref{E:h_define}, \eqref{E:f_define_1D}
with $\alpha_{11} = H_{iu} = 0$,

\begin{equation}
h(f) = \frac{f}{\delta(\beta-f)} =  \frac{\sqrt{\frac{2}{3}}\left( \sigma_{11} - \sigma_{ys} \right)}
{\delta \left[ \beta - \left(\sqrt{\frac{2}{3}} \left\{\sigma_{11} - 
\sigma_{ys} \right\}\right) \right] }~,
\end{equation}

\noindent which simplifies to

\begin{equation}
h(f) = \frac{\sigma_{11}-\sigma_{ys}}
{\delta \left[\sqrt{\textstyle{\frac{3}{2}}}\beta-
(\sigma_{11}-\sigma_{ys})\right]}~,
\end{equation}
\noindent where $\sigma_{11}\ge \sigma_{ys}$ is assumed for simplicity here to determine
the remaining material constants $\delta$ and $\beta$.

The separable differential equation 

\begin{equation}
{\textstyle{\frac{3}{2}}} \dot\epsilon^p_{11} = 
 \frac{{\sigma_{11}}-\sigma_{ys}}
{\delta \left[\sqrt{\textstyle{\frac{3}{2}}}\beta-
(\sigma_{11}-\sigma_{ys})\right]}
\;\dot\sigma_{11}
\end{equation}

\noindent is  integrated to compute

\begin{equation}
\epsilon^p_{11}=\frac{-1}{(3/2)\delta}\left[\sigma_{11}-\sigma_{ys} + 
\sqrt{\textstyle{\frac{3}{2}}}\beta \;\rm{ln} \left| \frac{\sigma_{11}-\sigma_{ys}-
\sqrt{\textstyle{\frac{3}{2}}}\beta}{\sqrt{\textstyle{\frac{3}{2}}}\beta} \right| \right]~.
\end{equation}

\noindent Thus, we can identify the 1-D equivalent definitions 
for $\beta$ and $\delta$ as $\beta_u \equiv \sqrt{3/2}\;\beta$ and $\delta_u\equiv(3/2)\delta$ such that

\begin{equation}
\epsilon^p_{11}=\frac{-1}{\delta_u}\left[\sigma_{11}-\sigma_{ys} +
\beta_u \; \rm{ln} \left| \frac{\sigma_{11}-\sigma_{ys}-\beta_u}{\beta_u} \right| \right]~, 
\quad \sigma_{11} \ge \sigma_{ys}~.
\end{equation}

The following table summarizes the 3-D algorithmic entities and their 1-D counterparts, for 
the Generalized Plasticity model. \textit{Note that WARP3D takes the 1-D equivalent material 
parameters as input values:} $H_{iu}, H_{ku}, \beta_u, \delta_u$.

\begin{table}[htb]	
	\centering
		\begin{tabular}{ | l | c |  c | }
		\hline
		Entity & 3-D algorithm & 1-D specialization \\
		\hline \hline
accumulated plastic strain & $\bar e^p = \int^t_0 \dot{\bar e}^p dt$ & 
$\bar e^p_u\equiv \sqrt{\textstyle{\frac{3}{2}}}\bar \epsilon^p_u;\quad 
\bar \epsilon^p_u = \int^t_0 |\dot \epsilon^p_{11}| dt $ \\

hardening parameter & $H$ & $H_{u}\equiv(3/2)H$ \\

limit function extent parameter & $\beta$ & 
$\beta_u\equiv\sqrt{\textstyle{\frac{3}{2}}}\beta$ \\

limit function rate parameter & $\delta$ & $\delta_u\equiv(3/2)\delta$ \\

current yield stress & $k(\bar e^p)=k_0+H_i \bar e^p$ & 
$k \left( \bar \epsilon^p_u \right) = 
\sqrt{ \textstyle {\frac{2}{3}} } 
\left( \sigma_{ys}+
H_{iu} \bar \epsilon^p_u \right)$ \\
\hline
		\end{tabular}
%  \caption{Properties for cyclic Material Model: nonlinear\_hardening option}
%	\label{table:nonlinear_hardening}
\end{table}


%--------------------------------------------------------------------------------
\subsubsection{Reducing 3-D constitutive equations to 1-D: Nonlinear Hardening}
%--------------------------------------------------------------------------------
The continuum 3-D equations for the Nonlinear Hardening (NH) model are summarized below:

\begin{equation} \label{E:yield_surface_nh}
f = \parallel \bmf{\sigma}' - \bmf{\alpha} \parallel - k(\bar e^p) ~,
\end{equation}

\begin{equation} \label{E:isotropic_nh}
k(\bar e^p) = k_0+ Q \left [ 1- exp \left( -b \bar e^p \right) \right ]~,
\end{equation}

\begin{equation}\label{E:n_define_nh}
\bmf n =  \frac {df/d\bmf{\sigma}'} {\parallel df/d\bmf{\sigma}' \parallel} =
\frac{\bmf \sigma' - \bmf \alpha} {\parallel \bmf \sigma' - \bmf \alpha \parallel }\;;
\quad \parallel \bmf n \parallel \equiv 1~,
\end{equation}

\begin{equation}\label{E:normal_flow_nh}
\dot{\bmf{\epsilon}}^p=\dot{\lambda}\bmf{n}~,
\end{equation}

\begin{equation} \label{E:kinematic_nh}
\dot{\bmf{\alpha}}=\dot{\lambda} \left( H\bmf{n} -\gamma \bmf{\alpha} \right)~,
\end{equation}

\noindent where the norm of $\bmf \epsilon^p$, the norm of its rate, 
$\dot{\bar e}^p$, and the
accumulated (monotonically increasing) plastic strain
measure, $\bar e^p$, are defined as in Eqs.\;\eqref{E:norm_epsilon_p_defined},
\eqref{E:e_bar_p_rate_define}, and \eqref{E:e_bar_p_define};
$k_0$ is the radius of the Mises yield cylinder 
at the onset of plastic flow; $H$ provides a constant rate of kinematic
hardening with increasing plastic flow; and $\gamma$ provides the nonlinearity
in the kinematic hardening rate. $Q$ and $b$ introduce an isotropic 
hardening response that saturates to a constant yield cylinder
radius, $k_0 + Q$, at
large values of the accumulated plastic strain. All values above are cast in terms 
of the 3-D formulation (uniaxial forms are introduced later in this section). 

To identify the isotropic hardening parameters $Q$ and $b$, 
Eq.\;\eqref{E:isotropic_nh} 
is specialized to the 1-D case using Eq.\;\eqref{E:k_sigma_11_define} 
and Eq.\;\eqref{E:e_bar_p_u_define}

\begin{equation}
k(\bar e^p_u) = k_0 + Q\left [ 1- exp \left( -b \bar e^p_u \right) \right ]~,
\end{equation}

\noindent which can be written in the form (using Eqs.\;\eqref{E:k_sigma_11_define},
\eqref{E:e_bar_p_u_define})

\begin{equation}
k(\bar \epsilon^p_u) = \sqrt{\textstyle {\frac{2}{3}}} \sigma_{ys} + 
\left(\sqrt{\textstyle {\frac{2}{3}}}\; Q_u \right)
\left[ 1- exp \left( -\sqrt{\textstyle {\frac{3}{2}}}\;b \bar \epsilon^p_u \right) \right ]~,
\end{equation}

\noindent where uniaxial definitions of the isotropic hardening parameters are
introduced such that
$Q=\sqrt{2/3}\;Q_u$ and in the following $b_u = \sqrt{3/2}\;b$. Here, $\sigma_{ys}$
and $Q_u$ denote the yield stress and saturation flow stress, respectively,
in uniaxial tension. The dimensionless term $b_u$ sets the rate at 
which the saturation flow stress is
attained with increasing (uniaxial) plastic strain. The 1-D form of the isotropic hardening 
function thus simplifies to 

\begin{equation}\label{E:Qu_bu_define}
k\left(\bar \epsilon^p_u\right) = 
\sqrt{ \textstyle {\frac{2}{3}} } 
\left( \sigma_{ys}+
Q_u \left[ 1- exp \left( -b_u \bar \epsilon^p_u \right) \right ]\right)~,
\end{equation}

\noindent where the 1-D forms $\sigma_{ys}$, $Q_u$ and $b_u$ are input by the user to define model properties. 


To determine the 1-D form of the kinematic hardening parameters $H$ and $\gamma$,
write the rate of backstress, Eq.\;\eqref{E:kinematic_nh}, in the 1-D form

\begin{equation}
\dot{\bmf{\alpha}}_u = \dot \lambda \left( H \bmf{n}_u - \gamma \bmf{\alpha}_u \right)~.
\end{equation}

\noindent Now use the 1-D specializations for $\dot{\bmf{\alpha}}_u$, $\bmf{n}_u$, and $\bmf{\alpha}_u$
from Eqs.\;\eqref{E:alpha_u_defined}, \eqref{E:n_u_define}, and \eqref{E:m_define}
to write

\begin{equation}
\dot\alpha_{11} \bmf{m} = \dot\lambda \left (
\sqrt{\textstyle{\frac{3}{2}}} H \text{sgn} \left(\sigma_{11}-\alpha_{11}\right) \bmf{m}-
 \gamma \alpha_{11}\bmf{m} \right)~.
\end{equation}

\noindent Equating terms applied to the constant tensor, $\bmf{m}$, leads to the simplified form
\begin{equation}
\dot\alpha_{11} = \dot\lambda \left (
\sqrt{\textstyle{\frac{3}{2}}} H \text{sgn} \left(\sigma_{11}-\alpha_{11}\right) -
 \gamma \alpha_{11}\right)~.
\end{equation}

\noindent Using the 1-D form for the rate of the consistency parameter, $\dot \lambda$, given by
Eq.\;\eqref{E:dot_lambda_u_defined} yields

\begin{equation}
\dot{\alpha}_{11} = \sqrt{\textstyle{\frac{3}{2}}}\; \dot\epsilon^p_{11}
 \text{sgn} \left(\sigma_{11}-\alpha_{11}\right)
\left (
\sqrt{\textstyle{\frac{3}{2}}} H \text{sgn} \left(\sigma_{11}-\alpha_{11}\right) -
 \gamma \alpha_{11}\right)~,
\end{equation}

\noindent which can be re-arranged to the form


\begin{equation}\label{E:define_Hu_gammau}
\dot{\alpha}_{11} = \left[ \textstyle{\frac{3}{2}} 
H  -
\sqrt{\textstyle{\frac{3}{2}}} \;\gamma \alpha_{11} \text{sgn} \left(\sigma_{11}-\alpha_{11}\right)\right] \dot\epsilon^p_{11}~.
\end{equation}

\noindent The 1-D forms for $H$ and $\gamma$ are seen to be
$H_u = (3/2)H$ and $\gamma_u = \sqrt{3/2}\;\gamma$. Recall also that 
$\dot\epsilon^p_{11}$ can be positive or negative. Let $\nu=\pm1$ set the sign of 
$\dot\epsilon^p_{11}$  based on the direction of loading (tension/compression). We also
see that $\text{sgn} \left(\sigma_{11}-\alpha_{11}\right)$ carries the same sign
as the plastic strain rate, \emph{e.g.} in compression yielding both
$\text{sgn} \left(\sigma_{11}-\alpha_{11}\right)$ and $\dot\epsilon^p_{11}$
are negative. For simplicity, drop the 11 subscripts and simplify

\begin{equation}\label{E:define_Hu_gammau1}
\dot\alpha = \nu H |\dot\epsilon^p| - \gamma \alpha |\dot\epsilon^p|~.
\end{equation}


The nonlinear kinematic hardening model with $\gamma >0$ defines a limit
surface by bounding the backstress magnitude at large plastic strains. The above equation can 
be re-arranged into a separable form such that 
\begin{equation}
\int^{\epsilon^p}_{\epsilon^p_0} \nu d\epsilon^p =
\int^{\alpha}_{\alpha_0} 
\frac{d\alpha}
{ \nu H_u   - 
\gamma_u \alpha  }~.
\end{equation}

\noindent Integrating both sides yields

\begin{equation}
\nu \left( \epsilon^p - \epsilon^p_0\right) = 
-\frac{1}{\gamma_u}\ln \left [ \nu H_u - \alpha \gamma_u \right]
 {|}^{\alpha}_{\alpha_0}~.
\end{equation}

\noindent Re-arrange, take the exponential of both sides, 
and solve for $\alpha$ to find

\begin{equation}
\alpha = \nu \frac{H_u}{\gamma_u} + 
\left( \alpha_0 -\nu \frac{H_u}{\gamma_u}\right )
\exp\left[-\nu \gamma_u \left ( \epsilon^p - \epsilon^p_0 \right)\right] ~,
\end{equation}

\noindent where at large plastic strains we have (see also Ref. [3])

\begin{equation}
\alpha = \alpha^{limit}_{11} = \pm1 \frac{H_u}{\gamma_u}~.
\end{equation}

\noindent In this limit expression, both $H_u$ and $\gamma_u$ refer to their definitions in the
1D formulation. Use this limiting expression for $\alpha_{11}$ in the 1-D form of the
yield function, Eq.\;\eqref{E:f_define_1D}, eliminate isotropic hardening for 
simplicity and re-arrange to find

\begin{equation}
k^{limit} = k_0 + \frac{H}{\gamma}~.
\end{equation}
\noindent In uniaxial terms,
\begin{equation}
\sigma_u^{limit} = \sigma_{ys} + \sqrt{\textstyle{ \frac{3}{2}}} \frac{H}{\gamma}~,
\end{equation}

\noindent and using the uniaxial definitions for $H$ and $\gamma$ from Eq.\;
\eqref{E:define_Hu_gammau} we have

\begin{equation}
\sigma_u^{limit} = \sigma_{ys} +\frac{H_u}{\gamma_u}~.
\end{equation}


The following table summarizes the 3-D algorithmic entities and their 1-D counterparts, for 
the Nonlinear Hardening model. \textit{Note that WARP3D takes the 1-D equivalent material 
parameters as input values:} $Q_u$, $b_u$, $H_u$, and $\gamma_u$.

\begin{table}[htb]	
	\centering
		\begin{tabular}{ | l | c |  c | }
		\hline
		Entity & 3-D algorithm & 1-D specialization \\
		\hline \hline
accumulated plastic strain & $\bar e^p = \int^t_0 \dot{\bar e}^p dt$ & 
$\bar e^p_u\equiv \sqrt{\textstyle{\frac{3}{2}}}\bar \epsilon^p_u;\quad 
\bar \epsilon^p_u = \int^t_0 |\dot \epsilon^p_{11}| dt $ \\

isotropic parameter & $Q$ & $Q_u \equiv \sqrt{3/2}\; Q$ \\

isotropic parameter & $b$ & $b_u \equiv \sqrt{3/2}\; b$ \\

kinematic parameter & $H$ & $H_u \equiv (3/2) H$ \\

kinematic parameter & $\gamma$ & $\gamma_u \equiv \sqrt{3/2}\; \gamma$ \\

current yield stress ($k, \sigma_u$) & $k_0+ Q \left[ 1 - exp \left( -b  \bar e^p\right)\right]$ & 
$\sigma_{ys}+
Q_u \left[ 1 - exp \left( -b_u  \bar \epsilon^p_u\right)\right] $ \\
\hline
		\end{tabular}
%  \caption{Properties for cyclic Material Model: nonlinear\_hardening option}
%	\label{table:nonlinear_hardening}
\end{table}

 
%*****************************************************
\subsection {References}
%*****************************************************
\small
[\refstepcounter{sectrefs}\label {R:FA1965}\thesectrefs]~P. J. Armstrong and C. O. Frederick. A mathematical representation of the
multi-axial Baushinger effect. Tech. report C.E.G.B. Report RD/B/N731,
Berkeley Nuclear Laboratories, R\&D Department, 1965.

\medskip
\noindent[\refstepcounter{sectrefs}\label {R:LTA1993}\thesectrefs]~J. Lubliner, 
R. L. Taylor, and F. Auricchio. A new model of generalized 
plasticity and its numerical implementation. \textit{International Journal of Solids 
and Structures}, 30:3171�3184, 1993.

\medskip
\noindent[\refstepcounter{sectrefs}\label {R:LC2000}\thesectrefs]~J. Lemaitre and 
J. L. Chaboche. \textit{Mechanics of Solid Materials}. Cambridge 
University Press, 2000.

\medskip
\noindent[\refstepcounter{sectrefs}\label {R:AT1995}\thesectrefs]~F. Auricchio 
and R. L. Taylor. Two material models for cyclic plasticity:
non-linear kinematic hardening and generalized plasticity. \textit{International Journal 
of Plasticity}, 11:65�96, 1995. 

\medskip
\noindent[\refstepcounter{sectrefs}\label {R:SH2000}\thesectrefs]~Simo, J.C., 
Hughes, T.J.R. \textit{Computational
Inelasticity}. Springer, 2000. ISBN: 978-0-387-97520-7.


\end{document}
