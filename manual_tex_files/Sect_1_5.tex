
\documentclass[11pt]{report}
\usepackage{geometry} 
\geometry{letterpaper}

%---------------------------------------------
\setlength{\textheight}{630pt}
\setlength{\textwidth}{450pt}
\setlength{\oddsidemargin}{14pt}
\setlength{\parskip}{1ex plus 0.5ex minus 0.2ex}


%----------------------------------------
\usepackage{amsmath}
\usepackage{layout}
\usepackage{color}
\usepackage{array}

%----------------------------------------------
\usepackage{fancyhdr} \pagestyle{fancy}
\setlength\headheight{15pt}
\lhead{\small{User's Guide - \textit{WARP3D}}}
\rhead{\small{Dynamic Analysis}}
\fancyfoot[L] {\small{\textit{Chapter {\thechapter}}\ \   (Updated: 1-26-2014)}}
\fancyfoot[C] {\small{\thesection-\thepage}}
\fancyfoot[R] {\small{\textit{Introduction}}}

%---------------------------------------------------
\usepackage{graphicx}
\usepackage[labelformat=empty]{caption}
\numberwithin{equation}{section}
\usepackage{bm}

%---------------------------------------------
%     --- make section headers in helvetica ---
%
\usepackage{sectsty} 
\usepackage{xspace}
\allsectionsfont{\sffamily} 
\sectionfont{\large}
\usepackage[small,compact]{titlesec} % reduce white space around sections
%---------------------------------------------->
%
%
%   which fonts system for text and equations. with all commented,
%   the default LaTex CM fonts are used
%
%
\frenchspacing
%\usepackage{pxfonts}  % Palatino text 
%\usepackage{mathpazo} % Palatino text
%\usepackage{txfonts}


%---------  local commands ---------------------


\newcommand{\bmf } {\boldsymbol }  %bold math symbol
\newcommand{\bsf } [1]{\textrm{\textit{#1}}\xspace}
\newcommand{\ul} {\underline}
\newcommand{\hv} {\mathsf}   %helvetica text inside an equation
\newcommand{\eg}{\emph{e.g.},\xspace}
\newcommand{\ti}{\emph}
\newcommand{\bardelta}{\bar \delta}
\newcommand{\vepsilon}{\varepsilon}
\newcommand{\etal}{\ti{et al.}\xspace}

\newenvironment{offsetpar}[1]%
{\begin{list}{}%
         {\setlength{\leftmargin}{#1}}%
         \item[]%
}
{\end{list}}

%
%
%        optional definition for bullet lists which
%        reduces white space.
%
\newcommand{\squishlist}{
 \begin{list}{$\bullet$}
  { \setlength{\itemsep}{0pt}
     \setlength{\parsep}{3pt}
     \setlength{\topsep}{3pt}
     \setlength{\partopsep}{0pt}
     \setlength{\leftmargin}{1.5em}
     \setlength{\labelwidth}{1em}
     \setlength{\labelsep}{0.5em} } }

\newcommand{\squishlisttwo}{
 \begin{list}{$\bullet$}
  { \setlength{\itemsep}{0pt}
     \setlength{\parsep}{0pt}
    \setlength{\topsep}{0pt}
    \setlength{\partopsep}{0pt}
    \setlength{\leftmargin}{2em}
    \setlength{\labelwidth}{1.5em}
    \setlength{\labelsep}{0.5em} } }

\newcommand{\squishend}{
  \end{list}  }
%


%-------------------------------------
\newcounter{sectrefs}
\setcounter{sectrefs}{0}
\setcounter{chapter}{1}
\setcounter{section}{4}
\setcounter{figure}{0}
\renewcommand{\thefigure}{\thesection.\arabic{figure}}
%
%--------------------------------------
%
%
%
%              start document 
%              ==========
%
%
\begin{document}

\section{Dynamic Analysis: Newmark \bm{$\beta$} Method}
\noindent 
Numerical integration of the equations of motion in WARP3D is performed 
using a method attributed to Newmark [\ref{R:N1959}]. This approach employs a two 
parameter family of equations that define the displacement, velocity, and acceleration 
at time $t_{n+1}$ in terms of the displacement increment from $t_n$ to $t_{n+1}$ 
and the kinematic 
state at time $t_n$. These equations derive from successive application of the extended 
mean value theorem of differential calculus. Consider first the velocities at time $t_n$ 
and $t_{n+1}$. Use of the extended mean value theorem for the first derivative leads 
to the equation 
%
\begin{equation}\label{E:veclocities1}
\dot {\bm{u}}_{n+1} = \dot {\bm{u}}_n + \Delta t\, \ddot{\bm{u}}_\gamma; ~~
\ddot{\bm{u}}_\gamma  \in \left [ \ddot{\bm{u}}_n , \ddot{\bm{u}}_{n+1} \right ]\ .
\end{equation}
%
\noindent Using the relationship
%
\begin{equation}\label{E:veclocities2}
\ddot{\bm{u}}_\gamma = \left (1-\gamma\right ) \ddot{\bm{u}}_n + 
\gamma \ddot{\bm{u}}_{n+1};~~ 0 \le \gamma \le 1
\end{equation}
%
\noindent Eq. (\ref{E:veclocities1}) can be rewritten as
%
\begin{equation}\label{E:veclocities3}
\dot{\bm{u}}_{n+1} = \dot{\bm{u}}_n + \left (1-\gamma\right ) \Delta t \,\ddot{\bm{u}}_n + 
\gamma \Delta t\, \ddot{\bm{u}}_{n+1}\ .
\end{equation}
%
\noindent Equation (\ref{E:veclocities3}) provides an exact result for a given time interval 
if the parameter $\gamma$ can be chosen correctly. Even so, the constant acceleration 
$\ddot{\bm{u}}_\gamma$ upon integration of Eq. (\ref{E:veclocities1}) does not necessarily produce the 
correct displacement at time $t_{n+1}$ in terms of the displacement and velocity at 
time $t_n$. Accordingly, the extended mean value theorem for the second derivative is invoked to yield 
%
\begin{equation}\label{E:displ1}
\bm{u}_{n+1} =\bm{u}_n + \Delta t \, \dot{\bm{u}}_n + \frac{\Delta t^2}{2}\ddot{\bm{u}}_\beta;
~~
\ddot{\bm{u}}_\beta  \in \left [ \ddot{\bm{u}}_n , \ddot{\bm{u}}_{n+1} \right ]\ .
\end{equation}
%
\noindent Again a relationship of the form
%
\begin{equation}\label{E:displ2}
\ddot{\bm{u}}_\beta = \left (1-2 \beta \right ) \ddot{\bm{u}}_n + 
2 \beta \ddot{\bm{u}}_{n+1};~~ 0 \le 2\beta \le 1
\end{equation}
%
\noindent is employed to recast Eq. (\ref{E:displ1}) as
%
\begin{equation}\label{E:displ3}
\bm{u}_{n+1} =\bm{u}_n + \Delta t \, \dot{\bm{u}}_n + \left( 1- 2 \beta \right )
\frac{\Delta t^2}{2}\ddot{\bm{u}}_n + \beta \Delta t ^2\, \ddot{\bm{u}}_{n+1}\ .
\end{equation}
%
Equation (\ref{E:displ3}) also provides an exact result for a given time interval 
with a correct choice of the parameter $\beta$. Of course, 
in general it is impossible to choose either $\gamma$ or $\beta$ correctly without 
knowing the solution in advance, thus the key approximation in the Newmark 
method lies in the choice of $\gamma$ and $\beta$. Newmark showed that  $\gamma=1/2$ 
avoids spurious damping in linear systems. 
The pertinent equations of the Newmark method then become
%
\begin{equation}\label{E:NM1}
\dot{\bm{u}}_{n+1} = \dot{\bm{u}}_n + \frac{\Delta t }{2} 
\left( \ddot{\bm{u}}_n  +\ddot{\bm{u}}_{n+1}  \right )
\end{equation}
%
%
\begin{equation}\label{E:NM2}
\bm{u}_{n+1} = \bm{u}_n + \Delta t \dot{\bm{u}}_n + \left( 1- 2 \beta \right )
\frac{\Delta t ^2}{2} \ddot{\bm{u}}_n + \beta \Delta t ^2 \ddot{\bm{u}}_{n+1} 
\end{equation}
%

A wide range of values for the parameter $\beta$ are possible. For example, setting $\beta=0$
leads to the second central difference method. A choice of $\beta=1/6$ defines the linear 
acceleration method, wherein the acceleration is assumed to vary  linearly over the time increment. 
The choice of $\beta=1/4$ produces the constant average acceleration method. 
Newmark demonstrated that $\beta=1/4$ renders the method unconditionally stable for linear problems; 
other choices must satisfy a time step constraint to maintain stability throughout the solution. For materially nonlinear problems, Schoeberle and Belytschko [\ref{R:SB1975}] established that the use of  $\beta=1/4$ leads to unconditional stability when nonlinear equilibrium iterations (Newton) are performed to satisfy an energy convergence criterion, and for nonlinear elastic problems Hughes [\ref{R:H2000}] found much the same situation. In WARP3D,  $\beta=1/4$ is the default value although users may modify this value in the nonlinear solution parameters input.

Use of the Newmark method leads to an implicit dynamic formulation which requires solution of a
set of simultaneous linear or nonlinear equations to compute a displacement increment. 
Assuming that $\beta\neq 0$, Eqs. (\ref{E:NM1}, \ref{E:NM2}) are manipulated to the form
%
\begin{equation}\label{E:NMF1}
\Delta \bm{u}_{n+1} = \bm{u}_{n+1} - \bm{u}_n
\end{equation}
%
\begin{equation}\label{E:NMF2}
\dot{\bm{u}}_{n+1} = \frac{1}{2\beta \Delta t}\Delta \bm{u}_{n+1} -
\frac{(1-2\beta)}{2\beta}\dot{\bm{u}}_n -
\frac{(1-4\beta)}{4\beta}\Delta t \,\ddot{\bm{u}}_n 
\end{equation}
%
\begin{equation}\label{E:NMF3}
\ddot{\bm{u}}_{n+1} = \frac{1}{\beta \Delta t^2}\Delta \bm{u}_{n+1} -
\frac{1}{\beta\Delta t}\dot{\bm{u}}_n -
\frac{(1-2\beta)}{2\beta}\ddot{\bm{u}}_n 
\end{equation}
\noindent where the implicit feature of the solution arises with the requirement
that $\bm{u}_{n+1}$ must be known to update the velocities and accelerations at
$t_{n+1}$.

Equations (\ref{E:NMF1}-\ref{E:NMF3}) are substituted into the equations of 
motion and into the 
chosen iterative nonlinear solution algorithm. The displacement 
increment for the current time step, $\Delta \bm{u}_{n+1}$, is computed from those
operations. Equations (\ref{E:NMF1}-\ref{E:NMF2}) then define the 
velocity and acceleration for the current estimate of the solution at time $t_{n+1}$. 

%*****************************************************
\subsection {References}
%*****************************************************
\small

\noindent[\refstepcounter{sectrefs}\label {R:N1959}\thesectrefs]~N. Newmark. A Method of 
Computation for Structural Dynamics. \ti{Journal of the Engineering Mechanics Division}, ASCE, Vol. 32, 
No. EM3, 1959, pp. 67?94.

\medskip
\noindent[\refstepcounter{sectrefs}\label {R:SB1975}\thesectrefs]~D. F. Schoeberle, D.F. and Belytschko, T.
On the Unconditional Stability of an Implicit Algorithm for Nonlinear Structural Dynamics.
\ti{Journal of Applied Mechanics}, Vol. 42, 1975, pp. 865?869.

\medskip
\noindent[\refstepcounter{sectrefs}\label {R:BLME2014}\thesectrefs]~T. Belytschko, W.K. Liu, B. Moran, K. Elkhodary.
Nonlinear Finite Elements for Continua and Structures. 2nd Edition. John Wiley \& Sons, Inc., 2014.

\medskip
\noindent[\refstepcounter{sectrefs}\label {R:H2000}\thesectrefs]~T.J. Hughes.
The Finite Element Method: Linear Static and Dynamic Finite Element Analysis. Dover, Inc., 2000.

\end{document}
