%
\documentclass[11pt]{report}
\usepackage{geometry} 
\geometry{letterpaper}

%---------------------------------------------
\setlength{\textheight}{630pt}
\setlength{\textwidth}{450pt}
\setlength{\oddsidemargin}{14pt}
\setlength{\parskip}{1ex plus 0.5ex minus 0.2ex}


%----------------------------------------
\usepackage{amsmath}
\usepackage{layout}
\usepackage{color}
\usepackage{hyphenat}

%----------------------------------------------
\usepackage{fancyhdr} \pagestyle{fancy}
\setlength\headheight{15pt}
\lhead{User's Guide - \textit{WARP3D}}
\rhead{\textit{Utility (*) Commands}}
\fancyfoot[L] {\textit{Chapter {\thechapter}}}
\fancyfoot[C] {\thesection-\thepage}
\fancyfoot[R] {\textit{Utility Commands}}

%---------------------------------------------------
\usepackage{graphicx}
\usepackage[labelformat=empty]{caption}
\numberwithin{equation}{section}

%---------------------------------------------
%     --- make section headers in helvetica ---
%
\frenchspacing
\usepackage{sectsty} 
\usepackage{xspace}
\allsectionsfont{\sffamily} 
\sectionfont{\large}
\usepackage[small,compact]{titlesec} % reduce white space around sections
%
%----------------------------------------------

%---------  local commands ---------------------

\newcommand{\tb} {\textbf}
\newcommand{\df} {\dotfill}
\newcommand{\nin} {\noindent}
\newcommand{\bmf } {\boldsymbol }  %bold math symbol
\newcommand{\bsf } [1]{\textrm{\textit{#1}}\xspace}
\newcommand{\ul} {\underline}
\newcommand{\hv} {\mathsf}   %helvetica text inside an equation
\newcommand{\eg}{\emph{e.g.},\xspace}
\newcommand{\ti}{\emph}
\newcommand{\noi}{\noindent}

%-------------------------------------
\newcounter{sectrefs}
\setcounter{sectrefs}{0}
\setcounter{chapter}{2}
\setcounter{section}{14}

%--------------------------------------
%--------------------------------------
%---------------------------------------

\begin{document}

\section{Utility (*) Commands}
%\layout
\noi Several utility commands are provided to manipulate input-output files, 
to control command echo, etc. Each command begins with an *. These
commands may be given at any time during input.

\noi {\bf{\ti{*echo Command}}}

\noi The \ti{*echo} command controls the ``echoing" of input 
commands to the current 
output device. By default, all commands are echoed. 
The \ti{*echo} command has the form

\begin{align*}
& \hv{{*\ \ul{echo}}\ }
\begin{Bmatrix}
\hv{\ul{on}} \\ \hv{\ul{off}}
\end{Bmatrix}
\end{align*}


\noi {\bf{\ti{*input Command}}}

\noi The \ti{*input} command controls the location (stream) from which input commands are 
read for processing. By default, the input stream is the user�s interactive display 
(window) or the 
Windows/Linux/Mac \ti{stdin} device. The input stream can be switched to a disk 
file or switched back to the interactive display

\begin{align*}
& \hv{{*\ \ul{input}\ (\ul{from})}\ }
\begin{Bmatrix}
\hv{\ul{display}} \\ \hv{(\ul{file}) <file\ name: label\ or\ string>}
\end{Bmatrix}
\end{align*}


\noi where the $<$string$>$ form is required with file names not meeting the definition 
of a $<$label$>$. \ti{*input from file ...} commands may be contained within referenced 
input files to create an input ``�stack" up to 10 levels deep. When an end-of-file 
condition is reached on the current file, the stack is popped to resume reading 
from the previous file. When reading of the last file completes, 
the input stream returns to the user�s display. In a batch (background) job, the program 
is terminated by the WARP3D command processor if an end-of-file condition 
(EOF) occurs at the highest level.


\noi {\bf{\ti{*output Command}}}

\noi The \ti{*output} command controls the location (stream) from which 
usual WARP3D output is directed. By default, the output stream is the user�s 
interactive display (window) or the 
Windows/Linux/Mac \ti{stdout} device. The output stream can be switched to a disk 
file or switched back to the interactive display

\begin{align*}
& \hv{{*\ \ul{output}\ (\ul{to})}\ }
\begin{Bmatrix}
\hv{\ul{display}} \\ \hv{(\ul{file}) <file\ name: label\ or\ string>}
\end{Bmatrix}
\end{align*}


\noi where the $<$string$>$ form is required with file names not meeting the definition 
of a $<$label$>$.




\noi {\bf{\ti{*time Command}}}

\noi The \ti{*time} command outputs the elapsed wall time in seconds for the
job. 

\newpage
\noi {\bf{\ti{*reset Command}}}

\noi When the WARP3D command processors read and interpret the 
command stream, errors of various types may be detected. 
When errors are encountered, the command processors
set an internal flag \ti{.true.} to prevent a \ti{compute} command 
from attempting a solution. This internal flag can be set to 
the ``no error" condition with the \ti{*reset} command, which has the form

\begin{align*}
& \hv{*\ \ul{reset} (\ul{error}) (\ul{flag})}
\end{align*}



\end{document}

 

