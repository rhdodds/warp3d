%
\documentclass[11pt]{report}
\usepackage{geometry} 
\geometry{letterpaper}

%---------------------------------------------
\setlength{\textheight}{630pt}
\setlength{\textwidth}{450pt}
\setlength{\oddsidemargin}{14pt}
\setlength{\parskip}{1ex plus 0.5ex minus 0.2ex}

%----------------------------------------
\usepackage{amsmath}
\usepackage{layout}
\usepackage{color}

%----------------------------------------------
\usepackage{fancyhdr} \pagestyle{fancy}
\setlength\headheight{15pt}
\lhead{\small{User's Guide - \textit{WARP3D}}}
\rhead{\small{Material: \textit{UMAT}}}
\fancyfoot[L] {\small{\textit{Chapter {\thechapter}}\ \   (Updated: 7-26-2012)}}
\fancyfoot[C] {\small{\thesection-\thepage}}
\fancyfoot[R] {\small{\textit{Elements and Material Models}}}

%---------------------------------------------------
\usepackage{graphicx}
\usepackage[labelformat=empty]{caption}
\numberwithin{equation}{section}
%\usepackage{epstopdf}

%---------------------------------------------
%     --- make section headers in helvetica ---
\usepackage{sectsty} 
\usepackage{xspace}
\allsectionsfont{\sffamily} 
\sectionfont{\large}
\usepackage[small,compact]{titlesec} % reduce white space around sections
%----------------------------------------------
%
%
%   which fonts system for text and equations. with all commented,
%   the default LaTex CM fonts are used
%
%
\frenchspacing
%\usepackage{pxfonts}  % Palatino text 
%\usepackage{mathpazo} % Palatino text
%\usepackage{txfonts}


%---------  local commands ---------------------

\newcommand{\bmf } {\boldsymbol }
\newcommand{\bsf } [1]{\textrm{\textit{#1}}\xspace}
\newcommand{\ul} {\underline}
\newcommand{\hv} {\mathsf}   %helvetica text inside an equation
\newcommand{\eg}{\emph{e.g.},\xspace}
\newcommand{\ie}{\emph{i.e.},\xspace}
\newcommand{\vs}{\emph{vs.}\xspace}
\newcommand{\ti}{\emph}
\newcommand{\veps}{\varepsilon}
\newcommand{\ol}{\overline}
\newcommand{\mdash}{\mbox{-}}
\newcommand{\clrr}{\color{red}}
\newcommand{\clrb}{\color{black}}
\newcommand{\Fig}{Fig.\xspace}
\newcommand{\Figs}{Figs.\xspace}
\newcommand{\umat}{\ti{umat}\xspace}


%
%
%        optional definition for bullet lists which
%        reduces white space.
%
\newcommand{\squishlist}{
 \begin{list}{$\bullet$}
  { \setlength{\itemsep}{0pt}
     \setlength{\parsep}{3pt}
     \setlength{\topsep}{3pt}
     \setlength{\partopsep}{0pt}
     \setlength{\leftmargin}{1.5em}
     \setlength{\labelwidth}{1em}
     \setlength{\labelsep}{0.5em} } }

\newcommand{\squishlisttwo}{
 \begin{list}{$\bullet$}
  { \setlength{\itemsep}{0pt}
     \setlength{\parsep}{0pt}
    \setlength{\topsep}{0pt}
    \setlength{\partopsep}{0pt}
    \setlength{\leftmargin}{2em}
    \setlength{\labelwidth}{1.5em}
    \setlength{\labelsep}{0.5em} } }

\newcommand{\squishend}{
  \end{list}  }
%


%-------------------------------------
\newcounter{sect

s}
\setcounter{sectrefs}{0}
\setcounter{chapter}{3}
\setcounter{section}{10}
\renewcommand{\thefigure}{\thesection.\arabic{figure}}

%--------------------------------------
%--------------------------------------
%---------------------------------------

\begin{document}

\section{Material Model Type: \textit{umat}}
%\layout
\noindent This model provides the complete material
constitutive response through computations performed by a
user-supplied subroutine \umat. The Fortran code for the \umat
and all supporting routines are placed in the file \ti{umats.f} for inclusion
in the WARP3D executable. The \umat programming interface
in WARP3D matches that used in Abaqus. Existing \umat routines 
developed for use with Abaqus can be used with very minimal changes in
WARP3D. Appendix J describes the \umat interface and
the mechanism to incorporate the routine(s) as part of the WARP3D
executable.

This section provides information needed by the user of the 
\umat to associate elements with the material and 
its user-definable properties.

The following example illustrates the process
\small
\begin{verbatim}
   structure cct
  c
    material inco718
      properties umat um_1 30000 um_2 0.3 um_3 -1.0 um_4 1.0e-04,
                              um_5 65.5 um_6 0.0 rho 1.2e-04
    material hastelloy
      properties umat um_1 28500 um_2 0.28 um_3 0.0 um_4 2.1e-04,
                              um_5 18.6 um_6 12.0 rho 0.0 
                             
    material steel_w_kinematic_hardening
      properties cyclic type nonlinear_hardening e 30000 nu 0.3 yld_pt 60,
        q_u 0.0 b_u 0.0 gamma_u 6000 H_u 5000 sig_tol 0.001 rho 1.1e-04
 \end{verbatim}
 \normalsize

\noindent In the above, three materials are defined for subsequent use to set 
the constitutive properties of finite elements. The first two materials
specify the \umat routine with different numerical values for the
six (6) \umat properties. The third material specifies the
built-in \ti{cyclic} material model included in WARP3D.

\ti{umats} may have up to fifty (50) user-definable property values. They
have names illustrated above in the form um\_1, um\_2, \dots um\_50. Properties
are all stored as float-point values, \ie um\_3 -4 is stored as -4.0. The
physical ordering of property values in the input data lines
is immaterial since each value is prefaced by the name. WARP3D stores the property
values in sequential, ascending order and provides them on each call to the
\umat routine. WARP3D is unaware of the number of property values
needed by the \umat or acceptable ranges of values. The \umat code is
responsible for such verification.


\subsection {Mass}

\noindent WARP3D handles computation of the element mass matrix and thus
requires the mass density value (rho) as with the built-in material
models.

The mass density is constant over the element with the value
set in the material definition as in the example above for 
inco718.


\subsection {Temperature Effects}

\noindent Two options exist for \umat developers: (1) the \umat
routine handles all processing of temperature effects with the
user specifying any required properties through
um\_1, um\_2, \dots; (2) the user specifies 
thermal expansion coefficients through the usual
property values of other (built-in) material models (alpha or alphax,
alphay, alphaz, ... alphayz) and WARP3D handles temperature
effects outside the \umat code. 

Option (1) offers the \umat developer an opportunity to 
make the temperature modeling as complex as necessary,
\eg temperature dependent values of thermal
expansion, values that vary with time or spatial
location of the material points, values read by the 
\umat from a separate data file, etc. 

Option (2) offers simplicity for \umat
developers when the thermal expansion coefficients are
invariant of temperature. The user assigns expansion 
coefficients as part of the material definition -- just include them
before or after all the um\_.. values. WARP3D input translators
set all thermal expansion values to zero by default and stores those
(zero) values unless the user specifies non-zero values.

%*****************************************************
\subsection {Model Output}
%*****************************************************
During computations, the \umat code may print information
to the current output device. The information, if any,
printed during computations is determined by the \umat
developer. The type and amount of such printed information
may be controlled by a \umat material property
as implemented by the developer.

WARP3D outputs the six (6) stress and strain values produced by the
\umat code. For the geometrically nonlinear
analysis, the strain values have the logarithmic definition;
stress values have the Cauchy definition.

The \umat code computes and returns to WARP3D the
updated work density, $U_0$, at each material point. This value is 
output by WARP3D with the label  ``eng\_dens''  following the
six stress components. WARP3D also computes and outputs the 
mises equivalent stress value from the stress components with the
label ``mises''

\begin{equation}
\sigma_{mises}= \frac {\sqrt{2}}{2}
\sqrt{ (\sigma_{11}-\sigma_{22})^2  +
(\sigma_{11}-\sigma_{33})^2  +
(\sigma_{22}-\sigma_{33})^2  + 6 ( \tau^2_{12} +
 \tau^2_{13} + \tau^2_{23} ) }
\end{equation}

\noindent where this value is computed independent of the \umat code.

In the strain output, WARP3D computes and outputs an ``eff\_eps''
value defined by

\begin{equation}
\veps_{eff}= \frac {\sqrt{2}}{3}
\sqrt{ (\veps_{11}-\veps_{22})^2  +
(\veps_{11}-\veps_{33})^2  +
(\veps_{22}-\veps_{33})^2  + 1.5 ( \gamma^2_{12} +
 \gamma^2_{13} + \gamma^2_{23} ) }
\end{equation}


\noindent The element stress output also contains three values for the \umat 
material model ``state" variables or any other values the
developer wishes to make available to the user. The output labels
are ``c1'', ``c2'' and ``c3''.

%*****************************************************
\subsection {Element Blocking}
%*****************************************************
As described in Section 2.6, elements are assigned to blocks
to facilitate parallel processing. For the computational material model \umat,
all elements in a block must refer to the same material defined
in an analysis model.

Consider an analysis model with 3 materials defined in the input
named \ti{mat\_a}, \ti{mat\_b} and \ti{mat\_c}. Each of these materials specifies
\umat as the computational material model. Different values for each of the
\umat properties (\eg \ti{um\_1}, \ti{um\_2}, $\dots$) may be defined in the materials
\ti{mat\_a}, \ti{mat\_b} and \ti{mat\_c}. The blocking requirement
described here requires that all elements assigned to a block have the same
associated material, either  \ti{mat\_a}, \ti{mat\_b} or \ti{mat\_c}.

%*****************************************************
\subsection {References}
%*****************************************************
\small
[\refstepcounter{sectrefs}\label {R:FA1965}\thesectrefs]~http://www.3ds.com/products/simulia/portfolio/abaqus/overview/

\end{document}
