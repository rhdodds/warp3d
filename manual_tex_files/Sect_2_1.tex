%!TEX TS-program=pdflatexmk
\documentclass[11pt]{report}
\usepackage{geometry} 
\geometry{letterpaper}

%---------------------------------------------
\setlength{\textheight}{630pt}
\setlength{\textwidth}{450pt}
\setlength{\oddsidemargin}{14pt}
\setlength{\parskip}{1ex plus 0.5ex minus 0.2ex}


%----------------------------------------
\usepackage{amsmath}
\usepackage{layout}
\usepackage{color}
\usepackage{array}

%----------------------------------------------
\usepackage{fancyhdr} \pagestyle{fancy}
\setlength\headheight{15pt}
\lhead{\small{User's Guide - \ti{WARP3D}}}
\rhead{\small{\ti{Model Name and Sizes}}}
\fancyfoot[L] {\small{\ti{Chapter {\thechapter}}\ \   (Updated: 12-13-2011)}}
\fancyfoot[C] {\small{\thesection-\thepage}}
\fancyfoot[R] {\small{\ti{Model Definition}}}

%---------------------------------------------------
\usepackage{graphicx}
\usepackage[labelformat=empty]{caption}
\numberwithin{equation}{section}

%---------------------------------------------
%     --- make section headers in helvetica ---
%
\usepackage{sectsty} 
\usepackage{xspace}
\allsectionsfont{\sffamily} 
\sectionfont{\large}
\usepackage[small,compact]{titlesec} % reduce white space around sections
%---------------------------------------------->
%
%
%   which fonts system for text and equations. with all commented,
%   the default LaTex CM fonts are used
%
%
\frenchspacing
%\usepackage{pxfonts}  % Palatino text 
%\usepackage{mathpazo} % Palatino text
%\usepackage{txfonts}


%---------  local commands ---------------------


\newcommand{\bmf } {\boldsymbol }  %bold math symbol
\newcommand{\bsf } [1]{\textrm{\ti{#1}}\xspace}
\newcommand{\ul} {\underline}
\newcommand{\hv} {\mathsf}   %helvetica text inside an equation
\newcommand{\eg}{\emph{e.g.},\xspace}
\newcommand{\ie}{\emph{i.e.},\xspace}
\newcommand{\vs}{\emph{vs.}\xspace}
\newcommand{\ti}{\emph}
\newcommand{\veps}{\varepsilon}
\newcommand{\ol}{\overline}
\newenvironment{offsetpar}[1]
{\begin{list}{}%
         {\setlength{\leftmargin}{#1}}%
         \item[]%
}
{\end{list}}

%
%
%        optional definition for bullet lists which
%        reduces white space.
%
\newcommand{\squishlist}{
 \begin{list}{$\bullet$}
  { \setlength{\itemsep}{0pt}
     \setlength{\parsep}{3pt}
     \setlength{\topsep}{3pt}
     \setlength{\partopsep}{0pt}
     \setlength{\leftmargin}{1.5em}
     \setlength{\labelwidth}{1em}
     \setlength{\labelsep}{0.5em} } }

\newcommand{\squishlisttwo}{
 \begin{list}{$\bullet$}
  { \setlength{\itemsep}{0pt}
     \setlength{\parsep}{0pt}
    \setlength{\topsep}{0pt}
    \setlength{\partopsep}{0pt}
    \setlength{\leftmargin}{2em}
    \setlength{\labelwidth}{1.5em}
    \setlength{\labelsep}{0.5em} } }

\newcommand{\squishend}{
  \end{list}  }
%


%-------------------------------------
\newcounter{sectrefs}
\setcounter{sectrefs}{0}
\setcounter{figure}{0}
\setcounter{chapter}{2}
\setcounter{section}{0}
\renewcommand{\thefigure}{\thesection.\arabic{figure}}

%
%--------------------------------------
%
%
%
%              start document 
%              ==========
%
%

\begin{document}


\LARGE
\hfill
\textbf{Chapter \thechapter}
\rule[0.15in]{450pt}{0.5mm}
\LARGE
\begin{flushright}
 \textbf{
{\fontfamily{phv}\selectfont Model Definition}}
\end{flushright}
\normalsize

This chapter describes the input commands to define a finite element model, 
to define a nonlinear/dynamic
solution algorithms, to request an analysis for a number of load steps, to request
computation of fracture mechanics parameters, to control crack-growth  and to request 
various types of output. 
Commands in this chapter are described in the recommended order of input:

\squishlist
\item	structure name and sizes (number of nodes and elements)
\item definition of ``materials" for association with elements in the model. 
Materials provide linear-elastic properties, material density, 
nonlinear properties and a ``type" of constitutive algorithm, 
\eg rate-dependent \ti{gurson} plasticity with damage by void nucleation and growth.
\item the type of each finite element in the model, the kinematic formulation for 
the element (large or small displacements) and the values of any 
properties for the element, \eg the order of numerical integration
\item	$X\mbox{-}Y\mbox{-}Z$ coordinates for all nodes in the model 
global coordinate system 
\item	connectivity of elements nodes to structure nodes -- termed 
\ti{incidences} 
\item assignment of contiguous lists of elements to ``blocks" for analysis. 
Blocking is required to support parallel execution on multi-core processors and on
supercomputers. All elements in a block must be the same type, have the same 
material model, the same type of kinematic formulation. Typical maximum number of
elements per block is often 64 or 128. For MPI-based parallel execution, element blocks
are assigned to domains.
\item absolute and relative displacement constraints imposed 
on nodes of the model, either zero or non-zero.
\item mesh tieing to connect geometrically compatible but
topologically mismatched regions in meshes
\item loading ``patterns" for the model. Loading patterns consist of nodal forces; 
element body forces, face tractions, face pressures which are converted to 
equivalent nodal forces; nodal and element temperature changes relative 
to a zero reference state.
\item initial velocities and displacements at nodes for dynamic analyses 
(otherwise assumed to all equal zero).
\item a nonlinear/dynamic loading which defines the increment of load to 
be applied during each load/time step. Loading increments for a step are 
defined using the loading patterns.
\item parameters to control the nonlinear/dynamic solution process, \eg 
the time increment for dynamic analysis, the type of equation solver 
(direct, iterative), number of threads, maximum number of Newton iterations, 
convergence tolerances, adaptive loading parameters etc.
\item rigid bodies of various shapes with assigned velocity vectors to
simulate frictionless, right-body contact
\item parameters to control the type of crack growth (node release, 
cell extinction, cohesive zones)
\item a request to compute displacements for a list of load steps
\item a request to output computed nodal and element results. Results 
for use by humans are directed to the current output device with 
appropriate pagination, headers, labels, etc.
\item a request to output computed nodal and element results in 
the format defined by the Patran modeling software. These results 
files are readable by Patran without further conversion.
\item requests to generate ``packet" files of results on a binary
sequential file -- very convenient for user-written specialized
post-processing of very large output sets
\item a request to compute and output values for the $J$-integral 
and various interaction integrals in fracture mechanics models 
\item	a ``save" command to write all current, essential data 
structures to a sequential binary file
for later use to restart an analysis. 
\item	a ``stop" command to terminate program execution.
\squishend
In typical analyses, multiple compute, output, $J$-integral and save commands 
appear in the input. Parameters to control the nonlinear/dynamic solution algorithm, \eg 
the time step, may be modified between analyses for sets of load steps. 
Constraints can be modified between analyses for load steps 
to effect incremental changes in the boundary conditions.
%
%------------------------------------------------------------------------------
\section{Model Name and Sizes}
%------------------------------------------------------------------------------
The definition of a finite element model begins with specification 
of an alphanumeric identifier for the model. The identifier appears on all pages 
of output. The command has the form
\begin{align*}
\hv{\ul{struct}ure\ <name:label> } 
\end{align*}
\noindent The first eight characters of model names are recognized
as unique.

\ti{The number of nodes and number of 
elements in the model must be specified prior to any 
other command related to nodal or elemental quantities}. WARP3D uses the specified
sizes to pre-allocate very large arrays, to support checking of the 
input data as it is entered and to support exhaustive 
consistency checking of the structural model for errors prior 
to the first compute request. An example of such 
an error is a node with no elements attached. The model 
sizes are defined with a command having the form
\begin{align*}
& \hv{\ul{num}ber\ (\ul{of})}\ \left [
\begin{Bmatrix}
\hv{\ul{node}s} \\ \hv{\ul{elem}ents}
\end{Bmatrix}
\hv{<size:integer>}(,) \right ]
\end{align*}

\noindent Examples of the above commands are
\small \begin{verbatim}
    structure bend_strip 
    number of nodes 3450 elements 4230
\end{verbatim}\normalsize

\noindent and 

\small \begin{verbatim}
    structure bend_strip 
    number of nodes 3450 number of elements 4230
\end{verbatim}\normalsize

All node and element identifiers are positive integers beginning with the value 1. 
\ti{Nodes and elements must each be numbered sequentially}.
Once specified, the number of nodes and elements cannot be modified 
through user commands.

\noindent \bf{\ti{Limits on Number of Nodes and Elements}}\rm

\noindent The maximum number of nodes and elements permitted in a 
model varies with the version of WARP3D being executed and the computer 
system executing the program. Typical limits are 2M nodes and 2M elements
on Linux and (Mac) OS X, and 500K nodes and 500K elements on Windows
(64-bit) systems. These limits are easily changed 
through one line in the source code followed by a re-compilation 
on the hardware platform.

\end{document}


