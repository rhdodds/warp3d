%

\input{common_preamble.tex}
%----------------------------------------------
%
%          --- header and footer contents ---
%
\usepackage{fancyhdr} \pagestyle{fancy}
\setlength\headheight{15pt}
\lhead{\small{User's Guide - \textit{WARP3D}}}
\rhead{\small{\textit{Parallel Execution}}}
\fancyfoot[L] {\small{\textit{Chapter {\thechapter}}\ \   (Updated: 5-10-2013)}}
\fancyfoot[C] {\small{\thesection-\thepage}}
\fancyfoot[R] {\small{\textit{Introduction}}}

%-------------------------------------
\newcounter{sectrefs}
\setcounter{sectrefs}{0}
\setcounter{chapter}{7}
\setcounter{section}{0}

%
%
%
%              start document 
%              ==========
%
%
\begin{document}

\LARGE
\hfill
\textbf{Chapter \thechapter}
\rule[0.15in]{450pt}{0.5mm}
\LARGE
\begin{flushright}
 \textbf{
{\fontfamily{phv}\selectfont Parallel Execution}}
\end{flushright}
\normalsize



\section{Introduction}
\noindent 
Parallel processing to perform the analysis of a large, 3-D finite element model offers the 
potential to reduce significantly the elapsed time needed to compute the solution. 
WARP3D offers two common parallel processing architectures: 
(1) threads and shared-memory, and (2) hybrid
systems which use message
passing between concurrently executing copies (ranks) of a program combined with
shared-memory and threads on each rank. Each architecture is described
briefly here as an introduction to further discussions in this chapter.


\subsection{Threads and Shared-Memory}
The simplest configuration is a processor chip that contains multiple cores, typically 
now 4, 6, or 8 cores but with an increased number expected in the future. Each of the cores is 
an independently executing computational unit. Each of the cores has 
effectively uniform, equal access to the large, main memory system, \ie the cores 
all share the memory (for example 4-64GB on a small desktop computer). Each 
core has its own local, high speed cache memory; 
the operating system and memory hardware work to ensure the coherency of 
data spread across the core cache memories and the main system memory. 

This design typifies current laptop, desktop and deskside 
systems running the Windows, Linux and OS X operating systems. 
The operating system dispatches programs for execution to 
a core and switches programs across the cores as needed. When a single program 
contains multiple tasks (called threads) that can be executed 
simultaneously, for example
iterations of certain do/for/while loops,
the operating system dispatches threads to the cores 
thus achieving parallel execution of the program. In larger compute servers, 
multiple processor chips will be included on the main board for the system, 
with each processor chip containing multiple cores (4, 6, 8, ..). Some hardware
designs extend the capability of each core to execute multiple, concurrent
threads. Thus a processor chip with 8 cores, each capable of running
2 threads offers the potential of running 16 concurrent tasks.

WARP3D utilizes the shared-memory (SM) capabilities with threads 
that can be dispatched and executed concurrently. 
This is accomplished in various parts of the code primarily (1) through 
the block structuring of elements (Section 2.6) defined in the 
input file and (2) through use of the Pardiso linear
equation solver (thread-based, with sparse direct and iterative
options). 
The WARP3D (Fortran) source code includes directives to create and 
manage threads during execution using the industry standard 
OpenMP framework. To illustrate this process, consider this 
simplified code fragment to drive the process of element 
stiffness computation using the block framework:

\small
\begin{verbatim}
    do in parallel block = 1, all_blocks_in_model
       call compute_elem_stiff_for_a_block( block, …..)
    end parallel do 
\end{verbatim}
\normalsize

Suppose a model has 100 blocks of elements defined in the WARP3D input file and the user 
has specified that WARP3D can use 5 threads during execution. Usually 
the number of threads allowed for use is less than or equal to the number of threads
available on the computer. In the simplest 
case of execution for the above code, thread 1 processes element block 1, 6, 11, 16, etc. 
Thread 2 processes blocks 2, 7, 12, 17, etc. The 5 threads run concurrently and 
can thus reduce significantly the elapsed time to compute all the element stiffness matrices. 
Inefficiencies arise when the amount of work for the blocks varies (some threads then 
run longer than others) and the code must wait for all threads to complete their work.
For example, before starting the stiffness assembly process which also runs in parallel,
stiffnesses must be completed for all blocks of elements.

The code implements the thread-based, parallel processing of element blocks illustrated 
above for the computation of: (1) initial linear stiffness and mass, (2) tangent 
stiffnesses, (3) updated strains and stresses, (4) element (internal) nodal forces 
for the updated stresses, and (5)  computation of domain integral quantities.
This high-level parallel execution achieved through 
element blocking achieves near linear scaling with increasing 
model sizes and number of threads.

A final, low-level of parallel execution is also achieved through the element 
blocking. The inner-most loops in all element computations for a block run 
from 1 through the number of elements in the block. The next higher loop runs over
integration points for elements in the block. Most serial codes, 
in contrast, have inner loops over the number of integration points of a single element. 
The element blocking structure, with much longer inner-loop 
lengths (64, 128, 256 for example), exposes significant opportunities 
for compilers to optimize local cache memory, registers and pipelining 
to achieve essentially vector processing efficiencies at the inner-loop level.

\subsection {Hybrid Systems (MPI + Threads)}
\nin Computers designed with the shared-memory architecture begin to 
lose efficiency and the hardware becomes exceedingly complex as the 
number of processor-chips and cores increases. The maintenance of fast 
access to the shared, main memory and coherency of local cache 
memories for the large number of cores requires increasingly complex 
(expensive) hardware. An alternative design uses a cluster of separate 
computers connected together most often with dedicated, very high-speed 
networking hardware. Each of the separate computers has the architecture 
of a shared-memory computer with its own memory system and disks. 
The design of a large, comprehensive program to execute in parallel 
in this type of environment with separate computers communicating 
over a high-speed network becomes very challenging. 

The approach now adopted as an industry standard for these type of computer 
systems relies on explicitly-coded message passing over the network. Suppose 
the cluster has 
8 separate computers, each with 8 cores running a single
thread. A copy of WARP3D is started on each core of each computer, 
with the first copy designated as the \ti{manager} process. The 
other 63 \ti{worker} copies of WARP3D start execution and then wait for 
a message from the manager process with a request for something to 
be computed for the model; \eg element stiffness matrices for a portion 
of the mesh. The execution continues with the manager and worker copies
of WARP3D sharing and exchanging data about the solution. The coding 
for this approach becomes far more complex than the shared-memory 
approach and requires design of the software from the beginning with 
parallel execution via message passing. The industry standard 
software to provide the message passing services is 
called MPI (Message Passing Interface). MPI provides a 
host of services to share data, to synchronize the 
independently executing copies of WARP3D, and to create and release additional
copies of WARP3D as needed during solution.

All advantages of the shared-memory, parallel execution are used by the 
manager and worker copies of WARP3D. This provides an efficient, 
second-level of parallel execution during an analysis, since each of 
the computers in a cluster is most often a multi-processor, shared-memory 
computer with each processor having multiple cores.

To support execution using the MPI approach, 
WARP3D employs a higher-level of 
model decomposition (also termed partitioning)
than is provided by the element blocking alone.
Blocks of elements are aggregated into domains, and there are as 
many domains as the number of MPI ranks set by the
user at the onset of execution. 
In this way, each worker rank owns a domain consisting of 
element blocks (a block of elements resides entirely in a single domain). 
When a worker receives an MPI message from the manager 
to compute element stiffness matrices, it does so for 
all the blocks it owns using the thread-based parallel concept 
illustrated in the previous code fragment. This two-level 
parallel execution for a worker appears as

\small
\begin{verbatim}
    do in parallel block = 1, all_blocks_in_model 
       domain_for_block = domain_list(block)
       if( domain_for_block .ne. my_mpi_rank ) cycle
       call compute_elem_stiff_for_a_block( block, ...)
    end parallel do 
\end{verbatim}
\normalsize
\nin where this same code fragment is executed in thread parallel 
over the blocks on the manager and on all workers concurrently.

The \ti{hypre}  iterative
equation solver is invoked to solve the linear equations in each Newton
iteration. Hypre uses the same MPI ranks during the equation solving
process (and threads on each rank).

The process of model decomposition into domains represents 
a topology optimization problem. A good decomposition into $n$ domains 
minimizes the number of shared nodes between domains to reduce the 
amount data exchanged during solutions. The \ti{patwarp} support program 
described in Appendix C provides the services to decompose a large model 
into domains and each domain into blocks. This requires that the 
analyst know early on the number of MPI ranks (= number of domains) 
that will be used for the analysis. 

\subsection{Summary \& Recommendations}
The WARP3D architecture provides multiple approaches to achieve
parallel execution during an analysis.

We expect the majority of WARP3D users will employ the threads-only
style of parallel execution on Windows, Linux and OS X systems. The
linear solver choices are then Pardiso (direct and iterative). The model
has blocks of elements but no partitioning into domains. The assignment of
elements to blocks is readily performed (automatically) by WARP3D (see Section 2.6).
Analyses of models approaching $10^6$ nodes are feasible with this
approach on current hardware (mid-2013).

For very large models most often analyzed on larger servers and clusters,
the preferred approach uses the hypre 
solver in a domain-decomposition + blocking setup of the model.
The hybrid (MPI + threads) version of WARP3D
is available only on Linux systems.
The \ti{patwarp} program may be used to perform the 
domain decomposition and element blocking during preparation of an
input file.

The remaining sections of the is chapter describe needed details
to utilize each of these approaches.

\end{document}
