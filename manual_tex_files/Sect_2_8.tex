 %!TEX TS-program = pdflatexmk
 %
\documentclass[11pt]{report}
\usepackage{geometry} 
\geometry{letterpaper}

%---------------------------------------------
\setlength{\textheight}{630pt}
\setlength{\textwidth}{450pt}
\setlength{\oddsidemargin}{14pt}
\setlength{\parskip}{1ex plus 0.5ex minus 0.2ex}


%----------------------------------------
\usepackage{amsmath}
\usepackage{layout}
\usepackage{color}
\usepackage{hyphenat}

%----------------------------------------------
\usepackage{fancyhdr} \pagestyle{fancy}
\setlength\headheight{15pt}
\lhead{User's Guide - \textit{WARP3D}}
\rhead{\textit{Loads}}
\fancyfoot[L] {\textit{Chapter {\thechapter}\ \   (Updated: 2-6-2015)        }}
\fancyfoot[C] {\thesection-\thepage}
\fancyfoot[R] {\textit{Model Definition}}

%---------------------------------------------------
\usepackage{graphicx}
\usepackage[labelformat=empty]{caption}
\numberwithin{equation}{section}

%---------------------------------------------
%     --- make section headers in helvetica ---
%
\frenchspacing
\usepackage{sectsty} 
\usepackage{xspace}
\allsectionsfont{\sffamily} 
\sectionfont{\large}
\usepackage[small,compact]{titlesec} % reduce white space around sections
%
%----------------------------------------------

%---------  local commands ---------------------

\newcommand{\nin} {\noindent}
\newcommand{\noi}{\noindent}
\newcommand{\HRule}{\rule{\linewidth}{0.5mm}}
\newcommand{\bmf } {\boldsymbol }
\newcommand{\bsf } [1]{\textrm{\textit{#1}}\xspace}
\newcommand{\ul} {\underline}
\newcommand{\hv} {\mathsf}   %helvetica text inside an equation
\newcommand{\eg}{\emph{e.g.},\xspace}
\newcommand{\ie}{\emph{i.e.},\xspace}
\newcommand{\vs}{\emph{vs.}\xspace}
\newcommand{\ti}{\emph}
\newcommand{\veps}{\varepsilon}
\newcommand{\ol}{\overline}
\newcommand{\mdash}{\mbox{-}}
\newcommand{\clrr}{\color{red}}
\newcommand{\clrb}{\color{black}}
\newcommand{\Fig}{Fig.\xspace}
\newcommand{\Figs}{Figs.\xspace}




%
%
%        optional definition for bullet lists which
%        reduces white space.
%
\newcommand{\squishlist}{
 \begin{list}{$\bullet$}
  { \setlength{\itemsep}{0pt}
     \setlength{\parsep}{3pt}
     \setlength{\topsep}{3pt}
     \setlength{\partopsep}{0pt}
     \setlength{\leftmargin}{1.5em}
     \setlength{\labelwidth}{1em}
     \setlength{\labelsep}{0.5em} } }

\newcommand{\squishlisttwo}{
 \begin{list}{$\bullet$}
  { \setlength{\itemsep}{0pt}
     \setlength{\parsep}{0pt}
    \setlength{\topsep}{0pt}
    \setlength{\partopsep}{0pt}
    \setlength{\leftmargin}{2em}
    \setlength{\labelwidth}{1.5em}
    \setlength{\labelsep}{0.5em} } }

\newcommand{\squishend}{
  \end{list}  }
%

%-------------------------------------
\newcounter{sectrefs}
\setcounter{sectrefs}{0}
\setcounter{chapter}{2}
\setcounter{section}{7}
\setcounter{figure}{0}
\renewcommand{\thefigure}{\thesection.\arabic{figure}}

%--------------------------------------
%--------------------------------------
%---------------------------------------

\begin{document}

\section{Loads (Including Temperatures \& Non-Zero Displacements)}

Loads and temperature changes may be applied to the nodes and elements of a model. 
Element loads, which are dependent on the type of finite element, and changes of nodal 
temperatures are converted to equivalent nodal forces by element processing routines. 

Nodal loads and element loads are grouped together to define loading \ti{patterns}. 
The loading patterns define the spatial variation and reference amplitudes of 
loads on a model. The displacement constraints defined on the model also represent 
a loading pattern but with a built-in
name, \ie \ti{constraints}. Examples of loading patterns include self-weight, an internal/external 
pressure and a localized temperature increase/decrease. 

Once loading patterns are specified, a \ti{nonlinear} loading condition is defined
[the term
\ti{dynamic} may be used as a synonym for \ti{nonlinear} if desired for clarity
in some analyses]. A nonlinear/dynamic loading
consists of a sequential number of load steps (can be many thousands of
steps as needed). An incremental-iterative solution procedure is performed
to obtain the displacements, velocities accelerations, strains, stresses, $\dots$ at
step $n+1$ starting with the the solution available at step $n$ and the user-specified,
\ti{incremental} (step) loading imposed on the model over $n \rightarrow n+1$.
For dynamic analyses, a load step ($n \rightarrow n+1$) is the same as a 
time step $t_n \rightarrow t_{n+1}$. 

The incremental load applied to the model over a load step
is formed by combining the equivalent nodal forces/displacements
from the loading patterns with pattern scaled by an
appropriate factor. For example,
suppose the pattern loading for the internal pressure has a magnitude of the
final, total pressure to be considered in the analysis.
To obtain a converged, nonlinear solution, the total pressure may need to be
increased gradually over some number of load steps -- say 100.  The
user then specifies 100 load steps with the reference (pattern)
loading for the pressure specified to have a 0.01 scale factor in 
each step. Alternatively, the pattern loadings are often defined to represent
a unit value of some loading on the model, \eg a 1 MPa internal
pressure over certain surfaces. This can simplify the specification of scale factors
in the definition of loading steps.

WARP3D computes the incremental solution for each load step 
assuming the model response is dynamic and nonlinear. 
A linear dynamic analysis has no material nonlinearities, geometric 
nonlinearities or contact present in the model
definition (computed residual forces at nodes following the first
Newton iteration of the step are all zero). A static linear or nonlinear analysis is accomplished as a 
dynamic analysis by specifying: (1) a very large time increment or (2) zero mass 
for the model. The user selects one
of the two procedures by setting the time increment or the model mass. A static, linear
analysis would most often be performed using just 1 load step.

The first sections below describe the commands to define nodal forces and element loads that
construct a loading pattern. Commands are then defined to specify load steps in a 
nonlinear/dynamic analysis (or step 1 of a static, linear analysis).


%--------------------------------------
\subsection{Loading Patterns}
%--------------------------------------
\noi  A new loading pattern is defined through a command of the form
loading

\begin{align*}
& \hv{\ul{loading}}\ \ \hv{<loading\ identifier:label>}\\
\end{align*}
\noi where the loading identifier is used in subsequent commands to identify the loading, for
example, in compute and output requests. Only the first eight characters of the identifier
are processed; all loading patterns must have unique identifiers.
When an existing loading pattern is referenced in this command, the input translators
delete all previously defined node and element loadings for the pattern.
Specified temperature values for nodes and elements in the pattern loading represent
incremental changes relative to the initial temperature distribution specified for time zero
(start of load step 1) using the initial conditions command (see Section 2.9).


%--------------------------------------
\subsection{Nodal Loads}
%--------------------------------------
\noi  A sequence of nodal load definitions has the form
\begin{align*}
& \hv{\ul{nodal}\ (\ul{loads})} \\
&\hv{<node\ list:list>}\ \left [
\begin{Bmatrix}
\hv{  \ul{force\_x}} \\ \hv{\ul{force\_y}}  \\ \hv{\ul{force\_z}} \\ \hv{\ul{temper}ature}
\end{Bmatrix}
\hv{(=)<value:numr>}(,) \right ]
\end{align*}

\noi Nodal loads are additive; if the same node and direction appear 
in two different loading commands within the pattern, the sum 
of the two loads is applied to the model. An example
sequence to define a loading condition and a set of nodal forces is

\small
\begin{verbatim}
   loading unit_pull
      nodal loads
           1-40 60-90 force_z -2.3 force_x 14 temperature -42.3
           3240 3671 4510-5000 force_z -3.12
           35 temperature 145.0 force_x 2
 \end{verbatim}
\normalsize

\noi In the above example, node 35 has a total force in the $X$-direction of
16 (14 from the first line 2 from the last line), in addition to a net temperarture
change of 102.7.
Nodal forces are always applied in the global coordinate system and are thus unaffected
by the deformed geometry.


\noi {\bf{\ti{User-Defined Nodal Loads Routine}}}

\noi The nodal forces and temperatures for a pattern loading may be defined through
a user-written function included in the WARP3D program. Appendix I provides
details of developing such a routine and how to include it in WARP3D.

To request that a user-routine be invoked to define nodal loads
rather than the type of input shown above, specify the command 
sequence 
\begin{align*}
& \hv{\ul{nodal}\ (\ul{loads})} \\
&\ \ \ \hv{user\_routine\ (file <file\ name:label or s<string>)}
\end{align*}

\noi The user-routine is called during the solution process for every load step
whenever the pattern multiplier is non-zero. This enables the user-routine to
modify the nodal forces and temperatures in the pattern loading as needed
for potentially complex modeling conditions. A file name may also be supplied as
indicated above. This file name is passed by 
WARP3D to the \ti{user\_nodal\_loads} subroutine. The file presumably will contain information
needed/used by the user subroutine to compute the nodal forces and/or temperatures.

%--------------------------------------
\subsection{Element Loads}
%--------------------------------------
\noi  A sequence of element load definitions has the form 
\begin{align*}
&\hv{\ul{element}\ (\ul{loads})}\\
&\ \ \hv{<elements:list>\ <type\ of\ element\ loading>}\\
&\ \ \hv{<elements:list>\ <type\ of\ element\ loading>}\\
&\ \ \ \ \ . \\
&\ \ \ \ \ . 
\end{align*}
\noi where the $<$type of element loading$>$ is either a body force, a face traction with constant
direction, a face pressure, or a uniform temperature change for the entire element. The
types of element loads and commands to define them are dependent on the type of element.
Refer to Chapter 3 for this information.

When the analysis includes geometric nonlinear effects (large displacements), 
equivalent loads for the incrementally applied surface tractions are re?computed at the beginning
of each load step using the current (deformed) geometry of the elements (see discussion in
Section 1.6).

%--------------------------------------
\subsection{Step Loads}
%--------------------------------------
\noi  The loading type designated dynamic or (equivalently) 
nonlinear defines the combinations
of pattern loads for each time step in a dynamic 
analysis or each load step in a static non?
linear analysis. These commands have the form
\begin{align*}
& \hv{\ul{loading} <loading\ identifier:label>} \\
&\ \ \hv{\ul{nonlinear}}\\
&\ \ \ \ \hv{\ul{step}s <steps:list> [ \ <pattern\ id:label>\ <multiplier:numr>(,)\ ]}
\end{align*}
\noi where the keyword dynamic may be substituted as a synonym for nonlinear. 
Nodal and element loads cannot be specified within a nonlinear/dynamic loading 
definition above. 
The multiplier value must follow each pattern id -- a multiplier value is required input. 
As indicated, multiple pattern loads may be combined with different multipliers to define a load
increment for a time step in a dynamic analysis or a load step in a static nonlinear analysis.

Only one loading condition of type nonlinear or dynamic should be 
defined for an analysis.

By default the existing constraint definitions are included in each load step 
with a multiplier of 1.0. The constraints used in solution for a load step 
are the constraints defined at the actual solution time for the step 
(users can redefine the constraints data at any time -- 
the 1.0 multiplier applies to the currently defined constraints at step solution time). The
user may include constraints as a loading pattern with a multiplier other than 1.0.

An example of this command sequence is
\small
\begin{verbatim}
   loading crush
      nonlinear
          steps 1-10 unit_pressure 2.3 unit_tens -1.2 constraints 1.0
          steps 11-200 pull 0.2 constraints 2.3
               . 
               .
 \end{verbatim}
\normalsize
\noi where the loading patterns \ti{unit\_pressure}, 
\ti{unit\_tens} and \ti{pull} have been defined previously.
Although the steps are defined in ascending sequence in the above example, the steps may
be defined in any order; the final set of steps must comprise a sequential list.

\noi {\bf{\ti{Modifying Step Definitions}}}

\noi During the course of a nonlinear or dynamic analysis, it is often necessary to define 
additional steps or to modify the definition of steps yet to be analyzed. For example, previously
defined, but unsolved, load steps may need to have a reduced multiplier based on current
convergence properties.

Unsolved steps within a nonlinear/dynamic loading condition may be redefined and
additional (new) steps defined at anytime by simply entering the above command sequence
with appropriate load step definitions. Only the steps requiring a change need to be entered.

%--------------------------------------
\subsection{Displacement Control Loading}
%--------------------------------------
\noi  A nonlinear/dynamic loading condition with appropriate step definitions must be always
be specified for a model. If the model is loaded only by non-zero imposed
displacements, there are no loading patterns defined. The keyword \ti{constraints} is
listed in the definition of load steps as in this example which gradually reduces the
magnitude of displacements imposed in each step as the analysis progresses.
\small
\begin{verbatim}
  loading crush
     nonlinear
       steps 1-10 constraints 2.0
       steps 11-50 constraints 1.5
       steps 51-100 constraints 0.5
       steps 101-200 constraints 0.25
           .
           .
 \end{verbatim}
\normalsize

\noi Unless explicitly included as in the above example, \ti{constraints}
included by default with a 1.0 multiplier. Note also that the \ti{constraints}
information may be re-defined between load steps (see Section 2.7).


           
%\noi {\bf{\ti{Effects of Step Multipliers}}}
%
%\noi The pattern multiplier (1.0 above) plays no role in the solution of displacement control 
%loadings unless the extrapolate option of the nonlinear solution algorithm is invoked 
%(extrapolate is on by default). When the extrapolate option is in effect, the incremental 
%displacements computed from the solution over step n?1 to n are scaled and applied to the model
%to start the iterative (Newton) solution from n to n+1. The displacement scaling factor is
%computed from the specified step multipliers for steps n (say fn) and n+1 (fn+1) as fn+1fn.
%Thus only the ratios of the multipliers are significant for displacement control with 
%extrapolate on. When the non?zero constraints are modified during a displacement control analysis,
%the loading step multipliers must be modified accordingly by the user; otherwise the 
%extrapolation ratio (fn+1fn) is computed incorrectly.
%
%To illustrate, consider the following example. Non?zero constraints are specified to load
%the model. The dummy loading pattern and nonlinear loading are defined as above with
%step multipliers of 1.0 for load steps 1-10. After step 10, the user modifies the constraints
%to reduced the imposed increment (uniformly) by one?half, possibly to reduce the number
%of Newton iterations for convergence in subsequent steps. Load steps 11, 12, 13, ... must
%have a multiplier of 0.5 for correct extrapolation. In step 11, the extrapolation multiplier
%is $0.5/1.0=0.5$ while in steps 12, 13, ... the multiplier again becomes 1.0.


% ----------------- end of real text ----------------------------------
\end{document}

 

