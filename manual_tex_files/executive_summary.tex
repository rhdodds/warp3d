%
\documentclass[10pt]{report}
\usepackage{geometry} 
\geometry{letterpaper}
%
%
%   --- margins and inter-paragraph spacing ---
%
%---------------------------------------------
\setlength{\textheight}{630pt}
\setlength{\textwidth}{450pt}
\setlength{\oddsidemargin}{14pt}
\setlength{\parskip}{1ex plus 0.5ex minus 0.2ex}
%
%----------------------------------------
\usepackage{amsmath}
\usepackage{layout}
\usepackage{xspace}
\usepackage{needspace}
%---------------------------------------------
%
%----------------------------------------------
%
%          --- header and footer contents ---
%
\usepackage[us,12hr]{datetime}
\usepackage{fancyhdr} \pagestyle{fancy}
%\setlength\headheight{15pt}
%\lhead{User's Guide - \ti{WARP3D}}
%\rhead{\ti{Revision History}}
\fancyfoot[L] {\small{\ti{Executive Summary}}}
\fancyfoot[C] {\small{\thepage}}
%\fancyfoot[R] {\small{\ti{Updated: 12-17-2015}}}
\fancyfoot[R] {  \small{\ti{Updated: \today\ at \currenttime}}}
\renewcommand{\headrulewidth}{0.0 pt}


%   
\frenchspacing
%
%   ---  local commands ---
%
\newcommand{\bmf } {\boldsymbol }
\newcommand{\bsf } [1]{\textrm{\ti{#1}}\xspace}
\newcommand{\HRule}{\rule{\linewidth}{0.5mm}}
\newcommand{\patwarp}{\ti{patwarp\xspace}}
\newcommand{\eg}{\ti{e.g.},\xspace}
\newcommand{\ie}{\ti{i.e.},\xspace}
\newcommand{\ul} {\underline}
\newcommand{\hv} {\mathsf}   %helvetica text inside an equation
\newcommand{\ti}{\emph}
%
%        optional definition for bullet lists which
%        reduces white space.
%
\newcommand{\squishlist}{
 \begin{list}{$\bullet$}
  { \setlength{\itemsep}{0pt}
     \setlength{\parsep}{3pt}
     \setlength{\topsep}{3pt}
     \setlength{\partopsep}{0pt}
     \setlength{\leftmargin}{1.5em}
     \setlength{\labelwidth}{1em}
     \setlength{\labelsep}{0.5em} } }

\newcommand{\squishlisttwo}{
 \begin{list}{$\bullet$}
  { \setlength{\itemsep}{0pt}
     \setlength{\parsep}{0pt}
    \setlength{\topsep}{0pt}
    \setlength{\partopsep}{0pt}
    \setlength{\leftmargin}{2em}
    \setlength{\labelwidth}{1.5em}
    \setlength{\labelsep}{0.5em} } }

\newcommand{\squishend}{
  \end{list}  }
%
%
%   ---  page numbering ---
%
\pagenumbering{roman}
\setcounter{page}{1}
%
%
%
%              start document 
%              ==========
%
%
\begin{document}
\noindent
\begin{center}
\large
 \textbf{
{\fontfamily{phv}\selectfont \ul{Executive Summary}}}\\[0.2in]
\Huge
\textbf{
{\fontfamily{phv}\selectfont WARP3D}}\\[0.2in]

\normalsize
\textit{
{An Open-Source Research Code for}}\\[0.2in]
\large
\textbf{
{\fontfamily{phv}\selectfont 3-D Nonlinear Finite Element Analysis of Solids\\
for Fracture and Fatigue Processes}}
\end{center}
\normalsize
%
%
WARP3D is under continuing development as a ready-to-run research code for the solution of large-scale, 
3-D solid models subjected to static and dynamic loads. The code includes
specific features oriented toward the investigation of fatigue and ductile fracture in metals.
\begin{center}
\line(1,0){250}
\end{center}


\noindent \textbf{Mechanics: Elements, Constitutive Models, Algorithms}
\squishlist
\item
a library of isoparametric hex, tet and interface-cohesive elements
\item
user defined multi-point constraints; absolute constraints in global and non-global coordinates
\item
 tied-contact capability to connect topologically dissimilar but geometrically congruent meshes using automatically constructed multi-point constraints
\item
a robust finite strain formulation for solid elements and interface-cohesive elements using rotation neutralized rates based on
polar decomposition of \textbf{F}
\item
a general $J$-integral computation facility with inertia, crack face loading, thermal loading, functionally graded and anisotropic materials
\item
a general interaction integral procedure to compute Mode I, II and III stress intensity factors and $T$-stress for cracks in homogeneous and non-homogeneous materials (\ie functionally graded materials, FGMs) 
\item3-D element extinction and node release facilities to model discrete crack growth
\item
general node release with relaxation of reaction forces -- separate from automated crack growth procedures
\item
linear and nonlinear modeling of functionally graded materials -- material properties defined at nodes of the model rather than conventional element-by-element
\item
nonlinear material models including viscoplastic and temperature effects; the 
Gurson-Tvergaard plasticity model for void growth (with rate and temperature effects), 
finite-strain plasticity including the effects of solute hydrogen on the micro-scale flow properties, an advanced model for cyclic plasticity of metals including nonlinear kinematic-isotropic hardening, more generalized complex cyclic behavior 
(Cottrell-Stokes) and temperature dependent cyclic properties; Norton-Bailey power creep
\item 
ability to integrate existing Abaqus  \ti{UMAT} routines for user-defined material behavior with only cosmetic changes needed. Other Abaqus compatible \ti{user} routines \eg UEXTERNALDB are supported in addition to WARP3D
specific user-routines not having a direct counterpart in Abaqus.
\item
a crystal plasticity material model for rate and temperature dependent 
simulation of microscale plastic flow in metals: fcc (12 slip systems), bcc (12 slip systems), bcc48 (48
slip systems), and a single slip system for teaching. A variety of constitutive models to
define shear strain-rates to  shear stresses are available. Includes options for
simple gradient-based geometric hardening, which incorporates the effect of
necessary dislocations on the hardening properties of the material.

\item
linear and nonlinear cohesive constitutive models for use with interface elements to model spontaneous crack formation and extension in 3D. The Paulino-Park-Roesler option provides a comprehensive treatment of mixed-mode fracture 
\item 
element body forces, face tractions, face pressures, temperatures, piston-theory unsteady face pressures, geometry dependent face pressures, user-defined nodal-loads routine
\item
displacement extrapolation, line search and automatic, adaptive solution strategies to enhance convergence of global Newton iterations, local constitutive updates and to control growth rates of material damage
\item
adaptive load control to facilitate extensive crack growth analyses using interface-cohesive elements, crack-tip opening angles (CTOA), and computational cell approaches using a Gurson constitutive model or stress-modified critical strain criteria.
\item
contact between the deformable finite element model and a library of rigid surfaces (planes, cylinders, spheres) which maybe assigned velocity vectors
\squishend

\needspace{4\baselineskip}
\noindent \textbf{Parallel Execution}
\squishlist
\item
parallel execution using either shared memory and OpenMP [Linux, Windows, OSX]
 or a hybrid mode with explicit message passing (MPI) and 
 OpenMP for very large models on distributed memory systems [Linux only]
\item
Pardiso (threaded) sparse direct and sparse iterative solver from Intel MKL [Linux, Windows, OSX]
\item
\ti{hypre} (iterative) solver from Lawrence Livermore National Laboratory for MPI-based 
solution of very large models. Supported on Linux platforms.
Stiffness assembly process is  performed using a
distributed approach across the MPI ranks. The \ti{hypre}
solver includes an option for the BoomerAMG pre-conditioner
(parallel implementation of algebraic multigrid).
\squishend

\noindent{\textbf{Pre- \& Post- Processing}}
\squishlist
\item
a translator program (\ti{patwarp}) written in Fortran is provided 
that generates a WARP3D input file from a Patran neutral file. For MPI-based
executions, \ti{patwarp} performs a domain decomposition 
of the mesh and includes the domain assignments in the input file;
\item
WARP3D outputs simple \ti{flat} files of node and element results in \ti{text} or
\ti{stream} formats. Displacements, velocities, accelerations, temperatures, strains, stresses may be
output at nodes. Strain, stresses and material state variables may be output for elements.
The files are readily processed by simple Python, C++, C, Fortran
programs and may be imported directly into Excel;
\item
a translator program (\ti{warp3d2exii})  written  in Python
is provided that builds an EXODUS II database from the flat results files and the flat model
description file (suitable for use with ParaView);
\item
a generalized \ti{packet} output facility in binary format to support rapid development 
of customized post-processing operations
\item WAR3D can write a Patran neutral file for the model and a simple flat file
describing the model (without the packet structuring of a Patran neutral file)
\item
WARP3D outputs Patran formatted and binary 
result files for displacements, velocities, accelerations, temperatures,
strains, stresses. These may
be imported directly into Patran and other post-processors that
support Patran result files;
\squishend

\noindent{\textbf{Supported Platforms}}

\noindent
Pre-compiled, ready-to-run  executables are included in open-source
distribution. All source code and driver scripts included for 
re-building to incorporate local modifications. Extensive verification problems 
are included in a simple automated testing suite.
\squishlist
\item
Linux: 64-bit. Threads-only version (OpenMP) and hybrid version (MPI + OpenMP)
\item
OS X: Threads-only version (OpenMP). 10.10,x, 1011.x
\item
Windows 64-bit.  Threads-only version (OpenMP). 
\squishend

\noindent{\textbf{License}}

\noindent
University of Illinois/NCSA, Open Source License. Copyright (c) 
2012 University of Illinois at Urbana-Champaign. (see last page of this Executive Summary
for details)

\begin{center}
\line(1,0){250}
\end{center}

\begin{center}
\bf{WARP3D SUMMARY}
\end{center}


WARP3D executes in a parallel on computers with multiple processors and 
with multiple cores per processor. The parallel implementation makes use of 
industry standards: (1) OpenMP for shared memory (threads), (2) Message Passing 
Interface (MPI) to support a multiple level hierarchy of parallel execution on 
distributed hardware (clusters) with local parallel execution using shared memory via OpenMP.

The nonlinear, dynamic equilibrium equations are solved using an 
incremental-iterative, implicit formulation with full Newton iterations, line search
and adaptive-substepping
to eliminate residual nodal forces. Time history integration of the nonlinear 
equations of motion is accomplished with Newmark�s $\beta$-method. 
Analyses with WARP3D thus exhibit the numerical stability for large time 
(load) steps provided by the implicit formulation. All computational 
aspects of the code (element stiffnesses, element strains, stress updating, 
element internal forces, contact, fracture parameter computation) are 
implemented in an element-by-element, blocked data structure and
algorithmic architecture. Such blocking  greatly improves parallel 
efficiency of these element level 
computations, with a thread assigned to perform all computations for a 
block of elements completely independent of other threads assigned to 
other blocks. The larger-size, inner-most loops made possible by the 
element blocking structure also create many opportunities for compilers 
to optimize local cache memory and register use and to employ 
pipelining (vector) hardware features
within each thread. For the very largest models, domain decomposition 
of the mesh provides yet another level of parallel
execution, with each domain partitioned into element blocks. WARP3D 
thus provides three 
levels of parallel execution from very coarse grain to very fine grain.

WARP3D executes in batch and interactive modes. Traditional batch mode 
execution is most useful for large analyses on supercomputers that 
enforce job queuing policies. On Linux, Mac and Windows workstations, the code 
is often executed in background (\&) mode for long jobs and then 
interactively during an analysis restart to obtain selected output. 
Options exist to write information files describing the solution 
status at completion of each Newton iteration during long 
analyses executed in batch mode.

WARP3D takes input data from a variety of sources under control of the user. 
A Patran-to-WARP3D translator program (\ti{patwarp}) is included to convert 
a Patran neutral file for the model into a WARP3D input file. Input commands 
to define the model, loading history, solution parameters, compute and output 
requests have a format-free, English-like structure. Input files may include 
extensive user comments and thus are generally self-documenting. Output 
consists of traditional printed displacements, strains, stresses, etc.; nodal 
and element results files in standard Patran format (binary or ascii) written 
directly by WARP3D; simpler flat files of node and element
results written in text or stream formats,
and a binary �packets� file of selected results to 
facilitate post-processing. A convenient restart capability provides the 
facility to segment a long job over multiple runs and to create analysis 
recovery files in the event of hardware failures or should the solution not converge. 

\clearpage
\begin{center}
\line(1,0){250}
\end{center}

\noindent \textbf{Details of University of Illinois/NCSA Open-Source License}

\small
\noindent Permission is hereby granted, free of charge, to any person obtaining a copy of
this software and associated documentation files (the "Software"), to deal with
the Software without restriction, including without limitation the rights to
use, copy, modify, merge, publish, distribute, sublicense, and/or sell copies
of the Software, and to permit persons to whom the Software is furnished to do
so, subject to the following conditions:
\squishlist
\item
Redistributions of source code must retain the above copyright notice,
this list of conditions and the following disclaimers.
\item
Redistributions in binary form must reproduce the above copyright notice,
this list of conditions and the following disclaimers in the
documentation and/or other materials provided with the distribution.
\item
Neither the names of the WARP3D Team, University of Illinois at
Urbana-Champaign, nor the names of its contributors may be used to
endorse or promote products derived from this Software without specific
prior written permission. 
\squishend
THE SOFTWARE IS PROVIDED "AS IS", WITHOUT WARRANTY OF ANY KIND, EXPRESS OR
IMPLIED, INCLUDING BUT NOT LIMITED TO THE WARRANTIES OF MERCHANTABILITY, FITNESS
FOR A PARTICULAR PURPOSE AND NON-INFRINGEMENT. IN NO EVENT SHALL THE
CONTRIBUTORS OR COPYRIGHT HOLDERS BE LIABLE FOR ANY CLAIM, DAMAGES OR OTHER
LIABILITY, WHETHER IN AN ACTION OF CONTRACT, TORT OR OTHERWISE, ARISING FROM,
OUT OF OR IN CONNECTION WITH THE SOFTWARE OR THE USE OR OTHER DEALINGS WITH THE
SOFTWARE. 

\noindent \textbf{Copyrights and Licenses for Third Party Software Distributed with WARP3D}

\noindent The WARP3D software contains code written by third parties.  Such software will
have its own individual license file in the directory in which it appears.
This file will describe the copyrights, license, and restrictions which apply
to that code.

The disclaimer of warranty in the University of Illinois Open Source License
applies to all code in the WARP3D Distribution, and nothing in any of the
other licenses gives permission to use the names of the WARP3D Team or the
University of Illinois to endorse or promote products derived from this
Software.

\noindent The following pieces of software have additional or alternate copyrights,
licenses, and/or restrictions:
\begin {verbatim}
Program                 WARP3D Directory
-------                 ----------------
hypre                   linux_packages/source/hypre-2.7.0b
metis                   linux_packages/source/metis-4.0
Intel MKL libraries     linux_packages/lib
\end{verbatim}

\normalsize
\begin{center}
\line(1,0){250}
\end{center}


\end{document}


