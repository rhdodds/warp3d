
\documentclass[11pt]{report}
\usepackage{geometry}
\geometry{letterpaper}

%-----------------------------------

%---------------------------------------------
\setlength{\textheight}{630pt}
\setlength{\textwidth}{450pt}
\setlength{\oddsidemargin}{14pt}
\setlength{\parskip}{1ex plus 0.5ex minus 0.2ex}


%----------------------------------------
\usepackage{amsmath}
\usepackage{layout}
\usepackage{color}
\usepackage{array}
\usepackage{bm}
\usepackage{graphicx}
\usepackage{longtable}
%
\usepackage[comma,numbers,sort,compress]{natbib} % numbered, sequential refs 
\renewcommand{\bibsection}{\subsection{\bibname}} %  makes bib a numbered subsection 
\renewcommand{\bibname}{References} % use References not Bibliography for section name
%----------------------------------------------

\usepackage[us,12hr]{datetime}
\usepackage{fancyhdr} \pagestyle{fancy}
\setlength\headheight{15pt}
\lhead{\small{User's Guide - \textit{WARP3D}}}
\rhead{\small{\textit{Example Analyses}}}
\fancyfoot[L] {\small{\  Updated:  \today\ at \currenttime}}

%\fancyfoot[L] {\small{\textit{Chapter {\thechapter}}\ \   (Updated: 8-10-2015. 9:00 AM)}}
\fancyfoot[C] {\small{\thesection-\thepage}}
\fancyfoot[R] {\small{\textit{Contact Procedures}}}

%---------------------------------------------------
\usepackage{graphicx}
% \usepackage[labelformat=empty]{caption}
\numberwithin{equation}{section}

%---------------------------------------------
%     --- make section headers in helvetica ---
%
\usepackage{sectsty} 
\usepackage{xspace}
\allsectionsfont{\sffamily} 
\sectionfont{\large}
\usepackage[small,compact]{titlesec} % reduce white space around sections
%---------------------------------------------->
%
%
%   which fonts system for text and equations. with all commented,
%   the default LaTex CM fonts are used
%
%
\frenchspacing
%\usepackage{pxfonts}  % Palatino text 
%\usepackage{mathpazo} % Palatino text
%\usepackage{txfonts}
\usepackage[stable]{footmisc}

%---------  local commands ---------------------

\newcommand{\ttt} {\texttt}  %typewriter text
\newcommand{\tb} {\textbf}
\newcommand{\nf} {\normalfont}
\newcommand{\df} {\dotfill}
\newcommand{\nin} {\noindent}
\newcommand{\bmf } {\boldsymbol }  %bold math symbol
\newcommand{\bsf } [1]{\textrm{\textit{#1}}\xspace}
\newcommand{\ul} {\underline}
\newcommand{\hv} {\mathsf}   %helvetica text inside an equation
\newcommand{\eg}{\emph{e.g.},\xspace}
\newcommand{\ie}{\emph{i.e.},\xspace}
\newcommand{\ti}{\emph}
\newcommand{\bardelta}{\bar \delta}
\newcommand{\barDelta}{\bar \Delta}
\newcommand{\veps}{\varepsilon}
\newcommand{\noi}{\noindent}

\newenvironment{offsetpar}[1]%
{\begin{list}{}%
         {\setlength{\leftmargin}{#1}}%
         \item[]%
}
{\end{list}}

%
%
%        optional definition for bullet lists which
%        reduces white space.
%
\newcommand{\squishlist}{
 \begin{list}{$\bullet$}
  { \setlength{\itemsep}{0pt}
     \setlength{\parsep}{3pt}
     \setlength{\topsep}{3pt}
     \setlength{\partopsep}{0pt}
     \setlength{\leftmargin}{1.5em}
     \setlength{\labelwidth}{1em}
     \setlength{\labelsep}{0.5em} } }

\newcommand{\squishlisttwo}{
 \begin{list}{$\bullet$}
  { \setlength{\itemsep}{0pt}
     \setlength{\parsep}{0pt}
    \setlength{\topsep}{0pt}
    \setlength{\partopsep}{0pt}
    \setlength{\leftmargin}{2em}
    \setlength{\labelwidth}{1.5em}
    \setlength{\labelsep}{0.5em} } }

\newcommand{\squishend}{
  \end{list}  }
%


%-------------------------------------
\newcounter{sectrefs}
\setcounter{sectrefs}{0}
\setcounter{chapter}{6}
\setcounter{section}{4}
\setcounter{figure}{0}
\renewcommand{\thefigure}{\thesection.\arabic{figure}}
%
%--------------------------------------
%
%
%		Editing
\newcommand{\medit}{\color{blue}}
\newcommand{\eedit}{\color{black}}
\definecolor{gold}{rgb}{0.83, 0.69, 0.22}
%
%
%              start document 
%              ==========
%
%
\begin{document}

\section{Example Analyses Using Contact}

To illustrate the contact capabilities in WARP3D, the following three 
examples describe 
representative analyses which use contact surfaces. The examples 
include the rolling of a metal bar, crushing of a pipe, and 
crack closure in a pin-loaded C(T) specimen. The problems are included
in the \ti{verification} subdirectory and are executed as part of the verification
script.
%
\begin{figure}[h]
\begin{center}
\includegraphics[trim=0.1in 5.0in 7.90in 1.5in, clip=true,scale=1.5,angle=0]{fig_6_5_1.pdf} 
\caption{{\small Mesh for rolling of a metal bar. Shaded rectangles indicate
planes of nodes with symmetry constraints. Arrows indicate path of cylinder
during simulation.}
\label{fig:z}}
%
\end{center}
\end{figure}
%

\subsection{Rolling of a Metal Bar}
This problem simulates the crushing of a metal bar using a rigid roller. 
The roller comes into contact with the $2\times 2\times 2$ in.  bar 
at  a distance 2 
in. from the end, moves downward 0.5 in., then moves along the bar 
at a constant height until it passes beyond the other end. Figure 6.5.1 shows 
the mesh and boundary conditions for the problem, as well as the path of 
the roller. The mesh contains 320 8-node brick elements, 525 
nodes, and uses the mises material model, with $E=30000$ ksi, $\nu=0.3$, 
$\sigma_0=60$ ksi, and $n=10$. The solution uses large deformation theory; 
and a total of 200 load steps, with 50 load steps for the initial crushing 
and 150 load steps for the rolling. The penalty stiffness of the rigid 
cylinder is $10^6$.

%
\begin{figure}
\begin{center}
\includegraphics[trim=0.1in 2.2in 4.90in 1.50in, clip=true,scale=1.3,angle=0]{fig_6_5_2.pdf} 
\caption{{\small Deformed shapes from rolling of a metal bar. Deformed
shapes shown at (a) step 50, (b) step 70, (c) step 150,
and (d) step 200. Part (e) shows a portion of the input file for this
simulation.}
\label{fig:z}}
%
\end{center}
\end{figure}
%

To start the contact smoothly and to avoid initial convergence problems, 
the roller moves down $10^{-5}$ in. per step over the first two steps, then 
$10^{-2}$ in. per step until step 50. After 50 steps, the total 
downward displacement of the roller is 0.48002 in. The roller 
stops for a load step, then moves across the bar, starting at a rate 
of $10^{-3}$ in. per step for steps 52 and 53, and at 0.05 in. 
per step for steps 54 to 200. Figure 6.5.2 shows deformation 
plots at several points during the analysis, and portions of the 
analysis input. Three load steps require adaptive step reduction 
to achieve convergence of the Newtons iterations.

\subsection{Crushing of a Pipe}

This example simulates crushing of a pipe located inside a rigid box.  The 
model has one-layer of 8-node solid element in the thickness (out-of-plane)
direction with plane-strain constraints.
The  model uses four contact planes; a horizontal contact plane 
descends from above the pipe, while the side and bottom planes remain 
stationary. As the top plane crushes  the pipe, the initially circular shape transforms 
to a rectangular shape. The outside radius of the pipe is 5 in., and the wall 
is 1 in. thick. The model is constructed with 186 8-node elements and 
496 nodes; it uses a mises material model with $E=30000$ ksi, $\nu=0.3$,
$\sigma_0=60$ ksi, and $n=10$. 

%
\begin{figure}
\begin{center}
\includegraphics[trim=0.1in 3.2in 5.90in 1.50in, clip=true,scale=0.8,angle=0]{fig_6_5_3.pdf} 
\caption{{\small Deformed shapes from crushing of a pipe. Deformed
shapes shown at 500 step increments between 0 and 2000 load steps.}
\label{fig:z}}
%
\end{center}
\end{figure}
%


The top plane translates down $10^{-6}$ in. in the first step, then 
$2.5\times 10^{-3}$ 
in. per step for the remainder of the analysis, providing a 5 in. height 
reduction after 2000 load steps. Fixed constraints on the topmost nodes 
traveling with the moving contact plane help stabilize the model. Figure 6.5.3 
displays a series of deformed shapes at various stages of the analysis. 

\subsection{Crack Closure in a Pin-Loaded C(T) Specimen}

Finite element analyses of fracture specimens typically do not include the 
actual boundary conditions as applied in experiments. For instance, 
experimental procedures dictate that loading of a compact tension, C(T), 
specimen use pins, while finite element analyses of the specimen usually 
load a single line of nodes or fill in the pin hole with linear elastic elements. 
While in general the discrepancies between experimental and modeled 
boundary conditions cause only minor differences in the crack front results, 
there may be cases in which the true boundary conditions can cause significant differences.


%
\begin{figure}
\begin{center}
\includegraphics[trim=0.4in 4.8in 5.90in 1.50in, clip=true,scale=1.2,angle=0]{fig_6_5_4.pdf} 
\caption{{\small Mesh for pin-loaded C(T) simulation.}
\label{fig:z}}
%
\end{center}
\end{figure}
%


To demonstrate the use of contact surfaces provided in WARP3D in simulation 
of fracture specimens, this section describes the large deformation, 
plane-strain analysis of a pin-loaded C(T) specimen, where a contact cylinder 
explicitly models the pin loading. Furthermore, the analysis includes both 
cyclic tensile and compressive loading, causing crack closure. To model crack 
closure, a rectangular contact surface congruent with the symmetry plane 
prevents penetration of the crack face nodes. The model has a width $W$ of 1.9685 
in., and a crack length to width ($a/W$) ratio of 0.6. An initially blunt notch 
with a radius of $7.6\times�10^{-4}$ in. allows significant blunting at the crack tip 
as deformation increases. The pin has a radius of 0.17 in., while the 
radius of the hole in the C(T) specimen is 0.18 in. The model contains 1-layer of
435 8-node elements and 1000 nodes, and uses a mises material model 
with $E=30000$ ksi, $\nu=0.3$, $\sigma_0=60$ ksi, and $n=10$.
Figure 6.5.4 illustrates the mesh, and Figure 6.5.5 shows the model deformation
at several points during the analysis.

%
\begin{figure}
\begin{center}
\includegraphics[trim=0.4in 3.8in 5.50in 1.50in, clip=true,scale=1.5,angle=0]{fig_6_5_5.pdf} 
\caption{{\small Mesh for pin-loaded C(T) simulation.}
\label{fig:z}}
%
\end{center}
\end{figure}
%


To initiate contact, the pin moves only $5.0\times 10^{-8}$ in. upwards during 
the first step. On step 2, the rate increases to $5.0\times�10^{-4}$ in.  per load step. 
This continues until the specimen experiences a total upward displacement 
of 0.0138 in. at step 30, after which the pin moves downwards. The 
downward movement begins at $5.0\times�10^{-8}$ in. for step 31, then increases 
to $1.0\times�10^{-4}$  in. per step. To assist computational stability during 
the initial loading and unloading of the specimen, explicit constraints 
on the nodes at the top of the loading hole ensure model displacement 
commensurate with the pin displacement. This is crucial in the 
unloading section between steps 31 and 61, where an analysis 
which does not include the explicit constraints requires a loading 
rate several orders of magnitude smaller. After step 61, the analysis 
moves into compressive loading, at which point the explicit constraints 
are removed. Once again, to ensure smooth convergence with the initial 
contact, the first step uses a displacement of $5.0\times�10^{-8}$ in., while 
movement of $5.0\times�10^{-4}$ in. per step ensues afterwards. The 
loading continues until step 130, achieving a total downwards displacement 
of 0.032 in.. Crack closure initiates on step 75, and by step 130, the 
crack closes almost completely except for a small region near the crack tip. 

A selected portion of the input file follows:
\small
\begin{verbatim}
       constraints
       *input from file �constraints�
       	33 34      v 5.0e-8
       nonlinear analysis parameters
       	time step 0.0001
       ! 
       contact surface
       	surface 1 cylinder
       		point -1.181 .75322844 -1
       		direction 0 0 1
       		length 2
       		radius .17
       		stiffness 10.0e5
       		rate 0.0 0.0005 0.0
       c
       compute displacements for loading test step 1
       c
       nonlinear analysis parameters
       	time step 1.0
       constraints
       *input from file �constraints�
       	33 34      v .0005
       ! 
        compute displacements for loading test step 2-30
       ! 
       constraints
       *input from file �constraints�
       	33 34      v -5.0e-8
       contact surface
       	surface 1 cylinder
       		point -1.181 .76772849 -1
       		direction 0 0 1
       		length 2
       		radius .17
       		stiffness 10.0e5
       		rate 0.0 -0.0005 0.0
       nonlinear analysis parameters
       	time step 0.0001
       c
        compute displacements for loading test step 31
       c
       constraints
       *input from file �constraints�
       	33 34      v -.0005
       nonlinear analysis parameters
       	time step 1.0
       c
        compute displacements for loading test step 32-60
       ! 
       constraints
       *input from file �constraints�
       contact surface
       	surface 1 cylinder
       		point -1.181 .7245 -1
       		direction 0 0 1
       		length 2
       		radius .17
       		stiffness 10.0e5
       		rate 0.0 -0.0005 0.0
       	surface 2 plane 
       		point -10 0 -10
       		point -10 0  10
       		point  10 0 -10
       		stiffness 10.0e5
       nonlinear analysis parameters
       	time step 0.0001
       c
       compute displacements for loading test step 61
       ! 
       nonlinear analysis parameters
       	time step 1.0
       c
        compute displacements for loading test step 62-130
       \end{verbatim}
\normalsize

\end{document}
