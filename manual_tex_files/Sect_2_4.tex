%!TEX TS-program=pdflatexmk
\documentclass[11pt]{report}
\usepackage{geometry} 
\geometry{letterpaper}

%---------------------------------------------
\setlength{\textheight}{630pt}
\setlength{\textwidth}{450pt}
\setlength{\oddsidemargin}{14pt}
\setlength{\parskip}{1ex plus 0.5ex minus 0.2ex}


%----------------------------------------
\usepackage{amsmath}
\usepackage{layout}
\usepackage{color}
\usepackage{array}

%----------------------------------------------
\usepackage{fancyhdr} \pagestyle{fancy}
\setlength\headheight{15pt}
\lhead{\small{User's Guide - \ti{WARP3D}}}
\rhead{\small{\ti{Nodal Coordinates}}}
\fancyfoot[L] {\small{\ti{Chapter {\thechapter}}\ \   (Updated: 12-15-2011)}}
\fancyfoot[C] {\small{\thesection-\thepage}}
\fancyfoot[R] {\small{\ti{Model Definition}}}

%---------------------------------------------------
\usepackage{graphicx}
\usepackage[labelformat=empty]{caption}
\numberwithin{equation}{section}

%---------------------------------------------
%     --- make section headers in helvetica ---
%
\usepackage{sectsty} 
\usepackage{xspace}
\allsectionsfont{\sffamily} 
\sectionfont{\large}
\usepackage[small,compact]{titlesec} % reduce white space around sections
%---------------------------------------------->
%
%
%   which fonts system for text and equations. with all commented,
%   the default LaTex CM fonts are used
%
%
\frenchspacing
%\usepackage{pxfonts}  % Palatino text 
%\usepackage{mathpazo} % Palatino text
%\usepackage{txfonts}


%---------  local commands ---------------------


\newcommand{\bmf } {\boldsymbol }  %bold math symbol
\newcommand{\bsf } [1]{\textrm{\ti{#1}}\xspace}
\newcommand{\ul} {\underline}
\newcommand{\hv} {\mathsf}   %helvetica text inside an equation
\newcommand{\eg}{\emph{e.g.},\xspace}
\newcommand{\ie}{\emph{i.e.},\xspace}
\newcommand{\vs}{\emph{vs.}\xspace}
\newcommand{\ti}{\emph}
\newcommand{\veps}{\varepsilon}
\newcommand{\ol}{\overline}
\newenvironment{offsetpar}[1]
{\begin{list}{}%
         {\setlength{\leftmargin}{#1}}%
         \item[]%
}
{\end{list}}

%
%
%        optional definition for bullet lists which
%        reduces white space.
%
\newcommand{\squishlist}{
 \begin{list}{$\bullet$}
  { \setlength{\itemsep}{0pt}
     \setlength{\parsep}{3pt}
     \setlength{\topsep}{3pt}
     \setlength{\partopsep}{0pt}
     \setlength{\leftmargin}{1.5em}
     \setlength{\labelwidth}{1em}
     \setlength{\labelsep}{0.5em} } }

\newcommand{\squishlisttwo}{
 \begin{list}{$\bullet$}
  { \setlength{\itemsep}{0pt}
     \setlength{\parsep}{0pt}
    \setlength{\topsep}{0pt}
    \setlength{\partopsep}{0pt}
    \setlength{\leftmargin}{2em}
    \setlength{\labelwidth}{1.5em}
    \setlength{\labelsep}{0.5em} } }

\newcommand{\squishend}{
  \end{list}  }
%


%-------------------------------------
\newcounter{sectrefs}
\setcounter{sectrefs}{0}
\setcounter{figure}{0}
\setcounter{chapter}{2}
\setcounter{section}{3}
\renewcommand{\thefigure}{\thesection.\arabic{figure}}

%
%--------------------------------------
%
%
%
%              start document 
%              ==========
%
%

\begin{document}

%
%------------------------------------------------------------------------------
\section{Nodal Coordinates}
%------------------------------------------------------------------------------
The coordinates of nodes are specified relative to the global 
Cartesian reference axes. During model definition, the command 
\ti{coordinates} initiates the translation of nodal coordinate data.
Any number of \ti{coordinates} commands may be given prior
to a compute request. The existing coordinates for nodes 
are simply overwritten by any newly specified values. 
The input syntax is
\begin{align*}
&\hv{\ul{coord}inates\ (\ul{clear}) } \\
& \qquad \hv{<node\ number:integer>}\ \left [
\begin{Bmatrix}
\hv{x} \\ \hv{y} \\ \hv{z}
\end{Bmatrix}
\hv{<value:number>}(,) \right ]\\ \\
&\qquad \hv{<node\ number:integer>\ \left[ <value:number> (,)\right]}
\end{align*}

\noindent where the second form applies the default ordering of entries 
$X\mbox{-}Y\mbox{-}Z$. 
\ti{When using the second form, coordinates not specified take on the 
last previously defined values.} For example, the sequence

\small \begin{verbatim}
    coordinates 
      4 3.2 5.2 6.4 
      10 4.1
\end{verbatim}\normalsize

\noindent defines the $Y$ coordinate of node 10 as 5.2 and 
the $Z$ coordinate of node 10 as 6.4. 
This feature may be suppressed by appending the word \ti{clear} to the 
coordinates command line. 
The default coordinates for every node are then 0.0 unless explicitly input. 
With this option for the above example, node 10 is assigned 
coordinates of 4.1, 0.0, 0.0 rather than 4.1, 5.2, 6.4.

The default $X\mbox{-}Y\mbox{-}Z$ ordering for the second 
input form may be modified by the default command

\begin{align*}
& \hv{\ul{coord}inates } \\
& \qquad \hv{\ul{default}}\ \left [
\begin{Bmatrix}
\hv{x} \\ \hv{y} \\ \hv{z}
\end{Bmatrix}
\right ] \\
& \qquad \hv{<node\ number:integer>\ \left[ <value:number> (,)\right]}
\end{align*}

\noindent where any number of \ti{default} commands may be given.


Some examples illustrating various options to define nodal coordinates are given below.
\small \begin{verbatim}
    coordinates
      4 x 2.5 y 3.0 z 4.1
     10 z �20 y 40 x 20
     11 �5.23 6.23 
    default z y x
      3 15.3 14.2 10.5 
    default x y z
     10 �13.5 10.5 �20.4
\end{verbatim}\normalsize

At any point during input of the coordinates, the \ti{dump} 
command is available to request a listing of current coordinates for 
all nodes of the model.

\small \begin{verbatim}
    coordinates 
      4 x 2.5 y 3.0 z 4.1 
     10 z �20 y 40 x 20 
     11 �5.23 6.23 
     dump 
     default z y x
        3 15.3 14.2 10.5 
     default x y z
       10 �13.5 10.5 �20.4 
     dump
\end{verbatim}\normalsize



\end{document}


