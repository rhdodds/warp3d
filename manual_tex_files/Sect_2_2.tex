%!TEX TS-program=pdflatexmk
%
\documentclass[11pt]{report}
\usepackage{geometry} 
\geometry{letterpaper}

%---------------------------------------------
\setlength{\textheight}{630pt}
\setlength{\textwidth}{450pt}
\setlength{\oddsidemargin}{14pt}
\setlength{\parskip}{1ex plus 0.5ex minus 0.2ex}


%----------------------------------------
\usepackage{amsmath}
\usepackage{layout}
\usepackage{color}
\usepackage{hyphenat}
\usepackage{array}
\usepackage{textcomp}
\raggedbottom
%----------------------------------------------
\usepackage{fancyhdr} \pagestyle{fancy}
\setlength\headheight{15pt}
\lhead{\small{User's Guide - \textit{WARP3D}}}
\rhead{\small{\textit{Material Definition}}}
\fancyfoot[L] {\textit{\small{Chapter {\thechapter}}\ \   (Updated: 8-20-2011)}}
\fancyfoot[C] {\small{\thesection-\thepage}}
\fancyfoot[R] {{\small{\textit{Model Definition}}}}

%---------------------------------------------------
\usepackage{graphicx}
\setlength{\textfloatsep}{1pt}
\setlength{\intextsep}{1pt}
\usepackage[labelformat=empty]{caption}
\numberwithin{equation}{section}

%---------------------------------------------
%     --- make section headers in helvetica ---
%
\frenchspacing
\usepackage{sectsty} 
\usepackage{xspace}
\allsectionsfont{\sffamily} 
\sectionfont{\large}
\usepackage[small,compact]{titlesec} % reduce white space around sections
%
%----------------------------------------------

%---------  local commands ---------------------
\newcommand{\tb} {\textbf}
\newcommand{\nf} {\normalfont}
\newcommand{\df} {\dotfill}
\newcommand{\nin} {\noindent}
\newcommand{\bmf } {\boldsymbol }  %bold math symbol
\newcommand{\bsf } [1]{\textrm{\textit{#1}}\xspace}
\newcommand{\ul} {\underline}
\newcommand{\hv} {\mathsf}   %helvetica text inside an equation
\newcommand{\eg}{\emph{e.g.},\xspace}
\newcommand{\ie}{\emph{i.e.},\xspace}
\newcommand{\ti}{\emph}
\newcommand{\bardelta}{\bar \delta}
\newcommand{\barDelta}{\bar \Delta}

\newenvironment{offsetpar}[1]%
{\begin{list}{}%
         {\setlength{\leftmargin}{#1}}%
         \item[]%
}
{\end{list}}

%
%
%        optional definition for bullet lists which
%        reduces white space.
%
\newcommand{\squishlist}{
 \begin{list}{$\bullet$}
  { \setlength{\itemsep}{0pt}
     \setlength{\parsep}{3pt}
     \setlength{\topsep}{3pt}
     \setlength{\partopsep}{0pt}
     \setlength{\leftmargin}{1.5em}
     \setlength{\labelwidth}{1em}
     \setlength{\labelsep}{0.5em} } }

\newcommand{\squishlisttwo}{
 \begin{list}{$\bullet$}
  { \setlength{\itemsep}{0pt}
     \setlength{\parsep}{0pt}
    \setlength{\topsep}{0pt}
    \setlength{\partopsep}{0pt}
    \setlength{\leftmargin}{2em}
    \setlength{\labelwidth}{1.5em}
    \setlength{\labelsep}{0.5em} } }

\newcommand{\squishend}{
  \end{list}  }
%

%-------------------------------------
\newcounter{sectrefs}
\setcounter{sectrefs}{0}
\setcounter{chapter}{2}
\setcounter{section}{1}

%--------------------------------------
%--------------------------------------
%---------------------------------------

\begin{document}


\section{Material Definitions}
%\layout

\noindent
Finite elements in a model are associated with ``materials" from which they
derive linear-elastic properties, mass density, thermal expansion and nonlinear
characteristics, if necessary. Through the \ti{material} command, the user specifies
a convenient name for the material, the type of constitutive model (\eg
rate-independent \ti{mises}) and the values of properties required by the
material model. Material definitions must precede the specification of element
properties during input.

Some constitutive models provide an option to specify nonlinear response in the
form of piecewise-linear descriptions, \eg uniaxial stress-strain
curves. The \ti{stress-strain curve} command describes points on the piecewise-linear
curve(s) for use by the material model. These curves also specify any
temperature and strain-rate dependencies of the elastic constants, thermal
expansion coefficient, etc.

To provide an initial capability to model the response of emerging functionally
graded materials, values of some key material properties (\eg Young's modulus,
Poisson's ratio, etc.) may be specified at nodes of the model rather than as
numerical constants associated with a specific material. Computational routines
then interpolate values of these material properties at points within elements
from the nodal values as needed during solutions. This new capability at present
supports modeling the linear and nonlinear response of functionally graded materials
(FGMs) which have
temperature and strain-rate invariant material properties. 

Anisotropic thermal expansion coefficients provide a very simple and convenient
approach to introduce eigenstrains for the purpose of imposing residual stresses
in a model during loading (by increased temperatures). These expansion
coefficients are intended for use with linear-elastic and elastic-plastic
constitutive models with temperature invariant material properties. These
expansion coefficients remain spatially constant over an element and are thus
specified for each element of the model involved in the eigenstrain generation.

This section describes the commands: \ti{material}, \ti{stress-strain curve}, \ti{functionally graded
material properties} and \ti{anisotropic thermal expansion coefficients}.
When a \ti{material} command references a \ti{stress-strain curve(s)}, the referenced
curve(s) must be defined previously in the input data. This requirement enables
additional consistency checks on the specified data.

\subsection{Material Command}
\nin
A \ti{material} command on a separate line initiates the material definition sequence.
 Any number lines may follow to define the properties required for the material
model. The definition of a new material has the command syntax
\begin{align*}
&\hv{\ul{material}\ <material\ id:label>} \\
&\hv{\ \ \ul{properties}\ <model\ type:label>\ [ <matl.\ prop:label>(<value>) (,)]}
\end{align*} 

\nin Input of the properties may be continued over multiple input lines by ending
lines with a comma. A material definition terminates with a line not ending in
a comma. Subsequent sections in Chapter 3 define the ``type" of
material constitutive models currently available and the properties required for
each model type. 

As an example, the following input lines specify a simple material for use in
the analysis: 
\small
\begin{verbatim}
    material al2024t3
       properties mises e 10350 nu 0.3 yld_pt 50.0 n_power 10,
                                rho 0.1254e-07 alpha 5.4e-06
\end{verbatim}
\normalsize

\nin In this example, the analyst specifies the name \ti{al2024t3} for the material.
Material names are unique for up to 24 characters. Note that
material names must satisfy the
definition of a \textless label\textgreater. The constitutive model type for the material
is \ti{mises}. This is one of the models described in Chapter 3. 
Keywords \ti{e, nu, n\_power,} etc.
are properties of the \ti{mises} model assignable by the user.

Later during input, users assign one such previously defined material to each
finite element in the model. An example of this process is 

\small
\begin{verbatim}
         .
   elements
         1-900 q3disop linear material al2024t3 order 2x2x2,
                             center_output short
         .
\end{verbatim}
\normalsize

\nin Here, elements 1-900 are type \ti{q3disop} (the 20-node
isoparametric brick element) with a small-displacement (\ti{linear}) formulation.
The material associated with these elements is \ti{al2024t3} as specified
previously. Remaining information on the input lines sets the element 
properties: integration order, output location, etc.

In the example above illustrating the definition of material \ti{al2024t3} and its
association with elements 1-900 in the model, the following actions result:

\small \squishlist
\item
the material properties \ti{e, nu}, etc. do not vary with temperature or
strain rate
\item
all integration points within elements 1-900 have the same
values for  \ti{e, nu}, etc.
\squishend \normalsize

More complex materials may be defined using additional options as described
below. \ti{Once defined, the specification for a material cannot be modified at any
further point in the analysis.}

\subsubsection{Stress-Strain Curves}
Stress-strain curves provide a mechanism to introduce a more complex variation
of material properties. In the simplest case described in Section 2.2.2,
specified points define a curve to represent the piecewise-linear equivalent
stress \ti{vs.} equivalent strain for use in a strain-rate and temperature
independent plasticity constitutive model. Section 2.2.3 describes more complex
stress-strain curves which include a temperature dependence of the scalar 
material properties and (possibly) the  stress-plastic flow curve. Section 2.2.4 describes
stress-strain curves that introduce a strain-rate dependence of the
stress-plastic flow curve. \ul{Combined temperature and}
\ul{strain-rate dependence is
\ti{not} yet supported within this stress-strain curve framework}.

The following example refers to stress-strain curve 3 for the piecewise-linear
description of the uniaxial, tensile stress-strain curve

\small
\begin{verbatim}
     material a36
           properties mises e 30000 nu 0.3 curve 3 rho 0.1254e-07
\end{verbatim}
\normalsize

The following example illustrates the use of several stress-strain curves to
define the temperature dependent material properties including Young's modulus,
Poisson's ratio, thermal expansion coefficient and nonlinear flow properties
(note the absence of \ti{e}, \ti{nu}, and \ti{alpha} -- the temperature dependent values are
defined with the stress-strain curves)

\small
\begin{verbatim}
     material a533b
          properties mises curves 1-5 rho 0.1254e-07
\end{verbatim}
\normalsize

\subsubsection{Nodal Values of Material Properties for FGMs}
\nin Functionally graded materials have property values that vary spatially
over the domain of the model according to a user-specified form. The
present modeling capability focuses on isotropic FGMs with properties
invariant of temperature and strain rate. The analyst specifies values of the
material properties at model nodes using commands described in Section 2.2.5.
Computations requiring these material properties interpolate values at points
within elements from the specified nodal values, using the same interpolation
function employed for temperatures.

The meaning of a material property \textless value\textgreater\ is 
extended to implement this
modeling capability. Rather than specifying a numerical value for a property as
in the above examples, the string `fgm' is given instead. This key triggers the
interpolation process whenever material property values are required.
Suppose for example that Young's modulus, Poisson's ratio, mass
density and (isotropic) thermal expansion coefficient all vary spatially for the
material and are specified at the model nodes. Then the material definition
could have the form

\small
\begin{verbatim}
     material graded_ceramic
          properties bilinear e 'fgm' nu 'fgm' alpha 'fgm',
	                     rho 'fgm' yld_pt 500
\end{verbatim}
\normalsize

\nin where the \ti{bilinear} constitutive model is employed in this example. 
In a simpler case, suppose that only Young's modulus and Poisson's ratio vary
spatially but the thermal expansion coefficient and mass density remain
spatially invariant for elements associated with the material. The above input
commands become 

\small
\begin{verbatim}
     material graded_ceramic
          properties bilinear e `fgm' nu 'fgm' alpha 0.3e-06,
	                     rho 0.000034 yld_pt 500
\end{verbatim}
\normalsize

\nin This FGM modeling capability may be used with the constitutive models
described in Chapter 3 compatible with solid elements.

%
\subsection{Stress-Strain Curve Command: \ti{Temperature Independent}}
%
\nin The uniaxial stress-strain response of certain materials requires a
general curve description for a realistic representation. Materials
that exhibit a sharp yield point, a Luder's band and then strain hardening are
common examples not amenable to modeling with power-law and 
exponential type curves. For
this form of the stress-strain curve definition, the material response remains
invariant of temperature. Figure 2.1 provides an example of a stress-strain
curve described with a piecewise-linear model.
%
\begin{figure}[htb]
\begin{center}
\includegraphics[trim=0in 5.2in 6.0in 0.4in,scale=1.1,angle=0]{Figs_1.pdf} 
\caption{{\small Fig. \thefigure: Example of piecewise-linear stress-strain curve for temperature 
independent response.\normalsize}
\label{fig:fig1}}
%
\end{center}
\end{figure}

Points on such curves are specified with the command sequence \ti{stress-strain
curve} where each such curve required in the analysis is assigned an integer
number for identification. The curve may then be referenced in a \ti{material}
command as described above. The command syntax is
\begin{align*}
&\hv{\ul{stress}(\mbox{-}\ul{strain})\ \ul{curve}\ <curve\ number:integer>} \\
&\hv{\ \ [\ <strain\ value:numr>\ <stress\ value:numr> (,) \ ]}
\end{align*}
\nin The following guidelines apply in defining such curves:
\small
\squishlist
\item Curve points are input as strain-stress pairs; 
\item Use as many lines as needed to specify the points; 
\item Multiple pairs may be specified on a line (up to 20 pairs total for a
curve);
\item All strain-stress values must be positive;
\item Strain values must increase monotonically. 
\item Stress values are not required to increase monotonically and may
actually decrease;
\item Do \ti{not} specify the (0,0) point on the curve. The first point defines the
yield strain and yield stress;
\item Young's modulus specified in the \ti{material} command \ti{must} match the value
implied by the yield strain-yield stress pair;
\item Input \ti{\ul{total}} strains (not the plastic
strains\ !)
\item Perfectly plastic response is employed after the last point.
\squishend
\normalsize
\nin For large-deformation analyses, the values should correspond to the logarithmic
strain--Cauchy stress; for small strain-analyses the values should be engineering
strain--engineering (nominal) stress.

The above curve is described with the command sequence
\small
\begin{verbatim}
       stress-strain curve 3
            0.0012 36,  0.01 36,  0.05 50,
            0.10 55, 0.30 60
\end{verbatim}
\normalsize
%
\subsection{Stress-Strain Curve Command: \ti{Temperature Dependent}}
\nin
The stress-strain curves provide the mechanism to define
materials with properties that vary with temperature. The material flow
properties, Young's modulus, Poisson's ratio, etc. 
may all vary with temperature through the definition of stress-strain
curves. During analyses with temperatures that vary over time (or load), these
curves provide the needed data to establish the current property values for the
material.

The user specifies material properties over a range of temperatures with
separate \ti{stress-strain curves}, where each curve 
defines values for a constant temperature. Up to 20 such curves may be defined.
The \ti{material}
definition then refers to a set of such temperature dependent stress-strain
curves. For example,

\small
\begin{verbatim}
        material a533b
            properties mises curves 1-5 rho 0.1254e-07
\end{verbatim} \normalsize

No requirements are imposed on the range of temperatures or the increments of
temperature between the specified curves. During analyses, the various processors
employ linear interpolation between specified temperatures and between
specified values on curves
to compute needed property data. When material point temperatures lie outside the
minimum-maximum temperatures specified for the stress-strain curves, the
corresponding minimum-maximum curves are used in computations -- no
extrapolation is performed outside user specified range of temperatures. Figure 2.2 shows an
example set of three isothermal stress-strain curves to describe the temperature dependence.
%

\begin{figure}[htb]
\begin{center}
\includegraphics[trim=0in 5.2in 6.0in 0.4in,scale=1.1,angle=0]{Figs_2.pdf} 
\caption{{\small Fig. \thefigure: Example of piecewise-linear stress-strain curve for temperature 
dependent response.\normalsize}
\label{fig:fig2}}
%
\end{center}
\end{figure}

%
Temperature dependent properties and points on temperature dependent 
stress-plastic strain curves are specified with the command sequence \ti{stress-strain
curve} where each such curve  is assigned an integer
number for identification. The curves may then be referenced in a \ti{material}
command as described above. The command syntax is
\begin{align*}
&\hv{\ul{stress}(\mbox{-}\ul{strain})\ \ul{curve}\ <curve\ number:integer>\ 
   \ul{temper}ature <numr>\ [<properties>(,)]} \\
&\hv{\ \ [\ <plastic\ strain\ value:numr>\ <stress\ value:numr> (,) \ ]}
\end{align*}\normalsize
\nin where the \textless properties\textgreater\ refers to scalar
quantities for a material, \eg Young's modulus, Poisson's ratio, etc. at that
temperature. Material models (\ti{bilinear, mises, cyclic,} ...) for solid elements
all support temperature dependent properties for \ti{e, nu, alpha}. Additional 
temperature dependent properties may be specified for some material
models -- refer to the manual section in Chapter 3 for each available model.

Initial temperatures in the model and subsequent changes 
with imposed temperature increments via load steps
all refer to a $T=0$ condition. The temperature specified 
here in the curve definition 
refers to a value above or below the $T=0$ zero condition. Users may
define a non-zero and (optional) spatially varying initial temperature field
for the model through the \ti{initial conditions} command described in Section 2.9.
Such temperatures are relative to the $T=0$ condition.
For example, consider again Fig. 2.2 which shows curves for three
temperatures of +100, +300 and +500 above the $T=0$
condition. Units for temperature must be consistent with units specified for the 
thermal expansion coefficients.
Suppose the user specifies an initial
temperature for the model of +200. Starting values for \ti{e, nu}, ...
then will be the average of user specified values in the curves for the +100 and +300
temperature. 

The ordering of stress-strain curves has no significance -- curve number 1
is not required to define the highest or lowest temperature. The input
translators re-arrange the internal ordering of curves as needed to support the
computational processors.

The following guidelines and requirements apply to define temperature dependent
property values and optionally the stress-plastic strain curves:
\small \squishlist
\item Curve points are input as strain-stress pairs;
\item Use as many lines as needed to specify the points;
\item Multiple pairs may be specified on a line (maximum of 20 pairs total for
the curve)
\item Use a comma to continue input lines of temperature dependent properties and
points on the curve;
\item Input \ul{plastic} strain values (not total strains\ !); 
\item All plastic strain and specified stress values must be positive;
\item The first point defines the temperature dependent yield stress. The
plastic strain for the first point on the curve \ti{must} be 0.0;
\item Plastic strain values must increase monotonically;
\item Stress values are not required to increase monotonically and may
actually decrease. 
\item As illustrated in Fig. 2.2, the specified plastic strain values must be identical
for all the curves that make up a set of curves to define the temperature
dependence;
\item The material response at plastic strain values larger than the last
point specified on the curve follows a perfectly plastic response.
\squishend \normalsize
For large-deformation analyses, the values should correspond to the logarithmic
strain-Cauchy stress; for small strain-analyses the values should be engineering
strain-engineering stress. 

Any values of material properties, \eg \ti{e, nu}, ..., specified during the \ti{material} definition are \ti{overwritten} during analysis with
the above values interpolated at the current temperature of the material point.

The following is a complete example which uses three curves to define the
temperature dependent response (see Fig. 2.2)

\small
\begin{verbatim}
   c   
     stress-strain curve 1 temperature 100 e 30000 nu 0.3 alpha 0.0001
        0.0   80,   $ these are plastic strain vs. stress values
        0.025 96,   $ the first point must have zero plastic strain...
        0.050 103,   0.10  108,  0.20  114, 0.30  118,
        0.40  122, 100   1000
   c
     stress-strain curve 2 temperature 300 e 28000 nu 0.28 alpha 0.0002
       0.0   70,   0.025 85,  0.050 91,  0.10  96,  0.20  102,  0.30  105,
       0.40  108,  100   1000
   c
     stress-strain curve 3 temperature 500  e 25000 nu 0.25 alpha 0.0003
       0.0   60,  0.025 74,   0.050 79,   0.10  83,   0.20  88,   0.30  91,
       0.40  93,     100   1000
   c 
   c 
       material steel
             properties mises curves 1-3
\end{verbatim}
\normalsize
%
\subsection{Stress-Strain Curve Command: \ti{Strain-Rate Dependent}}
\nin
The stress-strain curves provide one of several mechanisms 
to define nonlinear (flow) properties that vary with plastic strain-rate.
For these materials, the user specifies a series of stress \ti{vs.} plastic strain
curves at various (constant) plastic strain rates. Young's modulus, Poisson's
ratio and thermal expansion coefficient do not vary with loading rate and are
specified in the \ti{material} definition. During analyses with plastic strain rates
that vary over time (or load), these curves provide the needed data to establish
the instantaneous flow properties for the material. By numerical differentiation
of the curves at a fixed plastic strain value, the computational routines also
determine the rate of change of the uniaxial stress with plastic strain rate.

The user specifies material flow properties over a range of plastic strain rates
with separate \ti{stress-strain curves}. Up to 20 such curves may be defined. The
\ti{material} definition then refers to a \ti{set} of such rate dependent stress-strain
curves. For example,

\small \begin{verbatim}
      material a533b
           properties mises curves 1-5 e 30000 nu 0.3 alpha 1.2e-05,
                                            rho 0.1254e-07
\end{verbatim} \normalsize

No requirements are imposed on the range of plastic strain rates or the
increments of plastic strain rate between the specified curves. During an
analysis, the plastic strain rates at points within elements may not correspond
to one of the values specified through stress-strain curves. In such cases, the
material processors construct the stress \ti{vs.} plastic-strain curve
using (linear) interpolation between values on the specified curves. When the
plastic strain rate at a material point lies outside the minimum-maximum rates
specified for the stress-strain curves, the corresponding minimum-maximum curves
are used in computations -- no extrapolation is performed outside user
specified range. Figure 2.3 shows an example set of three stress-strain curves
to describe the dependence on plastic strain rate.
%
\begin{figure}[tb]
\begin{center}
\includegraphics[trim=0in 5.2in 6.0in 0.4in,scale=1.1,angle=0]{Figs_3.pdf} 
\caption{{\small Fig. \thefigure: Example of piecewise-linear stress-strain 
curve for response that varies with plastic strain rate.\normalsize}
\label{fig:fig2}}
%
\end{center}
\end{figure}
%

Points on such curves are specified with the command sequence \ti{stress-strain
curve} where each such curve required in the analysis is assigned an integer
number for identification. The curves may then be referenced in a \ti{material}
command as described above. The command syntax is
\begin{align*}
&\hv{\ul{stress}(\mbox{-}\ul{strain})\ \ul{curve}\ <curve\ number:integer>\ 
\ul{plastic}\ \ul{strain}\mbox{-}\ul{rate} <numr>\ <properties>} \\
&\hv{\ \ [\ <plastic\ strain\ value:numr>\ <stress\ value:numr> (,) \ ]}
\end{align*}\normalsize
\nin The ordering of stress-strain curves has no significance -- curve number 1
is not required to define the highest or lowest plastic strain rate. The input
translators re-arrange the internal ordering of curves as needed to support the
computational processors.

\nin Note the following guidelines and requirements to define strain-rate dependent
curves:
\small \squishlist
\item Curve points are input as strain-stress pairs;
\item Use as many lines continued with commas as needed to specify the points;
\item Multiple pairs may be specified on a line (maximum of 20 pairs total for
the curve);
\item The first point defines yield stress at the corresponding plastic strain
rate. 
\item The plastic strain for the first point on the curve \ti{must} be 0.0;
\item Input \ul{plastic} strain values;
\item All plastic strain-stress values must be positive;
\item Plastic strain values must increase monotonically;
\item Stress values are not required to increase monotonically and may
actually decrease;
\item As illustrated in Fig.  2.3, the plastic strain values must be identical
for all the curves that make up a set of curves to define the dependence on
plastic strain rate;
\item The material response at plastic strain values larger than the last
point specified on the curve follows a perfectly plastic response.
\squishend \normalsize

For large-deformation analyses, the values should correspond to the logarithmic
strain-Cauchy stress; for small strain-analyses the values should be engineering
strain-engineering stress. 

Any value of yield stress specified during the material definition is
overwritten during analysis with the above values interpolated at the current
plastic strain rate of the material point.

The following is a complete example which uses three curves to define the
plastic strain rate dependent response (see Fig. 2.3)
\small
\begin{verbatim}
   c
     stress-strain curve 1 plastic strain-rate 500
        0.0   80,   $ these are plastic strain vs. stress values
        0.025 96,   $ the first point must have zero plastic strain...
        0.050 103,   0.10  108,   0.20  114,  0.30  118, 
        0.40  122, 100   1000
   c
     stress-strain curve 2 plastic strain-rate 250
         0.0 70,  0.025 85,  0.050 91,  0.10  96,  0.20  102,  0.30  105,  
         0.40  108, 100   1000
   c
     stress-strain curve 3 plastic strain-rate 0
         0.0   60,  0.025 74,  0.050 79,  0.10  83,  0.20  88,  0.30  91,   
         0.40  93,  100   1000
   c 
   c 
      material steel
          properties mises curves 1-3 e 30000 nu 0.3,
                         alpha 1.2e-05 rho 0.1e-06
\end{verbatim} \normalsize

\subsection{Nodal Values of Material Properties for FGMs}
\nin
The nodal values for certain material properties may be specified with the
following command sequence
\begin{align*}
&\hv{\ul{function}ally\ \ul{grade}d\ \ul{mater}ial\ \ul{prop}erties} \\
&\hv{\ \ (\ul{node})\ <list\ of\ nodes:integerlist>\ [<material\ property:label>\ <value>]}
\end{align*}\normalsize
\nin where the currently supported \textless material properties\textgreater\ include \ti{e, nu, alpha, rho,
yld\_pt, tan\_e, and n\_power}. The capability of material models to use the
specified FGM properties varies with the computational models (see Chapter 3). 
Repeat the node data line as required to specify values. An example of
this command sequence is

\small
\begin{verbatim}
    functionally graded material properties
        nodes 1-200 e 30000 nu 0.3 alpha 0.3e-05 rho 0.1e-4 yld_pt 50
        nodes 201-1000 alpha 0.5e-06 nu 0.2 e 25000 rho 0.2e-04
        1200-1300 nu 0.1 alpha 0.7e-06 n_power 12.3 
        1500-8000 alpha 0.1e-06 tan_e 250.0
        1500-8000 e 28000.
           .
           .
\end{verbatim} \normalsize

\nin The command sequence may be given after the model sizes (number of nodes) have
been entered and before the element specifications. A complete example of
material, FGM and element specification is 
\small
\begin{verbatim}
    structure compact_tension
           .
           .
    material graded_ceramic
         properties bilinear e 'fgm' nu 'fgm' alpha 0.3e-06,
	                        rho 0.000034 yld_pt 1.0e20
           .
   number of nodes 19432 elements 15230
          .
          .

   functionally graded material properties
      nodes 1-200 e 30000 nu 0.3 alpha 0.3e-05 rho 0.1e-4
      nodes 201-1000 alpha 0.5e-06 nu 0.2 e 25000 rho 0.2e-04
      1200-1300 nu 0.1 alpha 0.7e-06 
           .
           .
   elements
        1-500 q3disop linear material graded_ceramic order 2x2x2,
                                           center_output short 
\end{verbatim} \normalsize

\nin Note the following options for specifying FGM nodal property values:
\small
\squishlist
\item property values for a list of nodes may be specified all on one line or
on separate lines. For example, all the \ti{e} values could be specified first, then
all the \ti{alpha} values, etc.
\item newly specified values overwrite previous values
\item nodal property values may be changed between load steps in the analysis
\squishend
\normalsize

\subsection{Anisotropic Thermal Expansion Coefficients}
\nin
Residual stresses are often conveniently modeled by imposing components of
initial strains that vary spatially over the model. Incompatibilities in the
specified initial strains generate mechanical (residual) stresses to re-establish strain
compatibility. To support this modeling strategy, WARP3D provides the capability
to define anisotropic thermal expansion coefficients for each element in the
model. When the imposed nodal/element temperatures have unit values, the expansion
coefficients then become the initial strains to drive the analysis.

Separate materials can be defined for each set of thermal expansion coefficients
but this leads to the input of much repeated information when many elements
(possibly thousands) with different initial strains are required. To streamline
the input, anisotropic thermal expansion coefficients also may be input in a
tabular manner once the elements and material definitions have been specified.
Materials are defined in the usual manner except for values of the anisotropic
thermal expansion coefficients. Those values are input using a command sequence
having the syntax is
\begin{align*}
&\hv{\ul{thermal}\ \ul{expa}nsion\ \ul{coeff}icients} \\
&\hv{\ \ <element\ nos.:list>\ [<alpha\ id:label>\ (<value>)\ (,)]}
\end{align*}\normalsize
\nin where the labels that specify the six expansion coefficients are: \ti{alphax, alphay,
 alphaz, alphaxy, alphaxz, alphayz}. Abbreviated labels may also be used: \ti{x, y, z,
 xy, xz, yz}. Values may be specified in any order; values not specified are set
to zero. Expansion coefficients are changed from the values specified in the
material definitions only for elements specified in this input sequence.

\nin An example of this command is 
\small
\begin{verbatim}
     thermal expansion coefficients
         1-40 alphaz 1.0e-4 alphaxy 2.1e-5 alphayz -3.7e-6
         41-1000 x 1.e-05 y -2.1e-4 xz 5.4e-4
\end{verbatim}
\normalsize

\nin The above command sequence may be input any point after the elements
specification.

\end{document}