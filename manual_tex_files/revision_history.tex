%
\documentclass[10pt]{report}
\usepackage{geometry} 
\geometry{letterpaper}
%
%
%   --- margins and inter-paragraph spacing ---
%
%---------------------------------------------
\setlength{\textheight}{630pt}
\setlength{\textwidth}{450pt}
\setlength{\oddsidemargin}{14pt}
\setlength{\parskip}{1ex plus 0.5ex minus 0.2ex}
%
%----------------------------------------
\usepackage{amsmath}
\usepackage{layout}
\usepackage{color}
\usepackage{hyphenat}

%---------------------------------------------
%
%   
\frenchspacing
\usepackage{sectsty} 
\usepackage{xspace}
\allsectionsfont{\sffamily} 
\sectionfont{\large}
\usepackage[small,compact]{titlesec} % reduce white space around sections
%
%          --- header and footer contents ---
%
\usepackage[us,12hr]{datetime}
\usepackage{fancyhdr} \pagestyle{fancy}
%\setlength\headheight{15pt}
%\lhead{User's Guide - \ti{WARP3D}}
%\rhead{\ti{Revision History}}
\fancyfoot[L] {\small{\ti{Revision History}}}
\fancyfoot[C] {\small{\thepage}}
\fancyfoot[R] {  \small{\ti{Updated: \today\ at \currenttime}}}
\renewcommand{\headrulewidth}{0.0 pt}
%
%   ---  local commands ---
%
\newcommand{\tb} {\textbf}
\newcommand{\df} {\dotfill}
\newcommand{\nin} {\noindent}
\newcommand{\bmf } {\boldsymbol }
\newcommand{\bsf } [1]{\textrm{\ti{#1}}\xspace}
\newcommand{\HRule}{\rule{\linewidth}{0.5mm}}
\newcommand{\patwarp}{\ti{patwarp\xspace}}
\newcommand{\eg}{\ti{e.g.},\xspace}
\newcommand{\ie}{\ti{i.e.},\xspace}
\newcommand{\ul} {\underline}
\newcommand{\hv} {\mathsf}   %helvetica text inside an equation
\newcommand{\ti}{\emph}
\newcommand{\noi}{\noindent}

%
%        optional definition for bullet lists which
%        reduces white space.
%
\newcommand{\squishlist}{
 \begin{list}{$\bullet$}
  { \setlength{\itemsep}{0pt}
     \setlength{\parsep}{3pt}
     \setlength{\topsep}{3pt}
     \setlength{\partopsep}{0pt}
     \setlength{\leftmargin}{1.5em}
     \setlength{\labelwidth}{1em}
     \setlength{\labelsep}{0.5em} } }

\newcommand{\squishlisttwo}{
 \begin{list}{$\bullet$}
  { \setlength{\itemsep}{0pt}
     \setlength{\parsep}{0pt}
    \setlength{\topsep}{0pt}
    \setlength{\partopsep}{0pt}
    \setlength{\leftmargin}{2em}
    \setlength{\labelwidth}{1.5em}
    \setlength{\labelsep}{0.5em} } }

\newcommand{\squishend}{
  \end{list}  }
%
%
%   ---  page numbering ---
%
\pagenumbering{roman}
\setcounter{page}{5}
%
%
%
%              start document 
%              ==========
%
%
\begin{document}
\noindent
\LARGE
\begin{center}
 \textbf{
{\fontfamily{phv}\selectfont Revision History}}
\end{center}
\normalsize
\vspace{-0.1in}
%
\begin{center}
\line(1,0){250}
\end{center}

\section* {Version 17.7.0 (January 20, 2016)}
\squishlist
\item A line search procedure is now available to augment global Newton iterations 
for simulations with difficult convergence. 
This implementation uses a backtracking algorithm to
compute a suitable step length for the corrective displacement vector. We now recommend these options
for routine nonlinear solutions: \ti{extrapolation on, line search on, adaptive solution on}.
\item The solution option \ti{linear stiffness on iteration one} has been deprecated and replaced by
actions of the (displacement) \ti{extrapolation on/off}.
\item New creep material model. The initial implementation of a creep model is now included in 
WARP3D. The model name is \ti{creep} and is
described in Section 3.13. This is a classic Norton-Bailey power-law model 
generally applied to characterize secondary creep.
\item The WARP3D interactions with UMAT routines (when they are called during the nonlinear solution)
now reflect more closely how Abaqus calls the UMAT for increment=1 of each
*STEP and in each subsequent increment for the *STEP. Updated documentation and code fragments describe 
the improvements. The procedures to handle thermal effects, initial 
strain (\eg creep)  and possibly initial stresses are much simplified.  Existing UMAT routines
written to be executed by Abaqus require very minimal changes to work properly with WARP3D.
\item A large-scale overhaul of the internal strategy for the sequencing of computations in
global Newton iterations has
substantially simplified the code, eliminated a noticeable fraction of the source code, and decreased 
run times. The number of global Newton iterations to reach a converged 
solution is often reduced. These modifications are transparent to the user
except for the changes made 
with the \ti{extrapolation on/off} nonlinear parameters (we recommend using \ti{extrapolation on}
as the default).
\item The \ti{crystal plasticity} model (CP) has been extended to include a new
slip-system type designated {bcc48} \{110\}$<$111$>$, \{112\}$<$111$>$, and \{123\}$<$111$>$ [48 slip systems]. Some internal algorithms in the CP model for stress updating have been revamped to improve 
computational efficiency and robustness. Professor Tim Truster at the University of Tennessee-Knoxville
is actively working to extend capabilities of the CP model.
\squishend

\noi Manual Sections updated: Table of Contents,  1.6.4 (line search, displacement extrapolation), 
2.10 (line search,
extrapolation), 3.12 (CP updated for new bcc48 slip systems), 3.13 (new), Appendix I (all sections).


\section* {Version 17.6.0 (July 30, 2015)}
\squishlist
\item Documentation describing the theory and input commands for the crystal plasticity 
model (CP) have been fully been re-written and expanded. Multiple new examples 
illustrating use of the model are included. A new simplified Voce work hardening 
(temperature and rate invariant) is now available in addition to the more complex MTS hardening model.
\item A significant new capability is now available to output key state variables for the material models,
including  user-provided UMATs. Such variables include, for example, the backstress components
for kinematic hardening in plasticity, elastic strain components for plasticity, equivalent plastic
strain rates, detailed slip system status for
the crystal plasticity model, etc. State variable output is provided in \ti{flat} files of element 
results (\ti{text} or \ti{stream} format). See updated Section 2.12 on Output.
\item The WARP3D interface to UMATs and other Abaqus compatible user-routines has been
expanded and re-organized. Appendix I on User-Routines has been re-written to describe these
improvements.
\item The \ti{warp3d2exii} post-processing program  has undergone significant 
updates and improvements. Earlier versions of this
program were named \ti{pat2exii}. This is a Python program that reads a WARP3D generated \ti{flat} file for the
model description and \ti{flat} files of nodal and elements results  (including new state variables mentioned above)
then produces an ExodusII (.exo) 
database file. We use this capability to enable post-processing with ParaView.
\item Processing of user-defined multi-point constraints and those generated automatically by tied mesh operations is
much improved. Earlier code versions forced an un-necessary increase on the number of global Newton iterations
required for convergence (solutions were correct by computations required un-necessary time).
Section 2.7 on constraints has been expanded for improved clarity in the description of multi-point
constraints.
\item Two new tests are now available for the global Newton iterations in each load (time) step. The new tests
examine absolute, rather than relative, values of the corrective displacements in each Newton iteration. Multiple
tests may be defined and all must be satisfied for convergence. The description of convergence 
tests has been expanded and clarified.
\item The command to request a user-routine to generate incremental nodal loads/temperatures for
a load(time) step may now supply a file name to be passed to the \ti{user\_nodal\_loads} routine.
\item The \ti{batch messages} command is extended to support a user-specified
file name in addition to the previously used default name based
on the structure name specified in the input.
\item The \ti{user list} command now has an option to display the coordinates of nodes
in the specified/generated list.
\item Much internal re-structuring, not visible to users, to improve maintainability of the source code.
\squishend

\noi Manual Sections updated: Revision History, Table of Contents,  2.7 (Nodal Constraints and Releases),
2.8 (Loads), 2.10 (Solution Parameters, Newton parameters), 2.12 (Output), 2.16 (User Lists), 3.12 (CP model), Appendix I.




\section* {Version 17.5.6 (October 1, 2014)}
\squishlist
\item The \ti{plane} option of the \ti{constraints} command now offers more
control over the distance proximity checks and provides additional
output on the distance checking. Similar improvements are included in the
\ti{list} command (Section 2.16).
\item A new output capability that simplifies significantly the specification of
analyses with many load(time) steps. A file of \ti{output} commands may be 
defined that will be executed automatically after each step in a 
specified list of steps. Such automatic output can also be mixed with
additional output commands given after specific load steps
as in the past.
\item Asymmetric equation solvers are now available for the systems of equations that
derive from the crystal plasticity model. Asymmetric systems have symmetry
of the non-zero locations in the assembled equations but non-symmetric numerical values.
The asymmetric solver requires twice the amount of memory and twice the equation
solving time for each global Newton iteration -- the savings in solution time can
arise from the reduced number of global Newton iterations. The asymmetric option is available
in threads only execution via the Intel Pardiso solver direct and iterative options (Section 2.10.2).
\item Output strain values for solid element in finite strain/rotation analyses are now 
given by the summed increments of \ti{unrotated} strain, $\Delta \mathbf{d}$,
rotated to the fixed-global axes using $\mathbf{R}$ from the polar decomposition,
$\mathbf{F}=\mathbf{RU}$. These values match closely those output by Abaqus 
which uses summed, incrementally rotated values over the loading history.
See Section 1.8.6 for additional details.
\item
A new output capability to write \ti{flat files} of nodal and element results.
These files have the simplest possible structure of a 2D array with no other
readable information. The flat files are provided in \ti{text} or
\ti{stream} (binary) forms.  The text files may be read directly using Python
(\ti{loadtxt}), 
imported into Excel, Matlab, Mathematica or read easily with
C, C++, Fortran programs. The stream files may be read with
\ti{fromfile} in Python; Matlab and Mathematica using 
binary stream I/O, and using standard stream operators
in C, C++ and Fortran (2003 and later).
\item
Similar to the flat results files, a new output capability is available
to write a simple, \ti{flat} file for the model description. The .text file
includes the number of nodes/elements, coordinates of nodes
and element data (type, material assigned to element and
nodes to which the element connects). The file structure is
designed to be read very efficiently by Python, imported into
spreadsheets, read by C, C++, ... programs. Appendix K includes sample 
Python programs to read this model description file and the
various flat results files.
\item Manual Chapter 1 has been re-written, updated and expanded with
a wider range of example analyses/discussion and an expanded discussion of 
theoretical developments implemented in the code.
\item
A new capability is available to \ti{release} absolute constraints
on nodes during the solution. The corresponding reactions are
relaxed uniformly to zero over the specified number of
load (time) steps. Node releases are continued across analysis restarts
if  required and are integrated into the adaptive step subdivision of
the nonlinear solution.
\item
The capability to write the equilibrium equations to a
file at any time during a nonlinear analysis is described in
Section 2.10.16. These files may become quite large (10s GB) for
large models. Moreover, users may want to read the equations for
processing with C and C++ programs. The solver file now
uses the \ti{access=stream} capability now available in
Fortran.
\item
The manual section on \ti{multipoint} constraints is updated to clarify
the more general command syntax, limitations that the equations must
be homogeneous, and assumptions the WARP3D equation solver makes about
dependent \ti{vs} independent displacement components. 
\squishend

\noi Manual Sections updated: Revision History, Table of Contents,
1.1, 1.2, 1.8, 1.9 (conversion to LaTeX and update), 
2.7 Nodal Constraints and Nodal Releases,
2.10 (Solution Parameters, Newton parameters),
2.12 (Output, new Sections 2.12.4, 2.12.6 on Flat Results and
Model Description Files),
2.16 (User-Defined Lists, updated tolerance definitions).
 A new Appendix K describes
skeleton Python programs to read flat files of results in \ti{text} and \ti{stream}
forms.
Sections 1.4, 1.5 and 1.6 have been re-written for
improved clarity. 



\section* {Version 17.5.0 (December 2013)}
\squishlist
\item
A new capability is available during the global Newton iterations to detect a likely
diverging solution. This option is ``on" by default. When a divergent sequence of iterations
is detected, the adaptive solution procedure is triggered immediately (if enabled).
\item
A \ti{user\_solution\_parameters} routine is now available. See
Section 2.10 and Appendix I for full details. This routine is called before
the analysis of each load (time) step. The routine may change
(1) values of variables that control the nonlinear solution process
and (2) the incremental loading definition or the upcoming step. May be
valuable, for example, in complex nonlinear simulations to modify
load (time) step definitions on-the-fly based on the convergence
history of the last few load steps.
\item
All user-routines are now collected into the \ti{user\_routines.f}  file
replacing the older \ti{umats.f} file.
The distributed version of this .f file has example 
implementations of all user routines. 
\item
The crystal plasticity (CP) model now has an exact consistent tangent to accelerate 
convergence of global Newton iterations. Computational efficiency in the CP routines is also
improved.
\item
The Pardiso equation solver now (automatically) uses threaded reordering to increase performance.
\item
The routines to write a restart file and to read a restart file now use a  \ti{segmental} structuring of
data which improves performance and eliminates occasional runtime issues with file buffering.
\item 
Increased to 5,000 the maximum number of loading conditions (patterns).
\squishend

\noi Manual Sections updated: Revision History, Table of Contents,
2.10 (Solution Parameters, Newton parameters),
Appendix I (Abaqus UMAT and other user routines.), Appendix J (Formulation and
algorithms for the CP model).


\section* {Version 17.4.1 (July 2013)}
\squishlist
\item
The Abaqus UEXTERNALDB user-written routine is now supported in
WARP3D for both threaded and MPI + threaded execution. See
Appendix I for full details.
\item
The material model \ti{cohesive} offers a linear-elastic response in addition to
nonlinear responses. The linear-elastic response now has the penalty
method implemented to prevent inter-penetration of the cohesive
surfaces under compressive loading. See Section 3.8 for details.
\squishend

\noi Manual Sections updated: Revision History, 2.12 (Output Requests - Patran Result 
Files, Patran Neutral File),
3.3 (interface-cohesive elements), 
Appendix I (Abaqus UMAT and other user routines - UEXTERNALDB and
user\_nodal\_loads).

\section* {Version 17.4 (June 2013)}
\squishlist
\item
New crystal plasticity material model for rate and temperature dependent 
simulation of microscale plastic flow in metals.  Includes options for
simple gradient-based geometric hardening, which incorporates the effect of
necessary dislocations on the hardening properties of the material.

\item 
Expanded solution capabilities during MPI-based execution:
(1) the stiffness assembly process is now performed using a
distributed approach across the MPI ranks, (2) the \ti{hypre}
solver now includes an option for the BoomerAMG pre-conditioner
(parallel implementation of algebraic multigrid).


\item
The interface-cohesive elements now support hardcopy and binary packet
file results for tractions and displacement jumps. Derived values applicable to 
each type of cohesive material option, \eg \ti{linear}, \ti{ppr}, etc. are also
provided in the output. Manual Section 3.3 has been re-written to
better describe the formulation and various options for the interface 
elements. 

\item
Nodal temperatures may now be requested in printed
output, Patran compatible result files and in binary
packet files.


\item
A new option for the \ti{linear stiffness on iteration 1} of a load
step has been added that often simplifies input. The new
option invokes the linear stiffness only for the next load step.

\item
The \ti{extrapolate displacements} option often improves convergence
characteristics of the global Newton solution. However, when the
compute processors  detect the next step is
non-proportional with the prior step, the displacement extrapolation is suspended 
just for the new step. The linear stiffness for iteration 1 is also
invoked. Both measures tend to improve convergence of the global
Newton iterations for strong load direction changes between steps.


\item 
The batch message system that writes messages to a file during execution
has been streamlined to compress the messages. The batch message file
summarizes the solution status during each global Newton iteration.
The file enables simple monitoring of long executions executed in batch
(background) job management systems often used on shared computer 
systems.

\item
The \ti{cpu time limit} option has been changed to the 
\ti{wall time limit} option to reflect that most all
modern analyses run in parallel. Wall clock time, not CPU
time, becomes the relevant measure of solution resources 
consumed. The \ti{batch message} files now provide wall clock time
rather than CPU time.


\item
Older solvers removed: element-by-element (EBE) conjugate gradient,
NASA, VSS sparse direct solvers. The \ti{hypre} iterative solver for  MPI-based
execution replaces the EBE solver. The MKL-Pardiso solver has direct and 
iterative capabilities for threads-only execution.

\item
With removal of older solvers, user input
requirements for solution parameters are much simpler.

\item Requirements for vectorized blocking of elements in certain situations has been
removed with simplification of options for equation solving.

\item
The startup process (Bash shell script) for MPI-based (+ threads)
executions has improved reliability and speed in starting MPI processes.

\item
\textit{patwarp} updated-simplified to remove vectorized blocking options and to
remove options for EBE solver.

\item
Significant cleanup, streamlining, error-fixes and re-writing of code throughout.
\squishend

\noi Manual Sections updated: Revision History, Executive Summary, 
Acknowledgements, Table of Contents,
1.7 (Equation Solvers), 
2.6 (Element Blocking), 2.10 (Solution Parameters),
3.3 (interface-cohesive elements), 
3.12 (the new crystal plasticity model), Chapter 7 (Parallel Execution),
Appendix C (\textit{patwarp}), Appendix F (binary packet types), 
Appendix J (Supported Platforms), Appendix I (umats)
Appendix J (details of crystal plasticity model).


\section* {Version 17.3.2 (September 2012)}
\squishlist
\item
The \ti{blocking} command has been extended to provide \ti{automatic}
assignment of elements to blocks for the most common
analyses performed with WARP3D. The automatic blocking feature
is applicable for: (1) threads-only parallel execution on Windows, Linux, Mac OS X,
(2) use of the Pardiso sparse equation solver (either direct or iterative). 
For other, much less common
situations (EBE solver and/or MPI+threads execution),
 the blocking information must be provided in the input as in
prior versions of the code.

\item The recently added UMAT feature provides a very convenient means to incorporate
new material constitutive behaviors. WARP3D expects the UMAT routine and all supporting
routines to be written using thread-safe principles -- parallel processing of element
blocks will invariably invoke multiple instances of the UMAT concurrently. Many
older UMATs use COMMON blocks (not thread-safe), are often exceedingly
complex and cannot be easily re-written to make them thread-safe. An option
in the \ti{nonlinear analysis
parameters} command is now available to request serial (1 thread) 
execution of element blocks using the UMAT
during only the stress update for non thread-safe UMATs. See Appendix J.

\squishend

\noi Manual Sections updated: Table of Contents, 2.6, 2.10.3,  2.10.6, 2.10.7, 2.10.8,
2.12, Appendix F, Appendix J.

\section* {Version 17.3.1 (September 2012)}
\squishlist
\item
Nodal forces and temperatures for  loading patterns may now be set by
a user-defined nodal loads subroutine. This option replaces the often
lengthy text-based definition of forces and temperatures in the model
definition. The user nodal routine is
invoked by the solution processor at the beginning of each load step which enables the
routine to provide a potentially complex history of nodal force/temperature
increments over the analysis.

\item For material models that require access to the deformation
gradient, [$\bmf F$], an [$\bmf {\bar F}$] formulation is now used for the
linear (8-node) brick element. The computations to set [$\bmf {\bar F}$] 
employ a mean-dilatation approach.

\item The displacement extrapolation algorithm now detects when
non-proportional, incremental loading exists for a step. The extrapolation
procedure is suspended temporarily just for that step. This simplifies
managing solution options when load reversals occur in cyclic loading
protocols.
\squishend

\noi Manual Sections updated: Table of Contents, 2.8, 2.10.8, 3.1, Appendix J.

\section* {Version 17.3 (July 2012)}
\squishlist
\item
WARP3D can now integrate existing Abaqus compatible UMAT 
routines to support user-defined material behavior.
The UMAT routines are invoked from within (thread) parallel
processing of element blocks. The UMAT thus has full benefit of 
parallel processing already designed into WARP3D. See Section 3.11 and
Appendix J.
\item
Major changes have been made to internal data structures and 
algorithms that impose non-zero displacement and temperature
increments over a load step. Testing reveals a generally reduced number of 
equilibrium iterations for convergence. No changes in the manual.
\item
Updated descriptions of element blocking requirements. See Section 2.6
specifically if you make use of computational material models \ti{cyclic}, 
\ti{mises\_hydrogen}, and \ti{umat}. 
\squishend

\section* {Version 17.2 (January 2012)}
\squishlist
\item
A new feature called user named lists of integers is now available as described in the
new Section 2.16. The \ti{list} command generates named lists of
nodes or elements using a variety of geometric 
procedures, \eg define a list containing the nodes that lie
on the surface of a cylinder. The named lists can be referenced in all
other commands that require input of an $\hv{<integerlist>}$ such as
constraints definition, output requests, loading definitions, domain integral
definitions, etc.
\item
An option to simplify the use of input files constructed partly from Abaqus models is
available. The definition of face loadings on hex elements and the definition of
surfaces for tied-meshing now have an \ti{abaqus} option to convert
Abaqus face numbers during input to WARP3D face numbers. See Sections 2.7.6
and 3.1.5.The ordering of element nodes is the same in Abaqus and WARP3D.
\item
Lines beginning with ! or \# in column one are now treated as comment lines
in the input. Lines beginning with c or C in column 1 followed by a blank continue
to be treated as comment lines as well. Blank lines in the input are ignored.
\item
The definition and interpretation of strain output for nonlinear geometry analyses
have been updated. Output values now correspond to the approximate logarithmic strain
on the \ul{current} configuration. See manual section 3.1.6.
\item
More sections of the manual are now in LaTeX format and include revisions
and corrections. Recently converted sections include 1.3, 1.7, 2.1-7,
2.10, 2.13-16, 3.1-3, 3.8, 3.10, 5.1, 5.4, Chapter 7 and the Appendices.
Eventually the entire manual
will be in LaTeX. Source files for the manual will be included in the distribution.
\squishend


\section* {Version 17.1 (December 2011)}
\squishlist
\item
A new feature called ``tables" is implemented with the \ti{table} command
(new Section 2.14). This is a general capability now
and for future to define large 2D tables of floating point
data values for use in various aspects of model definition.
For now, tables are used to define parameters for ``piston"
loading on surfaces of solid elements.
%
\item Piston theory loadings define unsteady aerodynamic
pressures applied on the surfaces of solid elements.
Piston theory provides a simplified
(quasi-1D) model to estimate local pressures (center of element surface)
for the current velocity of the surface and orientation relative
to airflow direction. The piston pressures are time varying and 
strongly dependent upon instantaneous motion of the element surface.
The availability of piston loadings can eliminate costly and
inconvenient coupling/iterations between structural analysis and 
separate loading programs in large-scale simulation work.
See updated descriptions of hex and tet elements in Sections 3.1 and 3.2
(and required use of the table command in Section 2.14).
%
\item The model solution time is now output with messages issued about
completion of the solution for loads steps. Model solution
time is the integral of the user-specified time increment, $\int dt$, specified
in nonlinear solution parameters.
%
\item Various messages issued during solution displayed the cpu time consumed
by the code to that point in execution. With nearly all analyses
running now in parallel, these messages have been changed to display
elapsed wall-clock time since the start of execution.

\squishend
%
%

\section* {Version 17.0 (August 2011)}
\squishlist
\item
A new approach to execution on hybrid (MPI+threads) computers is now available with
inclusion of the \ti{hypre} equation solver from Lawrence Livermore National Labs. Hypre solves the
linear equations for each global Newton iteration using variants of preconditioned conjugate gradients.
The solver is designed specifically to maintain very high scaling efficiency as the number of
MPI ranks increases into the hundreds and perhaps thousands. Our implementation runs WARP3D
on MPI rank 0 with threaded parallel execution and invokes hypre for each equation solve. Hypre
spawns/releases additional MPI ranks for just the equation solving phase. This approach maintains the
simplicity of WARP3D input files for threads-only execution (no domain decomposition) with the
opportunity for very high parallel efficiency across tens-to-hundreds (perhaps a thousand) of
processors during equation solving phases.
\item
The \ti{generalized\_plasticity} option of the \ti{cyclic} plasticity model has been extended to include
temperature-dependent material properties ($E$, $\nu$, $\sigma_0$, ...). This model offers
behavioral and computational advantages over the more traditional Armstrong-Frederick model (also 
available as an option of the \ti{cyclic} material model).
\item
Updated manual sections: 1.7 (Equation Solvers), 2.2 (Material Definitions),
2.6 (Element Blocking), 2.10 (Solution Parameters),
3.10 (Cyclic Plasticity Model), Chapter 7 is entirely re-written to introduce the hypre solver option
and the needed details to start execution of WARP3D, Appendix I now contains material describing the
element-by-element conjugate gradient solver, Executive Summary, Contents, etc. We are still in the
transition phase to eventually convert the manual to LaTeX.
\item
README files for each platform and the general file included in the distribution have been significantly
updated.
\item
A number of stubborn bugs that previously slowed threaded execution during stress update have been
eliminated (we had work around code to force serialization in a few locations). All work around code is
now removed.
\item
We are building interfaces between WARP3D and the Exodus II database system used by a number of pre-
post-processor codes. Watch for new developments.
\item
The compilation process on all three platforms (Windows, Linux, OS X) is updated to use the current
releases of the Intel compilers and the Math Kernel Library (which has the Pardiso solver).

\squishend

\section* {Version 16.3.1 (November 2010)}
\squishlist
\item
The \ti{cohesive} material model is expanded significantly to incorporate the Paulino-Park-Roesler 
formulation (\ti{ppr}) that properly models mixed-mode fracture. The model supports independent specification of the fracture energies and peak tractions for normal and shear deformation modes. The model couples normal and shear fracture processes through the use of a polynomial-based potential function.
\item
The interface-cohesive, crack growth processors and adaptive algorithms are updated to include the \ti{ppr} option.
\item
The binary packets file now includes results from interface-cohesive elements with the \ti{ppr} option
\item
The WARP3D interface to the Intel (Pardiso) solver is re-written to also support the Krylov iterative option in addition to the previously support, continually improving, sparse direct solver in Pardiso. The iterative version uses occasional sparse Choleski factorization of assembled equations to define the pre-conditioner. When the iterative process develops convergence issues, Pardiso switches to the direct solver automatically.
\item
The conversion of binary packets files to ASCII files has been deleted as no longer needed to move files across hardware platforms. 
\item
Updated manual sections: Appendix F (binary packets file), 5.1 Introduction to Crack Growth, 5.4 Crack Growth Using Interface-Cohesive Elements, 3.3 Interface-Cohesive Elements (re-written for improved clarity), 3.8 Material Model Type: cohesive, 2.10 Solution Parameters, 2.12 Output Requests, Table of Contents
\item
The compilation process for WARP3D on all three supported platforms (Linux, Mac OS X and Windows) has been moved to Intel Composer XE system. 
\squishend

%
%---------------------------------------------------------------------------
%        section
%---------------------------------------------------------------------------
%
\section*{Version 16.2.7 (July 2010)}
\squishlist
\item
The domain integral processor has a new feature that enables user-defined 
tangent vectors at crack front nodes. This option provides a means to remedy 
domain dependent $J$-values, for example, when the crack front in the discrete 
mesh intersects a symmetry plane with a kink angle, rather than smoothly at a right angle. 
This can occur when quadratic elements on the crack front are created with straight edges 
that define a �chordal� approximation to an otherwise smooth curve. Coarse meshes of linear 
elements also have this same issue. This feature will prove most useful for $J$ evaluation 
at symmetry planes and where the crack front intersects a free surface. For meshes 
with sufficient resolution and/or front elements with curved edges, the automatic 
tangent vectors computed by WARP3D are adequate.
\squishend

%
%---------------------------------------------------------------------------
%        section
%---------------------------------------------------------------------------
%
\section*{Version 16.2.4 (December 2009)}
\squishlist
\item
WARP3D for Mac OS X (Intel) systems is now available. The code runs in 64-bit mode although 
Mac OS X can be running in 32-bit (default) or 64-bit mode. This initial version for OS X 
supports parallel execution via threads and shared-memory. An MPI version for OS X is 
under consideration. Manual section 7.10 provides details of running 
WARP3D on OS X (essentially identical to running WARP3D in the threads-only 
version for Linux). The Intel (MKL) spares direct solver and the PCG 
solvers and all other capabilities of WARP3D are included in the OS X version.
\squishend

%
%---------------------------------------------------------------------------
%        section
%---------------------------------------------------------------------------
%
\section*{Version 16.2.2 (November 2009)}
\squishlist
\item
WARP3D for Mac OS X (Intel) systems is now available. The code runs in 64-bit mode although 
Mac OS X can be running in 32-bit (default) or 64-bit mode. This initial version for OS X 
supports parallel execution via threads and shared-memory. An MPI version for OS X is 
under consideration. Manual section 7.10 provides details of running 
WARP3D on OS X (essentially identical to running WARP3D in the threads-only 
version for Linux). The Intel (MKL) spares direct solver and the PCG 
solvers and all other capabilities of WARP3D are included in the OS X version.
\squishend


%
\section*{Version 16.2 (October 2009)}
\squishlist
\item
The cyclic plasticity model has been re-written to include new options and computational algorithms. 
The model now offers two options for representing the cyclic plasticity behavior of 
metals: nonlinear\_hardening and generalized\_plasticity. The generalized\_plasticity 
option provides the capability to yield on re-loading prior to the prior un-loading 
stress state and the option for non-zero asymptotic hardening at large plastic strains. 
Section 3.10 has been fully re-written and expanded.
\item
All processing of element ``blocks" now occurs with thread-based parallel execution. 
This significantly reduces solution elapsed time on computers with 
multiple cores and/or processors. This parallel execution capability is implement 
in both Windows and Linux versions.
\item
Chapter 7 of the manual has been re-written, updated and expanded to describe the 
additional thread-based parallel execution of element blocks. More details are now 
provided to run WARP3D on both Linux and Windows.
\item
Windows 64-bit versions (in addition to 32-bit versions) now 
supported on Intel and AMD (x86) processors. The Windows versions now 
support thread-based parallel execution in the (default) Intel MKL solver 
and for processing of element blocks in WARP3D. We are not considering 
MPI versions for Windows at this time.
\item
The Intel MKL solver on Windows, Linux and SGI versions now 
supports out-of-core solutions. The out-of-core solver is often necessary 
for large models computed on 32-bit Windows systems. 
Section 2.10 is revised to describe use of the out-of-core solver.
\item
Support for 32-bit Linux systems is discontinued. Going forward, the systems with
highest priority are 64-bit Linux, and 32- and 64-bit Windows.
\squishend






\end{document}


