
\documentclass[11pt]{report}
\usepackage{geometry} 
\geometry{letterpaper}

%---------------------------------------------
\setlength{\textheight}{630pt}
\setlength{\textwidth}{450pt}
\setlength{\oddsidemargin}{14pt}
\setlength{\parskip}{1ex plus 0.5ex minus 0.2ex}


%----------------------------------------
\usepackage{amsmath}
\usepackage{layout}
\usepackage{color}
\usepackage{array}

%----------------------------------------------
\usepackage{fancyhdr} \pagestyle{fancy}
\setlength\headheight{15pt}
\lhead{\small{User's Guide - \textit{WARP3D}}}
\rhead{\small{Finite Strain Plasticity}}
\fancyfoot[L] {\small{\textit{Chapter {\thechapter}}\ \   (Updated: 7-31-2014)}}
\fancyfoot[C] {\small{\thesection-\thepage}} 
\fancyfoot[R] {\small{\textit{Introduction}}}

%---------------------------------------------------
\usepackage{graphicx}
\usepackage[labelformat=empty]{caption}
\numberwithin{equation}{section}
\usepackage{bm}

%---------------------------------------------
%     --- make section headers in helvetica ---
%
\usepackage{sectsty} 
\usepackage{xspace}
\allsectionsfont{\sffamily}  
\sectionfont{\large}
\usepackage[small,compact]{titlesec} % reduce white space around sections
%---------------------------------------------->
%
%
%   which fonts system for text and equations. with all commented,
%   the default LaTex CM fonts are used
%
%
\frenchspacing
%\usepackage{pxfonts}  % Palatino text 
%\usepackage{mathpazo} % Palatino text
%\usepackage{txfonts}


%---------  local commands ---------------------
\newcommand {\Tab} { \ensuremath{ R_{11}^2}  }
\newcommand {\Tbc} { \ensuremath{ R_{12}^2}  }



\newcommand{\bmf } {\boldsymbol }  %bold math symbol
\newcommand{\bsf } [1]{\textrm{\textit{#1}}\xspace}
\newcommand{\ul} {\underline}
\newcommand{\hv} {\mathsf}   %helvetica text inside an equation
\newcommand{\eg}{\emph{e.g.},\xspace}
\newcommand{\ie}{\emph{i.e.},\xspace}
\newcommand{\ti}{\emph}
\newcommand{\vepsilon}{\varepsilon}
\newcommand{\etal}{\ti{et al.}\xspace}
\newcommand{\nid}{\noindent}
\newcommand{\vareps}{\varepsilon}
\newcommand{\cauchyu}{\boldsymbol{\sigma}_{u}}
\newcommand{\cauchy}{\boldsymbol{\sigma}}
\newcommand{\Dhat}{\hat{\boldsymbol{D}}}
\newcommand{\region}{\bm{\mathcal{R}}}
\newcommand{\dotR} {\dot{\mathbf{R}}}


\newenvironment{offsetpar}[1]%
{\begin{list}{}%
         {\setlength{\leftmargin}{#1}}%
         \item[]%
}
{\end{list}}

%
%
%        optional definition for bullet lists which
%        reduces white space.
%
\newcommand{\squishlist}{
 \begin{list}{$\bullet$}
  { \setlength{\itemsep}{0pt}
     \setlength{\parsep}{3pt}
     \setlength{\topsep}{3pt}
     \setlength{\partopsep}{0pt}
     \setlength{\leftmargin}{1.5em}
     \setlength{\labelwidth}{1em}
     \setlength{\labelsep}{0.5em} } }

\newcommand{\squishlisttwo}{
 \begin{list}{$\bullet$}
  { \setlength{\itemsep}{0pt}
     \setlength{\parsep}{0pt}
    \setlength{\topsep}{0pt}
    \setlength{\partopsep}{0pt}
    \setlength{\leftmargin}{2em}
    \setlength{\labelwidth}{1.5em}
    \setlength{\labelsep}{0.5em} } }

\newcommand{\squishend}{
  \end{list}  }
%
\newcounter{Lcount}
\newcommand{\squishnum}{
\begin{list}{\arabic{Lcount}. }
{ \usecounter{Lcount}
\setlength{\itemsep}{0pt}
\setlength{\parsep}{3pt}
\setlength{\topsep}{3pt}
\setlength{\partopsep}{0pt}
\setlength{\leftmargin}{1in}
\setlength{\labelwidth}{1em}
\setlength{\labelsep}{0.5em} } }

\makeatletter
\renewcommand*\env@matrix[1][\arraystretch]{%
  \edef\arraystretch{#1}%
  \hskip -\arraycolsep
  \let\@ifnextchar\new@ifnextchar
  \array{*\c@MaxMatrixCols c}}
\makeatother


%-------------------------------------
\newcounter{sectrefs}
\setcounter{sectrefs}{0}
\setcounter{chapter}{1}
\setcounter{section}{8}
\setcounter{figure}{0}
\renewcommand{\thefigure}{\thesection.\arabic{figure}}
%
%--------------------------------------
%
%
%
%              start document 
%              ==========
%
%
\begin{document}

\section{Finite Strain Plasticity}
\nid 
This section reviews the theoretical basis and numerical implementation of a constitutive 
architecture suitable for finite strains and rotations in metals. 
The constitutive equations employ a corotational framework with a
Cauchy stress tensor, denoted $\cauchyu$, defined
on an orthogonal 
frame of reference that rotates in an average sense uniquely with each material point. The rotation is
constructed through polar decompositions
of the deformation gradients, $\mathbf{F}$. The conventional (true)
Cauchy stress, $\cauchy$, has components relative to a fixed, global Cartesian
system common to all material points.
The $\cauchy$ stress tensor at a material point  thus represents the  $\cauchyu$ stress tensor
at the point expressed through a simple rotation on the fixed, global system.
Corresponding work-conjugate strain rates $\mathbf{D}$ and $\mathbf{d}$
are defined for $\cauchy$ and $\cauchyu$, respectively. 
Objectivity requires the adoption of a frame invariant stress rate in the
presence of finite rotations for hypoelastic\footnote{Hypoelastic constitutive 
models link objective stress rates 
to objective rates of deformation. Most often the link is 
an instantaneously, linear relationship
with effective (tangent) moduli that depend simply on current (objective)
stresses.}  
%  end of footnote
descriptions of material
response. Among the choices available, the Green-Naghdi rate
of Kirchhoff stress arises naturally as the rate of $\cauchyu$ projected 
onto the fixed global system. In the
absence of rotations at time $t$, the \ti{rates} of $\cauchy$   and $\cauchyu$ become identical.

This treatment of finite strains-rotations using  a corotational approach
for applications primarily in metal plasticity
evolved from the works of Dienes [\ref{R:D1979}], Hughes and Winget [\ref{R:HW1980}],
Johnson and Bammann [\ref{R:JB1984}], Taylor and Flanagan 
[\ref{R:FT1987},\ref{R:TF1989}], 
Hallquist \etal [\ref{R:H1984}], and others. Their
work focused on explicit codes with extraordinarily large numbers of small time steps
with the computational requirement to avoid costly polar decompositions during
updates of material point stresses. Such a
framework  remains the core technology for 
finite-strain plasticity in the explicit codes Abaqus-Explicit, DYNA3D,  
PRONTO3D and LS-DYNA (among
many others) and the older implicit code NIKE3D. The implicit
codes Abaqus Standard and ANSYS adopt the Jaumann rate of
Kirchhoff stress to approximate the effects of finite rotation on the 
updated Cauchy stress $\cauchy$ (and which also avoids the need
for polar decompositions of $\mathbf{F}$).
The discussion later in this section highlights a few effects of different objective
stress rates -- see also the extensive discussions by Bazant \etal [\ref{R:JWB2013}].

The corotational approach adopted here is often discussed with potentially confusing
terminology. The term \ti{unrotated} is frequently used to describe strains, stresses and their rates
expressed on the orthogonal system that rotates with a material point.  Perhaps this is
because increments of strain and stress expressed on those axes may always be simply added to obtain
the total values as in small-strain/rotation theory.

The  numerical algorithms  employed in WARP3D accommodate the moderately 
large strain increments 
and rotations
which may arise routinely during implicit solution of the global equilibrium 
equations.
The resulting computational framework  based on the
corotational approach with the Green-Naghdi rate a divorces finite rotation 
effects  from integration of the stress rates (including
internal tensor state variables) to update 
the material response over a load (time) step.  Consequently, the many 
constitutive models and numerical refinements developed for small-strain plasticity 
(radial return, closest projection, backstresses for kinematic hardening, 
consistent tangent operators, dilatant 
plasticity models for continuum descriptions of void growth) 
remain unchanged. Further, the elastic moduli and yield behavior are not required
to be isotropic in this framework -- the material orientations at time $t=0$ remain
unchanged when viewed relative to the axes that rotate with a material point.


The crystal plasticity formulation and material model available in WARP3D
(Section 3.12 and Appendix J) integrate the material stress using the same
corotational approach
as the other WARP3D material models.
However, the objective stress rate becomes different from the standard Green-Naghdi rate due to
non-zero plastic voriticty.


The connection to a user-defined material routine available in WARP3D opens the
capability to incorporate a variety of Cauchy elastic, hyperelastic and 
hyperelastic-plastic constitutive models 
that employ $\mathbf{F}$ and decompositions of
$\mathbf{F}$, \eg $\mathbf{F} =\mathbf{F}^e\mathbf{F}^p$, directly in the
stress update process. The WARP3D \ti{umat} interface is fully compatible with
the UMAT interface available in Abaqus Standard. The \ti{umat} is provided with 
$\mathbf{F}_n$ and $\mathbf{F}_{n+1}$ at the material point. 
The \ti{umat} may define any number of
scalar or tensor internal variables with array space allocated/maintained by WARP3D
for convenience. A \ti{umat} employing this constitutive technology
returns the updated 
Cauchy stress tensor $\cauchy_{n+1}$ (or $\cauchy_{u({n+1})}$) and the
consistent tangent matrix for the material  to WARP3D.

The computational framework  based on the
corotational approach relies on two key assumptions
which also lead to the primary points of criticism 
(see Simo and Hughes [\ref{R:SH1998}]):
(1) additive decomposition of elastic, plastic, thermal, creep, etc. strain 
rates, and (2) use of hypoelasticity to link
instantaneous (objective) stress rates with (objective) strain rates. 
The first assumption requires that 
plastic strains (and rates) exceed significantly elastic strains (and rates). Such 
conditions are realized approximately, for example, in the study of ductile fracture in metals 
which possess large $E/\sigma_0$ ratios.  The second assumption leads to
the anomaly of non-zero strains (and stresses) at completion  of finite magnitude 
deformations over a closed path (see Example 3.7 in [\ref{R:BLME2014}] for
an simple illustration of this effect). For other materials, such as polymers, 
this ad hoc treatment of elasticity becomes unsuitable. A multiplicative 
decomposition of the deformation gradient into elastic and plastic components, 
when coupled with a proper hyperelastic treatment of material elasticity, is  
more appropriate (see [\ref{R:SH1998}, \ref{R:BLME2014}] among others).

The following sections describe the hypoelastic constitutive 
framework and the step-by-step computations in WARP3D. 
Once the kinematic transformations have eliminated rotation effects 
on rates of tensorial quantities, the stress updating procedures for each 
constitutive model are those for the conventional small-strain formulation. 
Details of the usual small-strain computations are described in 
Chapter 5 for each of the conventional material models currently available.
The reader interested in an extensive description, further numerical implementation 
details and the criticism of this finite-strain plasticity framework is referred to the 
monograph of Simo and Hughes [\ref{R:SH1998}], specifically Chapters 6 and 7.
Belytschko \etal [\ref{R:BLME2014}] also provide a discussion of hypoelasticity within
the corotational  approach and numerical algorithms.  The description here draws
heavily on the particularly lucid presentations by 
Johnson and Bammann [\ref{R:JB1984}], Taylor and Flanagan 
[\ref{R:FT1987}] and Crisfield [\ref{R:C1997}].  Additional details to arrive at 
some expressions and alternative formulations may be found
in Belytschko, \etal [\ref{R:BLME2014}] (Chaps. 3 and 5) which have an extensive 
discussion of large-rotation, finite-strain mechanics.


\subsection{Kinematics, Strain-Stress Measures}

\nid Development of the finite strain plasticity model begins with consideration 
of the deformation gradient
%
\begin{equation}\label{E:KSSMa}
\mathbf{F} = \frac{\partial \bmf{x}}{\partial \bmf{X}} = \frac{\partial x_i}{\partial X_j},\ \ det(\mathbf{F}) = J > 0
\end{equation}
%
\nid where $\bmf{X}$ denotes the Cartesian position vectors for material points 
defined on the configuration at $t=0$. Position vectors for material points at time $t$ 
are denoted $\bmf{x}$ (configuration $\region$ in Fig. \ref{fig:motion}, 
after Flanagan and Taylor [\ref{R:FT1987}]). The 
displacements of material points are thus given by $\bmf{u = x- X}$. 
The polar decomposition of $\mathbf{F}$ yields
%
\begin{equation}\label{E:KSSMb}
\mathbf{F} = \mathbf{V}\mathbf{R}=\mathbf{R}\mathbf{U} 
\end{equation}
%
where $\mathbf{V}$ and $\mathbf{U}$ are the left- and right-symmetric, 
positive definite stretch tensors, 
respectively; $\mathbf{R}$ is a proper, orthogonal rotation tensor. The principal values 
$\mathbf{V}$ and $\mathbf{U}$ have identical principal values  $\lambda_i$ which are the stretch
ratios of the deformation. These two methods for decomposing the motion  
in the neighborhood of a material point are illustrated in Fig. \ref{fig:motion}.
In the initial configuration, $\region_0$, we define an orthogonal 
reference frame at each material point such that the motion relative to these axes is 
only deformation throughout the loading history. With the $\mathbf{RU}$ 
decomposition, for example, these axes are \ti{spatial} during the 
motion from $\region_0$ to $\region_u$ -- they are not altered in orientation
by deformation of the material. 
However, during the motion from $\region_u$ to $\region$ these axes 
are \ti{material} -- they rotate with the body in a local average sense 
at each material point. 
Strain-stress tensors and their rates referred to these axes are 
sometimes said to be defined in the \ti{unrotated} configuration  
(Johnson and Bammann [\ref{R:JB1984}]).

%
\begin{figure}[h]
\begin{center}
\includegraphics[trim=0.0in 2.0in 3.5in 0.0in, clip=true,scale=0.9,angle=0]{fig_decomposition.pdf} 
\caption{{\small Fig. \thefigure\ Decomposition of motion using polar 
decomposition of deformation gradient, $\mathbf{F}$ (after [\ref{R:FT1987}]).}
\label{fig:motion}}
%
\end{center}
\end{figure}
%

Velocities of material points with respect to the configuration at time $t$  are written as
$\bmf v = \dot{\bmf{x}}$. Here $t$ is physical time in a dynamic or other time dependent simulation
(\eg creep); in quasi-static, time invariant simulations, $t$ may be considered a scalar 
parameter associated with loading intensity. The spatial gradient of the velocities is given by
%
\begin{equation}\label{E:KSSMc}
\mathbf{L} = \frac{\partial \bmf{v}}{\partial \bmf{x}} = 
\frac{\partial \bmf{v}}{\partial \bmf{X}}\frac{\partial \bmf{X}}{\partial \bmf{x}}= 
\dot{ \mathbf{F}} \mathbf{F}^{-1}\ .
\end{equation}
%

The symmetric part of $\mathbf{L}$  is the spatial rate-of-deformation tensor, denoted
$\mathbf{D}$; the skew-symmetric part, denoted $\mathbf{W}$, is the rate-of-spin 
also termed the vorticity tensor. Thus,
%
\begin{equation}\label{E:KSSMd}
\mathbf{L = D + W}
\end{equation}
%
\nid where 
%
\begin{equation}\label{E:KSSMe}
\mathbf{D} = \frac{1}{2}\left ( \mathbf{L + L}^T \right)\,;\ \ 
\mathbf{W} = \frac{1}{2}\left ( \mathbf{L - L}^T \right)\ .
\end{equation}
%
\nid Components of $\mathbf{D}$ are aligned with the fixed, global axes
but with magnitudes relative to the current material coordinates $\bmf{x}$.
$\mathbf{W}$  defines the rate-of-rotation of the principal axes of $\mathbf{D}$.
When $\mathbf{W}=\mathbf{0}$ over the loading history, the
principal values of an integrated  $\mathbf{D}$ from $0\rightarrow t$
are recognized as the logarithmic (true) strains of infinitesimal fibers 
initially aligned with the global axes. Both $\mathbf{W}$  and $\mathbf{D}$  are
\ti{instantaneous} rates and have no sense of the deformation history. 

Using the $\mathbf{RU}$ decomposition of $\mathbf{F}$, the spatial velocity gradient
$\mathbf{L}$ may also be written in the form
%
\begin{equation}\label{E:KSSMf}
\mathbf{L} = \dot{\mathbf{R}}\mathbf{R}^T + 
\mathbf{R}\dot{\mathbf{U}} \mathbf{U}^{-1}\mathbf{R}^T
\end{equation}
%
\nid which makes use of these relations
 %
\begin{equation}\label{E:KSSMg}
\dot{\mathbf{F}}= \mathbf{R} \dot{\mathbf{U}} + \dot{\mathbf{R}} \mathbf{U}\ ,
\end{equation}
%
 %
\begin{equation}\label{E:KSSMh}
\mathbf{F}^{-1}= \left ( \mathbf{RU} \right )^{-1} = \mathbf{U}^{-1}\mathbf{R}^{-1}= 
\mathbf{U}^{-1}\mathbf{R}^T\ .
\end{equation}
%
  
The first term in the above definition for $\mathbf{L}$ is the rate of rigid-body
rotation at a material point and is denoted  $\mathbf{\Omega}$  (see Dienes [\ref{R:D1979}].
Note that some authors, \eg Crisfield [\ref{R:C1997}], reverse the notation
and use  $\mathbf{\Omega}$ for  $\mathbf{W}$).
The spin rate $\mathbf{W}$ and $\mathbf{\Omega}$ are identical when the 
principal axes of $\mathbf{D}$ coincide with principal axes of the current stretch $\mathbf{V}$  
(this observation plays a key role later in development of a linearized tangent operator).
Simple extension and pure rotation satisfy this condition. The symmetric part of the second term 
in $\mathbf{L}$ is called the \ti{unrotated} deformation rate tensor 
(sometimes the rotation neutralized or corotational deformation rate) and is denoted $\mathbf{d}$,
 %
\begin{equation}\label{E:KSSMi}
\mathbf{d} = \frac{1}{2} \left ( \dot {\mathbf{U}}\mathbf{U}^{-1} +  \mathbf{U}^{-1}\dot {\mathbf{U}}\right )
\end{equation}
%

The unrotated rate of deformation defines a material strain rate with components
relative to the 
orthogonal reference frame indicated on configuration $\region$ in Fig. 1.9.1.

Using the orthogonality property of $\mathbf{R}$ that $d(\mathbf{R}^T\mathbf{R})/dt=0$
%
\begin{equation}\label{E:KSSMj}
\mathbf{R}^T\dot {\mathbf{R}} + \dot {\mathbf{R}}^T \mathbf{R}\equiv  \mathbf{0}
\end{equation}
%
\nid the unrotated deformation rate may be expressed in the simpler form as
%
\begin{equation}\label{E:KSSMk}
\mathbf{d} = \mathbf{R}^T \mathbf{D}\mathbf{R}\ .
\end{equation}
%

The principle of virtual displacements (Section 1.4) demonstrates that the 
spatial rate of deformation,  $\mathbf{D}$, and the symmetric Cauchy (true) stress, $\cauchy$, 
are work conjugate in the sense that work rate per unit volume in the current configuration,
$\region$ of Fig. \ref{fig:motion},
is given by $\cauchy:\mathbf{D}$ ($\sigma_{ij}D_{ij}$) [components of  $\cauchy$ and $\mathbf{D}$
align with the global axes]. The work rate in the rotated system (axes shown in $\region$)
must be identical. Then, using the above expression to write $\mathbf{D}$ in terms of $\mathbf{d}$
yields the relationship between $\cauchy$ and $\cauchyu$ as
%
\begin{equation}\label{E:KSSMl}
\cauchyu = \mathbf{R}^T \cauchy \mathbf{R}
\end{equation}
%
\nid where $\cauchyu$ is termed the \ti{unrotated}  or corotational Cauchy stress, \ie
the unrotated Cauchy stress is simply the usual Cauchy stress transformed to the
orthogonal, rotated axes shown in configuration $\region$, where the rotation tensor
$\mathbf{R}$ may vary from material point to material point.

%%%%%%%%%%%%%
\subsection{Elastic-Plastic Decomposition}
%%%%%%%%%%%%%%%%%%%%%%%%
Further developments require kinematic decomposition of the total strain rate $\mathbf{d}$
into elastic and plastic components. The multiplicative decomposition of the deformation gradient
%
\begin{equation}\label{E:EPDa}
\mathbf{F} = \mathbf{F}^e\mathbf{F}^p
\end{equation}
%
\nid appears most compatible with the physical basis of elastic-plastic deformation in 
crystalline metals (see, for example, Lee [ \ref{R:L1969}] and Asaro [\ref{R:A1984}]). $\mathbf{F}^p$
represents plastic 
flow (dislocations) while $\mathbf{F}^e$ represents lattice distortion; rigid rotation 
of the material structure may be considered in either term. Substitution of this 
decomposition into the spatial rate of the displacement gradient Eq. (\ref{E:KSSMc}) yields
%
\begin{equation}\label{E:EPDb}
\mathbf{L} = \dot{\mathbf{F}}^e\mathbf{F}^{-e} +
\mathbf{F}^e\dot{\mathbf{F}^p}\mathbf{F}^{-p} \mathbf{F}^{-e} =
\mathbf{L}^e + \mathbf{F}^e \mathbf{L}^p \mathbf{F}^{-e} \ .
\end{equation}
%
We now impose the restriction that elastic strains remain vanishingly small 
compared to the unrecoverable plastic strains -- a deformation mode found
in ductile metals having an elastic modulus orders of magnitude 
greater than the flow stress. Consequently, $\mathbf{F}^p$ and 
$\mathbf{F}^e$ are uniquely 
determined by unloading from a plastic state. This considerably 
simplifies the above expression and permits separate 
treatment of material elasticity and plasticity. Using the left polar 
decomposition and writing the stretch as the product of elastic 
and plastic parts yields
%
\begin{equation}\label{E:EPDc}
\mathbf{F} = \mathbf{F}^e \mathbf{F}^p =
\mathbf{V}^e\mathbf{V}^p\mathbf{R}\ .
\end{equation}
%
\nid Identifying the elastic deformation as
%
\begin{equation}\label{E:EPDd}
\mathbf{F}^e = \mathbf{V}^e
\end{equation}
%
\nid and using the small elastic strain assumption, we have
%
\begin{equation}\label{E:EPDe}
\mathbf{F}^e = \mathbf{I}+\mathbf{e}^e\approx  \mathbf{I} .
\end{equation}
%
\nid Consequently, the expression for $\mathbf{L}$ is approximated by
%
\begin{equation}\label{E:EPDf}
\mathbf{L}\approx \mathbf{L}^e+\mathbf{L}^p\ .
\end{equation}
%
\nid As in Eq. (\ref{E:KSSMe}), the symmetric part of this approximation 
for $\mathbf{L}$ is taken as $\mathbf{D}$ with the result that
%
\begin{equation}\label{E:EPDg}
\mathbf{D}\approx \mathbf{D}^e+\mathbf{D}^p\  .
\end{equation}
%

Given the restriction of vanishingly small elastic strains, the multiplicative 
decomposition of the deformation gradient in Eq. (\ref{E:EPDa}) leads to the 
familiar additive decomposition of the spatial deformation rate $\mathbf{D}$ 
into elastic and
plastic components. The transformation of $\mathbf{D}$ to the unrotated configuration 
using Eq. (\ref{E:KSSMk}) provides the decomposition scheme needed for $\mathbf{d}$ as
%
\begin{equation}\label{E:EPDh}
\mathbf{d} = \mathbf{R}^T\left ( \mathbf{D}^e+\mathbf{D}^p \right )\mathbf{R}^T
= \mathbf{d}^e+\mathbf{d}^p \ .
\end{equation}
%

Once the above transformation of elastic and plastic strain rates onto the 
unrotated configuration is accomplished, the remaining steps in development 
of the finite-strain plasticity theory are identical to those for classical 
small-strain theory.


\subsection{Stress Update}

\nid The global solution is advanced from time (load step) $t_n\rightarrow t_{n+1}$
using an incremental-iterative
Newton method. Iterations at $t_{n+1}$ to remove unbalanced 
nodal forces are conducted under
fixed external loading and no change in the prescribed displacements.
Each such iteration, denoted $i$, provides a 
revised estimate for the total 
displacements at $t_{n+1}$, denoted $\bmf u_{n+1}^{(i)}$. Converged 
displacements at $t_n$ are denoted $\bmf u_n$. 
Following Pinsky, Ortiz and Pister [\ref{R:POP1984}] a 
mid-increment scheme is adopted in which 
deformation rates are evaluated on the intermediate configuration 
at $(1-\gamma)\bmf u_n + \gamma \bmf u_{n+1}^{(i)}$. The choice
of $\gamma = 1/2$ represents a specific form of the generalized 
trapezoidal rule that is unconditionally 
stable and second-order accurate. Key and Krieg [\ref{R:KK1982}] 
have demonstrated the optimality 
of the mid-point configuration for integrating the rate of deformation 
and the resulting 
correspondence with logarithmic strain (for pure stretch conditions).
The following sections describe the computational processes performed at each 
material (integration) point to update stresses from  $t_n\rightarrow t_{n+1}$. A brief 
discussion of the procedure 
to compute the polar decomposition of the deformation gradient is also provided.

Computational details for stress updating of the cohesive materials employed with 
interface elements and the crystal plasticity material model have variations on these
steps -- see Section 3.8 for cohesive materials with response defined 
by a traction-separation
relationship and Section 3.12 for the crystal plasticity model.

\nid Key steps to update stresses are
\small
\squishlist
\item \ti{Step 1}. Compute the deformation gradient at $n+1/2$ and $n+1$
%
\begin{equation}\label{E:SUa}
\mathbf{F}_{n+1/2}^{(i)} = \frac {\partial \left ( \bmf X + \bmf u_{n+1/2}^{(i)}\right )}  {\partial  \bmf X }\ ;
\end{equation}
\begin{equation}\label{E:SUb}
\mathbf{F}_{n+1}^{(i)} = \frac {\partial \left ( \bmf X + \bmf u_{n+1}^{(i)}\right )}  {\partial  \bmf X }\ .
\end{equation}
%
\item \ti{Step 2}. Compute polar decompositions  at $n+1/2$ and $n+1$
%
\begin{equation}\label{E:SUc}
\mathbf{F}_{n+1/2}^{(i)} = \mathbf{R}_{n+1/2}^{(i)} \cdot \mathbf{U}_{n+1/2}^{(i)}\ ;
\end{equation}
\begin{equation}\label{E:SUd}
\mathbf{F}_{n+1}^{(i)} = \mathbf{R}_{n+1}^{(i)} \cdot \mathbf{U}_{n+1}^{(i)}\  .
\end{equation}
%
\item \ti{Step 3}. Compute $i^{th}$ estimate of the spatial deformation increment
$\Delta \mathbf{D} =\mathbf{D}_{n+1/2}^{(i)}\,\Delta t$ over the step. Use the conventional small-displacement
$\mathbf{B}$ matrix for the element but with updated nodal coordinates at $n+1/2$
%
\begin{equation}\label{E:SUe}
\Delta \bmf \vareps^{(i)} = \mathbf{B}_{n+1/2}^{(i)} \left ( \bmf u_{n+1}^{(i)} -  \bmf u_{n} \right )
\end{equation}
%
\nid where $\Delta \bmf \vareps^{(i)}$ is the $6 \times 1$ vector form of the symmetric
tensor $\Delta \mathbf{D}$. Construct  $\Delta \mathbf{D}$ from the computed
$\Delta \bmf \vareps^{(i)}$ for use in the next step. Shear strain terms in the vector form
$\Delta \bmf \vareps$ are
twice the values in the tensor form $\Delta \mathbf{D}$.

This procedure is simpler and less computationally expensive than the
formal method described in Eqs. (\ref{E:KSSMc},\ref{E:KSSMe})
Further, this approach provides a straightforward method to utilize the 
$\overline{\mathbf{B}}$ formulation (to replace the conventional
$\mathbf{B}$ matrix) for finite strains thereby 
reducing volumetric locking in the linear-displacement
8-node isoparametric element.
\item \ti{Step 4}. Transform the increment of spatial deformation to the increment of 
deformation in the \ti{unrotated} configuration
%
\begin{equation}\label{E:SUf}
\Delta \mathbf{d}^{(i)} = \mathbf{R}_{n+1/2}^{(i)T} \cdot \Delta 
\mathbf{D} \cdot  \mathbf{R}_{n+1/2}^{(i)}\ .
\end{equation}
%
\item \ti{Step 5}. Terms of the symmetric tensor $\Delta \mathbf{d}^{(i)}$
define the strain increments for use by the conventional small-strain constitutive
models (subtract thermal, initial, etc. strain increments over the step to form the
increment of mechanical strain). Invoke the small-strain material routine to 
provide the \ti{unrotated} Cauchy stress, $\cauchyu$, at $t_{n+1}$
%
\begin{equation}\label{E:SUg}
{\cauchyu}_{(n+1)}^{(i)} \leftarrow \mathcal{C} \left (  {\cauchyu}_{(n)}, H_n^j, \mathbf{q}_n^k,
\Delta  \mathbf{d} \right ) \ .
\end{equation}
%
\nid where $\mathcal C$ denotes the integration process (\eg elastic-predictor radial return,
nearest point return, subincrements, etc.). The integration process requires the converged material
state at $n$: the unrotated Cauchy stress ${\cauchyu}_{(n)}$, a set of scalar
state variables $H_n^j$ ($j=1, 2, \dots$), and possibly a set of tensor state variables  $\mathbf{q}_n^k$
($k=1, 2, \dots$).
\item \ti{Step 6}. The updated (spatial) Cauchy stress at $n+1$  is then given by
%
\begin{equation}\label{E:SUh}
\cauchy_{n+1}^{(i)} = \mathbf{R}_{n+1}\cdot  {\cauchyu}_{(n+1)}^{(i)}\cdot\mathbf{R}_{n+1}^T
\end{equation}
%
\squishend
\normalsize

\nid This stress updating procedure offers a number of advantages:  
\small
\squishlist
\item routines to perform
the computations represented above by  the operator $\mathcal{C}$ may be written without
consideration of the finite strain-rotation setting, 
\item tensor state variables (\eg backstresses),
$\mathbf{q}_n^k$, are defined and maintained on the \ti{unrotated}
configuration and are thus unaffected by the finite strain-rotation setting,  
\item issues
with objective stress rate become quite simple -- the new, (total) Cauchy stress at $n+1$ is found from
rotation of the new (total) unrotated Cauchy stress at $n+1$, and 
\item
material elasticity and thermal expansion coefficients
may be anisotropic with values defined on $\bmf X$ and used without
change in  $\mathcal{C}$ operations performed on the \ti{unrotated} configuration, \ie
the properties referenced to the undeformed axes in $\region_0$ remain
unchanged with reference to the orthogonal axes shown on $\region$ in Fig. \ref{fig:motion}.
%
\squishend
\normalsize

\nid{\bf{\ti{Polar Decomposition}}}

\nid The polar decomposition $\mathbf{F=RU}$ is a key step in the stress updating algorithm 
and must be performed twice for each integration point for each stress update, \ie
 at $t_{n+1/2}$ and $t_n$\footnote{for processing blocks of elements
 that use the \ti{umat} material, $\mathbf{F}_n$ is also computed and
 made available to the \ti{umat}}. For  their explicit code, 
 Flanagan and Taylor [\ref{R:FT1987}] developed an algorithm for the integration of 
 $\dot{\mathbf{R} }= \mathbf{\Omega}\mathbf{R}$ that 
avoids the polar decomposition and that
maintains orthogonality of  $\mathbf{R}$  for the small displacement 
increments characteristic of explicit solutions. For the 
WARP3D stress update, the polar decompositions are computed
directly  to remove potential issues with larger displacement increments
experienced in implicit solutions.
 
\small
\squishlist
\item \ti{Step 1}. Compute the right Cauchy-Green tensor
%
\begin{equation}\label{E:PDa}
\mathbf{C} = \mathbf{F}^T \mathbf{F}
\end{equation}
%
\nid and the square
%
\begin{equation}\label{E:PDb}
\mathbf{C}^2 = \mathbf{C}^T \mathbf{C}
\end{equation}
%
\nid where only the upper-triangular form of the symmetric products (6 terms) are
computed and stored.
\item \ti{Step 2}. Compute the eigenvalues $\lambda_1^2$,
$\lambda_2^2$, and $\lambda_3^2$ of $\mathbf{C}$. A Jacobi 
transformation procedure specifically coded for $3\times3$ matrices extracts 
the eigenvalues. Loops are unrolled with vector-style code and
transcendental function evaluations avoided to compute
the eigenvalues at an integration point for all elements in the block
(multiple blocks of elements are processed concurrently over threads).
\item \ti{Step 3}. Compute invariants of $\mathbf{U}$ and the $det (\mathbf{F})$
%
\begin{equation}\label{E:PDc}
I_U = \lambda_1  + \lambda_2 + \lambda_3
\end{equation}
%
\begin{equation}\label{E:PDd}
II_U = \lambda_1 \lambda_2  + \lambda_2\lambda_3  + \lambda_1\lambda_3 
\end{equation}
%
%
\begin{equation}\label{E:PDe}
III_U = \lambda_1 \lambda_2\lambda_3 = J = det( \mathbf{F} )
\end{equation}
%
\item \ti{Step 4}. Form the upper triangle of the symmetric, right stretch, $\mathbf{U}$, and its 
symmetric inverse, $\mathbf{U}^{-1}$ (see Hoger and Carlson [\ref{R:HC1984}])
%
\begin{equation}\label{E:PDf}
\mathbf{U} = \beta_1 \left ( \beta_2 \mathbf{I}+  \beta_3  \mathbf{C}  - \mathbf{C}^2 \right )
\end{equation}
%
\nid where $\mathbf{I}$ denotes a unit tensor with the $\beta$ coefficients defined by
%
\begin{equation}\label{E:PDg}
\beta_1 = 1/\left ( I_UII_U - III_U\right),\ \beta_2 = I_IIII_U, \ \beta_3 = I_U^2 - {II}_U\ .
\end{equation}
%
\nid Similarly, $\mathbf{U}^{-1}$  is formed directly as
%
\begin{equation}\label{E:PDh}
\mathbf{U}^{-1} = \gamma_1 \left ( \gamma_2  \mathbf{I} + \gamma_3 \mathbf{C}+\gamma_4 \mathbf{C}^2 \right )
\end{equation}
%
\nid where the $\gamma$ coefficients are given by
%
\begin{equation}\label{E:PDi}
\gamma_1 = 1/III_U\left ( I_U II_U - III_U\right),\ \gamma_2 = I_U II_U^2 -III_U \left ( I_U^2 + II_U\right ),
\end{equation}
%
\begin{equation}\label{E:PDj}
\gamma_3 = - III_U - I_U \left ( I_U^2 - 2 II_U \right ), \ \gamma_4 = I_U 
\end{equation}
%
\item \ti{Step 5}. Form $\mathbf{R}$ as the product
\begin{equation}\label{E:PDk}
\mathbf{R}= \mathbf{F}\mathbf{U}^{-1}\ .
\end{equation}


\squishend
\normalsize



\subsection{Stress Rates and Consistent Tangent Operators}

\nid The global solution is advanced from time (load) $t_n\rightarrow t_{n+1}$
using an incremental-iterative Newton procedure which requires the linearized, rate 
form of the element nodal forces (repeated here from Section 1.8.6)
%
\begin{equation}\label{E:SRCTa}
 \dot{ \bmf{I}}_{e(3ne\times1)} = 
 \int_{Ve} \mathbf{B}^T \dot{\bmf{\sigma}} \, dV_e  +  \int_{Ve} \dot{\mathbf{B}}^T \bmf{\sigma} \, dV_e =
\left ( \left [ \mathbf{K}_{mat} \right ]_e+ \left [ \mathbf{K}_{geo} \right ]_e \right )  \dot{\bmf{u}}_e
= [ \mathbf{K}_T ]_e   \dot{\bmf{u}}_e
\end{equation}
%
where the $(\dot\,)$ derivative (rate) refers to real time for dynamic and other time
dependent (\eg creep) analyses or to a load-like
parameter for quasi-static analyses. The first term leads to the \ti{material} stiffness with the second term
defining the \ti{geometric} (or \ti{initial}) stiffness matrix. The second term contains the current 
($i^{th}$) estimate during global iterations for the Cauchy stress
$\bmf{\sigma}$ at $t_{n+1}$ computed by the stress-updating
procedure described in the previous subsections. Equation (1.8.30) provides the final form
of the element geometric stiffness which involves no stress rates. Also recall that the stiffness term from
$\dot J$ is neglected from the assumption of incompressibility. 

The  $\mathbf{B}$ matrix appearing above, see also Eq. (\ref{E:SUe}), has the usual linear-displacement
form where derivatives are taken with respect to
the current ($i^{th}$) estimate of the deformed element geometry $\bmf{x}_{n+1}^{(i)}$.  Thus
terms of the product $\mathbf{B}\dot{\bmf{u}}_e$ correspond to the 
rate deformation tensor $\mathbf{D}_{n+1}^{(i)}$ via 
Eqs. (\ref{E:KSSMc}-\ref{E:KSSMe}).
Integrations in Eq. (\ref{E:SRCTa}) occur over the current estimate of the
deformed element configuration $V_{e(n+1)}^{(i)}$. For simplicity, the $n+1$ and $i^{th}$ current 
estimate notations are understood in the remaining discussion of this section.

For small-strain formulations with all computations based on the undeformed element 
geometry, completion of the material stiffness for the element is straightforward:
(1) the stress rate couples linearly to the strain rate through 
a \ti{consistent} constitutive matrix $\mathbf{E}_{6 \times 6}$ that reflects small-strain, nonlinear behavior of the material,
(2) the linear strain rate is $\dot{\bmf{\vareps}} = \mathbf{B} \,\dot{\bmf{u}}_e$. Then with
integrations over the undeformed element geometry,
%
\begin{equation}\label{E:SRCTb}
 \left [ \mathbf{K}_{mat} \right ]_e =  
 \int_{V0} \mathbf{B}^T\, \mathbf{E}\, \mathbf{B} \, dV_0\ .
\end{equation}
%
\nid For small-strains, \ti{consistency} here
implies that stress increments during numerical computations 
predicted by the constitutive (tangent) matrix 
$\mathbf{E}_{6 \times 6}\ \Delta \bmf{\vareps}_{6 \times 1}$ are equivalent to the 
stress increments determined by the stress-updating procedures in the material computations, \ie
%
\begin{equation}\label{E:SRCTc}
 \Delta \bmf{s}_{6 \times 1} = \mathbf{E}_{6 \times 6}\ \Delta \bmf{\vareps}_{6 \times 1}
=\int_{t_n}^{t_{n+1}} \dot{\bmf{s}}(\bmf{s},\bmf{\vareps}, \bar H, \mathbf{q},\dots) \, dt
\end{equation}
%
\nid where $\bmf{s}$ denotes stress in the small-strain formulation to prevent confusion with the
Cauchy stress for finite strain/rotation conditions. Since $\dot{\bmf{s}}$ is (usually) nonlinear with
strain over $t_n \rightarrow t_{n+1}$, the \ti{consistent} $\mathbf{E}$ defines a \ti{secant} rather
than a tangent relationship (of course the secant becomes the tangent as $\Delta \bmf{\vareps}
\rightarrow d\bmf{\vareps}$ with vanishingly small numerical time steps). Use of the consistent
$\mathbf{E}$ proves essential to maintain strong convergence rates of the global Newton iterations.
For metal plasticity, this was recognized in the 1970s and early 1980s with special forms constructed
for a so-called \ti{secant}  $\mathbf{E}$ (see [\ref{R:NV1984}] and others).
Simo and Taylor [\ref{R:ST1985}]
first developed a general, formal procedure to construct the newly named \ti{consistent} tangents for 
various material hardening models, yield surfaces, etc. In the finite-strain/rotation formulations
described next, the (small-strain based) consistent $\mathbf{E}$ computed on the unrotated
configuration continues to play a 
key role in maintaining strong convergence rates of the global Newton iterations.

Returning to the the finite-strain/rotation case  
%
\begin{equation}\label{E:SRCTd}
 \left [ \mathbf{K}_{mat} \right ]_e= \int_{Ve} \mathbf{B}^T \dot{\bmf{\sigma}} \, dV_e
\end{equation}
%
\nid where $\dot{\bmf{\sigma}}$ is the time rate-of-change of the Cauchy stress.
The Cauchy stress and its rate $\dot{\bmf{\sigma}}$ have components relative to the fixed,
global system. Consequently,  motion  $\dot{\bmf{u}}$ in the neighborhood of
a material point that corresponds to only 
rotation leads to a non-zero  $\dot{\bmf{\sigma}}$. For example,
consider a material point at time $t_n$ that
has $\sigma_{xx}$ as the only non-zero component of
the Cauchy stress. A rotation of material at the point without deformation
from $t_n\rightarrow t_{n+1}$  can readily cause
a full tensor $\bmf{\sigma}_{n+1}$  -- in this case the unrotated Cauchy stress 
$\bmf{\sigma}_{u_{(n+1)}}$ has only the non-zero $\sigma_{xx}$ on the orthogonal axes shown in
configuration $\region$ in Fig. \ref{fig:motion}. Then  $\bmf{\sigma}_{n+1}$ is given by 
Eq. (\ref{E:SUh}) where the
rotation matrix $\mathbf{R}_{n+1}$ is found from the polar decomposition of the deformed
configuration at $n+1$. Further details of this example with figures are given 
in Sections 3.4.3, 3.7 of  [\ref{R:BLME2014}] and Section 10.2 in [\ref{R:C1997}].

This example illustrates the difficulty in proceeding to find 
an expression comparable to Eq. (\ref{E:SRCTb}) for the finite-strains/rotations. If the motion over
$t_n\rightarrow t_{n+1}$ is only rotation, then both $\mathbf{D}$ and 
$\mathbf{d}$ vanish at the mid-point of the time step -- 
$\Delta \bmf{\vareps}_{n+1/2}=\bmf{0}$ from Eq. (1.8.26) and
$\mathbf{L}= \mathbf{W}=\dot{\mathbf{R}}\mathbf{R}^T$ since $\dot{\mathbf{U}}=\mathbf{0}$
from Eqs. (\ref{E:KSSMe}, \ref{E:KSSMf}). 
Consequently,  from Eq. (\ref{E:SUg})  $\dot{\bmf{\sigma}}_u\equiv\bmf{0}$
since $\mathbf{d}=\mathbf{0}$ but  $\dot{\bmf{\sigma}}\ne\bmf{0}$ from the rotational motion
of the material point, \ie it is clear that 
$\dot{\bmf{\sigma}}\ne\mathbf{R}\,\dot{\bmf{\sigma}}_u\,\mathbf{R}^T$.

The relationship between $\dot{\bmf{\sigma}}$ and $\dot{\bmf{\sigma}}_u$ is required.  Taking
time derivatives of Eq. (\ref{E:KSSMl}), we have
%
\begin{equation}\label{E:SRCTe}
\dot {\bmf{\sigma}}_u = \dotR^T \bmf{\sigma}\mathbf{R} + \mathbf{R}^T\dot{\bmf{\sigma}}\mathbf{R} +
\mathbf{R}^T\bmf{\sigma}\dotR
\end{equation}
%
\nid Now pre- and post- multiply both sides by $\mathbf{R}$ and $\mathbf{R}^T$
%
\begin{equation}\label{E:SRCTf}
\mathbf{R}\dot {\bmf{\sigma}}_u \mathbf{R}^T= \mathbf{R}\dotR^T \bmf{\sigma}\mathbf{R}\mathbf{R}^T +\mathbf{R} \mathbf{R}^T\dot{\bmf{\sigma}}\mathbf{R} \mathbf{R}^T+
\mathbf{R}\mathbf{R}^T\bmf{\sigma}\dotR\mathbf{R}^T
\end{equation}
\nid and simplify using $\mathbf{R}\mathbf{R}^T=\mathbf{I}$, $\mathbf{R}^T=-\mathbf{R}$, 
$\mathbf{\Omega}=\dotR\mathbf{R}^T$ and $\mathbf{R}\dotR^T=
-\dotR\mathbf{R}^T =-\mathbf{\Omega}$ to find
%
\begin{equation}\label{E:SRCTg}
\mathbf{R}\dot {\bmf{\sigma}}_u \mathbf{R}^T= -\mathbf{\Omega}\bmf{\sigma} +\dot{\bmf{\sigma}}+
\bmf{\sigma}\mathbf{\Omega}\ .
\end{equation}
\nid When the material point motion is rigid rotation, $\dot {\bmf{\sigma}}_u =\mathbf{0}$ with the
rate of Cauchy stress given by  $\dot{\bmf{\sigma}}=\mathbf{\Omega}\bmf{\sigma} -
\bmf{\sigma}\mathbf{\Omega}$. Including deformation and rotation,  we have
%
\begin{equation}\label{E:SRCTh}
\dot{\bmf{\sigma}}=\underbrace{\mathbf{\Omega}\bmf{\sigma} -
	\bmf{\sigma}\mathbf{\Omega}}_{\text{rotation}} +
	\underbrace{\mathbf{R}\dot {\bmf{\sigma}}_u \mathbf{R}^T}_{\text{deformation}}
	=\mathbf{\Omega}\bmf{\sigma} -
	\bmf{\sigma}\mathbf{\Omega} +\bmf{\sigma}^{\nabla GN}
\end{equation}
%
\nid where the last term  is called the \ti{Green-Naghdi} stress rate. In tensor form we write
$\dot {\bmf{\sigma}}_u= \overline{\mathbf{E}}_u:\mathbf{d}$, where 
$\overline{\mathbf{E}}_u$ is the ($4^{th}$)
order consistent tangent in the
unrotated system. The rate of Cauchy stress needed in Eq. (\ref{E:SRCTd}),
expressed on the fixed-global axes, thus consists of a
rotation part and a material (deformation) part. 

The frequently used Jaumann rate of Cauchy stress follows from the above by
adopting a corotational coordinate system at $t$ coincident 
with the fixed-global axes ($\mathbf{R}=\mathbf{I}$)
and rotating with a  spin $\mathbf{W}=\dotR$. Then
%
\begin{equation}\label{E:SRCTi}
\dot{\bmf{\sigma}}=\underbrace{\mathbf{W}\bmf{\sigma} -
	\bmf{\sigma}\mathbf{W}}_{\text{rotation}} +
	\underbrace{\overline{\mathbf{E}}_D:\mathbf{D}}_{\text{deformation}}
\end{equation}
%
\nid where the two above expressions lead to different rates of $\dot{\bmf{\sigma}}$
unless certain conditions are met and, consequently, different evolutions of Cauchy stress
over the loading history. Finite simple shear has become the classic illustration of 
the differences as described in 
Example 3.13 of [\ref{R:BLME2014}] (and many earlier papers) when 
$\overline{\mathbf{E}}_D\equiv \overline{\mathbf{E}}_u$.


In WARP3D we use Eq. (\ref{E:SRCTh}) to form the material stiffness but will need to
adopt some concepts from Eq. (\ref{E:SRCTi}) to develop the required form 
%
\begin{equation}\label{E:SRCTj}
 \left [ \mathbf{K}_{mat} \right ]_e =  
 \int_{Ve} \mathbf{B}^T\, \tilde {\mathbf{E}}\, \mathbf{B} \, dV_e\ .
\end{equation}
%
\nid where  $\tilde {\mathbf{E}}$ contains $\mathbf{E}_u$
and contributions from the rotation terms to $\dot{\bmf{\sigma}}$.

To begin, let $\dot{\bmf{\vareps}}_{6 \times 1}$ denote the vector form of $\mathbf{D}$
where the $\mathbf{B}_{6 \times 3n_e}$ matrix is used to construct $\dot{\bmf{\vareps}}$
directly from
%
\begin{equation}\label{E:SRCTk}
\dot{\bmf{\vareps}} = \mathbf{B} \,\dot{\bmf{u}}_e\ .
\end{equation}
%
Here $\dot{\bmf{\vareps}} ^T =
\left\{\vareps_{xx},\vareps_{yy},\vareps_{zz}, \gamma_{xy}, \gamma_{yz}, \gamma_{xz},\right\}$.
The shear terms $\gamma_{ij} = 2 D_{ij}$.
Then the vector-matrix form of Eq. (\ref{E:KSSMk}),
$\mathbf{d} = \mathbf{R}^T \mathbf{D}\mathbf{R}$,
can be written
%
\begin{equation}\label{E:SRCTl}
\bmf{d}_{6 \times 1} =  \mathbf{T}_{6 \times 6}\,\dot{\bmf{\vareps}}_{6 \times 1}
\end{equation}
%
\nid where $\bmf{d}$ is the vector form of the symmetric tensor $\mathbf{d}$ and the transformation
matrix $\mathbf{T}$ is constructed from terms of $\mathbf{R}_{n+1}$ as
%
\scriptsize
\begin{equation}\label{E:SRCTm}
 \mathbf{T} = \begin{bmatrix} [1.3]
R_{11}^2 & R_{21}^2 & R_{31}^2 &R_{11}R_{21}& R_{21}R_{31} & R_{11}R_{31} \\
 R_{13}^2 & R_{22}^2 &R_{32}^2&R_{12}R_{22} &R_{32}R_{22} &R_{12}R_{32} \\
 R_{13}^2 &R_{23}^2 &R_{33}^2 &R_{13}R_{23}  & R_{23}R_{33}  & R_{13}R_{33}  \\
  2R_{11}R_{12} & 2R_{21}R_{22} & 2R_{31}R_{32} & \left (  R_{11}R_{22}+R_{21}R_{12} \right ) &
           \left (  R_{21}R_{32}+R_{22}R_{31} \right )    & \left (  R_{11}R_{32}+R_{12}R_{31} \right ) \\ 
2 R_{13}R_{12}  & 2 R_{23}R_{22}  & 2 R_{33}R_{32} & \left (  R_{12}R_{23}+R_{13}R_{22} \right )   & 
       \left (  R_{22}R_{33}+R_{32}R_{23} \right ) & \left (  R_{12}R_{33}+R_{13}R_{32} \right )  \\ 
2R_{11}R_{13}  & 2R_{21}R_{23}   & 2R_{31}R_{33}&\left (  R_{11}R_{23}+R_{13}R_{21} \right )&
         \left (  R_{21}R_{33}+R_{23}R_{31} \right ) & \left (  R_{11}R_{33}+R_{31}R_{13} \right ) \\
\end{bmatrix}
\end{equation}
\normalsize
%
\nid The Green-Naghdi stress rate is constructed as
%
\begin{equation}\label{E:SRCTn}
\dot{\bmf{\sigma}}_{u( 6 \times 1)} =  \mathbf{E}_{6\times 6}\,\bmf{d}_{6\times 1}=\mathbf{E}\,\mathbf{T}\, \dot{\bmf{\vareps}}\ ,
\end{equation}
%
then multiplying through by $\mathbf{T}^T$
%
\begin{equation}\label{E:SRCTn}
\underbrace{\mathbf{T}^T\,\dot{\bmf{\sigma}}_{u( 6 \times 1)} }_{\text{= Greeen-Naghdi rate}}= 
\bmf{\sigma}^{\nabla GN}=\mathbf{T}^T\, \mathbf{E}\,\mathbf{T}\, \dot{\bmf{\vareps}}=
 \mathbf{E}^*\, \dot{\bmf{\vareps}}
\end{equation}
%
\nid where the left side is the matrix-vector form of the tensor  
$\mathbf{R}\,\dot {\bmf{\sigma}}_u \,\mathbf{R}^T$. Here, $\mathbf{E}$ is the
consistent tangent for the material state referenced to the orthogonal axes
shown on configuration $\region$ in Fig. \ref{fig:motion}. It is the conventional small-strain 
(hypoelastic) constitutive relationship since all rotational effects are imposed outside the construction
of  $\mathbf{E}$. The $\bmf{\sigma}^{\nabla GN}$ rate retains symmetry of $\mathbf{E}$ if present through the
symmetry preserving coordinate transformation above. And as discussed earlier, the linear-elastic
term of $\mathbf{E}$ may be 
anisotropic since the material orientation relative to the orthogonal axes shown in $\region$ is
unchanged from that in $\region_0$.

To complete the material stiffness, a matrix-vector expression must be found such that
%
\begin{equation}\label{E:SRCTo}
\mathbf{\Omega}\bmf{\sigma} -
	\bmf{\sigma}\mathbf{\Omega}=\dotR \mathbf{R}^T\bmf{\sigma} -
	\bmf{\sigma}\dotR \mathbf{R}^T \quad \Rightarrow \quad 
	-\mathbf{Q}_{6\times 6}\,  \dot{\bmf{\vareps}}_{6\times 1}
\end{equation}
%
\nid to enable writing the matrix-vector form of Eq. (\ref{E:SRCTh}) as
%
\begin{equation}\label{E:SRCTp}
\dot{\bmf{\sigma}}_{6\times 1}= \left [  \mathbf{E}^* -\mathbf{Q} \right ]_{6\times 6} 
\dot{\bmf{\vareps}}_{6\times 1}
\end{equation}
%
\nid and then the material stiffness will have the final, familiar form Eq. (\ref{E:SRCTd})
%
\begin{equation}\label{E:SRCTq}
 \left [ \mathbf{K}_{mat} \right ]_{e(3ne \times 3ne)}= \int_{Ve} \mathbf{B}^T_{3ne\times 6} \,
 \left [  \mathbf{E}^* -\mathbf{Q} \right ]_{6\times 6} \,\mathbf{B}_{6\times 3ne}\, dV_e
\end{equation}
%
\nid where it is also desirable that $\mathbf{Q}$ be symmetric. Use of $-\mathbf{Q}$ above
is one of simple convenience as becomes apparent below.

Construction of an exact form of $\mathbf{Q}$ is extraordinarily complex and leads to a
non-symmetric matrix. This is discussed by Simo and Hughes  [\ref{R:SH1998}] (pg. 273) and
Belytschko, \etal [\ref{R:BLME2014}] (pg. 244), both referring to the earlier work of 
Mehrabadi and Nemat-Nasser [\ref{R:MN1987}].

To reach the desired form of a symmetric $\mathbf{Q}$ for use in Eq. (\ref{E:SRCTp}),
we assume: (1) the incremental deformation rate is incompressible, \ie $trace(\mathbf{D})=$$D_{11} + D_{22} +
D_{33}$$=\dot{\vareps}_{1}+\dot{\vareps}_{2}+\dot{\vareps}_{3}=0$ [this assumption
was adopted in Section 1.8 enabling the use of Cauchy ($\bmf{\sigma}$) rather than Kirchhoff ($\bmf{\tau}$)
stresses],
and (2) that $\mathbf{W}$ $\approx$ $\mathbf{\Omega}$. The first assumption holds readily for ductile metals
undergoing large plastic strains.  The second assumption requires $\mathbf{W}\approx\dotR \mathbf{R}^T$.
This approximation becomes exact for  rigid-rotation $\dot{\mathbf{U}}=\mathbf{0}$ and for 
proportional deformation when that 
$\dot{\mathbf{U}} \mathbf{U}^{-1} = \mathbf{U}^{-1} \dot{\mathbf{U}}$.
Under these assumptions the Green-Naghdi and Jaumann rates of Cauchy stresses become equal. Thus,
we approximate rotational rate terms   in the tangent stiffness by those for the Jaumann stress.
The derivation to find the rotational terms $\mathbf{Q}$ for the Jaumann rate of Cauchy
stress remains rather tedious and lengthy but is worked out
in detail by Crisfield  [\ref{R:C1997}] in Sections 12.4 and 12.5 leading to 
 %
\begin{equation}\label{E:SRCTt}\mathbf{Q} =  \begin{bmatrix} [1.5]
2 \sigma_{11} &0 & 0& \sigma_{12}& 0 & \sigma_{13}  \\
0 & 2\sigma_{22} & 0 & \sigma_{12}& \sigma_{23}& 0 \\
0 &0& 2\sigma_{33} &0&\sigma_{23} & \sigma_{13} \\
\sigma_{12} & \sigma_{12} &0& \frac{1}{2}(\sigma_{11}+\sigma_{22})&  \frac{1}{2}\sigma_{13}&\frac{1}{2}\sigma_{23}\\
0 & \sigma_{23}&\sigma_{23} & \frac{1}{2}\sigma_{13} & \frac{1}{2}(\sigma_{22}+\sigma_{33}) &  \frac{1}{2}\sigma_{12}\\
\sigma_{13} & 0 & \sigma_{13} &  \frac{1}{2}\sigma_{23} & \frac{1}{2}\sigma_{12} & \frac{1}{2}(\sigma_{11}+\sigma_{33})
\end{bmatrix}_{6\times 6} \ .
\end{equation}
\normalsize

To reach this form, Crisfield begins with the virtual work expression and then the rate of internal nodal
forces for an element as
(using a mix of matrix and tensor notation)
%
\begin{equation}\label{E:SRCTu}
 \delta W_{int} = \delta \bmf{u}^T_e  \bmf{I}_e    = \int_{V_0} \delta \mathbf{D}: \bmf{\tau} \, dV_0
\end{equation}
%
\nid and,
%
\begin{equation}\label{E:SRCTv}
 \delta \bmf{u}^T_e   \dot{ \bmf{I}}_e =  \int_{V_0} \left [  \delta \mathbf{D}: \dot{\bmf{\tau}} +
 \delta \dot{\mathbf{D}} : \bmf{\tau} \right ]  \, dV_0\ .
\end{equation}
%
\nid Now switch to Jaumann rate of Cauchy stress from the rate of Kirchhoff stress where 
$\bmf{\tau}=J\bmf{\sigma}$ and 
$\dot{\bmf{\tau}}=J\dot{\bmf{\sigma}}+ \dot J \bmf{\sigma}$
%
\begin{equation}\label{E:SRCTw}
 \dot{ \bmf{\tau}} = J \left [ \overline{\mathbf{E}}^*:\mathbf{D} + \mathbf{W} \bmf{\sigma} -
  \bmf{\sigma}\mathbf{W} + \bmf{\sigma}\, tr(\mathbf{D}) \right ]
\end{equation}
%
\nid and take $J=1$ and $tr(\mathbf{D})=0$. Note the use of the $4^{th}$ order tensor
form $\overline {\mathbf{E}}^*$ of $\mathbf{E}^*$ here from 
the Green-Naghdi formulation. Then
%
\begin{equation}\label{E:SRCTx}
 \delta \bmf{u}^T_e   \dot{ \bmf{I}}_e =
  \int_{V_e} \left [  \delta \mathbf{D}: \overline{\mathbf{E}}^*:\mathbf{D} +
 \delta \mathbf{D}: ( \mathbf{W} \bmf{\sigma} )-
  \delta \mathbf{D}:( \bmf{\sigma}\mathbf{W} )  +\delta \dot{\mathbf{D}} : \bmf{\sigma} \right ]  \, dV_e\ .
\end{equation}
%
\nid The last 3 of the 4 terms combine (with much manipulation) to yield the (symmetric)
geometric stiffness, $\left [\mathbf{K}_{geo}\right ]_e$, with details shown in  Eq. (1.8.29) and the
$\mathbf{Q}$ matrix defined above.  Use is made of symmetry for $\mathbf{D}$  and $\bmf{\sigma}$,
substitution of $\mathbf{W}=\mathbf{L}-\mathbf{D}$, and the definition of shear strains (\eg $\gamma_{xy} =
u,_y + v,_x$) to express $\mathbf{D}, \delta \mathbf{D}$ in $6\times 1$ vector 
forms $\dot{\bmf{\vareps}},\, \delta \bmf{\vareps}$ suitable for writing in $\mathbf{B}$ matrix form, \ie
$\dot{\bmf{\vareps}}= \mathbf{B} \,\dot{\bmf{u}}_e$. Nagtegaal and Veldpaus [\ref{R:NV1984}]
derived the same expressions earlier but without the detailed presentation of Crisfield.

The separation of terms placing $\mathbf{Q}$  into the material 
stiffness is one of convenience in WARP3D
since the we have $\mathbf{B}^T\mathbf{Q}\mathbf{B}$.
However, terms of $\mathbf{Q}$ 
are linear in the Cauchy stresses as are terms of the geometric stiffness, $[K_{geo}]_e$. When an
eigenvalue buckling analysis is performed, the rotational terms (linear in stress)
leading to $\mathbf{Q}$ should be moved to the geometric stiffness (see [\ref{R:JWB2013}]).

The complete tangent stiffness for an element used in the global
Newton iterations thus consists of the exact
geometric stiffness, $\left [ \mathbf{K}_{geo}\right]_e$,
and an approximate material tangent, $\left [ \mathbf{K}_{mat}\right]_e$.
Multiple simplifications are made to develop
this tractable expression for the material tangent. 
The part of $\dot{\bmf{\sigma}}$ from deformation follows from the 
Green-Naghdi term, Eq. (\ref{E:SRCTn}), using the exact
algorithmic tangent, $\mathbf{E}^*$, on the unrotated configuration [this is 
considered the most critical aspect of the element tangent stiffness to compute exactly]. The
effects of material rotation on $\dot{\bmf{\sigma}}$ over $\Delta t$ are approximated by the
Jaumann rate description. Further,  incompressible behavior is assumed even though
the stress update models generally do not also make this assumption, \eg plasticity
in the presence of strain hardening. 
The potential impact of  $\mathbf{Q}$ becomes clear, for example, when the diagonal
terms of $\mathbf{E}^*$
under continued plastic straining decrease towards the magnitude of the Cauchy stresses. 

Given this approximate form used for the material part of the element tangent stiffness matrix, we have
observed some solutions that exhibit improved (global) convergence rate when $\mathbf{Q}$ is omitted. In contrast,
some solutions will not converge unless $\mathbf{Q}$ is included -- in most of the models analyzed over the
years with WARP3D, including the  $\mathbf{Q}$ term improves global convergence. 
The \ti{nonlinear solution
parameters} for model input (Section 2.10) provides an option to include/exclude the $\mathbf{Q}$  contribution.
The default is to \ti{include} $\mathbf{Q}$.

The material stiffness defined by Eq. (\ref{E:SRCTq}) appears in the NIKE2D/3D codes
for implicit nonlinear solutions based on the same Green-Naghdi stress  rate. However,
the  $\mathbf{Q}$ term is omitted.


%
%*****************************************************
\subsection {References}
%*****************************************************
\small 

\noindent[\refstepcounter{sectrefs}\label {R:KK1982}\thesectrefs]~S.W. Key and R.D. Krieg.
On the Numerical Implementation of Inelastic Time Dependent and Time Independent, 
Finite Strain Constitutive Equations in Structural Mechanics. \ti{Computer Methods in Applied Mechanics and
Engineering}, Vol. 33, 1982, pp. 439-452.

\medskip
\noindent[\refstepcounter{sectrefs}\label {R:NV1984}\thesectrefs]~J.C. Nagtegaal, and F.E. Veldpaus.
On the Implementation of Finite Strain Plasticity Equations in a Numerical Model.
In Numerical Analysis of Forming Processes (edited by J.F. Pittman, O. C. Ziekiewicz, R. D. Wood 
and J. M. Alexander), p. 351. John Wiley and Sons, New York, 1984.

\medskip
\noindent[\refstepcounter{sectrefs}\label {R:HRZ1976}\thesectrefs]~E. Hinton, T. Rock, and O.C. Zienkiewicz.
A Note on Mass Lumping and Related Processes in the Finite Element Method. \ti{Earthquake Engineering and
Structural Dynamics}, Vol. 4, No. 3, 1976, pp 245-249.

\medskip
\noindent[\refstepcounter{sectrefs}\label {R:ZTZ2013}\thesectrefs]~O.C. Zienkiewicz, R.L. Taylor, 
and J.Z. Zhu. The Finite Element Method: Its Basis \& Fundamentals. 7th Ed. Butterworth-Heinemann, 
Waltham, MA, 2013.

\medskip
\noindent[\refstepcounter{sectrefs}\label {R:BLME2014}\thesectrefs]~T. Belytschko, W.K. Liu,
B. Moran, K.I. Elkhodary. Nonlinear Finite Elements for Continua and Structures. 2nd Ed. John Wiley \& Sons,
New York, 2014.

\medskip
\noindent[\refstepcounter{sectrefs}\label {R:C1997}\thesectrefs]~M. Crisfield. Nonlinear 
Finite Element Analysis 
of Solids and Structures.Volume 2: Advanced Topics. John Wiley \& Sons,
New York, 1997.

\medskip
\noindent[\refstepcounter{sectrefs}\label {R:D1979}\thesectrefs]~J.K. Dienes. On the Analysis of 
Rotation and Stress Rate in Deforming Bodies. \ti{Acta Mechanica}, Vol. 32, 1979, pp. 217-232.

\medskip
\noindent[\refstepcounter{sectrefs}\label {R:H1984}\thesectrefs]~J.O. Hallquist. NIKE 3D - A Vectorized, 
Implicit, Finite Deformation, Finite-Element Code for Analyzing the Static and 
Dynamic Response of 3-D Solids. Lawrence Livermore Laboratory Report UCID-18822, 1984.

\medskip
\noindent[\refstepcounter{sectrefs}\label {R:JB1984}\thesectrefs]~C.C. Johnson and D.J. Bammann.
A Discussion of Stress Rates in Finite Deformation Problems. \ti{International Journal for Solids and Structures},
 Vol. 20, 1984, pp. 725-737.

\medskip
\noindent[\refstepcounter{sectrefs}\label {R:HW1980}\thesectrefs]~T.J.R. Hughes and J. Winget.
FInite Rotation Effects In Numerical Integration of Rate Constitutive 
Equations Arising In Large-Deformation Analysis.
\ti{International Journal for Numerical Methods in Engineering},
 Vol. 15, No. 12, 1980, pp. 1862-1867.
 
\medskip
\noindent[\refstepcounter{sectrefs}\label {R:FT1987}\thesectrefs]~D.P. Flanagan and L.M. Taylor.
An Accurate Numerical Algorithm For Stress Integration With Finite Rotations.
\ti{Computer Methods in Applied Mechanics and Engineering},
 Vol. 62, 1987, pp. 305-320.

\medskip
\noindent[\refstepcounter{sectrefs}\label {R:TF1989}\thesectrefs]~L.M. Taylor and D.P. Flanagan.
PRONTO 3D: A Three-Dimensional Transient Solid Dynamics Program. SAND-87-1912 ON: DE89010517,
OSTI ID: 6212624, 1989.

\medskip
\noindent[\refstepcounter{sectrefs}\label {R:SH1998}\thesectrefs]~J.C. Simo and T.J.R. Hughes.  
Computational Inelasticity (Interdisciplinary Applied Mathematics) (v. 7).  Springer-Verlag New York, Inc. 1998.
ISBN 0-387-97520-9.

\medskip
\noindent[\refstepcounter{sectrefs}\label {R:JWB2013}\thesectrefs]~W. Ji, A.M. Waas, and M.Z. Bazant. 
On the Importance of Work-Conjugacy and Objective Stress Rates in
Finite Deformation Incremental Finite Element Analysis.
\ti{Journal of Applied Mechanics},  Vol. 80(4), 2013. 9pgs.
 
\medskip
\noindent[\refstepcounter{sectrefs}\label {R:L1969}\thesectrefs]~E.H.Lee.
Elastic-plastic Deformation at Finite Strain.
\ti{Journal of Applied Mechanics},  Vol. 36, pp. 1-6,1969.
 
\medskip
\noindent[\refstepcounter{sectrefs}\label {R:A1984}\thesectrefs]~R.J. Asaro.
Crystal plasticity.\ti{Journal of Applied Mechanics},  Vol. 50, pp. 1-12,1984.

\medskip
\noindent[\refstepcounter{sectrefs}\label {R:POP1984}\thesectrefs]~P.M. Pinsky, M. Ortiz, and K.S. 
Pister, K. S.
Numerical Integration of Rate Constittive Equations in Finite Deformation Analysis.
\ti{Computer Methods in Applied Mechanics and Engineering}, Vol. 40, 1983, pp. 137-158.
 
\medskip
\noindent[\refstepcounter{sectrefs}\label {R:HC1984}\thesectrefs]~A. Hoger and D.E. Carlson.
Determination of the Stretch and Rotation in the Polar Decomposition of the 
Deformation Gradient. \ti{Quarterly of Applied Mathematics}, Vol. 42, 1984, pp. 113-117.


\medskip
\noindent[\refstepcounter{sectrefs}\label {R:ST1985}\thesectrefs]~J.C. Simo and R.L. Taylor, R.L.
Consistent tangent operators for rate Independent elastoplasticity,
\ti{Computer Methods in Applied Mechanics and Engineering}, Vol. 35, 1985, pp. 101-118.


\medskip
\noindent[\refstepcounter{sectrefs}\label {R:MN1987}\thesectrefs]~M.M. Mehrabadi and S. Nemat-Nasser.
Some basic kinematical relations for finite deformations of continua,
ti{Mechanics of Materials}, Vol. 6, No. 2 1987, pp. 167-138.

\medskip
\noindent[\refstepcounter{sectrefs}\label {R:JWB2013}\thesectrefs]~W. Ji, A.M. Waas, and M.Z. Bazant. 
On the importance of work-conjugacy and objective stress rates in finite deformation incremental finite element analysis.
\ti{Journal of Applied Mechanics},  Vol. 80(4), 2013. 9pgs.




\end{document}
