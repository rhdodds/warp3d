%!TEX TS-program=pdflatexmk

\documentclass[11pt]{report}
\usepackage{geometry} 
\geometry{letterpaper}

%---------------------------------------------
\setlength{\textheight}{630pt}
\setlength{\textwidth}{450pt}
\setlength{\oddsidemargin}{14pt}
\setlength{\parskip}{1ex plus 0.5ex minus 0.2ex}


%----------------------------------------
\usepackage{amsmath}
\usepackage{layout}
\usepackage{color}
\usepackage{array}

%----------------------------------------------
\usepackage{fancyhdr} \pagestyle{fancy}
\setlength\headheight{15pt}
\lhead{\small{User's Guide - \ti{WARP3D}}}
\rhead{\small{\ti{Interface-Cohesive Elements}}}
\fancyfoot[L] {\small{\ti{Chapter {\thechapter}}\ \   (Updated: 6-5-2013)}}
\fancyfoot[C] {\small{\thesection-\thepage}}
\fancyfoot[R] {\small{\ti{Elements and Material Models}}}

%---------------------------------------------------
\usepackage{graphicx}
\usepackage[labelformat=empty]{caption}
\numberwithin{equation}{section}

%---------------------------------------------
%     --- make section headers in helvetica ---
%
\usepackage{sectsty} 
\usepackage{xspace}
\allsectionsfont{\sffamily} 
\sectionfont{\large}
\usepackage[small,compact]{titlesec} % reduce white space around sections
%---------------------------------------------->
%
%
%   which fonts system for text and equations. with all commented,
%   the default LaTex CM fonts are used
%
%
\frenchspacing
%\usepackage{pxfonts}  % Palatino text 
%\usepackage{mathpazo} % Palatino text
%\usepackage{txfonts}


%---------  local commands ---------------------


\newcommand{\bmf } {\boldsymbol }  %bold math symbol
\newcommand{\bsf } [1]{\textrm{\ti{#1}}\xspace}
\newcommand{\ul} {\underline}
\newcommand{\hv} {\mathsf}   %helvetica text inside an equation
\newcommand{\eg}{\emph{e.g.},\xspace}
\newcommand{\ie}{\emph{i.e.},\xspace}
\newcommand{\vs}{\emph{vs.}\xspace}
\newcommand{\ti}{\emph}
\newcommand{\bardelta}{\bar \delta}
\newcommand{\barDelta}{\bar \Delta}
\newcommand{\NI}{\noindent}

\newenvironment{offsetpar}[1]%
{\begin{list}{}%
         {\setlength{\leftmargin}{#1}}%
         \item[]%
}
{\end{list}}

%
%
%        optional definition for bullet lists which
%        reduces white space.
%
\newcommand{\squishlist}{
 \begin{list}{$\bullet$}
  { \setlength{\itemsep}{0pt}
     \setlength{\parsep}{3pt}
     \setlength{\topsep}{3pt}
     \setlength{\partopsep}{0pt}
     \setlength{\leftmargin}{1.5em}
     \setlength{\labelwidth}{1em}
     \setlength{\labelsep}{0.5em} } }

\newcommand{\squishlisttwo}{
 \begin{list}{$\bullet$}
  { \setlength{\itemsep}{0pt}
     \setlength{\parsep}{0pt}
    \setlength{\topsep}{0pt}
    \setlength{\partopsep}{0pt}
    \setlength{\leftmargin}{2em}
    \setlength{\labelwidth}{1.5em}
    \setlength{\labelsep}{0.5em} } }

\newcommand{\squishend}{
  \end{list}  }
%


%-------------------------------------
\newcounter{sectrefs}
\setcounter{sectrefs}{0}
\setcounter{figure}{0}
\setcounter{chapter}{3}
\setcounter{section}{2}
\renewcommand{\thefigure}{\thesection.\arabic{figure}}

%
%--------------------------------------
%
%
%
%              start document 
%              ==========
%
%

\begin{document}
%------------------------------------------------------------------------------
\section{Interface-Cohesive Elements:\ti{ inter\_8, trint6, trint12}}
%------------------------------------------------------------------------------

Interface elements provide a very general three-dimensional framework to 
model spontaneous crack formation and/or crack extension. Mode I crack formation/growth
over symmetry planes becomes a simplified application of the more
general 3-D, mixed-mode fracture configuration. The interface elements consist of two 
generally non-planar, quadrilateral or triangular surfaces 
which connect the congruent faces of two adjacent 
solid elements (hexahedrals and tets). The higher-order, triangular interface element
may have curved edges as well.  When the surfaces of the two adjacent solid 
elements deform, the interface element between them experiences 
displacement jumps across the two surfaces which 
generate tractions following a user specified, linear or nonlinear
traction-separation relationship. In the initial undeformed configuration, 
the nodes on the two surfaces of the interface elements (\ie $S^-$ and $S^+$) 
are most often defined to be coincident although this is not a requirement
(relatively thin interface elements may facilitate mesh
generation). The interface elements are intended for use with 
the \ti{cohesive} constitutive model described in Section 3.8 and with the crack growth
processor defined in Section 5.4. 

Figure \ref{fig:inter_def} shows the three different types of interface 
elements presently available in WARP3D. The element \ti{inter\_8 }consists 
of two, 4-node bilinear isoparametric surfaces and connects the faces of 
two compatible solid elements (\eg the \ti{l3disop}). The two surfaces 
of element \ti{trint6} have three nodes each and this element joins 
two 4-node tetrahedral 
(\ti{tet4}) elements. Finally, the element \ti{trint12} comprises 
two 6-node quadratic, parametric surfaces and connects the faces 
of two 10-node tetrahedral (\ti{tet10}) elements. The triangular interface 
elements, together with tetrahedral solid elements, provide WARP3D 
with the capability to simulate complex, 3-D fracture processes with
crack turning and branching.

Finite element models with interface-cohesive elements may or 
may not have pre-existing cracks. Interface-cohesive elements often
are employed to model spontaneous crack formation and then growth 
across multiple locations in the model under increased loading. Pre-exisiting
cracks in the model introduce no difficulties in the use of interface-cohesive
elements to simulate their subsequent extension, branching, etc. and their
influence on the formation-extension of other cracks in nearby
material.

Essential concepts of the element formulation for finite deformation analyses are
described by Ortiz and Pandolfi [\ref{R:OP1999}] and others [\ref{R:SL2004}].
The present formulation casts 
the governing equations on the current rather than initial configuration.

Much of the notation adopted here follows Section 3.8 which describes the
cohesive constitutive models.
%
\begin{figure}[htb]
\begin{center}
\includegraphics[scale=0.75,angle=0]{fig_1.pdf} 
\caption{{\small Fig. \thefigure\ Assembly of various interface-cohesive elements
with two solid elements. Note that the \ti{trint12} element may have curved edges.}
\label{fig:inter_def}}
%
\end{center}
\end{figure}

%------------------------------------------------------------------------------

\subsection{Element Formulation}
%------------------------------------------------------------------------------

This section describes details of the formulation of a general interface element 
consisting of $2 \times nnode$ nodes, where 
$nnode=$ 4, 3 and 6, for \ti{inter\_8}, \ti{trint6} and \ti{trint12}, respectively. 
The interface-cohesive element formulation supports both geometrically 
linear and nonlinear (large displacement) analyses. Linear or nonlinear 
traction-separation, cohesive relationships (Section 3.8) are 
assigned to the elements to define the 
constitutive response. For geometrically nonlinear elements, the true 
traction (Cauchy traction computed in the current configuration) defines 
the stress measure for the interface constitutive behavior. For the 
geometrically linear (small-displacement) formulation, the traction-separation 
law employs a stress measure based on the initial, undeformed configuration. 

 Consider an interface element with $2 \times nnode$ nodes. 
 Nodes $1 \rightarrow nnode$ lie on the \ti{bottom} surface of the 
 element, while nodes $(nnode+1) \rightarrow 2 \times nnode$ reside 
 on the \ti{top} surface. The mesh generation process should make each 
 pair of top and bottom surface nodes geometrically coincident in the undeformed 
 configuration (this is not strictly required). Each node 
 has three translational displacements $(u_1, u_2, u_3)$ in the 
 global Cartesian coordinate system of the model $(X_1, X_2, X_3)$. 
 The $6 \times nnode $ 
 nodal displacement vector, $\bmf{d}$, for the element in global 
 coordinates  may be written 
%
\begin{equation}
\bmf{d} = \left ( u_1, u_2, ... u_{2 \times nnode}, 
v_1, v_2, ... v_{2 \times nnode}, w_1, w_2, ... w_{2 \times nnode}
\right )^T\ .
\end{equation}
%
Note the ordering of nodal displacements: $u$ for all nodes with bottom nodes first,
then $v, w$ for all nodes in the same ordering.

The use of standard interpolation functions 
enables construction of the continuous displacement 
fields in global coordinates at each point on the bottom surface,
$\bmf{u}^b_{3 \times 1}$, and on the top surface, $\bmf{u}^t_{3 \times 1}$.
%
Writing the displacement functions as 
$\bmf{\tilde u}_{6 \times 1}=\left ( \bmf{u}^b,\bmf{u}^t \right )^T$, we have 
%
\begin{equation}
\bmf{\tilde u}_{6 \times 1} ( \xi_1, \xi_2)= \bmf{N}_{6 \times \left ( 6 \times nnode \right )}
\,\bmf{d}_{\left ( 6 \times nnode \right ) \times 1 }
\end{equation}
%
\noindent where the matrix $\bmf{N}$ contains the usual interpolation 
functions expressed in parametric (surface) coordinates, 
$\bmf{\mathcal{N}}\left( \xi_1, \xi_2 \right ) =
 \left[ N_1\left( \xi_1, \xi_2 \right ), N_2\left( \xi_1, \xi_2 \right ), ...,
 N_{nnode}\left( \xi_1, \xi_2 \right )\right ]$, according to
%
\begin{equation}
\bmf{N}= \left [
\begin{matrix}
\bmf{\mathcal{N}} & \bmf{0} & \bmf{0} & \bmf{0} & \bmf{0} & \bmf{0} \\ 
\bmf{0} & \bmf{0} & \bmf{\mathcal{N}}  & \bmf{0} & \bmf{0} & \bmf{0} \\ 
\bmf{0} & \bmf{0} & \bmf{0} & \bmf{0} &\bmf{\mathcal{N}}  & \bmf{0} \\ 
\bmf{0} & \bmf{\mathcal{N}} & \bmf{0} & \bmf{0}& \bmf{0}& \bmf{0} \\ 
\bmf{0} & \bmf{0} & \bmf{0} & \bmf{\mathcal{N}} &\bmf{0} & \bmf{0} \\ 
\bmf{0} & \bmf{0} & \bmf{0} & \bmf{0} & \bmf{0} & \bmf{\mathcal{N}} \\ 
\end{matrix} \right ]_{6 \times ( 6 \times nnode) }
\end{equation}
where $\bmf{0} = \left [ 0, 0, ..., 0_{nnode} \right ]$.
%
\noindent Figures  \ref{fig:quad_def},  \ref{fig:tri6_def}, and  \ref{fig:tri12_def} 
show the parametric coordinates adopted to describe the interface elements. 
The \ti{inter\_8}
element uses parametric coordinates $(\xi_1, \xi_2)$ to define surface locations. 
The triangular interface
elements use natural area coordinates such that the three parametric coordinates 
$(\xi_1, \xi_2, \xi_3)$ are coupled by the usual relation $\xi_1 + \xi_2 + \xi_3 = 1$.

Global coordinates for points $(\xi_1, \xi_2)$ on a surface of  the element are found with 
usual interpolation of nodal coordinates
\begin{eqnarray}
X_1(\xi_1, \xi_2) &=& \sum^{nnode} N_i(\xi_1, \xi_2)\, X_{1(i)} \\
X_2(\xi_1, \xi_2) &=& \sum^{nnode} N_i(\xi_1, \xi_2) \,X_{2(i)}\\
X_3(\xi_1, \xi_2) &=& \sum^{nnode} N_i(\xi_1, \xi_2)\, X_{3(i)}\ 
\end{eqnarray}
where the  coordinates for nodes $i:1\rightarrow nnode$ refer to those on the bottom surface,
the top surface or for the large displacement formulation the average of
top and bottom surface nodal coordinates to construct a mid-surface.

With these (interpolated) continuous displacement fields at each point
on the bottom and top surfaces, the \ti{jump} displacement fields (top surface -- bottom surface)
corresponding to common points in parametric coordinates on the two
surfaces is defined by
%
\begin{equation}
\Delta\bmf{u}_{3 \times 1}(\xi_1, \xi_2) = \bmf{L}_{3 \times 6}
\, \bmf{\tilde u}_{6 \times 1 }(\xi_1, \xi_2)
\end{equation}
%
\noindent where the operator matrix $\bmf{L}$ has the simple form
\begin{equation}
\bmf{L}= \left [
\begin{matrix}
-1 &  0 & 0 & +1 & 0 & 0 \\ 
0 & -1 &  0 & 0 & +1 & 0 \\ 
0 & 0 & -1 & 0 & 0 & +1 \\ 
\end{matrix} \right ]
\end{equation}
where the ordering of terms in each row of $\bmf{L}$  sets the positive 
sense of jump displacements
as top -- bottom surface values.
Using the above expressions, the relation between element nodal 
displacements and the continuous (jump) displacement fields 
over the element in \ti{global} coordinates has the form
%
\begin{equation}
\Delta \bmf{u}(\xi_1, \xi_2) = \bmf{L N d}\ .
\end {equation}
%

Cohesive constitutive models link: (1)
displacement jumps normal
to the top and bottom surfaces with a corresponding normal traction, and 
(2)  displacement jumps (sliding)  in the tangent plane to shear tractions.
In general, the identification of normal and sliding displacements
requires a transformation of global displacements at nodes on the top and bottom
surfaces of the element
to a local (orthogonal) tangent-normal coordinate system (the same as required in beam
and shell elements to identify axial and shear displacements).

Let matrix $\bmf{R}_{3 \times 3}$ define the 
rotational transformation for vectors expressed in the global reference frame  $\bmf{X}$
$(X_1,X_2,X_3)$ to an element specific, local orthogonal system  $(\bmf{t}_1,
\bmf{ t}_2,\bmf{ n})$ 
[see Figs.\ \ref{fig:quad_def},  \ref{fig:tri6_def}, and  \ref{fig:tri12_def}]. 
The direction $\bmf{n}$ lies normal to a tangent plane for the element  constructed at
location $(\xi_1 =\xi_2 = 0)$ and the (orthogonal)
directions $\bmf{t_1}, \bmf{t_2}$ lie in the tangent plane. Positive senses are indicated on the 
figure with $\bmf{n}$ directed (positive) from $S^-$ to $S^+$. 
For \ti{trint6} and \ti{trint12}, the tangent plane constructed at
 $(\xi_1 =\xi_2 = \xi_3 = \textstyle{1/3})$
is used to form $\bmf{R}$. 

\NI Construction of $\bmf{R}$ follows this sequence:
\squishlist
\item  translate element nodes such that
element node 1 lies at $\bmf{X}=\bmf{0}$;
\item define a set of element orthogonal coordinates $( x_1, x_2,  x_3)$
where $\bmf{x} $ $=$ $  \bmf{\lambda} \bmf{X}$.
Construct the rotation matrix $  \bmf{\lambda} $
such that $\bmf{x}_3$ lies normal to the plane containing element nodes 1, 2 and 3.
$\bmf{x}_2$ is in direction of $\bmf{x}_3 \times \bmf{L}_{1\mbox{-}2}$; with 
$\bmf{x}_1$ $=$ $\bmf{L}_{1\mbox{-}2}/ || \bmf{L}_{1\mbox{-}2} ||$ where
$\bmf{L}_{1\mbox{-}2}$ denotes a vector from node 1 to 2;
\item compute the set of coordinates $\bmf{x}$ for each element node $i$, 
$\bmf{ x}_{(i)}$ $=$ $ \bmf{\lambda}  \bmf{X}_{(i)}$
\item  compute
the coordinate Jacobian, $ \bmf {J}_{2\times 3}$  (in terms of  $ x_1, x_2, x_3$)
at parametric location $(0,0)$ where $J_{i,j}= \partial x_j / \partial \xi_i$ .
This yields two vectors $(\tilde {\bmf{t}}_1,
\tilde {\bmf{t}}_2)$ in the
tangent plane but that are not necessarily orthogonal, \eg $\tilde {\bmf{t}}_1=J_{1,1} \bmf{i}_1 $
$+J_{1,2} \bmf{i}_2$ $+J_{1,3} \bmf{i}_3$ where $\bmf{i}_1$, $\bmf{i}_2$, and $\bmf{i}_3$
are unit vectors aligned with the $\bmf{x}$ axes.
\item compute $\bmf{n} = \tilde{\bmf{t}}_1 \times \tilde{\bmf{t}}_2 / || \tilde{\bmf{t}}_1 \times \tilde{\bmf{t}}_2 ||$
\item compute $\bmf{t}_1 = \tilde {\bmf{t}}_1 / ||\tilde {\bmf{t}}_1 ||$; then $\bmf{t}_2 =
\bmf{n} \times \bmf{t}_1 $
\item define rotation matrix $ \bar { \bmf \lambda } = 
\left [ \bmf{t}_1 \, \bmf{t}_2 \, \bmf{n} \right ]^T $ 
\item $ \bmf{R} =  \bar { \bmf \lambda } \, \bmf{\lambda}$
defines the global - to - tangent plane rotation for displacements.
\squishend
 
\NI  This same $\bmf{R}$ is
used for all parametric locations within the element. This introduces a small
approximation for initially non-planar interface elements. For the geometrically 
linear formulation, all element level computations refer to the undeformed nodal
coordinates and thus $\bmf{R}$ remains fixed over the loading history.

In geometrically nonlinear analyses, the deformed shape-orientation
of the interface element at time $t$ is employed in computations, 
$\bmf{X}_t= \bmf{X}_0 + \bmf{u}_t$. The computations construct
a \ti{middle} surface of the cohesive element in the current 
configuration using averaged displacements of 
corresponding $S^-$ and $S^+$
nodes. The middle surface provides the basis to compute 
$ \bmf{t}_1, \bmf{t}_2, \bmf{n} $ and then $\bmf{R}$ as outlined above
(see Fig. 
\ref{fig:top_bottom_def}).
%
\begin{figure}[htb]
\begin{center}
\includegraphics[trim=0.95in 3.5in 0.in 0.85in, clip=true,scale=0.82,angle=0]{fig_2.pdf} 
\caption{{\small Fig. \thefigure: Local node numbering for \ti{inter\_8}. 
Isoparametric coordinates for the element corner nodes and 
location of integration points are listed.} \label{fig:quad_def}}
%
\end{center}
\end{figure}
%

Finally, the vector of displacement jumps  defined relative to the tangent plane and
normal direction at each pair of top and bottom
surface nodes, 
$\bmf{\Delta}=\left ( \Delta_{t1}, \Delta_{t2}, \Delta_n \right )$,  is given by 
%
\begin {equation}\label{E:B_define}
\bmf{\Delta} = \bmf{B d} = \bmf{R L N d}\ .
\end {equation}%
%
Rates (increments) of the displacement jumps and cohesive tractions may be 
coupled in standard form through a tangent modulus matrix, 
$\bmf{D}_{3 \times 3}$, such that
%
\begin {equation}
\bmf{\dot T} = \bmf{D \dot \Delta}
\end {equation}
%
\noindent where $\bmf{\dot T}= \left(\dot  T_{t1}, \dot T_{t2}, \dot T_n\right )^T$
and $D_{ij}= \partial T_i / \partial \Delta_j$. 

%------------------------------------------------------------------------------
\subsection{Geometrically Linear Elements}
%------------------------------------------------------------------------------

The finite element equilibrium equations are derived from the 
principle of virtual displacements. In the absence of inertia forces, the virtual work 
expression for a body containing a cohesive surface can be expressed in the form,
%
\begin{equation}
\int_{V_0} \delta \bmf{\varepsilon}^T \bmf{\sigma} \;\mathrm{d}V +
\int_{S_{0\mbox{-}coh}} \delta \bmf{\Delta}^T \bmf{T} \;\mathrm{d}S_{coh} =
\int_{S_{0\mbox{-}ext}} \delta \bmf{d}^T \bmf{T}_{ext} \;\mathrm{d}S+
\int_{V_0} \delta \bmf{d}^T \bmf{F}_{ext} \;\mathrm{d}V\ . 
\end {equation}
%
\noindent Here $\bmf{\sigma}$ denotes the stress tensor, 
$\bmf{T}_{ext}$ the ${3 \times 1}$ the externally applied tractions on the portion of the boundary
$S_{0\mbox{-}ext}$ in the undeformed configuration, with body forces $\bmf{F}_{ext}$. $V_0$
and $S_{0\mbox{-}coh}$ denote volume and internal cohesive areas in the undeformed
configuration, respectively. 
$\delta \bmf{d}$ represents the virtual displacement vector and
$\delta \bmf{\varepsilon}$ the corresponding variations of the usual
(small) strain tensor.  For the interface-cohesive elements, 
$\delta \bmf{\Delta} = \bmf{B} \delta \bmf{d}$.

Application of standard finite element procedures yields the (global) nodal force vector 
for each interface element as
%
\begin{equation}
\bmf{f}_{(6 \times nnode) \times 1} = \int_{-1}^1  \int_{-1}^1 
\bmf{B}^T \bmf{T} \; J_0  \;\mathrm{d}\xi_1  \;\mathrm{d}\xi_2 \ 
\ \mathrm{for\ \ti{inter\_8}} 
\end {equation}
%
\begin{equation}
\bmf{f}_{(6 \times nnode) \times 1} = \int_{0}^1  \int_{0}^{1} 
\bmf{B}^T \bmf{T} \; J_0  \;\mathrm{d}\xi_1  \;\mathrm{d}\xi_2   
\ \ \mathrm{for\ \ti{trint6}\ and\ \ti{trint12}} 
\end {equation}
%
\noindent Gauss or Newton-Cotes integration is applied for the quadrilateral 
element defined over the parametric, bi-unit square (Fig. \ref{fig:quad_def}); 
integration over the triangle elements employs area coordinates over the parametric,
unit triangle
(Figs. \ref{fig:tri6_def}, \ref{fig:tri12_def}).  The area scale factor
$J_0(\xi_1,\xi_2)= || \tilde{\bmf{t}}_1(\xi_1,\xi_2) \times \tilde{\bmf{t}}_2(\xi_1,\xi_2) ||$.

For the implicit solution procedures used in WARP3D, the global tangent 
stiffness matrix of dimensions $(6 \times nnode) \times (6 \times nnode )$ for 
the interface elements is given by
%
\begin{equation}
\bmf{K}_T = \int_{-1}^1  \int_{-1}^1 
\bmf{B}^T \bmf{D} \bmf{B} \; J_0  \;\mathrm{d}\xi_1  \;\mathrm{d}\xi_2
 \ \ \mathrm{for\ \ti{inter\_8}} 
\end {equation}
%
%
\begin{equation}
\bmf{K}_T = \int_{0}^1  \int_{0}^{1} 
\bmf{B}^T \bmf{D} \bmf{B} \; J_0  \;\mathrm{d}\xi_1  \;\mathrm{d}\xi_2   
\ \ \mathrm{for\ \ti{trint6}\ and\ \ti{trint12}} \ .
\end {equation}
\NI Since $\bmf{B}$ contains $\bmf{R}$ (Eq. \ref{E:B_define}),
the nodal force vector $\bmf{f}$ 
and element stiffness $\bmf{K}_T$
are computed directly in the global $ \bmf{X}$ coordinate frame.
%
%------------------------------------------------------------------------------
\subsection{Geometrically Nonlinear Elements}
%------------------------------------------------------------------------------
The virtual work expression for a body containing a cohesive surface can 
be written in the form,
%
\begin{equation}
\int_{V} \delta \bmf{\varepsilon}^T \bmf{\sigma} \;\mathrm{d}V +
\int_{S_{coh}} \delta \bmf{\Delta}^T \bmf{T} \;\mathrm{d}S_{coh} =
\int_{S_{ext}} \delta \bmf{d}^T \bmf{T}_{ext} \;\mathrm{d}S+
\int_{V} \delta \bmf{d}^T \bmf{F}_{ext} \;\mathrm{d}V \ .
\end {equation}
%
\noindent Here $\bmf{\sigma}$ denotes the (symmetric) Cauchy stress tensor, 
$\bmf{T}_{ext}$ the ${3 \times 1}$ the externally applied tractions on the portion of the boundary,
$S_{ext}$ in the deformed configuration, with body forces $\bmf{F}_{ext}$. $V$
and $S_{coh}$ are the \ti{current} volume and \ti{current} internal cohesive areas, respectively. 
The tractions, $\bmf{T}$, acting across internal cohesive surfaces now 
refer to the \ti{current} configuration.

The (global) nodal force vector for the nonlinear interface element is given by
%
\begin{equation}
\bmf{f}_{(6 \times nnode) \times 1} = \int_{-1}^1  \int_{-1}^1 
\bmf{B}^T\left( \bmf{x}_{n+1}\right ) \bmf{T}_{n+1} 
\; J_{n+1}  \;\mathrm{d}\xi_1  \;\mathrm{d}\xi_2 \ \ \mathrm{for\ \ti{inter\_8}} 
\end {equation}
%
\begin{equation}
\bmf{f}_{(6 \times nnode) \times 1} = \int_{0}^1  \int_{0}^{1} 
\bmf{B}^T\left( \bmf{x}_{n+1}\right ) \bmf{T}_{n+1} 
\; J_{n+1}  \;\mathrm{d}\xi_1  \;\mathrm{d}\xi_2 \ \ \mathrm{for\ \ti{trint6}\ and\ \ti{trint12}} 
\end {equation}
%
\noindent where $J_{n+1}$ denotes the Jacobian of the transformation for
differential surface areas (see detailed notes in previous section).  Further,  $\bmf{t}$
represents the tractions acting on the internal cohesive 
surfaces having size, shape and orientation at $n+1$. The tractions 
$\bmf{T}_{n+1}$ to corresponding the 
displacement jumps $\bmf{\Delta}_{n+1}$ between $S^-$ and 
$S^+$ surfaces in the current configuration 
are provided by the cohesive constitutive model. The incremental 
displacement jumps $\bmf{\Delta}$ that drive updating 
of cohesive tractions from load step $n \rightarrow n+1$ are given by
%
\begin{equation}
\bmf{\Delta}  = \bmf{B}\left( \bmf{x}_{n+1}\right) \Delta \bmf{d} =
\bmf{R}_{n+1}\; \bmf{L} \;\bmf{N} \;\Delta \bmf{d}
\end {equation}
%
\noindent where $\bmf{R}$  is evaluated following details provided earlier
using the end of step configuration of the reference surface.

The global tangent stiffness matrix 
$(6 \times nnode ) \times (6 \times nnode)$ 
for the nonlinear interface elements is given by
%
\begin{equation}
\bmf{K}_T = \int_{-1}^1  \int_{-1}^1 
\bmf{B}^T\left( \bmf{x}_{n+1}\right) \bmf{D} 
\bmf{B}\left( \bmf{x}_{n+1}\right) \; J_{n+1} 
\;\mathrm{d}\xi_1  \;\mathrm{d}\xi_2 \ \ \mathrm{for\ \ti{inter\_8}} 
\end {equation}
%
%
\begin{equation}
\bmf{K}_T = \int_{0}^1  \int_{0}^{1} 
\bmf{B}^T\left( \bmf{x}_{n+1}\right) \bmf{D} \bmf{B}
\left( \bmf{x}_{n+1}\right) \; J_{n+1}  \;\mathrm{d}\xi_1  \;\mathrm{d}\xi_2   
\ \ \mathrm{for\ \ti{trint6}\ and\ \ti{trint12}} \ .
\end {equation}
%
\noindent where numerical experiments indicate that 
omission of initial stress and non-symmetric terms in $\bmf{K}_T$  do not 
adversely degrade convergence rates of the global Newton iterations -- load steps 
must necessarily be quite small to track realistically the change in cohesive tractions
with deformation.
%

\begin{figure}[htb]
\begin{center}
\includegraphics[trim=0.84in 4.1in 0.in 0.85in, clip=true,scale=0.88,angle=0]{fig_3.pdf} 
\caption{{\small Fig. \thefigure: Local node numbering for \ti{trint6}. 
Natural (area) coordinates for the element corner nodes and 
location of integration points are listed.}\label{fig:tri6_def}}
%
\end{center}
\end{figure}

%------------------------------------------------------------------------------
\subsection{Treatment of Interface Compression for Nonlinear Elements}
%------------------------------------------------------------------------------
For geometrically nonlinear analyses, the rotation matrix $\bmf{R}$ 
is computed (by default) using a middle surface constructed
by averaging the deformed $S^-$ and $S^+$ surfaces of the interface element. 
When the interface element experiences a net compressive traction, 
the  $S^-$ and $S^+$ surfaces  may experience interpenetration. 
For such elements, the middle surface 
and corresponding normal direction are computed using the 
element nodal coordinates in the undeformed configuration. However, 
the traction is computed using the deformed, middle surface area 
in the current configuration. 

To detect interpenetration of the top and bottom surfaces, the code treats
the interface element as an equivalent solid element. The Jacobian 
matrix of the transformation from parametric to (deformed) 
Cartesian coordinates is computed at the parametric center of the element. 
Elements with a negative determinant for the Jacobian 
matrix are treated as having interpenetration of the top and bottom surfaces. 
The stiffness and internal forces for these elements are 
calculated assuming a linear normal traction 
(compressive) \vs normal relative displacement 
(penetration) response. The stiffness of this linear 
response  is taken 
as a user-defined factor times the slope of the 
tensile traction-separation curve at the origin (see
Section 3.8). Even though some interpenetration may actually occur
prior to detection, the solution process does attempt to remove/correct
the situation through the application of restoring (penalty) forces. The use of
suitably small load steps is thus advised when conditions of interpenetration are
expected in regions of the applied loading history.

These additional checks for interpenetration take place only for 
the element formulation with large displacements. 
For the small-displacement formulation, 
element computations are performed using the 
initial, undeformed configuration. In such cases, 
negative normal displacements at an integration 
point lead to use of the compressive stiffness of the constitutive model.

%------------------------------------------------------------------------------
\subsection{Node and Integration Point Ordering}
%------------------------------------------------------------------------------
The stiffness matrix and internal forces of the interface element 
are evaluated using numerical quadrature over the reference surface.  For geometrically
linear elements, all computations take place using the
initial (undeformed) configuration -- the  $S^-$ and $S^+$ surfaces  
remain identical  and define the unique reference surface. For geometrically nonlinear
elements, integration takes place over the middle surface, unless
the user overrides this default in the presence of symmetry planes (see subsequent
section). Figures  \ref{fig:quad_def},  
\ref{fig:tri6_def}, and  \ref{fig:tri12_def} 
show the node ordering and orientation of the parametric axes $(\xi_1,\xi_2)$ 
for the quadrilateral element and 
or $(\xi_1,\xi_2, \xi_3)$ for the triangular elements.  

 Two integration schemes are available for the 
 \ti{inter\_8} element: (a) $2 \times 2$ Gauss 
 quadrature rule '2x2gs', or (b)
 a $2 \times 2$ Newton-Cotes rule '2x2nd' 
 which has sampling points located at 
 the element nodes. Figure \ref{fig:quad_def} shows the locations of integration points 
 in parametric coordinates for these integration schemes. Experience shows the best choice of 
 an integration rule depends on features of the cohesive constitutive model for the 
 interface. (see [\ref{R:SDB1993}] for an extensive
 discussion of this issue).
 For a linear-elastic, cohesive model with very high stiffness, the 
 Newton-Cotes scheme provides better numerical stability and accuracy. 
 For a nonlinear traction-separation model, the Newton-Cotes scheme 
 may lead to oscillatory displacement profiles behind a crack front. The 
 Gauss integration scheme eliminates this undesirable response. 
 
The  rules and the sampling points for numerical integration 
over the triangular elements \ti{trint6} and \ti{trint12} are shown 
in Figs. \ref{fig:tri6_def}, and  \ref{fig:tri12_def}, respectively. Again, some
experimentation may be required to set the most suitable integration scheme
for (very) stiff linear cohesive models and for the various nonlinear
cohesive models available.
 
%
%------------------------------------------------------------------------------
\subsection{Ensuring Proper Element Orientation}
%------------------------------------------------------------------------------
The interface-cohesive elements consist of two, initially coincident (or very
nearly coincident) surfaces 
with a key role played in the formulation by the \ti{normal} direction to the surfaces. 
Consequently, users must specify the incidences for the element 
(node ordering) to insure proper orientation of the normal. 
The following rules apply:
%
\squishlist
\item The incidence list for the element must have \ti{all} 
nodes of one surface listed first, followed by those nodes of the other surface.
\item The nodes on the first surface should appear in an order 
such that the normal vector to the surface, defined by a right-hand 
rule, points \ti{away} from the corresponding face of the attached 
solid element (or symmetry plane) and \ti{into} the interface element.
\item The first node of the second surface must coincide with 
the first node of the first surface. The remaining nodes of the second 
surface should follow the same order as that of the first.
\squishend

With reference to \ti{inter\_8} element
shown in Fig. \ref{fig:inter_def}, the ordering rule can be interpreted as follows. 
Consider 2 vectors: one constructed using the first node listed in 
the element incidences to the second 
node in the list (vector $ab$); the second vector 
from the first node to the last node in 
the incidence list (vector  $ad$). The cross product of these two 
vectors ($ab \times ad$) must point \ti{away} from the attached solid (with 
surface \ti{a-b-c-d} in the figure) or \ti{away} from a symmetry plane. 
This interpretation also holds 
for the triangular interface elements. 
%
\begin{figure}[htb]
\begin{center}
\includegraphics[trim=1.4in 2.50in 0.in 0.85in, clip=true,scale=0.8,angle=0]{fig_4.pdf} 
\caption{{\small Fig. \thefigure: Local node numbering for \ti{trint12}. 
Natural (area) coordinates for the element corner nodes and 
location of integration points are listed.}\label{fig:tri12_def}}
%
\end{center}
\end{figure}
%------------------------------------------------------------------------------
\subsection{Special Procedure for Mode I Loading (Geometric Nonlinear)}
%------------------------------------------------------------------------------
Consider mode I crack growth over a symmetry plane in a 
finite element mesh. Interface-cohesive elements specified over the crack 
plane define the remaining (uncracked) ligament and constrain 
the crack to grow only in the symmetry plane, \eg planar surface breaking or
buried cracks, and planar through-thickness cracks. Interface-cohesive elements
readily model non-uniform extension along the front of a surface/buried crack or
tunneling in through-thickness  crack.

%
\begin{figure}[htb]  
\begin{center}
\includegraphics[trim=0.84in 4.1in 0.in 1.3in, clip=true,scale=0.88,angle=0]{fig_5.pdf} 
\caption{{\small Fig. \thefigure: By default, the local coordinate 
system for a geometrically nonlinear cohesive element 
is defined with respect to the mid-plane (surface 'middle') of the deformed 
configuration. For
symmetric models with the geometric nonlinear formulation, specify
which surface 'top' or 'bottom' of the interface element connects to the
adjacent solid element. }\label{fig:top_bottom_def}}
%
\end{center}
\end{figure}

In such a model, nodes of an interface-cohesive element either 
on the ``bottom" ( $S^-$) surface (nodes $1\rightarrow nnode$) or the
``top'' ($S^+$) surface (nodes 
$nnode+1 \rightarrow  2 \times nnode$) are incident (connected)
to the adjacent solid elements. The surface of an interface-cohesive
element connected to an adjacent solid element is referred to  
as the \ti{reference}
surface in this discussion.
Those nodes on the symmetry plane 
must have user-specified constraints to enforce zero displacements 
normal to the crack plane over the remaining ligament (other constraints on 
displacement components of nodes in the symmetry
plane may be needed based on mesh layout and model features).

During mode I growth, the top and bottom faces of the interface elements 
gradually separate. In small displacement analyses (geometrically
linear), all computations for 
the interface elements occur in the initial (undeformed) configuration. The 
top and bottom surfaces are geometrically identical in the initial
configuration. This is also the reference surface for all subsequent 
computations.  No issues arise in the definition and
computations of models with symmetry planes. 

For general meshes and loading with large displacement solutions 
(geometrically nonlinear), the current element 
geometry is defined (by default) using the deformed \ti{middle} 
surface as indicated in Fig. \ref{fig:top_bottom_def}. This
surface is constructed from the
average of deformed coordinates for each pair of top and bottom surface
nodes. But with models having an interface element connected to
a symmetry plane, computations must adopt the 
reference surface as the one  
connected to the adjacent solid element. The interface nodes on 
this surface experience
displacement changes caused by Poisson effects in the solid 
element. WARP3D cannot determine which 
surface of an interface element is connected to the adjacent solid element
without additional user input. The user must specify the reference surface 
as either \ti{bottom} (nodes: $1 \rightarrow nnode$) or \ti{top }(nodes: 
$nnode+1 \rightarrow 2 \times nnode$) depending on which surface of the 
interface element is connected to the adjacent solid element as defined by the 
element orientation in the mesh through the incidence list. 
The keywords \ti{surface 'top'} or \ti{surface 'bottom'} in the element
property values are used to indicate the 
reference surface for interface elements with the geometrically nonlinear
formulation. This option is ignored for geometrically linear elements.

In view of the above discussion, we recommend the following procedure to 
model mode I crack growth: (1) constrain the displacement normal to 
the symmetry plane as necessary to define the initial crack size/shape
or other model features,
(2) for geometrically \ti{nonlinear} analyses, specify the surface 
of the interface elements connected to the adjacent solid elements
(\ti{surface 'top' }or \ti{surface 'bottom'}), (3) the cohesive material
must have a \ti{non-zero} resistance to sliding displacement jumps to preclude
singularities in the equilibrium equations, and (4) properties of the 
cohesive material (see Section 3.8)
may require adjustment to reflect the symmetry conditions (\ie only
one-half of the opening displacement is computed by the interface element).
When mode I growth is modeled with a finite element mesh 
that does not take advantage of symmetry conditions, this 
issue of reference surfaces never arises. 
%=================================================
\begin{table}[htb]	
\centering
{
\setlength{\extrarowheight}{4pt}
\begin{tabular}{ | l | c |  c | c | }
\hline
Element property & Keyword & Mode & Default Value \\
\hline \hline
Geometrically \ti{linear} formulation &linear &logical  &True  \\ \hline
Geometrically \ti{nonlinear} formulation &nonlinear &logical  &False  \\ \hline
Reference surface (\ti{nonlinear} formulation only)	& surface& string	& 'middle'$^\dag$ \\ \hline
Integration rule for $\bmf{K}_T$  and $\bmf{f}$: inter\_8 & order & string & '2x2gs'$\,^*$ \\
 &&& '2x2nd' \\
  \hline
 Integration rule for $\bmf{K}_T$  and $\bmf{f}$: trint6 & order & string & '1pt\_rule' \\
  & & & '3pt\_rule'$\,^*$\\ 
 & & & '3mpt\_rule' \\ \hline
 Integration rule for $\bmf{K}_T$  and $\bmf{f}$: trint12 & order & string & '3pt\_rule' \\
   & & & '3mpt\_rule' \\ 
 & & & '4pt\_rule' \\
 & & & '6pt\_rule' \\
 & & & '7pt\_rule'$\,^*$ \\ \hline
 Output of tractions and displacement jumps at center & center& logical & False \\ \hline
\end{tabular}}

\small {
\vspace{0.1in}\dag When interface elements are connected to a symmetry plane
in finite element models with the \ti{nonlinear} formulation, the
reference surface must be declared either `top' or `bottom' and defines which element
surface connects to the adjacent solid element.

$^*$default integration order }
\normalsize
	
%
\caption{Table \thesection.1 
Properties for the interface-cohesive elements.}
\label{table:elem_props}
\end{table}


%------------------------------------------------------------------------------
\subsection{Element Properties}
%------------------------------------------------------------------------------
Table 3.3.1 summarizes the user-assignable values 
that control the response of interface elements. Available integration schemes
for each element type are listed with (*) indicating the default
value. 
%------------------------------------------------------------------------------
\subsection{Output}
%------------------------------------------------------------------------------

Printed strain-stress results may be obtained at the integration points (default), 
or at the parametric center-point of the element. For this element, stresses refer to the
\ti{tractions} ($T_{t1}, T_{t2}, T_{n}$) and strains refer to the \ti{displacement jumps} ($\Delta_{t1},
\Delta_{t2},\Delta_n$) both in the local mid-plane 
coordinate system ($\bmf{t}_1, \bmf{t}_2,\bmf{ n}$) shown in the
figures of this section.

To provide a more complete description of results, additional output quantities are printed based on the 
type of cohesive model associated with the element -- 
cohesive models \ti{linear\_intf},  \ti{exp1\_intf} and \ti{ppr} described in Section 3.8. 
Tables below summarize the output values.

%=================================================
\begin{table}[htb]	
\centering
\small
{
\setlength{\extrarowheight}{4 pt}
\begin{tabular}{ | p{1.0in} | >{\centering\arraybackslash}m{1.5in} | | p{1.0in} | >{\centering\arraybackslash}m{1.5in} |  }
 \hline
  Stress label & Quantity& Strain label& Quantity \\
\hline \hline
\hspace{0.25in}shear-1 & $T_{t1}$& \hspace{0.25in}shear-1 & $\Delta_{t1}$ \\
\hspace{0.25in}shear-2 & $T_{t2}$ & \hspace{0.25in}shear-2 & $\Delta_{t2}$ \\ 
\hspace{0.25in}shear & $T_s = \sqrt{T_{t1}^2 + T_{t2}^2}$ & \hspace{0.25in}shear & $ \Delta_s = \sqrt{\Delta_{t1}^2 + \Delta_{t1}^2}$ \\ 
\hspace{0.25in}normal & $T_n$ & \hspace{0.25in}normal & $\Delta_n$ \\

\hspace{0.25in}gamma &$ \Gamma = 1/2\ ( \bmf{T}^T \cdot \bmf{\Delta})$ &  & \\ 
\hspace{0.25in}gamma\_ur & $\equiv 0$& &  \\ \hline 
\end{tabular}}
%
\begin{minipage}[c]{6in}
\caption{\\ \small Table \thesection.2
Output values for interface-cohesive elements using 
cohesive material option \ti{linear\_intf}.  }
\end{minipage}
\label{table:linear_model_output}
\end{table}
\normalsize
%=================================================

%=================================================
\begin{table}[htb]	
\centering
\small
{
\setlength{\extrarowheight}{4 pt}
\begin{tabular}{ | p{1.0in} | >{\centering\arraybackslash}m{1.5in} | | p{1.0in} | >{\centering\arraybackslash}m{1.5in} |  }
 \hline
  Stress label & Quantity& Strain label& Quantity \\
\hline \hline
\hspace{0.25in}shear-1 & $T_{t1}$& \hspace{0.25in}shear-1 & $\Delta_{t1}$ \\
\hspace{0.25in}shear-2 & $T_{t2}$ & \hspace{0.25in}shear-2 & $\Delta_{t2}$ \\ 
\hspace{0.25in}shear & $T_s = \sqrt{T_{t1}^2 + T_{t2}^2}$ & \hspace{0.25in}shear & $ \Delta_s = \sqrt{\Delta_{t1}^2 + \Delta_{t1}^2}$ \\ 
\hspace{0.25in}normal & $T_n$ & \hspace{0.25in}normal & $\Delta_n$ \\

\hspace{0.25in}eff & $\bar T = \sqrt{\beta^{-2} T_s^2 + T_n^2}$  &\hspace{0.25in}eff & $\bar \Delta = \sqrt{\beta^2 \Delta_s^2 + \Delta_n^2} $ \\
\hspace{0.25in}eff/peak & $\bar T / \bar T_p$ &\hspace{0.25in}eff/peak & $\bar \Delta / \bar \Delta_p$  \\ \hline

\hspace{0.25in}gamma &$  \Gamma = \int_0^{\bmf{\Delta}} \bmf{T}^T d\bmf{\Delta}$ & & \\ 
\hspace{0.25in}gamma\_ur & $\Gamma - 1/2\ ( \bmf{T}^T \cdot \bmf{\Delta})$& &  \\ \hline 
\end{tabular}}
%
\begin{minipage}[c]{6in}
\caption{\\ \small Table \thesection.3
Output values for interface-cohesive elements using 
cohesive material option \ti{exp1\_intf}.  See Section 3.8 for definitions
of $\beta$, $\bar T_p$ and $\bar \Delta_p$.}
\end{minipage}
\label{table:linear_model_output}
\end{table}
\normalsize
%=================================================

%=================================================
\begin{table}[htb]	
\centering
\small
{
\setlength{\extrarowheight}{4 pt}
\begin{tabular}{ | p{1.0in} | >{\centering\arraybackslash}m{1.5in} | | p{1.0in} | >{\centering\arraybackslash}m{1.5in} |  }
 \hline
  Stress label & Quantity& Strain label& Quantity \\
\hline \hline
\hspace{0.25in}shear-1 & $T_{t1}$& \hspace{0.25in}shear-1 & $\Delta_{t1}$ \\
\hspace{0.25in}shear-2 & $T_{t2}$ & \hspace{0.25in}shear-2 & $\Delta_{t2}$ \\ 
\hspace{0.25in}shear & $T_s = \sqrt{T_{t1}^2 + T_{t2}^2}$ & \hspace{0.25in}shear & $ \Delta_s = \sqrt{\Delta_{t1}^2 + \Delta_{t1}^2}$ \\ 
\hspace{0.25in}normal & $T_n$ & \hspace{0.25in}normal & $\Delta_n$ \\

\hspace{0.25in}shr/peak & $T_s / T_{t-p}$ &\hspace{0.25in}shr/peak & $\Delta_s / \Delta_{t-p}$ \\ 

\hspace{0.25in}nml/peak & $T_n / T_{n-p}$ & \hspace{0.25in}nml/peak & $\Delta_n / \Delta_{n-p}$ \\
 

\hspace{0.25in}gamma &$  \Gamma = \int_0^{\bmf{\Delta}} \bmf{T}^T d\bmf{\Delta}$ & & \\ 
\hspace{0.25in}gamma\_ur & $\Gamma - 1/2\ ( \bmf{T}^T \cdot \bmf{\Delta})$& &  \\ \hline 
\end{tabular}}
%
\begin{minipage}[c]{6in}
\caption{\\ \small Table \thesection.4
Output values for interface-cohesive elements using 
cohesive material option \ti{ppr}. See Section 3.8 for definitions of
$T_{t-p}$, $T_{n-p}$, $\Delta_{t-p}$, and  $\Delta_{n-p}$.}
\end{minipage}
\label{table:linear_model_output}
\end{table}
\normalsize
%=================================================

The center-point values of  are the simple numerical average of 
integration point values. Binary packet quantities when requested 
are those shown in these tables. See also Appendix E.

When the interface element is made \ti{killable} in the analysis 
through the associated, cohesive material model (Section 3.8) and crack growth 
procedures (Section 5.4), the crack growth processors output a summary 
of similar quantities for each interface element at the beginning 
of each load step.
%
%------------------------------------------------------------------------------
\subsection{Examples}
%------------------------------------------------------------------------------
The following examples illustrate the specification 
of interface elements in a model.
\small \begin{verbatim}
      structure ct
   
      material background type mises  
           properties ....
   
      material interface type cohesive 
          properties ....
             .
             .   
      number of elements 36000 element 60000 
 
      elements 
      
         1-34000 type l3disop linear material background ...
  
         34001-36000 type inter_8 linear material interface order '2x2gs'
                                     
\end{verbatim}\normalsize
And,
\small \begin{verbatim}
      structure mt
   
      material background type mises  
           properties ....
   
      material interface type cohesive 
          properties ....
             .
             .   
      number of elements 100000 element 60000 
 
      elements 
      
        1-45000 type tet10 nonlinear material background ...
        
        45001-54000 type trint12 nonlinear material interface,
                order '7pt_rule' surface 'middle' center

\end{verbatim}\normalsize
%

 
%*****************************************************
\subsection {References}
%*****************************************************
\small
[\refstepcounter{sectrefs}\label {R:OP1999}\thesectrefs]~M. Ortiz 
and A. Pandolfi. Finite-Deformation Irreversible Cohesive Elements
for Three-Dimensional Crack-Propagation Analysis. 
\textit{International Journal for Numerical
Methods in Engineering}, 44, pp. 1267-1282, 1999.

\medskip
\noindent[\refstepcounter{sectrefs}\label {R:SL2004}\thesectrefs]~J. Segurado and 
J. Llorca. A New Three-Dimensional Interface Finite Element to Simulate Fracture in Composites.
\textit{International Journal of Solids and Structures}, 41, pp. 2977-2993, 2004.

\medskip
\noindent[\refstepcounter{sectrefs}\label {R:SDB1993}\thesectrefs]~J.C.J. Schellekens and R. De Borst. On the Numerical Integration of Interface Elements.\textit{International Journal for Numerical
Methods in Engineering}, 36, pp. 43-66, 1993.

\end{document}

