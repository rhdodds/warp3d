
\documentclass[11pt]{report}
\usepackage{geometry}
\geometry{letterpaper}

%-----------------------------------

%---------------------------------------------
\setlength{\textheight}{630pt}
\setlength{\textwidth}{450pt}
\setlength{\oddsidemargin}{14pt}
\setlength{\parskip}{1ex plus 0.5ex minus 0.2ex}


%----------------------------------------
\usepackage{amsmath}
\usepackage{layout}
\usepackage{color}
\usepackage{array}
\usepackage{bm}
\usepackage{graphicx}
\usepackage{longtable}
%
\usepackage[comma,numbers,sort,compress]{natbib} % numbered, sequential refs 
\renewcommand{\bibsection}{\subsection{\bibname}} %  makes bib a numbered subsection 
\renewcommand{\bibname}{References} % use References not Bibliography for section name
%----------------------------------------------

\usepackage[us,12hr]{datetime}
\usepackage{fancyhdr} \pagestyle{fancy}
\setlength\headheight{15pt}
\lhead{\small{User's Guide - \textit{WARP3D}}}
\rhead{\small{\textit{Contact Tips}}}
\fancyfoot[L] {\small{\  Updated:  \today\ at \currenttime}}

%\fancyfoot[L] {\small{\textit{Chapter {\thechapter}}\ \   (Updated: 8-10-2015. 9:00 AM)}}
\fancyfoot[C] {\small{\thesection-\thepage}}
\fancyfoot[R] {\small{\textit{Contact Procedures}}}

%---------------------------------------------------
\usepackage{graphicx}
% \usepackage[labelformat=empty]{caption}
\numberwithin{equation}{section}

%---------------------------------------------
%     --- make section headers in helvetica ---
%
\usepackage{sectsty} 
\usepackage{xspace}
\allsectionsfont{\sffamily} 
\sectionfont{\large}
\usepackage[small,compact]{titlesec} % reduce white space around sections
%---------------------------------------------->
%
%
%   which fonts system for text and equations. with all commented,
%   the default LaTex CM fonts are used
%
%
\frenchspacing
%\usepackage{pxfonts}  % Palatino text 
%\usepackage{mathpazo} % Palatino text
%\usepackage{txfonts}
\usepackage[stable]{footmisc}

%---------  local commands ---------------------

\newcommand{\ttt} {\texttt}  %typewriter text
\newcommand{\tb} {\textbf}
\newcommand{\nf} {\normalfont}
\newcommand{\df} {\dotfill}
\newcommand{\nin} {\noindent}
\newcommand{\bmf } {\boldsymbol }  %bold math symbol
\newcommand{\bsf } [1]{\textrm{\textit{#1}}\xspace}
\newcommand{\ul} {\underline}
\newcommand{\hv} {\mathsf}   %helvetica text inside an equation
\newcommand{\eg}{\emph{e.g.},\xspace}
\newcommand{\ie}{\emph{i.e.},\xspace}
\newcommand{\ti}{\emph}
\newcommand{\bardelta}{\bar \delta}
\newcommand{\barDelta}{\bar \Delta}
\newcommand{\veps}{\varepsilon}
\newcommand{\noi}{\noindent}

\newenvironment{offsetpar}[1]%
{\begin{list}{}%
         {\setlength{\leftmargin}{#1}}%
         \item[]%
}
{\end{list}}

%
%
%        optional definition for bullet lists which
%        reduces white space.
%
\newcommand{\squishlist}{
 \begin{list}{$\bullet$}
  { \setlength{\itemsep}{0pt}
     \setlength{\parsep}{3pt}
     \setlength{\topsep}{3pt}
     \setlength{\partopsep}{0pt}
     \setlength{\leftmargin}{1.5em}
     \setlength{\labelwidth}{1em}
     \setlength{\labelsep}{0.5em} } }

\newcommand{\squishlisttwo}{
 \begin{list}{$\bullet$}
  { \setlength{\itemsep}{0pt}
     \setlength{\parsep}{0pt}
    \setlength{\topsep}{0pt}
    \setlength{\partopsep}{0pt}
    \setlength{\leftmargin}{2em}
    \setlength{\labelwidth}{1.5em}
    \setlength{\labelsep}{0.5em} } }

\newcommand{\squishend}{
  \end{list}  }
%


%-------------------------------------
\newcounter{sectrefs}
\setcounter{sectrefs}{0}
\setcounter{chapter}{6}
\setcounter{section}{3}
\setcounter{figure}{0}
\renewcommand{\thefigure}{\thesection.\arabic{figure}}
%
%--------------------------------------
%
%
%		Editing
\newcommand{\medit}{\color{blue}}
\newcommand{\eedit}{\color{black}}
\definecolor{gold}{rgb}{0.83, 0.69, 0.22}
%
%
%              start document 
%              ==========
%
%
\begin{document}

\section{Tips for Analyses Using Contact}

Contact analyses within an implicit solution framework
can be very challenging; convergence of the global Newton iterations
is by no means 
guaranteed, and may be difficult to achieve. Also, improperly defined 
contact surfaces cause severe problems and may prove formidable to find. 
However, a variety of solution strategies and modeling techniques can 
significantly improve the likelihood of a successful analysis. This 
section presents some tips and suggestions for improving the 
performance of contact analyses in WARP3D. 

\noindent \bf{Choosing a penalty parameter}\rm

\noi Choosing an appropriate penalty parameter (contact stiffness) 
is one of the most important factors in the success of a contact analysis. 
Too small of a contact stiffness allows excessive penetration, 
while too large a value causes convergence 
problems when iterative solvers are used for the linearized
equilibrium equations. Also, the choice 
of a good penalty parameter depends on the material model, the element 
type, mesh size, the type of loading, and even the contact surface geometry. 
A few guidelines:
\small
\squishlist
\item The contact stiffness should be several orders of magnitude 
greater than the local stiffness of the model at the point of 
contact. To evaluate the model stiffness at a node, run an 
analysis with a unit force at the node in the direction
normal to the anticipated 
contact surface. The local stiffness of the structure at the node 
is simply the force divided by the resulting displacement. A contact 
stiffness many magnitudes more than this value may cause 
ill-conditioning and large errors in the results. 
\item To assess the adequacy of an appropriate penalty parameter 
in a new analysis, try a small value first, then re-run the analysis 
several times, each time increasing the penalty parameter. A good 
choice is a penalty parameter which maintains strong convergence 
properties, but which would cause convergence problems if increased 
somewhat. Once a successful parameter is found, analyses with 
similar properties (loading, material, etc.) can use a similar value.
\item Experience indicates that curved contact surfaces, such as 
clinders and spheres, need a lower contact stiffness than rectangular contact surfaces.
\squishend
\normalsize
\noindent \bf{Convergence of the first step at contact}\rm

\noi Achieving global convergence of contact analyses on the 
first step in which contact takes place can be especially challenging. 
Oscillations may occur where the contact springs force penetrating 
nodes completely out of contact. Without any contact to restrain them, 
the nodes penetrate on the next Newton iteration, repeating the cycle 
and impeding convergence. To avoid this problem:
\small
\squishlist
\item Reduce the size of the first step in which contact 
occurs; this is particularly necessary for moving contact surfaces. 
An effective size for the first step with contact may need to 
be $10^{-3}$ of the step size used for the remainder of 
the analysis. After the first step converges and contact 
initiates successfully, the step size can increase significantly. 
A reduced loading for a step or two may be necessary 
each time the contact surface experiences a significant 
shift in direction. Re-definition of the time increment per 
load step provides an effective means of reducing the 
movement rate for a contact surface; see Section 2.10 
for information on the time step increment. Make sure 
that any explicitly defined, non-zero constraints also reflect 
the change in movement rate.
\item Decrease the contact stiffness by several 
orders of magnitude for the first load step with contact. Then 
redefine the contact surface with a much higher stiffness after a step or two. 
\item If the analysis contains a contact surface with a non-zero 
velocity, insure that the time increment per load step is reasonable. 
Many convergence problems arise from errors specified 
time increments. See Section 2.10 for information on the time step parameter.
\squishend
\normalsize

\noindent \bf{Improving general convergence}\rm

\noi A number of techniques can improve general convergence 
of contact analyses. A few are listed below:
\small
\squishlist
\item Always use the nonlinear solution option \ti{line search} (see Section 2.10)
\item Take smaller computational load(time) steps.
\item Use a smaller contact stiffness.
\item Models which have symmetry planes are significantly more robust than similar models
 which represent an entire structure (and thus often have more rigid body motion).
\item Make sure that all rigid body motions are prevented using explicit constraints. 
Contact enforcement using the penalty method is equivalent to supplying force 
boundary conditions, which are typically less robust than explicit constraint 
(displacement) boundary conditions.
\item If elimination of rigid body motions is not possible, try including mass 
in the analysis with small time steps. The addition of inertia can help 
stabilize the analysis.
\item If the analysis involves moving contact surfaces, then application of 
explicit (displacement) constraints on nodes which move along with the contact surface 
can improve convergence significantly. See the pin-loaded C(T) specimen 
in the examples section. 
\item Incorrect specification of contact surfaces can cause hidden 
difficulties; see Section 6.3 for more details on contact input and possible solutions.
\item Turn adaptive load reduction off (adaptive off in the nonlinear 
solution parameters command).
\item Extrapolation of the nodal displacements at the start of new load (time) step 
(Section 2.10) can cause convergence
difficulties if new contact detection occurs -- try turning this option 
off if other measures fail to improve convergence
\squishend
\normalsize

\end{document}
