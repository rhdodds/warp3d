%
\documentclass[10pt]{report}
\usepackage{geometry} 
\geometry{letterpaper}
%
%
%   margins and inter-paragraph spacing
%
%---------------------------------------------
\setlength{\textheight}{630pt}
\setlength{\textwidth}{450pt}
\setlength{\oddsidemargin}{14pt}
\setlength{\parskip}{1ex plus 0.5ex minus 0.2ex}


%----------------------------------------
\usepackage{amsmath}
\usepackage{layout}
\usepackage{color}
%\usepackage{hyphenat}
%\usepackage{listings}

%----------------------------------------------
%
%          --- header and footer contents ---
%
\usepackage{fancyhdr} \pagestyle{fancy}
\setlength\headheight{15pt}
\lhead{\small{User's Guide - \ti{WARP3D}}}
\rhead{\small{\ti{Binary Packets File}}}
\fancyfoot[L] {\small{\ti{Appendix F}\ \   (Updated: 3/15/2013)}}
\fancyfoot[C] {\small{F.\thepage}}
\fancyfoot[R] {\small{\ti{Binary Packets File}}}



%---------------------------------------------------
\usepackage{graphicx}
\usepackage[labelformat=empty]{caption}
\numberwithin{equation}{section}

%---------------------------------------------
%
%     --- make section headers in helvetica ---
%
\frenchspacing
\usepackage{sectsty} 
\usepackage{xspace}
\allsectionsfont{\sffamily} 
\sectionfont{\large}
\usepackage[small,compact]{titlesec} % reduce white space around sections
%
%
%   which fonts system for text and equations. with all commented,
%   the default LaTex CM fonts are used
%
%
%\usepackage{pxfonts}  % Palatino text 
%\usepackage{mathpazo} % Palatino text
%\usepackage{txfonts}

%----------------------------------------------

%   ---  local commands ---

\newcommand{\bmf } {\boldsymbol }
\newcommand{\bsf } [1]{\textrm{\ti{#1}}\xspace}
\newcommand{\HRule}{\rule{\linewidth}{0.5mm}}
\newcommand{\patwarp}{\ti{patwarp\xspace}}
\newcommand{\eg}{\ti{e.g.},\xspace}
\newcommand{\ie}{\ti{i.e.},\xspace}
\newcommand{\ul} {\underline}
\newcommand{\hv} {\mathsf}   %helvetica text inside an equation
\newcommand{\ti}{\emph}

%
%        optional definition for bullet lists which
%        reduces white space.
%
\newcommand{\squishlist}{
 \begin{list}{$\bullet$}
  { \setlength{\itemsep}{0pt}
     \setlength{\parsep}{3pt}
     \setlength{\topsep}{3pt}
     \setlength{\partopsep}{0pt}
     \setlength{\leftmargin}{1.5em}
     \setlength{\labelwidth}{1em}
     \setlength{\labelsep}{0.5em} } }

\newcommand{\squishlisttwo}{
 \begin{list}{$\bullet$}
  { \setlength{\itemsep}{0pt}
     \setlength{\parsep}{0pt}
    \setlength{\topsep}{0pt}
    \setlength{\partopsep}{0pt}
    \setlength{\leftmargin}{2em}
    \setlength{\labelwidth}{1.5em}
    \setlength{\labelsep}{0.5em} } }

\newcommand{\squishend}{
  \end{list}  }
%
%
%     --- set chapter number or appendix letter ---
%
%-------------------------------------
\def\thechapter  {\Alph{chapter}}
\setcounter{chapter}{6} % means 6th alphabet letter
\newcounter{sectrefs}
%
%
%
%              start document 
%              ==========
%
%
\begin{document}
\LARGE
\hfill
\textbf{Appendix F}
\rule[0.15in]{450pt}{0.5mm}
\LARGE
\begin{flushright}
 \textbf{
{\fontfamily{phv}\selectfont Binary Packets File}}
\end{flushright}
\normalsize
%
%---------------------------------------------------------------------------
%        section
%---------------------------------------------------------------------------
%
Section 2.12.2 outlines the concepts of the \ti{binary packets file} and defines 
commands to control the type and amount of data
written to the file during WARP3D execution. This 
appendix describes each of the available packets. The source
code for a complete Fortran
program illustrate reading and processing a binary packets file
is included in the WARP3D distribution.
%
%---------------------------------------------------------------------------
%        section
%---------------------------------------------------------------------------
%
\section{File Characteristics }

WARP3D creates and writes the packets file using the standard capabilities in Fortran 
for sequential, unformatted I$/$O. The file thus has the structure of \ti{logical} records, 
where each read$/$write statement in Fortran processes at least one logical 
record. Logical records in a file may have variable length (number of bytes)
as required to store the data located 
in the record. The Fortran statement to open the binary packets file is:
\begin{verbatim}
        open(unit=packet_file_no, file=packet_file_name,   
       &     access='sequential',form='unformatted',status='old' )
\end{verbatim}
Integer values in the binary file are all 32-bits. All floating point values are
64-bit (\ie double precision).

The binary packets 
file produced on one hardware platform may or may not be readable on 
another platform. For example, files produced on a Linux system
likely will not be readable on a Windows system (and vice-versa) even though
WARP3D is built with the Intel Fortran compiler system on
both platforms. Files produced
on two different Linux systems (\eg RedHat and SuSE) are most often 
compatible. Linux and OS X binary files are compatible.

Fortran unformatted files have the record
lengths stored within the file. Each logical record
may have many physical records as dictated by features of the storage device
and the features of the runtime, Fortran I/O system on that platform. The
existence of physical records is fully hidden from Fortran codes as are details
of the additional data in the file to identify-separate logical records. C and C++ 
unformatted files \ti{do not} have this notion of logical records -- the file is just
a flat linear sequence of bytes. Consequently, C and C++ programs written for post-processing
the packets file must handle the additional file data included by the Fortran I$/$O system
to define logical record lengths.

%
%---------------------------------------------------------------------------
%        section
%---------------------------------------------------------------------------
%
\section{Packet Header Record}
Each packet has an identical header record with the following data values:
\squishlist
\item the packet type:  integer
\item number of additional \ti{logical} records of data for the packet:  integer
\item the load step number:  integer
\item the Newton iteration number:  integer
\squishend

%
%---------------------------------------------------------------------------
%        section
%---------------------------------------------------------------------------
%
\section{Packet Type 1: \ti{nodal displacements}}
The packet contains the $u,v,w$ global displacements for each node 
appearing in the $<$integerlist$>$ of the \ti{output packets 
displacements nodes $<$integerlist$>$ } command. 
Results for each node appear on a separate logical record. The number of additional 
packet records on the header record thus indicates the number of 
nodes with results in the packet.
\squishlist
\item the node number:  integer
\item $u$ displacement:  floating point
\item $v$ displacement:  floating point
\item $w$ displacement:  floating point
\squishend

%
%---------------------------------------------------------------------------
%        section
%---------------------------------------------------------------------------
%
\section{Packet Type 2: \ti{nodal velocities}}
The packet contains the $\dot u, \dot v, \dot w$ global velocities for each 
node appearing in the $<$integerlist$>$ of the \ti{output packets 
velocities nodes $<$integerlist$>$ } command. Results for each 
node appear on a separate logical record. The number of additional 
packet records on the header record thus indicates the number 
of nodes with results in the packet.
\squishlist
\item the node number:  integer
\item $\dot u$ velocity:  floating point
\item $\dot v$ velocity:  floating point
\item $\dot w$ velocity:  floating point
\squishend

%
%---------------------------------------------------------------------------
%        section
%---------------------------------------------------------------------------
%
\section{Packet Type 3: \ti{nodal accelerations}}
The packet contains the $\ddot u, \ddot v, \ddot w$ global accelerations for 
each node appearing in the $<$integerlist$>$ of the \ti{output packets 
accelerations nodes $<$integerlist$>$ } command. Results for each node 
appear on a single logical record. The number of additional packet records 
on the header record thus indicates the number of nodes with results in the packet.
\squishlist
\item the node number:  integer
\item $\ddot u$ acceleration:  floating point
\item $\ddot v$ acceleration:  floating point
\item $\ddot w$ acceleration:  floating point
\squishend

%
%---------------------------------------------------------------------------
%        section
%---------------------------------------------------------------------------
%
\section{Packet Type 4: \ti{nodal reactions}}
The packet contains the $X, Y, Z$ global reactions for each node appearing 
in the $<$integerlist$>$ of the \ti{output packets reactions 
nodes $<$integerlist$>$ } command. Results for each node appear on 
a single logical record. The last logical record of the packet provides the 
algebraic sums for each reaction component for the corresponding nodes. 
The number of additional packet records on the header record thus indicates 
the number of nodes+1 with results in the packet.  The number of additional 
packet records is 1 when the \ti{totals only} option is used within 
the \ti{output} command.
\squishlist
\item the node number:  integer
\item $R_x$ reaction:  floating point
\item $R_y$ reaction:  floating point
\item $R_z$ reaction:  floating point
\squishend
\noindent Last record of packet
\squishlist
\item $\sum R_x$ reactions for nodes in packet:  floating point
\item $\sum R_y$ reactions for nodes in packet:  floating point
\item $\sum R_z$ reactions for nodes in packet:  floating point
\squishend

%
%---------------------------------------------------------------------------
%        section
%---------------------------------------------------------------------------
%
\section{Packet Type 5: \ti{gurson elements w/o adaptive load control}}
At the end of each load step in analyses which have ``gurson" elements for 
crack growth, WARP3D provides a printed table of key values for each 
gurson element. A ``gurson" element is any solid
element in the model which has the gurson plasticity model and
which is marked for possible extinction (\ie \ti{killable}) in the 
material properties. The same table is written concurrently as a packet if the 
user has authorized packet file operations. The number of packet records 
equals the number of gurson elements appearing in the printed list following 
each load step. A gurson element appears in the list when the current 
porosity ($f$) exceeds the initial porosity ($f_0$). This packet outputs data for 
gurson elements when the adaptive load control option is not employed 
in the \ti{crack growth parameter }input, \ie \ti{adaptive 
load control off }(default setting in WARP3D when not specified). 
The packet record for each gurson element provides the following data.
\squishlist
\item the element number:  integer
\item $f_0$:  initial porosity:  floating point
\item $f$:  current porosity (averaged over element integration points):  floating point
\item $\bar \varepsilon^p$:  average plastic strain in the matrix material:  floating point
\item $\bar \sigma$:  average equivalent stress in the matrix material:  floating point
\item $\sigma_{m}$:  average macroscopic mean stress:  floating point
\item the status of the mean stress: integer. ( =0, the mean 
stress increased during the current load step;
=1, the mean stress decreased during the current load step)
\item $\sigma_{e}$:  average macroscopic mises stress:  floating point
\item the status of the mises stress:  integer.
( =0, the mises stress increased during the current load step; 
=1, the mises stress decreased during the current load step)
\squishend
%
%
%---------------------------------------------------------------------------
%        section
%---------------------------------------------------------------------------
%
\section{Packet Type 6: \ti{gurson elements: adaptive load control}}
At the end of each load step in analyses which have gurson 
elements for crack growth, WARP3D provides a printed table of key values 
for each gurson element. A ``gurson" element is any solid
element in the model which has the gurson plasticity model and
which is marked for possible extinction (\ie \ti{killable}) in the 
material properties. The same table is written concurrently as a packet 
if the user has authorized packet file operations. The number of packet records 
equals the number of gurson elements appearing in the printed list following 
each load step. A gurson element appears in the list when the current porosity 
($f$) exceeds the initial porosity ($f_0$). This packet outputs data for gurson 
elements when the adaptive load control option is employed in the 
\ti{crack growth parameter }input, \ie \ti{adaptive load control on}. 
The packet record for each gurson element provides the following data.
\squishlist
\item the element number:  integer
\item $f_0$:  initial porosity:  floating point
\item $f$:  current porosity (averaged over element integration points):  floating point
\item $\bar \varepsilon^p$:  average plastic strain in the matrix material:  floating point
\item $\bar \sigma$:  average equivalent stress in the matrix material:  floating point
\item $\sigma_{m}$:  average macroscopic mean stress:  floating point
\item the status of the mean stress: integer. ( =0, the mean 
stress increased during the current load step;
=1, the mean stress decreased during the current load step)
\item $\sigma_{e}$:  average macroscopic mises stress:  floating point
\item the status of the mises stress:  integer.
( =0, the mises stress increased during the current load step; 
=1, the mises stress decreased during the current load step)
\item the increase in average porosity ($f$) during the current load step:  floating point 
\squishend

%
%---------------------------------------------------------------------------
%        section
%---------------------------------------------------------------------------
%
\section{Packet Type 7: \ti{interface-cohesive elements }}
After each load step in analyses having interface-cohesive elements for crack 
growth, WARP3D provides a printed table of key values for each cohesive element. 
The same table is written concurrently as a packet if the user has authorized packet 
file operations. The number of packet records equals the number 
of interface-cohesive elements 
appearing in the printed list following each load step. 

See the descriptions of packet types 31 and 32 for stress (\ie tractions) 
and strains (\ti{displacement jumps across the interface}) for interface-cohesive
elements produced by usual \ti{output} commands.

\noindent \ul{\ti{exp1\_intf} formulation} (elements are included only if the
$\bar T \ge 0.5 \bar T_p$)
\squishlist
\item the element number:  integer
\item the cohesive option:  integer ($=4$ for the \ti{exp1\_intf} option)
\item normal component of cohesive traction ($||\bmf{T}_n||$):  floating point
\item total shear component of cohesive traction ($||\bmf{T}_t||$:  floating point
\item normal component of displacement jump in cohesive element ($\Delta_n$):  floating point
\item shear component of displacement jump in cohesive element($||\bmf{\Delta}_t||$:  floating point
\item effective traction in cohesive element ($\bar T$) :  floating point
\item effective displacement jump in cohesive element ($\bar \Delta$) :  floating point
\item marker character (usually an *) :  1 character , \ie character *1 marker
\squishend
\noindent The traction and displacement jump values for the element are the 
average values for the the integration points of the element. The ``special'' character 
is ether (1) blank or (2) an * when the averaged, element effective 
displacement jump has exceeded the value 
corresponding to peak effective stress. 

\noindent \ul{\ti{PPR} formulation} (elements are included if any 
integration points $> 50\%$ peak traction values)
\squishlist
\item the element number:  integer
\item the cohesive option:  integer ($=6$ for the \ti{PPR} option)
\item normal component of cohesive traction ($T_n$):  floating point
\item shear component of cohesive traction ($T_t$):  floating point
\item normal component of displacement jump in cohesive element ($\Delta_n$):  floating point
\item shear component of displacement jump in cohesive element($\Delta_t$):  floating point
\item averaged normal traction over element / $T_{n-p}$ :  floating point
\item averaged shear traction over element  / $T_{t-p}$ :  floating point
\item number of element integration points at which $T_n \ge T_{n-p}$: integer
\item number of element integration points at which $T_t \ge T_{t-p}$: integer
\squishend
\noindent The traction and displacement jump values for the element are the 
average values for the the integration points of the element. 

%
%---------------------------------------------------------------------------
%        section
%---------------------------------------------------------------------------
%
\section{Packet Type 8: \ti{ctoa growth: constant front summary}}
At the end of each load step in analyses which have crack growth 
using the constant front CTOA algorithm, WARP3D provides a printed table that 
summarizes the status of each current ``master'' node. There exists one master 
node for each active crack front in the analysis. This same table of information is 
also written to this packet type (when packet output is on). The number of packet 
records equals the number of active master nodes and equals the number of 
active crack fronts. The packet record for each master node provides the following data.
\squishlist
\item the master node number:  integer
\item the current opening angle (total) in degrees:  floating point
\item that status of the master node:  integer. ( =0, the front has yet to grow and the 
critical angle refers to the initiation angle;  =1, the front is growing and the critical angle 
refers to the growth angle)
\item $100\ \times$ the current opening angle $/$  critical angle:  floating point
\squishend
\noindent The packet is generated \ti{after} the load step finishes 
so the results in the packet refer to the load step just completed.

%
%---------------------------------------------------------------------------
%        section
%---------------------------------------------------------------------------
%
\section{Packet Type 9: \ti{ctoa growth: non-constant front summary}}
At the end of each load step in analyses which have crack growth using the 
general CTOA algorithm, WARP3D provides a printed table that summarizes 
the status of each current node along the crack front(s). Each node along a 
front has one or more ``neighbor'' nodes which leads to computation of the 
current opening angles. Growth at the node occurs when any of the angles 
formed with neighboring nodes exceeds the critical angle. This same table 
of information is also written to this packet type (when packet output is on).
The number of packet records equals the sum of [ crack front nodes $\times$ (number of 
neighbor nodes on the crack front -1) ]. The -1 factor arises since each crack 
front node has at least one neighbor node. The packet record for each crack 
front node provides the following data.
\squishlist
\item the crack front node number:  integer
\item the neighbor node number: integer
\item the current opening angle (total) in degrees:  floating point
\item that status of the master node:  integer. ( =0, the front has yet to grow and the 
critical angle refers to the initiation angle;  =1, the front is growing and the critical angle 
refers to the growth angle)
\item $100\ \times$ the current opening angle $/$  critical angle:  floating point
\squishend
\noindent The packet is generated \ti{after} the load step finishes so the results in the packet 
refer to the load step just completed.

%
%---------------------------------------------------------------------------
%        section
%---------------------------------------------------------------------------
%
\section{Packet Type 10: \ti{newly extincted interface-cohesive elements}}
At the end of each load step in analyses having interface-cohesive elements 
for crack growth, WARP3D indicates which, if any, cohesive elements 
were newly extincted during the step (\ie killed). The user must indicate whether or not
an interface-element may be extincted in an analysis through the keyword
\ti{killable} in the specified
properties of the cohesive material definition. WARP3D essentially removes the 
element from the model, 
thereby causing crack growth equal to the surface area of the interface element. 
When the analysis extincts an element, the indication 
of removal is written concurrently as a packet if the user has authorized 
packet file operations. 
The number of packet records equals the number of newly extincted elements.
The packet record for each newly ``extincted'' cohesive 
element provides the following data.
\squishlist
\item the element number:  integer
\item cohesive model formulation: integer ($=4$ for simple exponential model; 
$= 6$ for PPR model)
\item two data values, both floating point, for exponential model:
  (1) ratio of the average effective displacement to the peak effective 
  displacement, $avg(\bar \Delta)/ \bar \Delta_p$;  
  (2) ratio of the average cohesive traction to the peak cohesive 
  traction, $avg(\bar T) / \bar T_p$
\squishend
\ \ or,
\squishlist
\item two data values, both floating point, for PPR model: (1) average 
normal traction for 
element integration points / peak value on the normal traction-separation 
curve, $avg(T_n) / T_{n-p}$;
  (2) average shear traction for element integration points / peak value on 
  the shear traction-separation curve, $avg(T_t) / T_{t-p}$
\squishend

%
%---------------------------------------------------------------------------
%        section
%---------------------------------------------------------------------------
%
\section{Packet Type 11: \ti{displacements for element nodes}}
The packet contains the $u, v, w$ global displacements for each node 
for elements appearing in the $<$integerlist$>$ of the \ti{output 
packets displacements elements  $<$integerlist$>$ } command. 
Results for each node of an element appear on a single logical record. 
The number of additional packet records on the header record thus 
indicates the number of elements $\times$ the number of nodes on those elements.
\squishlist
\item the element number:  integer
\item the structure node number for the element:  integer
\item $u$ displacement:  floating point
\item $v$ displacement:  floating point
\item $w$ displacement:  floating point
\squishend

%
%---------------------------------------------------------------------------
%        section
%---------------------------------------------------------------------------
%
\section{Packet Type 12: \ti{velocities for element nodes}}
The packet contains the $\dot u, \dot v, \dot w$ global velocities for each node 
for elements appearing in the $<$integerlist$>$ of the \ti{output 
packets velocities elements  $<$integerlist$>$ } command. Results for 
each node of an element appear on a single logical record. The number 
of additional packet records on the header record thus indicates the 
number of elements $\times$ the number of nodes on those elements.
\squishlist
\item the element number:  integer
\item the structure node number for the element:  integer
\item $\dot u$ velocity:  floating point
\item $\dot v$ velocity:  floating point
\item $\dot w$ velocity:  floating point
\squishend

%
%---------------------------------------------------------------------------
%        section
%---------------------------------------------------------------------------
%
\section{Packet Type 13: \ti{accelerations for element nodes}}
The packet contains the $\ddot u, \ddot v, \ddot w$ global accelerations for 
each node for elements appearing in the $<$integerlist$>$ of 
the \ti{output packets accelerations elements  $<$integerlist$>$ } 
command. Results for each node of an element appear on a separate 
logical record. The number of additional packet records on the header 
record thus indicates the number of elements $\times$ the number of nodes 
on those elements.
\squishlist
\item the element number:  integer
\item the structure node number for the element:  integer
\item $\ddot u$ acceleration:  floating point
\item $\ddot v$ acceleration:  floating point
\item $\ddot w$ acceleration:  floating point
\squishend

%
%---------------------------------------------------------------------------
%        section
%---------------------------------------------------------------------------
%
\section{Packet Type 14: \ti{reactions for element nodes}}
The packet contains the $R_x, R_y, R_z$ global reactions for each node 
for elements appearing in the $<$integerlist$>$ of the 
\ti{output packets reactions elements $<$integerlist$>$ } command. 
Results for each node appear on a separate logical record. The last 
logical record of the packet provides the algebraic sums for each reaction component 
for the corresponding nodes. The number of additional packet records on the 
header record thus indicates 1 + the number of elements~the number 
of nodes on those elements. The number of additional packet records is 1 
when the \ti{totals only} option is used within the \ti{output} command.
\squishlist
\item the element number:  integer
\item the structure node for the element:  integer
\item $R_x$ reaction:  floating point
\item $R_y$ reaction:  floating point
\item  $R_x$ reaction:  floating point
\squishend
\noindent Last record of packet:
\squishlist
\item  $\sum R_x$ reactions for nodes in packet:  floating point
\item  $\sum R_y$ reactions for nodes in packet:  floating point
\item  $\sum R_z$ reactions for nodes in packet:  floating point
\squishend

%
%---------------------------------------------------------------------------
%        section
%---------------------------------------------------------------------------
%
\section{Packet Type 15: \ti{element stresses}}
The packet contains stress values for elements appearing in 
the $<$integerlist$>$ of the \ti{output packets stresses
 elements $<$integerlist$>$ }command. The location of stress 
 output is specified by the element logical properties: 
 \ti{gausspts}, \ti{nodpts} or \ti{center\_output}
 (see for example, Section 3.1.3). Twenty-six stress values are 
 output for each Gauss point, element node or element center point. 
 These 26 values are stored in one logical record on the file. 
 Figure 2.9 summarizes the 26 values. 

\noindent The number of additional packet records indicated on the 
header record for output at the:
\squishlist
\item \ti{gausspts} -- equals the number of elements $\times$ the 
number of Gauss points within those elements. 
\item \ti{nodpts} -- equals the number of elements $\times$ the number of 
nodes on those elements.
\item \ti{center\_output} -- equals the number of elements within 
the $<$integerlist$>$. 
\squishend
\noindent Elements appearing in the $<$integerlist$>$ may have 
a different type, material model, etc. The values on each record (in order) are:   
\squishlist
\item the element number:  integer
\item the Gauss point number, the node number or `0' indicating \ti{center\_output}:  integer
\item $\sigma_x$:  floating point
\item $\sigma_y$:  floating point
\item $\sigma_z$:  floating point
\item $\sigma_{xy}$:  floating point
\item $\sigma_{yz}$:  floating point
\item $\sigma_{xz}$:  floating point
\item $U_0$ (work density):  floating point
\item $\sigma_{vm}$ (\ti{mises} effective stress):  floating point
\item $c_1$ (state variable from material model):  floating point
\item $c_2$ (state variable from material model):  floating point
\item $c_3$ (state variable from material model):  floating point
\item $I_1$  (first stress invariant):  floating point
\item $I_2$  (second stress invariant):  floating point
\item $I_3$  (third stress invariant):  floating point
\item $\sigma_1$ (principal stress, $\sigma_1 \le \sigma_2 \le \sigma_3$):  floating point
\item $\sigma_2$( principal stress, $\sigma_1 \le \sigma_2 \le \sigma_3$):  floating point
\item $\sigma_3$ (principal stress, $\sigma_1 \le \sigma_2 \le \sigma_3$ ):  floating point
\item $l_1$ (cosine for direction 1):  floating point
\item $m_1$  (cosine for direction 1):  floating point
\item $n_1$  (cosine for direction 1):  floating point
\item $l_2$  (cosine for direction 2):  floating point
\item $m_2$  (cosine for direction 2):  floating point
\item $n_2$  (cosine for direction 2):  floating point
\item $l_3$  (cosine for direction 3):  floating point
\item $m_3$  (cosine for direction 3):  floating point
\item $n_3$  (cosine for direction 3):  floating point
\squishend

%
%---------------------------------------------------------------------------
%        section
%---------------------------------------------------------------------------
%
\section{Packet Type 16: \ti{element strains}}
The packet contains strain values for elements appearing in 
the $<$integerlist$>$ of the \ti{output packets strains elements 
$<$integerlist$>$ }command. The location of strain output is specified 
by the element logical properties: \ti{gausspts}, \ti{nodpts} or 
\ti{center\_output} (see for example, Section 3.1.3). Twenty-six strain 
values are output for each Gauss point, element node or element center point. 
These 26 values are stored in one logical record on the file. Figure 2.8 
summarizes the 26 values. 

\noindent The number of additional packet records indicated on the 
header record for output at the:
\squishlist
\item \ti{gausspts} -- equals the number of elements $\times$ the 
number of Gauss points within those elements. 
\item \ti{nodpts} -- equals the number of elements $\times$ the number of 
nodes on those elements.
\item \ti{center\_output} -- equals the number of elements within 
the $<$integerlist$>$. 
\squishend
\noindent Elements appearing in the $<$integerlist$>$ may have 
a different type, material model, etc. The values on each record (in order) are:   
\squishlist
\item the element number:  integer
\item the Gauss point number, the node number or `0' indicating \ti{center\_output}:  integer
\item $\varepsilon_x$:  floating point
\item $\varepsilon_y$:  floating point
\item $\varepsilon_z$:  floating point
\item $\gamma_{xy}$:  floating point
\item $\gamma_{yz}$:  floating point
\item $\gamma_{xz}$:  floating point
\item $\varepsilon_{eff}$:  (effective strain):  floating point
\item $I_1$  (first strain invariant):  floating point
\item $I_2$  (second strain invariant):  floating point
\item $I_3$  (third strain invariant):  floating point
\item $\varepsilon_1$ (principal strain, $\varepsilon_1 \le \varepsilon_2 \le \varepsilon_3$):  floating point
\item $\varepsilon_2$( principal strain, $\varepsilon_1 \le \varepsilon_2 \le \varepsilon_3$):  floating point
\item $\varepsilon_3$ (principal strain, $\varepsilon_1 \le \varepsilon_2 \le \varepsilon_3$ ):  floating point
\item $l_1$ (cosine for direction 1):  floating point
\item $m_1$  (cosine for direction 1):  floating point
\item $n_1$  (cosine for direction 1):  floating point
\item $l_2$  (cosine for direction 2):  floating point
\item $m_2$  (cosine for direction 2):  floating point
\item $n_2$  (cosine for direction 2):  floating point
\item $l_3$  (cosine for direction 3):  floating point
\item $m_3$  (cosine for direction 3):  floating point
\item $n_3$  (cosine for direction 3):  floating point
\squishend

%
%---------------------------------------------------------------------------
%        section
%---------------------------------------------------------------------------
%
\section{Packet Type 17: \ti{J-integral results}}
The packet contains results for the currently-defined domain constructed using 
the \ti{domain} command. Domain definition data is contained in the first 
logical record following the header record. Domain integral results for 
individual rings (domains) are contained in the following logical records. When 
the domain definition employs automatic rings (domains) to define multiple domains,
the packet contains the average, minimum and maximum domain-integral 
totals. Following output of $J$-values, WARP3D writes stress intensity factor 
values $K_I, K_{II}$, and $K_{III}$ to the packet for each domain.

\noindent The number of additional packet records indicated on the header record for:
\squishlist
\item \ti{a single ring (automatic or user-defined)} $=3$.
\item \ti{multiple automatically-defined rings} $= (2\ \times$ the number of rings $)+\ 2$.
\squishend
\noindent The values in the first record after the header are:   
\squishlist
\item the number of rings: integer
\item the domain ID: character (24 chars)
\item the structure name: character (8 chars)
\item the loading name: character (8 chars)
\item the load step number: integer
\squishend

\noindent The values in the record(s) corresponding to each ring are (see Chapter 4):   
\squishlist
\item domain integral term 1: floating point
\item domain integral term 2: floating point
\item domain integral term 3: floating point
\item domain integral term 4: floating point
\item domain integral term 5: floating point
\item domain integral term 6: floating point
\item domain integral term 7: floating point
\item domain integral term 8: floating point
\item domain integral total: floating point
\item number of elements skipped or killed: integer
\squishend
\noindent For multiple rings, the values in the next record of the packet are:   
\squishlist
\item average of domain integral totals: floating point
\item minimum domain integral total: floating point
\item maximum domain integral total: floating point
\squishend
\noindent Additional record(s) corresponding to each ring include:  
\squishlist
\item pure mode-I stress intensity factor for plane stress: floating point
\item pure mode-I stress intensity factor for plane strain: floating point
\item pure mode-II stress intensity factor for plane stress: floating point
\item pure mode-II stress intensity factor for plane strain: floating point
\item pure mode-III stress intensity factor: floating point
\squishend

%
%---------------------------------------------------------------------------
%        section
%---------------------------------------------------------------------------
%
\section{Packet Type 18: \ti{ctoa growth: non-constant front node release event}}
After each load step in analyses which have crack growth using the 
general CTOA algorithm, WARP3D checks for crack front nodes that 
exceed the critical angle (for initiation or continued growth). The code provides 
a printed table that summarizes the status of each current node along the 
crack front(s) that exceeds the critical angle and is to be 
released at that time. This same table of information is also written 
to this packet type (when packet output is on). The number of packet 
records equals the 1 + the number of crack front nodes being released at this point.

\noindent  The values in the first record after the header are:   
\squishlist
\item (1) the critical angle (degrees) for continued growth, (2)
the critical angle (degrees) for initiation of growth, and 
(3) the (+,-) angle variation (a percentage of the critical value) within which 
nodes are released: 3 floating point numbers
\squishend
\noindent  The packet record for each crack front node provides the following data.
\squishlist
\item the crack front node number:  integer
\item the current opening angle (total) in degrees:  floating point
\squishend

%
%---------------------------------------------------------------------------
%        section
%---------------------------------------------------------------------------
%
\section{Packet Type 19: \ti{ctoa growth: constant front node release    event}}
After each load step in analyses which have crack growth using the 
constant-front CTOA algorithm, WARP3D checks the ``master'' node for 
each current front that exceeds the critical angle (for initiation or continued 
growth). The code provides a printed table that summarizes the status of 
each master node that exceeds the critical angle and the associated front 
to be released at that time. This same table of information is also written 
to this packet type (when packet output is on). The number of packet records 
equals the 1 + the number of crack fronts being released at this point and 
the associated interim nodes for the master node. Interim nodes appear 
when the crack is grown forward more than one element. This occurs 
when the user-specified criterion of $L_c > L_e$ is in effect (see Section 5.3).

\noindent  The values in the first record after the header are:   
\squishlist
\item (1)the critical angle (degrees) for continued growth, (2) the 
critical angle (degrees) for initiation of growth, and (3) the (+,-) 
angle variation (a percentage of the critical value) within which 
nodes are released: 3 floating point numbers
\squishend
\noindent  The packet record for each crack front node provides the following data.
\squishlist
\item the crack front master or interim node number:  integer ( =0, the node is a 
master node; =1, the node is an ``interim'' node associated with the most 
recent record for a master node)
\item the current opening angle (total) in degrees:  floating point
\squishend

%
%---------------------------------------------------------------------------
%        section
%---------------------------------------------------------------------------
%
\section{Packet Type 20: \ti{newly extincted gurson elements}}
After each load step in analyses having ``gurson" elements 
for crack growth, WARP3D indicates which, if any, additional gurson elements 
were extincted (\ie killed). A gurson element is any solid element
in the model with the \ti{gurson} plasticity model and which has the 
material property \ti{killable} invoked. The user must indicate whether or not
a solid element may be extincted in an analysis through the keyword
\ti{killable} in the specified
properties of the gurson material definition.
WARP3D essentially removes the element from the model, 
causing crack growth. When the analysis extincts an element, the indication of 
removal is concurrently written as a packet if the user has authorized packet 
file operations. The number of packet records equals the number 
of newly extincted elements.
The packet record for each newly extincted gurson 
element provides the following data.
\squishlist
\item the element number:  integer
\item $f_0$:  initial porosity defined in the material:  floating point
\item $f$:  porosity (averaged over element integration points):  floating point
\item $\sigma_m$: average macroscopic mean stress:  floating point
\item $\sigma_e$:  average macroscopic mises stress:  floating point
\squishend

%
%---------------------------------------------------------------------------
%        section
%---------------------------------------------------------------------------
%
\section{Packet Type 21: \ti{newly extincted stress-modified critical strain elements}}
After each load step in analyses having ``smcs" elements 
for crack growth, WARP3D indicates which, if any, additional smcs elements 
were extincted (\ie killed). The user must indicate whether or not
a solid element may be extincted in an analysis through the keyword
\ti{killable} in the specified
properties of the plasticity material definition.
WARP3D essentially removes the element from the model, 
causing crack growth. When the analysis extincts an element, the indication of 
removal is concurrently written as a packet if the user has authorized packet 
file operations. The number of packet records equals the number 
of newly extincted elements.
The packet record
 for each newly extincted ``smcs" element provides the following data.
\squishlist
\item the element number:  integer
\item $\bar \varepsilon^p$:  plastic strain (averaged over element 
integration points):  floating point
\item $\bar \varepsilon^p_c$: critical plastic strain:  floating point
\squishend

%
%---------------------------------------------------------------------------
%        section
%---------------------------------------------------------------------------
%
\section{Packet Type 22: \ti{stress-modified critical strain elements: 
no automatic load reduction}}
Aftger each load step in analyses having ``smcs" elements for crack growth, 
WARP3D provides a printed table of key values for each such smcs element. The 
same table is written concurrently as a packet if the user has authorized 
packet file operations. The number of packet records equals the number 
of smcs elements appearing in the printed list following each load step. A 
smcs element appears in the list when the current average plastic strain 
($\bar \varepsilon^p$) exceeds 0. This packet outputs data for smcs elements 
when the automatic
load reduction option is not employed in the \ti{crack growth 
parameter }input, \ie \ti{automatic load reduction off }(default setting 
in WARP3D when not specified). The packet record for each smcs element 
provides the following data.
\squishlist
\item the element number:  integer
\item $\bar \varepsilon^p$:  average plastic strain (over element 
integration points):  floating point
\item $\bar \varepsilon^p_c$:  current critical plastic strain: 
 floating point
\item $\sigma_m$:  average mean stress (over element 
integration points):  floating point
\item $\sigma_e$: average mises equivalent stress (over 
element integration points):  floating point
\squishend

%
%---------------------------------------------------------------------------
%        section
%---------------------------------------------------------------------------
%
\section{Packet Type 23: \ti{stress-modified critical strain elements: 
automatic load reduction}}
After each load step in analyses which have ``smcs" elements for crack 
growth, WARP3D provides a printed table of key values for each smcs element. 
The same table is written concurrently as a packet if the user has authorized 
packet file operations. The number of packet records equals the number of 
smcs elements appearing in the printed list following each load step. A 
smcs element appears in the list when the current average plastic 
strain ($\bar \varepsilon^p$) exceeds 0. This packet outputs data for 
smcs elements when 
the automatic load reduction option is employed in the \ti{crack 
growth parameter} input, \ie \ti{automatic load reduction on}. 
The packet record for each smcs element provides the following data.
\squishlist
\item the element number:  integer
\item $\bar \varepsilon^p$:  plastic strain (averaged over element 
integration points):  floating point
\item $\bar \varepsilon^p_c$:  current critical plastic strain: 
 floating point
\item $\sigma_m$:  averaged mean stress (over element 
integration points):  floating point
\item $\sigma_e$: averaged mises equivalent stress (over 
element integration points):  floating point
\item the averaged increase in plastic strain ($\bar \varepsilon^p$) during 
the current load step:  floating point 
\squishend

%
%---------------------------------------------------------------------------
%        section
%---------------------------------------------------------------------------
%
\section{Packet Type 24: \ti{accumulated loading pattern factors for load step}}
The user specifies the contribution of one or more loading patterns and 
constraints to establish the incremental load for each analysis step. 
When the crack growth processors, for example, reduce the magnitude 
of the loading increments, the user often has difficulty tracking the 
actual loads applied on the model. WARP3D updates the \ti{actual} 
multipliers for loading patterns and constraints imposed on the model. 
The code prints this table following each load step. This packet type 
provides the same information. The number of packet records equals 
the number of loading patterns with non-zero \ti{accumulated} factors, 
including the ``constraints''. Each packet record contains the following data.
\squishlist
\item the loading pattern name as a character string (a12) -- imposed 
displacements through constraints are named ``constraints''
\item the accumulated multiplier for this pattern: floating point
\squishend

%
%---------------------------------------------------------------------------
%        section
%---------------------------------------------------------------------------
%
\section{Packet Type 27: \ti{interaction-integral results: stress intensity factors}}
The packet contains results for the currently-defined domain constructed 
using the \ti{domain} command. Domain definition data is contained in 
the first logical record following the header record. Interaction integral 
results for individual rings (domains) are contained in the following logical records. 
The final records contain $J$-values computed for each ring (domain) using 
the stress intensity factors obtained through the interaction integral procedure.
(Note: \ti{rings} and \ti{domains} are synonymous.)

\noindent The number of additional packet records indicated on the header record for:
\squishlist
\item \ti{a single ring (automatic or user-defined)} $= 7$.
\item \ti{multiple automatically-defined rings} $= 1 + 5 \times$(the number of 
rings $+1$) $+$ the number of rings.
\squishend
\noindent The values in the first record after the header are:
\squishlist
\item the number of rings: integer
\item the domain ID: character (24 chars)
\item the structure name: character (8 chars)
\item the loading name: character (8 chars)
\item the load step number: integer
\squishend
\noindent Although the domain command in Chapter 4 may suppress either plane-stress 
or plane-strain output to the WARP3D output file, packet output 
contains data for plane stress and plane strain. Therefore, all of the 
following data for this packet type is repeated five times. Each 
repetition corresponds, respectively, to interaction-integral output for: 
mode-I plane stress, mode-I plane strain, mode-II plane stress, 
mode-II plane strain, and mode-III (anti-plane shear). For each of 
these five sets of data, the values in the record(s) corresponding to 
each ring are (see Chapter 4):   
\squishlist
\item interaction integral term 1:  floating point
\item interaction integral term 2:  floating point
\item interaction integral term 3:  floating point
\item interaction integral term 4:  floating point
\item interaction integral term 5:  floating point
\item interaction integral term 6:  floating point
\item interaction integral term 7:  floating point
\item interaction integral term 8:  floating point
\item interaction integral total:  floating point
\item stress intensity factor:  floating point
\item number of elements skipped or killed:  integer
\squishend
\noindent For multiple rings, the next record of the packet stores:
\squishlist
\item average of interaction integral totals:  floating point
\item minimum interaction integral total:  floating point
\item maximum interaction integral total:  floating point
\squishend
\noindent Additional record(s) corresponding to each ring include:  
\squishlist
\item $J$-value for plane stress: floating point
\item $J$-value for plane strain: floating point
\squishend

%
%---------------------------------------------------------------------------
%        section
%---------------------------------------------------------------------------
%
\section{Packet Type 28: \ti{interaction-integral results: T-stresses}}
The packet contains results for the currently-defined domain constructed 
using the \ti{domain} command. Domain definition data is contained 
in the first logical record following the header record. Interaction integral 
results for individual rings (domains) are contained in the following
logical records. (Note: \ti{rings} and \ti{domains} are synonymous.)

\noindent The number of additional packet records indicated on the header record for:
\squishlist
\item \ti{a single ring (automatic or user-defined)} $=3$.
\item \ti{multiple automatically-defined rings} $=1+2\times$(the number of rings$+2$).
\squishend
\noindent The values in the first record after the header are:   
\squishlist
\item the number of rings: integer
\item the domain ID: character (24 chars)
\item the structure name: character (8 chars)
\item the loading name: character (8 chars)
\item the load step number: integer
\squishend
\noindent All of the following data for this packet type is 
repeated twice. Repetitions correspond to $T$-stress 
interaction-integral output for plane stress and plane strain, 
respectively. For each of these two sets of data, the values in the 
record(s) corresponding to each ring are (see Chapter 4):   
\squishlist
\item interaction integral term 1:  floating point
\item interaction integral term 2:  floating point
\item interaction integral term 3:  floating point
\item interaction integral term 4:  floating point
\item interaction integral term 5:  floating point
\item interaction integral term 6:  floating point
\item interaction integral term 7:  floating point
\item interaction integral term 8:  floating point
\item interaction integral total:  floating point
\item $T$-stress $T_{11}$:  floating point
\item $T$-stress $T_{33}$:  floating point
\item number of elements skipped or killed:  integer
\squishend
\noindent For multiple rings, the next record of the packet stores:
\squishlist
\item average of $T_{11}$-values:  floating point
\item minimum of $T_{11}$-values:  floating point
\item maximum of $T_{11}$-values:  floating point
\item average of $T_{33}$-values:  floating point
\item minimum of $T_{33}$-values:  floating point
\item maximum of $T_{33}$-values:  floating point
\squishend

%
%---------------------------------------------------------------------------
%        section
%---------------------------------------------------------------------------
%
\section{Packet Type 29: \ti{nodal temperatures}}
The packet contains the temperatures for each node 
appearing in the $<$integerlist$>$ of the \ti{output packets 
temperatures nodes $<$integerlist$>$ } command. 
Results for each node appear on a separate logical record. The number of additional 
packet records on the header record thus indicates the number of 
nodes with results in the packet.
\squishlist
\item the node number:  integer
\item current temperature ($T$) --  temperature $T_0$ 
specified in the \ti{initial conditions} for the model +
all incremental changes ($\Delta T$) defined through load steps:  floating point
\item $T_0$ temperature specified in the \ti{initial conditions} for the model (see Section 2.9). 
Values are zero if no
initial conditions specified
\squishend

%
%---------------------------------------------------------------------------
%        section
%---------------------------------------------------------------------------
%
\section{Packet Type 30: \ti{temperatures for element nodes}}
The packet contains the temperatures for each node 
of elements appearing in the $<$integerlist$>$ of the \ti{output 
packets temperatures elements  $<$integerlist$>$ } command. 
Results for each node of an element appear on a single logical record. 
The number of additional packet records on the header record thus 
indicates the number of elements $\times$ the number of nodes on those elements.
\squishlist
\item the element number:  integer
\item the structure node number for the element:  integer
\item current temperature ($T$) --  temperature $T_0$ 
specified in the \ti{initial conditions} for the model +
all incremental changes ($\Delta T$) defined through load steps:  floating point
\item $T_0$ temperature specified in the \ti{initial conditions} for the model (see Section 2.9). 
Values are zero if no
initial conditions specified
\squishend

%
%---------------------------------------------------------------------------
%        section
%---------------------------------------------------------------------------
%
\section{Packet Type 31: \ti{tractions for interface-cohesive elements}}
The packet contains traction values for interface elements appearing in 
the $<$integerlist$>$ of the \ti{output packets stresses
 elements $<$integerlist$>$ }command. By default, traction 
 values are output at the element integration points -- to request a single
 set of averaged values at the element center list the keyword \ti{center\_output}
 in the element proeprties.
 
\noindent The number of additional packet records indicated on the header 
record for output at the:
\squishlist
\item integration points -- equals the number of elements $\times$ the number of integration 
 points within those elements. \ti{This the default output location.}.
 \item center\_output --  equals the number of elements within the $<$integerlist$>$.
 \squishend

Key traction values output at each integration point or the center are: $T_{t1}$, 
$T_{t2}$, $T_s$ and $T_n$.  Additional values, based on the type of cohesive 
material model associated with the element, follow these common four
values. See tables in Section 3.3 for listing and
definition of all output values. 

The following summarizes the values at each location 
based on the cohesive model type.


\noindent \ti{\ul{linear\_intf} (6 values per integration point or 6 values for the center)}

\squishlist
\item $T_{t1}$, $T_{t2}$, $T_s$ and $T_n$
\item $\Gamma = 1/2 \, ( \bmf{T}^T \cdot \bmf{\Delta} )$
\item $\Gamma_{ur} \equiv 0$
\squishend


\noindent \ti{\ul{exp1\_intf} (8 values per integration point or 8 values for the center)}

\squishlist
\item $T_{t1}$, $T_{t2}$, $T_s$ and $T_n$
\item $\bar T$ and $\bar T / \bar T_p$
\item $\Gamma = \int_0^{\bmf{\Delta}} \bmf{T}^T\, d \bmf{\Delta}$
\item $\Gamma_{ur} = \Gamma -1/2 \, ( \bmf{T}^T \cdot \bmf{\Delta} ) $
\squishend


\noindent \ti{\ul{ppr\_intf} (8 values per integration point or 8 values for the center)}

\squishlist
\item $T_{t1}$, $T_{t2}$, $T_s$ and $T_n$
\item $T_s/T_{s-p}$ and $T_n / T_{n-p}$
\item $\Gamma = \int_0^{\bmf{\Delta}} \bmf{T}^T\, d \bmf{\Delta}$
\item $\Gamma_{ur} = \Gamma -1/2 \, ( \bmf{T}^T \cdot \bmf{\Delta} ) $
\squishend
%
%---------------------------------------------------------------------------
%        section
%---------------------------------------------------------------------------
%
\section{Packet Type 32: \ti{displacement jumps for interface-cohesive elements}}
The packet contains displacement jump (\ie strain) values for interface elements appearing in 
the $<$integerlist$>$ of the \ti{output packets stresses
 elements $<$integerlist$>$ }command. By default, displacement jump 
 values are output at the element integration points -- to request a single
 set of averaged values at the element center list the keyword \ti{center\_output}
 in the element proeprties.
 
\noindent The number of additional packet records indicated on the header 
record for output at the:
\squishlist
\item integration points -- equals the number of elements $\times$ the number of integration 
 points within those elements. \ti{This the default output location.}.
 \item center\_output --  equals the number of elements within the $<$integerlist$>$.
 \squishend

Key displacement jump values output at each integration point or the center are: $\Delta_{t1}$, 
$\Delta_{t2}$, $\Delta_s$ and $\Delta_n$.  Additional values, based on the type of cohesive 
material model associated with the element, follow these common four
values. See tables in Section 3.3 for listing and
definition of all output values. 

The following summarizes the values at each location 
based on the cohesive model type.


\noindent \ti{\ul{linear\_intf} (4 values per integration point or 4 values for the center)}

\squishlist
\item $\Delta_{t1}$,  $\Delta_{t2}$, $\Delta_s$ and $\Delta_n$
\squishend


\noindent \ti{\ul{exp1\_intf} (6 values per integration point or 6 values for the center)}

\squishlist
\item $\Delta_{t1}$,  $\Delta_{t2}$, $\Delta_s$ and $\Delta_n$
\item $\bar \Delta$
\item $\bar \Delta / \bar \Delta_p$
\squishend


\noindent \ti{\ul{ppr\_intf} (6 values per integration point or 6 values for the center)}

\squishlist
\item $T_{t1}$, $T_{t2}$, $T_s$ and $T_n$
\item $T_s/T_{s-p}$ and $T_n / T_{n-p}$
\item $\Delta_s / \Delta_{s-p}$
\item $\Delta_n / \bar \Delta_{n-p}$
\squishend


%
%---------------------------------------------------------------------------
%        section
%---------------------------------------------------------------------------
%
\section{Packet Reader (Fortran Program)}
The Fortran source code for an example packet reader program is included in the
WARP3D distribution in the directory \ti{packet\_dir}. This example
program is designed to run interactively in a shell window. On startup, it queries for
a list of packet types to be processed during execution (other packet types
in the binary packets file are ignored). The routines to process each
packet type read values from the packet file and write some or all of
the data values. We expect most users will modify extensively this example
program to suit their needs.

The example program may be compiled simply, for example, using the Intel Fortran 
compiler using (the entire program is in a single text file)
\begin{verbatim}
     ifort  packet_reader.f -o packet_reader.go
\end{verbatim}

\end{document}


