%! program = pdflatex

\documentclass[11pt]{report}
\usepackage{geometry} 
\geometry{letterpaper}

%---------------------------------------------
\setlength{\textheight}{630pt}
\setlength{\textwidth}{450pt}
\setlength{\oddsidemargin}{14pt}
\setlength{\parskip}{1ex plus 0.5ex minus 0.2ex}


%----------------------------------------
\usepackage{amsmath}
\usepackage{layout}
\usepackage{color}
\usepackage{hyphenat}
\usepackage{listings}

%----------------------------------------------
\usepackage{fancyhdr} \pagestyle{fancy}
\setlength\headheight{15pt}
\lhead{\small{User's Guide - \textit{WARP3D}}}
\rhead{\small{\textit{patwarp}}}
\fancyfoot[L] {\small{\textit{Appendix C}\ \   (Updated: 5-12-2013)}}
\fancyfoot[C] {\small{C.\thepage}}
\fancyfoot[R] {\small{\textit{patwarp}}}

%---------------------------------------------------
\usepackage{graphicx}
\usepackage[labelformat=empty]{caption}
\numberwithin{equation}{section}

%---------------------------------------------
%     --- make section headers in helvetica ---
\frenchspacing
\usepackage{sectsty} 
\usepackage{xspace}
\allsectionsfont{\sffamily} 
%----------------------------------------------

%---------  local commands ---------------------

\newcommand{\bmf } {\boldsymbol }
\newcommand{\bsf } [1]{\textrm{\textit{#1}}\xspace}
\newcommand{\HRule}{\rule{\linewidth}{0.5mm}}
\newcommand{\patwarp}{\textit{patwarp }}
\newcommand{\eg}{\textit{e.g.,}}
\newcommand{\ie}{\textit{i.e., }}


%
%        optional definition for bullet lists which
%        reduces white space.
%
\newcommand{\squishlist}{
 \begin{list}{$\bullet$}
  { \setlength{\itemsep}{0pt}
     \setlength{\parsep}{3pt}
     \setlength{\topsep}{3pt}
     \setlength{\partopsep}{0pt}
     \setlength{\leftmargin}{1.5em}
     \setlength{\labelwidth}{1em}
     \setlength{\labelsep}{0.5em} } }

\newcommand{\squishlisttwo}{
 \begin{list}{$\bullet$}
  { \setlength{\itemsep}{0pt}
     \setlength{\parsep}{0pt}
    \setlength{\topsep}{0pt}
    \setlength{\partopsep}{0pt}
    \setlength{\leftmargin}{2em}
    \setlength{\labelwidth}{1.5em}
    \setlength{\labelsep}{0.5em} } }

\newcommand{\squishend}{
  \end{list}  }
%


%-------------------------------------
\newcounter{sectrefs}
%
%      Replace section number with letter C
%
\renewcommand\thesection{C}
%--------------------------------------
%--------------------------------------
%---------------------------------------

\begin{document}

\LARGE
\hfill
\textbf{Appendix C}
\rule[0.15in]{450pt}{0.5mm}
\LARGE
\begin{flushright}
 \textbf{
{\fontfamily{phv}\selectfont Patran-to-WARP3D Translator (\textit{patwarp})}}
\end{flushright}
\normalsize


\subsection{Concept}
\noindent 
This appendix describes the procedures to communicate data between the Patran
modeling post-processing code and the WARP3D analysis code. This writeup assumes
that the reader is familiar with the use of both Patran and WARP3D. The information
and procedures here apply equally as well to ``neutral" files that 
conform the the 2.5 style of Patran neutral file format generated by many other 
programs (for example, FEMAP).

Figure C.1 illustrates the general flow of data between these codes. Analysts
execute Patran to create interactively the geometric model and the finite element model.
One form of output from Patran is denoted the ``neutral file" (have Patran
produce a 2.5 style version neutral file for the model). This is a sequential
(ASCII) file of line images that describes essential features of the finite
element model in a manner independent of any specific analysis system. After
building the finite element model, the user requests that Patran create the
neutral file. Outside of Patran, the user initiates the program denoted \textit{patwarp}
in Figure C.1. This interactive program is a ``forward translator." It reads the
neutral file, conducts a dialog with the user to define various options, and
produces an ASCII (text) input file for WARP3D. The input file for WARP3D generally requires
minor changes and additions by the user to include supplemental information, \eg\ material 
properties, solution parameters might require modification. Any text
editor available on the computer system may be employed to make the required
changes to the input file. The modified file should then be suitable for input
WARP3D to perform the analysis.
%--------------------------
\begin{figure}[htb]
\begin{center}
\includegraphics[scale=1.1,angle=-90]{Figure_1.eps} 
\caption*{\small 
Fig. \thesection.1: Typical order of operations to convert a Patran neutral file (2.5 format) to WARP3D input,
run an analysis, and post-process results files using Patran.\normalsize}
\label{F:cyclic-capabilities}
\end{center}
\end{figure}
%--------------------------

Patran has many features for post-processing 
of the analysis results. Analysis
results in WARP3D (displacements, strains, stresses, etc.) are written directly
to Patran compatible files at the user's request through \textit{output patran ...} 
commands. Each of these results files
contains the strains, stresses, displacements, etc. for a single load step. The
file format can be either formatted (ASCII) or unformatted (binary) as directed in the
output request given to WARP3D (refer to Section 2.12.3). A file written with
the formatted option can be examined with a text editor but the structure of the
file is not readily discernible by the user. A binary file is a sequential,
unformatted data file. It cannot be examined with a text editor. Binary files
are considerably smaller than formatted files but are not generally
transportable between different computer architectures, especially 
to-from Linux and Windows. WARP3D offers the option
to write Patran nodal results files or element results files (see Section
2.12.3).

After analysis by WARP3D and output of the results files, users continue Patran
execution and the post-processing operations to read and process the results
files. Because WARP3D directly generates Patran compatible results files, there
is no need for a ``reverse translator" program. The results files are also
available to users of WARP3D for input to other special-purpose programs (the
binary packets file may be more convenient for this purpose, see Appendix F).

Section 2.12 describes the ordering of results for model nodes and elements in
the Patran compatible results files. Appendix A provides a description of the
data formatting in these files and small program fragments to read these files. 

\subsection{Patran-to-WARP3D Translation}
\noindent 
The Patran-to-WARP3D translator program handles the most frequently used
modeling features of WARP3D. In some cases, Patran has modeling capabilities for
which there is no corresponding analysis capability in WARP3D, \textit{e.g.} alternate
coordinate systems. Despite these differences, the use of Patran (and other similar
programs) dramatically
reduces the effort for model generation and results post-processing.

The following model data in a neutral file (Patran 2.5 format) are supported by
the Patran-to-WARP3D translator program (\patwarp):
\squishlist
\item Structure name (a default name is generated by the translator program).

\item Structure size (number of nodes, elements).

\item Nodal coordinates.

\item Element incidences (connectivity of element to structure nodes).

\item Element types.

\item Nodal constraints (absolute and multi-point [MPCs])

\item Nodal loads (forces and temperatures).

\item 	Element loads (uniform face pressures on elements)
\squishend
\patwarp writes an example set of material definitions, nonlinear compute
commands and output commands into input file for potential use in completing the
model definitions. The translator ignores other types of data contained in the
Patran generated neutral file such as material properties.

The above data constitute a large majority of the input for most finite element
analyses.  A list of modeling data \textit{not} currently supported by \patwarp
includes:
\squishlist
\item Groups other than the default group.
\item Material properties for elements.
\item Physical properties for elements (however, the Config Id is recognized by \patwarp)
\item Alternate coordinate systems as defined in Patran.
\item Solution parameters which control a dynamic or a nonlinear analysis. 
\squishend
The processing of material properties and physical properties will be
implemented in later versions of \patwarp. 

\subsection{Executing the Translator Program}
\noindent 
The \patwarp program performs neutral file translation. The WARP3D
distribution contains source code, compile scripts and separate,
ready-to-run executables for Windows, Linux and Mac OS X. The Windows and Mac
versions prepare input files to execute in parallel using the threads-only version of 
WARP3D. The Linux version prepares input files to execute with the threads-only
or the MPI + threads versions of WARP3D.

The program prints an identifying message
followed by a prompt for the name of the Patran neutral file. The names of
neutral files are assigned by the user when prompted by Patran. \patwarp verifies
that the neutral file exists and that it can be processed. If the file cannot be
accessed by the translator program, the prompt is re-issued. After the neutral
file name has been defined, the name of the desired WARP3D input file is
requested. The file may have any name and need not already exist. \patwarp then
continues a question-answer dialog with the user to drive various options in the
translation process.

\patwarp performs the translation process in two steps, although these steps are
transparent to the program user. In step one, the neutral file is read and the
data stored internally to \textit{patwarp}. In step two, the actual WARP3D input file is
produced. A log of the processing is displayed for the user. After
\textit{patwarp} has terminated, the user should edit the generated WARP3D input file to
supply the element properties, modify the default output requests, etc.

By default, \textit{patwarp} writes all of the WARP3D input data into a single file with a name
provided by the user. The file becomes quite large in size for models having
many nodes, elements and constraints. \patwarp provides an option to place the
coordinates, incidences and constraint data into three separate files. The user
supplies a ``prefix" for these file names, \textit{i.e.}, for a prefix named pcvn, the
files pcvn.coordinates, pcvn.incidencess, and pcvn.constraints are written. The
main WARP3D input file then has *input from file ... commands inserted by
\textit{patwarp} to read the three data files at the correct location during input
processing (see Section 2.14 for additional description of these various utility
commands available).

\subsection{Element Mapping}
\noindent
Within Patran, elements are generated using the MESH menu form under the \textbf{Finite
Elements} radio button on the main (top) menu). Elements have generic types such
as Hex8, Hex20, etc. For example, the Hex8 element implies a 3-D solid element
that has 12 edges, 6 faces and 8 nodes. 

\textit{patwarp} supports only elements that conform to the Hex8, Hex20, Tet4 and Tet10
type in Patran. The 9, 12 and 15 node (Hex) transition elements available in
WARP3D are created automatically by \textit{patwarp} from the 8-node elements that share
common faces-edges with 20-node elements (see Section C.9 for details on
processing models with transition elements). To distinguish between different
``groups" of elements (\textit{e.g.} those with common material properties), the user can
employ the numeric ``configuration" code provided by Patran. The configuration
code assigned to each element is passed through in the neutral file for use by
\textit{patwarp}. The configuration code numbers are assigned to elements in Patran using
the \textbf{Properties} menu button on the main form. This brings up the \textbf{Element
Properties} menu in Patran. Click on the \textbf{Input Properties} button to bring up the
menu form. The first listed item is \textbf{Config Id}.

All elements of the same type (\textit{l3disop, tet10,} etc.) with the same configuration
code are grouped in clearly identified lists in the WARP3D file generated by
\textit{patwarp}. This feature simplifies considerably the assignment of options and
material properties to elements following execution of \textit{patwarp}.

\subsection{Element Blocking and Domain Decomposition}
\subsubsection{Blocking for Thread-Based Execution}
\noindent Section 2.6 describes how WARP3D processes elements in \textit{blocks} 
to increase numerical efficiency. A block of elements must have the same type
(\textit{l3disop, q3disop,} ...), material model, integration order, etc. Elements within
each block must have sequential numbering. 

\textit{patwarp} writes the command \textit{blocking automatic size=128} into the input file
which requests that WARP3D assign elements to blocks while processing input.


No domains are defined for threads-based execution.



\subsubsection{Domain Decomposition - Linux Only}
\noindent For execution using MPI + threads, elements in the model must first be
assigned to \textit{domains} prior to blocking, where the number of domains equals the
number of processors (MPI ranks) available to perform the analysis. The assignment of
elements to domains defines a complex optimization problem. Domains may contain
hex, tet or a mix of hex and tet elements (interface elements appear to this
process as hex or tet elements). A good allocation of elements to domains
maintains load balance of computational effort across processors and minimizes
communication of data among them. \textit{patwarp} includes the Metis software system to
create domains of elements for optimal processing during the finite element
analysis. Once elements are assigned to domains, elements within each domain are
re-numbered to maintain sequential ordering. Then the blocking process is
applied, independently, to the elements in each domain as described in the
previous paragraph. The blocking data written to the WARP3D input file includes
the domain number for each block.

\textit{patwarp} uses several criteria to determine which elements may properly appear
within a block. First, \textit{patwarp} recognizes different element types by the number
of nodes on the element as provided in the neutral file. Second, users may
indicate further groupings of common elements from which blocks are constructed
through the \textit{configuration code} (config id is the term used in Patran, see
Section C.4) for the elements as given in the neutral file. Only elements of the
same type and with the same configuration code will be assigned to the same block
by \textit{patwarp}.
Thus, all 8-node elements that have a common material model, for example, can
be given the same config id in Patran. Users can assign any number of different
config ids for the model in the Patran neutral file.

As an example, consider a model that has a mix of 8-node and 20-node elements in
the Patran neutral file (the automatic feature in patwarp redefines existing
8-node elements as various 9, 12, 15-node transition elements to maintain
displacement compatibility). There are two different material models that should
be assigned to elements, \textit{e.g.}, the von Mises plasticity model and the
Gurson-Tvergaard (GT) plasticity model. Let config id 1 be used during building
of the Patran model to denote elements with von Mises plasticity and config id 2
be used for elements having GT plasticity. \textit{patwarp}first redefines some 8-node
elements as transition elements to maintain displacement compatibility. Then
separate lists of elements for each unique element type and user assigned config
id are created, resulting in say 6 lists of elements where each element in a
list meets all requirements to reside in the same block. \textit{patwarp} processes each
list separately to renumber elements sequentially as it assigns them to blocks.
This process results in blocks of
various size containing sequentially numbered elements where all elements within
a block satisfy all the above requirements.


To complete the translation, \textit{patwarp} offers to: (1) print a correspondence table
between the original (Patran) element numbers and the blocked element numbers,
(2) generate a new neutral file to reflect the new element numbering, (3)
generate an ASCII elements results file where the domain number for the element
is the only data value. This last file can be used with Patran post-processing
capabilities to show element assignments to domains by coloring elements on the
display by domain number.

\subsection{Constraint Processing}
\noindent Absolute nodal constraints on the three translations may be specified in Patran
under the \textbf{BCs/Loads} radio button on the main (top) menu. The translator program
builds the corresponding \textit{constraints} data for WARP3D. 

Multi-point constraints (MPCs) may be imposed on the three translations in
Patran under the \textbf{Elements} menu. \textit{patwarp} supports 
``Explicit" and ``Rigid" type
MPCs as defined in Patran. The Rigid MPC has two subtypes" \textit{Pinned} 
and \textit{Fixed}.
Both subtypes may be specified, but if a \textit{Fixed} MPC is defined, the rotation
terms are ignored, \textit{i.e.}, \textit{patwarp} inserts the same MPC equations 
in the WARP3D input file for both \textit{Pinned} and \textit{Fixed} subtypes.

In Patran, constraints may be imposed within a loading set (condition) or
external to all loadings. The \textit{patwarp} translator combines all constraint sets in
the neutral file and writes a single set of nodal constraints in the input file.

When the user requests writing of constraint data to a separate file, the MPC
equations are included in that file.

\subsection{Loads Processing}
\noindent Nodal and element loads applied to the model in Patran under the \textbf{BCs/Loads} 
radio button on the main (top) menu). In Patran, the user groups these cases together
(along with imposed displacements) to define loading cases, \textit{e.g.}, the
``default\_case". \patwarp processes applied nodal forces, applied
nodal temperatures and pressure loads applied to element faces.

When Patran writes the neutral file for the model, the loading cases are
converted to loading ``sets" with assigned numbers (1, 2, 3, ...). \patwarp processes 
applied nodal forces, temperatures and element face
pressures for any number of loading sets.  It builds the loading
condition name for WARP3D as \textit{set\_n}, where \textit{n} refers to the Patran loading set
number. 

\patwarp recognizes pressure loads applied to the faces of
elements. They are converted to element load commands in the WARP3D input file.
When \textit{patwarp} automatically converts an 8-node element into one of the transition
elements, the user-defined pressure loads are carried forward onto the
transition element as well.

\subsection{Solution and Output Commands}
\noindent \textit{patwarp} writes the following commands in the WARP3D
input file to request analysis and output.
These commands request a solution for load step 1 of the model and output of
Patran binary results files. The analyst should edit these example
commands for the specific needs of the solution.
\small
\begin{verbatim}
 nonlinear analysis parameters
   solution technique sparse direct 
c   solution technique sparse iterative 
c   solution technique hypre
c   hypre tolerance 0.000001
   maximum iterations 5 $  newton iterations
   minimum iterations 1
   convergence test norm res tol 0.01
   nonconvergent solutions stop
   adaptive on
   linear stiffness for iteration one off
   batch messages off
   cpu time limit off
   material messages off
   bbar stabilization factor 0.0
   consistent q-matrix on
   time step 1.0e06
   trace solution on
   extrapolate on
   display tied mesh mpcs off
\end{verbatim}
\normalsize

\subsection{Processing of Models Containing Transition Elements}
\noindent The 9, 12 and 15 node elements (\textit{ts9isop, ts12isop, ts15isop}) enable development
of models containing 8 and 20 node elements which maintain complete displacement
compatibility. However, Patran does not directly support such transition
elements. Models containing these elements can be constructed and post-processed
using Patran with support of the patwarp program.

\squishlist
\item Create the Patran model using Hex/8 and Hex/20 type elements. At
this stage there will be mismatches in the number of nodes on common faces/edges
shared between the 8 and 20-node elements. Apply nodal constraints and element
pressure loadings as necessary to the model. Configuration ids to signify
different materials should be included at this point.

\item Create the Patran neutral file for the model.

\item Run the \textit{patwarp} program. Once the neutral file has been read,
\patwarp will ask the user if the creation of transition elements is
desired. If yes, \patwarp searches the user-defined 8 and 20-node elements to
find all the shared faces/edges. It then redefines some 8-node elements as one of
the transtion elements (9, 12, 15 node elements) needed to maintain full
displacement compatibility in the model. In this process, no new nodes or
elements are added to the model. Some 8-node elements are simply refined as
transition elements by appending existing nodes (shared with 20-node elements)
to their incidence list and then re-ordering the incidences to conform with the
ordering requirements for nodes in WARP3D.

\item \textit{patwarp} then executes the normal blocking strategy to renumber
elements into blocks of common element types and writes the WARP3D input file.

\item Patran compatible node and element result files are thus
unaffected by the conversion of some 8-node elements into transition elements.
Patran believes it is processing a mesh of only 8 and 20-node elements. When the
user requests that patwarp write a new neutral file for the model to reflect the
blocked element re-ordering, the transition elements are written as standard
8-node elements.
\squishend

\subsection{Limitations and Recommendations}
\noindent The WARP3D code requires that \textit{elements and nodes be numbered sequentially}.
Patran performs this task through the \textbf{Renumber} option of the main \textbf{Finite
Elements} menu. After the model is generated, the user should always perform a
node and element compaction within Patran.

\subsection{Example - Windows and Mac OS X (Threads Only Execution)}
\noindent The following example illustrates the process of translating a Patran model into
a WARP3D input file for execution on a Windows or Mac OS X system. The Windows and
Mac OS X versions of WARP3D
operate in a threads-only mode. In this example, the Paradiso sparse direct or
iterative solver is selected with the automatic blocking command inserted into 
the generated input file. 

\small
\begin{verbatim}
[rdodds]$ patwarp

 ****************************************************
 *                                                  *
 *     PATRAN to WARP3D Neutral File Translator     *
 *               (Mac OS X, Windows)                *
 *                                                  *
 *            MSC.Patran 2003 & later               *
 *       (2000000 nodes - 4000000 elements)         *
 *           Build Date:   5-12-2013                *
 *                                                  *
 * includes:                                        *
 *  o support for 8, 9, 12, 15, 20-node hexs        *
 *  o support for 4, 10 node tets                   *
 *  o support for 6, 15 node wedges                 *
 *  o output of blocking and partitioning info      *
 *       in Patran-readable (element) results files *
 *  o tet4 and tet10 elements now supported         *
 *  o MPCs defined in Patran model now supported    *
 *                                                  *
 ****************************************************


 >> patran neutral file name (default: patran.out.1) ? cvn_model_pat.out

 >> warp3d input file name (default: warp3d_input) ? cvn_model.inp

 >> coordinates, incidences-blocking, and constraints
 >> input data placed in separate files, (y/n, default = n)? n

 >> create transition elements (y/n, default=y)? n

        >> user title as read from neutral file is:

 STRUCTURE cvn

        >> neutral file created on: Mar  8 1999  time: 08:25:00

        >> patran version:  2.5-1

        >> model size parameters:
              >> number of nodes ...............   22140
              >> number of elements ............   19334
              >> number of materials ...........       0
              >> number of physical properties .       0

        >> begin processing nodal data
              >> processing data for node.......    5000
              >> processing data for node.......   10000
              >> processing data for node.......   15000
              >> processing data for node.......   20000

        >> begin processing element data
              >> processing data for element....    5000
              >> processing data for element....   10000
              >> processing data for element....   15000

        >> begin processing nodal displacement data

        >> use WARP3D automatic blocking assignment
        >> begin warp3d input file generation
              >> model title and sizes written
              >> element types written
              >> nodal coordinates written
              >> element incidences written
              >> blocking command written
              >> nodal and element loads written
              >> constraints written
              >> warp3d input file completed

        >> analysis file generation completed.
        >> job terminated normally.
\end{verbatim}
\normalsize

\subsection{Example - Linux for MPI + Threads Execution}
\noindent The following example illustrates the process of translating a Patran model
for MPI execution on a Linux system using the MPI + threads version of WARP3D.
The model is decomposed into domains and blocks for
parallel operations at the domain level and lower level parallel execution via
threads. 

\small
\begin{verbatim}
[rdodds]$ patwarp

 ****************************************************
 *                                                  *
 *     PATRAN to WARP3D Neutral File Translator     *
 *                                                  *
 *            MSC.Patran 2003 & later               *
 *       (2000000 nodes - 4000000 elements)         *
 *            Build Date:  5-12-2013                *
 *                                                  *
 * includes:                                        *
 *  o support for 8, 9, 12, 15, 20-node hexs        *
 *  o support for 4, 10 node tets                   *
 *  o support for 6, 15 node wedges                 *
 *  o output of blocking and partitioning info      *
 *       in Patran-readable (element) results files *
 *  o tet4 and tet10 elements now supported         *
 *  o MPCs defined in Patran model now supported    *
 *  o Domain decomposition using METIS for MPI      *
 *       based, parallel analyses in WARP3D.        *
 *       Hex, tet & mixed hex-tet meshes supported. *
 *                                                  *
 ****************************************************


 >> patran neutral file name (default: patran.out.1) ? cvn_model.out

 >> warp3d input file name (default: warp3d_input) ? cvn_model.inp

 >> execution procedure: 
      (1) threads-only, Pardiso sparse solver (direct/iterative)
      (2) MPI + threads, hypre solver 

 >> choice: 2

 >> number of MPI processes (same as
    the number of model domains (default = 1)? 8

 >> element block size used in domains: (default = 128)? 128

 >> domain decomposition may change element numbers
    print the new => old element listing (y/n, default=n)? n

 >> make an updated patran neutral file (y/n, default=n)? n

 >> make a patran-readable element results file to display the block
    assignments for the elements (y/n, default=n)? n

 >> coordinates, incidences-blocking, and constraints
    input data placed in separate files, (y/n, default = n)? n

 >> create transition elements (y/n, default=y)? n

        >> user title as read from neutral file is:

 STRUCTURE cvn                                                                   

        >> neutral file created on: Mar  8 1999  time: 08:25:00

        >> patran version:  2.5-1

        >> model size parameters:
              >> number of nodes ...............   22140
              >> number of elements ............   19334
              >> number of materials ...........       0
              >> number of physical properties .       0

        >> begin processing nodal data
              >> processing data for node.......    5000
              >> processing data for node.......   10000
              >> processing data for node.......   15000
              >> processing data for node.......   20000

        >> begin processing element data
              >> processing data for element....    5000
              >> processing data for element....   10000
              >> processing data for element....   15000

        >> begin processing nodal displacement data

        >> begin element reordering

        >> using standard blocking -- element reordering skipped
        >> using metis to build graph of mesh
        >> total number of corner nodes:   22140
        >> completed graph of mesh
        >> using metis to patition graph
        >> completed graph partitioning

        >> Notes:
              >> Multiple element configurations caused a renumbering process.
              >> Elements of the same configuration  are now numbered sequentially.
              >> Patran element results files should be displayed using newly
                     provided Patran neutral file for model.

        >> begin warp3d input file generation
              >> model title and sizes written
              >> element types written
              >> nodal coordinates written
              >> element incidences written
              >> blocking command written
              >> nodal and element loads written
              >> constraints written
              >> warp3d input file completed

        >> analysis file generation completed.
        >> job terminated normally. 
\end{verbatim}
\normalsize


\end{document}

