%
\documentclass[11pt]{report}
\usepackage{geometry} 
\geometry{letterpaper}

%---------------------------------------------
\setlength{\textheight}{630pt}
\setlength{\textwidth}{450pt}
\setlength{\oddsidemargin}{14pt}
\setlength{\parskip}{1ex plus 0.5ex minus 0.2ex}


%----------------------------------------
\usepackage{amsmath}
\usepackage{layout}
\usepackage{color}
\usepackage{hyphenat}
\usepackage{hanging} 

%----------------------------------------------
\usepackage{fancyhdr} \pagestyle{fancy}
\setlength\headheight{15pt}
\lhead{\small{User's Guide - \textit{WARP3D}}}
\rhead{\small{\textit{Manual Conventions}}}
\fancyfoot[L] {\small{\textit{Chapter {\thechapter}}\ \   (Updated: 1-3-2012)}}
\fancyfoot[C] {\small{\thesection-\thepage}}
\fancyfoot[R] {\small{\textit{Introduction}}}

%---------------------------------------------------
\usepackage{graphicx}
\usepackage[labelformat=empty]{caption}
\numberwithin{equation}{section}

%---------------------------------------------
%     --- make section headers in helvetica ---
%
\frenchspacing
\usepackage{sectsty} 
\usepackage{xspace}
\allsectionsfont{\sffamily} 
\sectionfont{\large}
\usepackage[small,compact]{titlesec} % reduce white space around sections
%
%----------------------------------------------

%---------  local commands ---------------------
\newcommand{\tb} {\textbf}
\newcommand{\df} {\dotfill}
\newcommand{\nin} {\noindent}
\newcommand{\bmf } {\boldsymbol }  %bold math symbol
\newcommand{\bsf } [1]{\textrm{\textit{#1}}\xspace}
\newcommand{\ul} {\underline}
\newcommand{\hv} {\mathsf}   %helvetica text inside an equation
\newcommand{\eg}{\emph{e.g.},\xspace}
\newcommand{\ti}{\emph}
\newcommand{\noi}{\noindent}
\newcommand{\ie}{\emph{i.e.},\xspace}
\newcommand{\vs}{\emph{vs.}\xspace}


\newenvironment{offsetpar}[1]%
{\begin{list}{}%
         {\setlength{\leftmargin}{#1}}%
         \item[]%
}
{\end{list}}

%
%
%        optional definition for bullet lists which
%        reduces white space.
%
\newcommand{\squishlist}{
 \begin{list}{$\bullet$}
  { \setlength{\itemsep}{0pt}
     \setlength{\parsep}{3pt}
     \setlength{\topsep}{3pt}
     \setlength{\partopsep}{0pt}
     \setlength{\leftmargin}{1.5em}
     \setlength{\labelwidth}{1em}
     \setlength{\labelsep}{0.5em} } }

\newcommand{\squishlisttwo}{
 \begin{list}{$\bullet$}
  { \setlength{\itemsep}{0pt}
     \setlength{\parsep}{0pt}
    \setlength{\topsep}{0pt}
    \setlength{\partopsep}{0pt}
    \setlength{\leftmargin}{2em}
    \setlength{\labelwidth}{1.5em}
    \setlength{\labelsep}{0.5em} } }

\newcommand{\squishend}{
  \end{list}  }
%

%-------------------------------------
\newcounter{sectrefs}
\setcounter{sectrefs}{0}
\setcounter{chapter}{1}
\setcounter{section}{2}

%--------------------------------------
%--------------------------------------
%---------------------------------------

\begin{document}

\section{Manual Conventions}
%\layout
The input translators for WARP3D provide a convenient, free-form 
command structure to simplify specification of model and solution parameters. 
This section describes the conventions and notation employed 
throughout the manual to explain commands.
The appearance within a WARP3D command of a descriptor of the form 

\small
\begin{verbatim}
      <integer>
\end{verbatim}
\normalsize

\noi implies that the user is to enter an item of data within that position in 
the statement of the class described by the descriptor (in the above example 
an integer). The command

\small
\begin{verbatim}
     number of nodes <integer>
\end{verbatim}
\normalsize

\noi implies that the word \ti{nodes} is to be followed by an integer, such as 
1000 or 68970, and that the statement entered as input data should be of the form

\small
\begin{verbatim}
     number of nodes 68970
\end{verbatim}
\normalsize

\noi The following are definitions of the most common descriptors used within the 
language. Those not described below are explained when they first occur in the text.

\begin{hangparas}{1.0in}{1}
\texttt{<integer>\ \ \   } a series of digits optionally preceded by a plus or minus sign. 
Examples are 121, +300, -410.

\texttt{<real>\ \ \ \ \ \ } a series of digits with a decimal point included, or 
series of digits with a decimal point followed by an exponential indicating a power of 10. 
Real numbers may be optionally signed. Examples are 1.0, -2.5, 4.3e-01.

\texttt{<number>\ \ \ \ } is either a \texttt{<real>} or an \texttt{<integer>}. The input 
translator performs mode conversion as needed for internal storage. The notation
\texttt{<numr>} is also used as an alternate name for this descriptor.

\texttt{<label>\ \ \ \ \ } is a series of letters and digits. The sequence must begin with a letter. 
Input translators also accept the character underbar, \_ , as a valid letter. 
Labels may have the form big\_cylinder, for example, to give the appearance 
of multiple words for readability.

\texttt{<string>\ \ \ \ } is any textual information enclosed in apostrophes (') or quotes ("). 
An example is 'this is a string'.

\texttt{<list>\ \ \ \ \ \ } sometimes termed \texttt{<integer list>} is 
the notation used to indicate a sequence 
of positive integer values --
usually nodes or element numbers. Lists generally contain two forms of data 
that may be intermixed with the same list. The first form of data is 
a series of integers optionally separated by commas. An example is 1, 3, 6, 
10, 12. The second common form of a list implies a consecutive sequence 
of integers and consists of two integers separated by a hyphen. An example is 1-10, 
which implies all integers in the sequence 1 through 10. An extension of this form 
implies a constant increment, \eg 1-10 by 2 implies 1, 3, 5, 7, 9. A third form, 
\ti{all}, is sometimes permitted, and implies all physically meaningful integers. 
The forms of lists are often combined as 
in ... \ti{nodes 1-100 by 3, 200-300, 500-300 by - 3}.

\end{hangparas}

Input to WARP3D thus appears as a sequence of English-like commands. Many 
of the words or phrases in these commands are optional and 
are permitted for readability or to specify options with a command. 
In the definition of each command, underlined words are required 
for proper operation of the input translators. If a portion of a word is underlined, 
only the underlined portion is required input. Items such as \texttt{<integer>} shown 
in the command definitions are not underlined but must always be 
replaced by an item of the specified class. For example, the command phrase defined by
\begin{align*}
& \hv{\ul{numb}er\ (\ul{of})\ \ul{node}s\ <integer>}
\end{align*}
\noi can be shortened to

\small
\begin{verbatim}
     numb of node 10
\end{verbatim}
\normalsize

\noi if the user so desires.

In many instances, more than one word is acceptable at a given 
position within a command. The choices are listed one above the other in the 
command definition and enclosed in \{\ \}. The command definition
\begin{align*}
& \hv{\ul{comp}ute}\   
\begin{Bmatrix}
\hv{\ul{displ}acements} \\ \hv{\ul{domain}}
\end{Bmatrix}
\end{align*}
\noi indicates that each of the following commands is acceptable


\small
\begin{verbatim}
      compute domain
      compute displacements
      comp displa
\end{verbatim}
\normalsize

\noi Optional words and phrases are enclosed with parentheses, (\ ). In some commands, 
items may be repeated and/or multiple phrases may be combined on one data line. 
This is indicated in the command definition by enclosing the repeatable entries 
within  brackets, [\ ]. The command
\begin{align*}
& \hv{<integer>} \left [
\begin{Bmatrix}
\hv{\ul{x}} \\ \hv{\ul{y}}\\ \hv{\ul{z�}}
\end{Bmatrix}
\hv{<number>}\ (,) \right ]
\end{align*}
\noi implies that the following sequences are valid:
\small
\begin{verbatim}
      1 x 10 y 10 z 15.3
      2 x 15 z 30
      30 z -42.5
\end{verbatim}
\normalsize

To be more descriptive within the command definitions, 
actual data items (those denoted with \texttt{<\ >} in the definition) are 
sometimes described in terms of their physical meaning and 
followed by the type or class of data item which can 
be used in the command. For example the command,
\begin{align*}
& \hv{\ul{struct}ure\ <name\ of\ structure:label>}
\end{align*}
\noi implies that the data item following the word \ti{structure} is the name of the 
structure and must a descriptor of type \texttt{<label>}. Examples of acceptable commands are
 
\small
\begin{verbatim}
    structure cylinder
    struct big_block
\end{verbatim}
\normalsize

\noi while 

\small
\begin{verbatim}
    structure 1a
\end{verbatim}
\normalsize

\noi is not acceptable since the name of the structure is not 
a label (\ti{labels must begin with a letter}).

\vspace{5 mm}
\noi \textbf{\textit{User Named Lists}}

User named lists of integers described in Chapter 2 as part of
model definition provide an often convenient approach
to manage otherwise repetitive input of long lists of integers. The \ti{list} command
enables construction of lists as in

\small
\begin{verbatim}
    list 'ex_1'  1-10000 by 3,  23100-40000,  69250-119500 by 13,
                   142590-168900, 489000-64000 by 15
\end{verbatim}
\normalsize

\noi Then the list \texttt{'ex\_1'} may be used in any subsequent command that
requires a \texttt{<list>} or \texttt{<integer list>}.

\vspace{5 mm}
\noi \textbf{\textit{Continuation Lines}}

A comma (,) placed at the end of a line causes the subsequent data line to be considered a 
logical continuation of the current line. There is no limit on the number of 
continuation lines. Continuation can be invoked at any point in any command.

\vspace{5 mm}
\noi \textbf{\textit{Comment Lines}}

Comments may be placed at any point in the input. Comment lines may be
indicated with the three forms: (a) a \# in column 1, (b) an ! in
column 1, or (c) the letter ``c" or ``C" appearing in column 1 
of the data line  followed
by one or more blanks marks 
it as a comment line. The line is read and (possibly) echoed by the input translator. 
The content is ignored and the next data line read.

\ul{Blank lines} may appear at any point in the input for readability. The input
translators ignore blank lines.

\vspace{20 mm}
\noi \textbf{\textit{Line Termination}}

Line termination is accomplished in one of three ways. First, the last column examined by 
the input translators is column 72. Secondly, after encountering the first data item on a card, 
the translators count blanks between data items. If 40 successive blanks are found, 
the remainder of the line is assumed blank. Finally, a \$ indicates an end of line. 
Space following the \$ is ignored by the input translators and is often used to
include short comments.


\end{document}

 

