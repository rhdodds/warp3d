 %!TEX TS-program = pdflatexmk
 %
\documentclass[11pt]{report}
\usepackage{geometry} 
\geometry{letterpaper}

%---------------------------------------------
\setlength{\textheight}{630pt}
\setlength{\textwidth}{450pt}
\setlength{\oddsidemargin}{14pt}
\setlength{\parskip}{1ex plus 0.5ex minus 0.2ex}


%----------------------------------------
\usepackage{amsmath}
\usepackage{layout}
\usepackage{color}
\usepackage{hyphenat}

%----------------------------------------------
\usepackage{fancyhdr} \pagestyle{fancy}
\setlength\headheight{15pt}
\lhead{User's Guide - \textit{WARP3D}}
\rhead{\textit{Initial Conditions}}
\fancyfoot[L] {\textit{Chapter {\thechapter}\ \   (Updated: 12-12-2012)        }}
\fancyfoot[C] {\thesection-\thepage}
\fancyfoot[R] {\textit{Model Definition}}

%---------------------------------------------------
\usepackage{graphicx}
\usepackage[labelformat=empty]{caption}
\numberwithin{equation}{section}

%---------------------------------------------
%     --- make section headers in helvetica ---
%
\frenchspacing
\usepackage{sectsty} 
\usepackage{xspace}
\allsectionsfont{\sffamily} 
\sectionfont{\large}
\usepackage[small,compact]{titlesec} % reduce white space around sections
%
%----------------------------------------------

%---------  local commands ---------------------

\newcommand{\nin} {\noindent}
\newcommand{\noi}{\noindent}
\newcommand{\HRule}{\rule{\linewidth}{0.5mm}}
\newcommand{\bmf } {\boldsymbol }
\newcommand{\bsf } [1]{\textrm{\textit{#1}}\xspace}
\newcommand{\ul} {\underline}
\newcommand{\hv} {\mathsf}   %helvetica text inside an equation
\newcommand{\eg}{\emph{e.g.},\xspace}
\newcommand{\ie}{\emph{i.e.},\xspace}
\newcommand{\vs}{\emph{vs.}\xspace}
\newcommand{\ti}{\emph}
\newcommand{\veps}{\varepsilon}
\newcommand{\ol}{\overline}
\newcommand{\mdash}{\mbox{-}}
\newcommand{\clrr}{\color{red}}
\newcommand{\clrb}{\color{black}}
\newcommand{\Fig}{Fig.\xspace}
\newcommand{\Figs}{Figs.\xspace}




%
%
%        optional definition for bullet lists which
%        reduces white space.
%
\newcommand{\squishlist}{
 \begin{list}{$\bullet$}
  { \setlength{\itemsep}{0pt}
     \setlength{\parsep}{3pt}
     \setlength{\topsep}{3pt}
     \setlength{\partopsep}{0pt}
     \setlength{\leftmargin}{1.5em}
     \setlength{\labelwidth}{1em}
     \setlength{\labelsep}{0.5em} } }

\newcommand{\squishlisttwo}{
 \begin{list}{$\bullet$}
  { \setlength{\itemsep}{0pt}
     \setlength{\parsep}{0pt}
    \setlength{\topsep}{0pt}
    \setlength{\partopsep}{0pt}
    \setlength{\leftmargin}{2em}
    \setlength{\labelwidth}{1.5em}
    \setlength{\labelsep}{0.5em} } }

\newcommand{\squishend}{
  \end{list}  }
%

%-------------------------------------
\newcounter{sectrefs}
\setcounter{sectrefs}{0}
\setcounter{chapter}{2}
\setcounter{section}{8}
\setcounter{figure}{0}
\renewcommand{\thefigure}{\thesection.\arabic{figure}}

%--------------------------------------
%--------------------------------------
%---------------------------------------

\begin{document}

\section{Initial Conditions for Model}

Users may specifiy certain initial conditions present in the model at time $t=0$ (\ie at the start 
of load step 1). These include: nodal displacements, nodal velocities and nodal temperatures. 
When initial conditions are not specified, all initial displacements, velocities and 
temperatures are set to zero. Initial displacements and velocities may often be 
required to specify properly the starting conditions for a time-dependent (dynamic) 
analysis. Similarly, the specification of initial temperatures proves convenient to 
set up the use of temperature dependent material properties. All temperature 
changes specified in the loading thus are relative to the specified initial nodal 
temperatures.

The command initiate the specification of initial conditions is:
\begin{align*}
& \hv{\ul{initial}}\ \ \hv{\ul{cond}itions}
\end{align*}
\nin This command is followed by a series of statements to specify initial nodal displacements, 
velocities and/or temperatures. The ordering of commands is immaterial and 
only the last values entered are retained.

To specify initial nodal displacements, use the command sequence
\begin{align*}
& \hv{\ul{displace}ments} \\
&\ \ \hv{(\ul{node}s)\ <node\ list:list>}\ \left [
\begin{Bmatrix}
\hv{  \ul{u}} \\ \hv{\ul{v}}  \\ \hv{\ul{w}} 
\end{Bmatrix}
\hv{(=)<value:numr>}(,) \right ]
\end{align*}
\nin where the value \ti{all} for the $<$node list:list$>$ is acceptable 
in the initial conditions. To specify initial velocities, use the command sequence 
\begin{align*}
& \hv{\ul{vel}ocities} \\
&\ \ \hv{(\ul{node}s)\ <node\ list:list>}\ \left [
\begin{Bmatrix}
\hv{  \ul{u}} \\ \hv{\ul{v}}  \\ \hv{\ul{w}} 
\end{Bmatrix}
\hv{(=)<value:numr>}(,) \right ]
\end{align*}
To specify initial temperatures, use the command sequence 
\begin{align*}
& \hv{\ul{temp}eratures} \\
&\ \ \hv{(\ul{node}s)\ <node\ list:list>}\ \hv{\ul{temp}erature (=) <value:number>}
\end{align*}

\nin An example of the initial conditions command is
\small
\begin{verbatim}
     initial conditions
       temperatures
         1-800 temperature 300
         900-10000 by 2, 12000-14000 by 3 temperature 600
      velocities
         all w 100 v 200 u -200
      displacements
        900-1400 by 2, 1600-2300 by 5 v -2.1
\end{verbatim}
 \normalsize

% ----------------- end of real text ----------------------------------
\end{document}

 

