
\documentclass[11pt]{report}
\usepackage{geometry} 
\geometry{letterpaper}

%---------------------------------------------
\setlength{\textheight}{630pt}
\setlength{\textwidth}{450pt}
\setlength{\oddsidemargin}{14pt}
\setlength{\parskip}{1ex plus 0.5ex minus 0.2ex}


%----------------------------------------
\usepackage{amsmath}
\usepackage{layout}
\usepackage{color}
\usepackage{array}

%----------------------------------------------
\usepackage{fancyhdr} \pagestyle{fancy}
\setlength\headheight{15pt}
\lhead{\small{User's Guide - \textit{WARP3D}}}
\rhead{\small{Material: \textit{cohesive}}}
\fancyfoot[L] {\small{\textit{Chapter {\thechapter}}\ \   (Updated: 7-3-2013)}}
\fancyfoot[C] {\small{\thesection-\thepage}}
\fancyfoot[R] {\small{\textit{Elements and Material Models}}}

%---------------------------------------------------
\usepackage{graphicx}
\usepackage[labelformat=empty]{caption}
\numberwithin{equation}{section}

%---------------------------------------------
%     --- make section headers in helvetica ---
%
\usepackage{sectsty} 
\usepackage{xspace}
\allsectionsfont{\sffamily} 
\sectionfont{\large}
\usepackage[small,compact]{titlesec} % reduce white space around sections
%---------------------------------------------->
%
%
%   which fonts system for text and equations. with all commented,
%   the default LaTex CM fonts are used
%
%
\frenchspacing
%\usepackage{pxfonts}  % Palatino text 
%\usepackage{mathpazo} % Palatino text
%\usepackage{txfonts}


%---------  local commands ---------------------


\newcommand{\bmf } {\boldsymbol }  %bold math symbol
\newcommand{\bsf } [1]{\textrm{\textit{#1}}\xspace}
\newcommand{\ul} {\underline}
\newcommand{\hv} {\mathsf}   %helvetica text inside an equation
\newcommand{\eg}{\emph{e.g.},\xspace}
\newcommand{\ti}{\emph}
\newcommand{\bardelta}{\bar \delta}
\newcommand{\barDelta}{\bar \Delta}

\newenvironment{offsetpar}[1]%
{\begin{list}{}%
         {\setlength{\leftmargin}{#1}}%
         \item[]%
}
{\end{list}}

%
%
%        optional definition for bullet lists which
%        reduces white space.
%
\newcommand{\squishlist}{
 \begin{list}{$\bullet$}
  { \setlength{\itemsep}{0pt}
     \setlength{\parsep}{3pt}
     \setlength{\topsep}{3pt}
     \setlength{\partopsep}{0pt}
     \setlength{\leftmargin}{1.5em}
     \setlength{\labelwidth}{1em}
     \setlength{\labelsep}{0.5em} } }

\newcommand{\squishlisttwo}{
 \begin{list}{$\bullet$}
  { \setlength{\itemsep}{0pt}
     \setlength{\parsep}{0pt}
    \setlength{\topsep}{0pt}
    \setlength{\partopsep}{0pt}
    \setlength{\leftmargin}{2em}
    \setlength{\labelwidth}{1.5em}
    \setlength{\labelsep}{0.5em} } }

\newcommand{\squishend}{
  \end{list}  }
%


%-------------------------------------
\newcounter{sectrefs}
\setcounter{sectrefs}{0}
\setcounter{chapter}{3}
\setcounter{section}{7}
\setcounter{figure}{0}
\renewcommand{\thefigure}{\thesection.\arabic{figure}}
%
%--------------------------------------
%
%
%
%              start document 
%              ==========
%
%
\begin{document}

\section{Material Model Type: \textit{cohesive}}
\noindent This material model idealizes the fracture process in solids as the gradual
separation across initially thin and coincident surfaces which may 
in general be non-planar. The loss of cohesion and
thus crack formation-extension within a solid may be viewed as the progressive
decay of otherwise intact tension and shear tractions across the adjacent
surfaces. The introduction of interface constitutive models that determine the
current normal and shear tractions
from relative displacement jumps across the surfaces provides a description for the
progressive fracture process. At sufficiently large displacement jumps
across the interface surfaces, the models degrade the tractions to
zero thereby creating new, traction-free surfaces within the 
volumetric finite element model.
Further, the cohesive zone model introduces an
intrinsic length-scale in the local fracture process via the specified cohesive-fracture
energy which enables fracture
process zones on the specimen-component scale to evolve as a natural outcome of
the computations. 

The following sections provide details of the formulations for
a linear-elastic and two types of nonlinear, cohesive constitutive models intended 
for use at the integration points in linear-displacement interface 
elements (\ti{inter\_8}, \ti{trint6}) and
the quadratic-displacement interface element \ti{trint12} (see
Sec.3.3). The interface elements provide both small-displacement and
large-displacement formulations. The cohesive 
constitutive models described here define a full 3-D response across an
interface for mixed-mode fracture compatible with both the small
and large-displacement formulations.
The following briefly summarizes the available cohesive
formulations.
%
\begin{offsetpar}{0.2in}
\noindent{\bf{Linear-elastic}}. The user specifies the constant 
stiffness values (units of stress/displacement) that connect: 
(a) normal (opening) traction to the relative (opening) displacement jump; 
(2) shear traction ($t_1$) and the relative (longitudinal) sliding
displacement jump, and (3) shear traction ($t_2$) and the 
relative (transverse) sliding displacement jump
between the surfaces at an integration point within an interface element. Most often, the
shear traction-sliding displacement jump is considered isotropic; the
user just assigns equal stiffness values for the shear terms. The model does not
couple the normal and sliding response. Interpenetration under compressive loading is 
suppressed with a penalty term.

\noindent{\bf{Park-Paulino-Roesler option (\emph{ppr})}}. This nonlinear
model provides the option 
for different cohesive fracture energies in normal and shear modes, different values
of maximum (peak) stress on the traction-separation curves for normal and shear
loading, and explicit control over the initial stiffness of the traction-separation
curves. In the post peak-stress region of the response, the user may specify a variety
of behaviors from effectively brittle to very ductile, and separately for each mode. The model
formulation follows from a rigorous, consistent development of a mixed-mode
potential function for the nonlinear cohesive behavior 
(see Park, Paulino and Roesler [\ref{R:PPR2009}]).

\noindent{\bf{Exponential option (\ti{exp1\_intf})}}. This older nonlinear model
treats mixed-mode, cohesive fracture using a single, ``effective" traction connected to
an ``effective" displacement jump across the interface using a single, 
traction-separation curve. The user specifies one value of the
cohesive fracture energy and one value for the ``effective" peak stress on the
(single) traction-separation curve. A user-defined value for a
scalar, mode-mixity parameter ($\beta$) sets the relative weighing of
normal and shear modes to generate the ``effective" values
in the response. The traction-separation curve follows
a smooth exponential form that rises (nonlinearly) from zero traction-separation
to peak stress, and then decreases the effective stress exponentially 
towards zero with increasing relative effective separation. The initial
stiffness at zero traction-separation follows from the defined shape of the
exponential curve, the peak stress and fracture energy. The user does not have
explicit (independent) control over the initial stiffness.
Interpenetration under compressive loading is 
suppressed with a penalty term.
This formulation
draws heavily on the work of 
Needleman ([\ref{R:AN1990a}], [\ref{R:AN1990b}]) and 
Ortiz and co-workers ([\ref{R:CO1996}], [\ref{R:APO1999}], [\ref{R:OP1999}]).
\end{offsetpar}
 
\noindent In all three of these options, the compression part 
of the normal traction-separation 
response is penalized to prevent interpenetration of interface element
surfaces. To accomplish this, the user specifies a ``compression"
multiplier ($>1$) imposed on the
initial (normal) stiffness response when the cohesive updating algorithms detect the
onset of interpenetration.

At present, the cohesive constitutive models are temperature and loading rate
invariant. Future developments are expected to include these effects.
 


%*****************************************************
\subsection{Potential Functions}
%*****************************************************
For isothermal and rate-independent conditions, the
normal and shear (tangential) tractions in a local Cartesian system, 
$\bmf{T}(T_n, T_{t1}, T_{t2})$, across 
the interface-cohesive surfaces
derive from an energy density function, $\psi$, per unit area 
in the form 
%
\begin{equation}\label{E:traction_potential_a}
\bmf{T}=\frac {\partial \psi }{\partial \bmf{\Delta}} 
\left( \bmf{\delta}, \bmf{q} \right )
\end{equation}
%
where 
$\bmf{\Delta}=\bmf{\Delta}\left(\Delta_n, \Delta_{t1}, \Delta_{t2}\right)$
denotes the normal ($\Delta_n$) and shear ($\Delta_{t1}$,
$\Delta_{t2}$) displacement
jumps across the cohesive surface; $\bmf{q}$ defines 
a set of internal variables which describe the
inelastic processes of decohesion. Here, $\bmf{\Delta}$ vanishes for
a superposed rigid body motion. We define the orientation of the unit 
normal ($\bmf{n}$) at each material point
such that $\Delta_n=\bmf{\Delta} \cdot
\bmf{n} >0$ leads to an opening separation across the
interface. The (total) sliding displacements across
the interface, $\bmf{\Delta}_t$, may be written 
%
\begin{equation}
\bmf{\Delta}_t = \bmf{\Delta} - \Delta_n \bmf{n} = 
( \bmf{I} - \bmf{n} \otimes \bmf{n} ) \bmf{\Delta}; ~~\Delta_t = \Vert \bmf{\Delta}_t \Vert~,
\end{equation}
%
where $\bmf{\Delta}_t$ is the projection of $\bmf{\Delta}$ onto the tangent plane.
In the present treatment, the
shearing process on cohesive surfaces is taken as isotropic
and thus requires a single sliding displacement jump, $\Delta_t$,
and work-conjugate traction, $T_t$. For 3-D implementation,
the decomposition of
$\bmf{\Delta}_t$ into two orthogonal 
vectors, $\bmf{\Delta}_t = \bmf{\Delta}_{t1} + \bmf{\Delta}_{t2}$, 
at a material point in the tangent plane 
(normal to $\bmf{n}$) of the
interface becomes arbitrary and defined most conveniently by the parameterized
definition of the interface geometry (see Section 3.3).

The internal variables, $\bmf{q}$,
evolve according to a set of kinetic relations of the form
%
\begin{equation}
\bmf{\dot q}= \bmf{f}\left(\bmf{\Delta},\bmf{q}\right)\,.
\end{equation}
%
The potential structure of the cohesive traction-separation relationship 
follows as a consequence of the first and second laws of thermodynamics.
Adoption of this potential structure reduces the formulation of the
cohesive relations from two independent functions into a single scalar
function, $\psi(\Delta_n, \Delta_t,\bmf{q})$.

The tractions follow from the potential function directly as
%
\begin{equation}\label{E:traction_potential_b}
T_n = \frac {\partial \psi} {\partial \Delta_n} ; \quad 
T_t = \frac {\partial \psi} {\partial \Delta_t}\ .
\end{equation}
%
For Mode I loading and crack formation-extension over a symmetry
plane, the contribution of $T_t$ and $\Delta_t$ to $\psi$ vanish, and the 
formulation and needed material-fracture properties simplify 
considerably. Similarly for Mode II loading, the contribution 
of $T_n$ and $\Delta_n$ to $\psi$ vanish and the formulation
again simplifies considerably. The proper modeling of mixed-mode fracture
requires the careful construction of $\psi$ to yield $T_n-\Delta_n$ and
$T_t-\Delta_t$ traction-separation curves that support different
fracture energies
%
\begin{equation}
\Gamma_n = \int_0^{\delta_{n}} 
T_n\left ( \Delta_n,\Delta_t=0 \right) \, d \Delta_n\  ,\ \text{and}
\end{equation}
\begin{equation}
\Gamma_t = \int_0^{\delta_{t}} 
T_t\left ( \Delta_n=0,\Delta_t, \right) \, d \Delta_t\  .
\end{equation}
%
Here, and subsequently, $\delta_n$ denotes the value of $\Delta_n$ when the
normal traction $T_n$ degrades to zero. Similar, $\delta_t$ 
denotes the value of $\Delta_t$ when the
shear traction $T_t$ degrades to zero.
The various approaches to include mixed-mode behavior differentiate 
among the proposed 
formulations for cohesive traction-separation models. The \ti{PPR}
model implemented in WARP3D reflects a complete treatment of 
mixed-mode fracture within
the potential framework. 

The simpler \ti{exp1\_intf} formulation combines the
mixed-mode quantities (tractions, displacement jumps) into an ``effective" 
traction-displacement jump pair ($\bar T-\barDelta$) and uses that single pair to define the 
potential function, \ti{i.e.} 
$\scriptstyle{\barDelta= \sqrt {\beta^2 \Delta_t^2 + \Delta_n^2} }$
where $\beta$ is a scalar constant defined as part of the model input to
assign a weight factor to the shear-mode contribution.




%*****************************************************
\subsection{Linear-Elastic Option}
%*****************************************************

Linear-elastic relationships between the individual traction components and the
respective displacement jumps provide the 
simplest cohesive interface model. The keyword to request this
option is \ti{linear\_intf}. Users must specify stiffness values (units: stress/displacement)
of the interface in
the normal (\ti{stiffn}), longitudinal-shear (\ti{stiffs}),
and transverse-shear (\ti{stifft}) directions.
The longitudinal and transverse orientations are shown as the $t_1$ and 
$t_2$ directions, respectively in Figs. 3.8-3.10 (configurations of the
interface elements, which use the alternate notation 
$s_1$, $s_2$ rather than $t_1$, $t_2$).

An isotropic shear-sliding response is obtained by setting the
two shear stiffness coefficients to the have the same value,
\ti{stiffs}$=$\ti{stifft}.

To prevent interpenetration in the normal direction, the
\ti{compression\_multiplier} has a default value of 10.0 applied
to \ti{stiffn} when $\Delta_n < 0 $. This
value may be modified as necessary. 

\noindent A complete example is:
\small
\begin{verbatim}
   structure cct
  c
    material thin_layer
      properties cohesive type linear_intf stiffn 1000,
                     stifft 500 stiffs 500,
                     compression_multiplier 15
            .
            .
    elements 50-85 type inter_8 material thin_layer,
           order 2x2gs surface top
           .
           .
\end{verbatim}
\normalsize




%*****************************************************
\subsection{PPR Option (Park-Paulino-Roesler)}
%*****************************************************

The \ti{PPR} option defines a nonlinear, cohesive constitutive relationship with
a complete and rigorous treatment for mixed-mode fracture under
isothermal and rate-independent conditions. The traction-separation
curves have the form of polynomials which prescribe finite values of
normal and sliding displacement jumps when the tractions degrade to zero.
Mode I (normal) only behavior and Mode II/III (shear) only behavior represent
special cases of the general mixed-mode, nonlinear behavior.

The normal and shear traction-separation curves derived from a unified
potential for mixed-mode fracture [\ref{R:PPR2009}] have the forms
and associated terminology shown in Fig. \ref{fig:ppr_tract_sep}.
%
\begin{figure}[htb]
\begin{center}
\includegraphics[scale=0.65,angle=-90]{fig_ppr_traction_sep_a.eps} 
\includegraphics[scale=0.65,angle=-90]{fig_ppr_traction_sep_b.eps} 
\caption{{\small Fig. \thefigure: (a) Mode I (normal) traction-separation
curve, and (b)  Mode II/III shear mode traction separation curve for the 
\ti{PPR} cohesive formulation. Peak tractions and fracture energies
may be specified independently for the two modes.\normalsize}
\label{fig:ppr_tract_sep}}
%
\end{center}
\end{figure}
%
%\begin{offsetpar}{0.0in}\noindent{\bf{Key Features of Response}}\end{offsetpar}

\noindent \bf{Key Features of Response}\rm

\noindent Key features of the behavior modeled with these traction-separation
curves include:
\squishlist
\item Complete normal failure occurs ($T_n = 0$)
when the normal or tangential separation reaches a 
certain set of values ($\delta_n$, $\bar{\delta_t}$),
called the normal final crack opening width
and the tangential conjugate final crack opening width, respectively (see
subsequent section on Mode Interactions),
%
\begin{equation}
T_n (\delta_{n}, \Delta_t) = 0
\ , \ \
T_n(\Delta_n, \bar{\delta_t}) = 0 \ .
\end{equation}
\item Similarly, complete shear failure occurs ($T_t = 0$)
either when the normal separation reaches the normal 
conjugate final crack opening width ($\bar{\delta_n}$)
or when the tangential separation reaches the 
tangential final crack opening width ($\delta_t$),
\begin{equation}
T_t (\bar{\delta_{n}}, \Delta_t) = 0
\ , \ \
T_t(\Delta_n, \delta_t) = 0 \ .
\end{equation}
\item The area, $\Gamma_n$, under the normal traction-separation 
curve in the absence of shear loading defines the material's opening-mode
fracture energy (Mode I toughness). Similarly, the area, $\Gamma_t$, 
under the shear traction-separation 
curve in the absence of normal loading defines the material's shear-mode
fracture energy (Mode II/III toughness). Here, $\Gamma_n$ and
$\Gamma_{t}$ are given by
\begin{equation}
\Gamma_{n}= \int^{\delta_n}_{0} T_n(\Delta_n, 0) \, d\Delta_n
\ , \ \
\Gamma_t = \int^{\delta_t}_{0} T_t(0, \Delta_t) \, d\Delta_t \ .
\end{equation}
\item The normal and tangential tractions are maximum
when the separations reach the peak opening displacements
($\delta_{n-p}$, $\delta_{t-p}$),
\begin{equation}
{\partial T_n \over \partial{\Delta_n}}\bigg\arrowvert_{\Delta_n = \delta_{n-p}} = 0
\ , \ \
{\partial T_t \over \partial{\Delta_t}}\bigg\arrowvert_{\Delta_t = \delta_{t-p}} = 0 \ .
\end{equation}
\item The maximum tractions correspond to the cohesive strengths ($T_{n-p}$, $T_{t-p}$),
\begin{equation}
T_n(\delta_{n-p}, 0) = T_{n-p}
\ , \ \
T_t(0, \delta_{t-p}) = T_{t-p} \ .
\end{equation}
\item The shape parameter indices ($\alpha$, $\beta$) are introduced to characterize material softening responses,
\ti{e.g.} brittle, plateau and quasi-brittle.
\item
The normal and tangential tractions satisfy basic symmetry 
and anti-symmetry requirements (with respect to $\Delta_t$), \ti{i.e.}
\begin{equation}
T_n(\Delta_n, \Delta_t) = T_n(\Delta_n, -\Delta_t) \ , \ \ \  T_t(\Delta_n, \Delta_t) = - T_t(\Delta_n, -\Delta_t) \ ,
\end{equation}
respectively.
\item The value of $T_t (\Delta_n, \Delta_t)$ at $\Delta_t = 0$ exists in the limit sense, \ti{i.e.}
\begin{equation}
\lim\limits_{\Delta_t \rightarrow 0^+} T_t(\Delta_n, \Delta_t) = 0 \ ,  
\ \ \ \lim\limits_{\Delta_t \rightarrow 0^-} T_t(\Delta_n, \Delta_t) = 0 \ .
\end{equation}
\item Unloading follows a linear-secant response. Reloading follows the linear-secant 
response until the prior maximum displacements are again reached. The 
unloading-reloading behavior is computed independent of the energy potential.
\item
On initial loading from zero tractions, the normal and shear mode response is not coupled.
Further, the orthogonal components ($T_1$, $T_2$) of the shear traction, 
$T_t^2=T_1^1 + T_2^2$, are uncoupled on initial loading
from zero tractions.
\squishend
%
%\begin{offsetpar}{0.0in}
\noindent \bf{Potential, Tractions, Parameters}\rm

\noindent With these desired behaviors for a mixed-mode, cohesive fracture model, \ti{PPR} 
developed the potential function
%
\begin{eqnarray}
\psi (\Delta_n, \Delta_t) &=& \min(\Gamma_n, \Gamma_t)  + \left[ \bar \Gamma_{n} \left(1 - {\Delta_n\over\delta_n} \right)^{\alpha}
                                   \left({ m \over \alpha} + {\Delta_n\over\delta_n} \right)^{m} + \langle \Gamma_n - \Gamma_t \rangle \right] \nonumber \\
    &  &  \ \ \ \ \ \ \ \ \ \ \ \times \left[ \bar \Gamma_{t} \left(1-{\left| \Delta_t \right| \over\delta_t} \right)^{\beta}
                                   \left({ n \over  \beta} + {\left| \Delta_t \right|\over\delta_t} \right)^{n} + \langle \Gamma_t-\Gamma_{n} \rangle \right] \ ,
\label{e-ummp}
\end{eqnarray}
% 
where $\langle \cdot \rangle$ denotes the \ti{Macauley bracket}, \ti{i.e.}
%
\begin{equation}
\langle x \rangle= 
\begin{cases} 0, & (x \le 0)
\\
x, &(x > 0)~.
\end{cases}
\end{equation}
%
The parameters ($m$, $n$) and ($\delta_n$, $\delta_t$) are internal to the model and
computed to satisfy the boundary conditions on macroscopic fracture (more discussion
below). The energy constants above are given by
%
\begin{equation}
\bar\Gamma_{n} = (-\Gamma_{n})^{ \langle \Gamma_{n} - \Gamma_{t} \rangle \over \Gamma_{n}-\Gamma_{t}}
    \left( {\alpha \over m} \right)^m
\, , \ \
\bar\Gamma_{t} = (-\Gamma_{t})^{ \langle \Gamma_{t} - \Gamma_{n} \rangle \over \Gamma_{t}-\Gamma_{n}}
    \left( {\beta \over n} \right)^n \ \ \ \ \mathrm{for}
\ \ (\Gamma_n \neq \Gamma_{t}) \ .
\end{equation}
If the normal and shear mode fracture energies are the same, the energy constants 
have the simpler form
\begin{equation}
\bar\Gamma_{n} = -\Gamma_{n} \left( {\alpha \over m} \right)^m
\, , \ \
\bar\Gamma_{t} = \left( {\beta \over n} \right)^n \ \ \ \ \mathrm{for}
\ \ (\Gamma_n = \Gamma_{t}) \ .
\end{equation}
%
The user may specify \ti{initial stiffness indicators} for the traction-separation curves 
as the ratio of the opening and sliding displacements at the peak traction values to the
values when the tractions degrade to zero,
%
\begin{equation}
\lambda_n = {\delta_{n-p} / \delta_n}
\, , \ \
\lambda_t = {\delta_{t-p} / \delta_t} \ .
\end{equation}
%
The initial stiffness indicators provide explicit user control of the initial, linear-elastic behavior.
Smaller values of $\lambda_n$, $\lambda_t$ (or $\delta_{n-p}$, $\delta_{t-p}$) increase the initial slope,
and decreased artificial elastic deformation.
Therefore, $\lambda_{n-p}$ and $\lambda_{t-p}$ are generally selected to 
be ``small'' values within the range of numerical stability.

The internal, non-dimensional  exponents, $m$ and $n$, are given by
\begin{equation}
m = {\alpha(\alpha-1) {\lambda_n}^2 \over (1-\alpha {\lambda_n}^2)}
 \ , \ \ \
n = {\beta (\beta-1)  {\lambda_t}^2 \over (1-\beta  {\lambda_t}^2)} \ .
\end{equation}
%
and the final tangential and normal crack opening widths
($\delta_n$ and $\delta_t$),
%
\begin{eqnarray}
\delta_n &=& {\Gamma_{n} \over T_{n-p}} \alpha \lambda_n \left(1-\lambda_n \right)^{\alpha-1}
    \left( {\alpha \over m} + 1 \right)  \left({\alpha \over m }\lambda_n + 1 \right)^{m-1} \ , \nonumber \\
\delta_t &=& {\Gamma_t  \over T_{t-p}} \beta \lambda_t  \left(1-\lambda_t \right)^{\beta-1}
    \left( {\beta \over n} + 1 \right)   \left({\beta \over n }\lambda_t + 1 \right)^{n-1} \ .
\end{eqnarray}
%

Finally, the traction-separation curves follow directly from gradients of the 
potential and have the form,
%
\begin{equation}\label{E:ppr_Tn}
T_n (\Delta_n, \Delta_t) = {\bar\Gamma_{n} \over \delta_n}
\left[ \mathcal{A} - \mathcal{B}\right]  \times  f_t \left ( \Delta_t \right )~,
\end{equation}
%
\begin{equation}\label{E:ppr_Tt}
T_t (\Delta_n, \Delta_t)= {\bar\Gamma_{t} \over \delta_t}
\left[ \mathcal{C} - \mathcal{D} \right]  \times  f_n\left ( \Delta_n \right ) {\Delta_t \over \left|\Delta_t \right|}
\end{equation}
%
where,
\begin{equation}
\mathcal{A}=m \left(1 - {\Delta_n\over\delta_n} \right)^{\alpha}
 \left({ m \over \alpha} + {\Delta_n\over\delta_n} \right)^{m-1}~,
\end{equation}
%
\begin{equation}
\mathcal{B}=\alpha \left(1 - {\Delta_n\over\delta_n} \right)^{\alpha-1}
 \left({ m \over \alpha} + {\Delta_n\over\delta_n} \right)^{m}~,
\end{equation}
%
\begin{equation}
\mathcal{C}= n \left(1-{\left| \Delta_t \right|\over\delta_t} 
\right)^{\beta} \left({ n \over  \beta} +
 {\left| \Delta_t \right|\over\delta_t} \right)^{n-1} ~,
\end{equation}
%
\begin{equation}
\mathcal{D}= \beta \left(1-{\left| \Delta_t \right|\over\delta_t} 
\right)^{\beta-1} \left({ n \over  \beta} + 
{\left| \Delta_t \right|\over\delta_t} \right)^{n}~.
\end{equation}
And the \ti{mode-interaction} functions are
\begin{equation}\label{E:ppr_ft}
f_t\left( \Delta_t \right )=  \bar\Gamma_{t} \left(1-{\left| \Delta_t \right|\over\delta_t} \right)^{\beta}
          \left({ n \over  \beta} + {\left| \Delta_t \right|\over\delta_t} \right)^{n} +
           \langle \Gamma_t-\Gamma_{n} \rangle  ~,
\end{equation}
\begin{equation}\label{E:ppr_fn}
f_n \left( \Delta_n \right )= \bar\Gamma_{n} \left(1 - {\Delta_n\over\delta_n} \right)^{\alpha}
    \left({ m \over \alpha} + {\Delta_n\over\delta_n} \right)^{m} +
     \langle \Gamma_{n}-\Gamma_t \rangle ~.
\end{equation}
%
%
%   PPR Mode Interactions
%\begin{offsetpar}{0.0in}\noindent{\bf{Mode Interactions}}\end{offsetpar}
\noindent \bf{Mode Interactions}\rm
%

\noindent The normal traction-separation curve in Eq.\ (\ref{E:ppr_Tn}) has an multiplicative, 
mode-interaction term, denoted $f_t(\Delta_t)$; the shear traction in Eq.\ (\ref{E:ppr_Tt})
has the mode-interaction term $f_n(\Delta_n)$. These terms arise naturally in 
derivatives to compute tractions from the cohesive potential
function -- they describe the impact of loading in one mode on the traction-separation
response of the other mode. The presence of normal loading reduces the shear
capacity and the presence of a shear loading reduces the normal mode
capacity. 

This interaction is described through the shaded, rectangular regions
illustrated in Fig.\ \ref{fig:ppr_tract_interact}. 
The normal traction $T_n(\Delta_n,\Delta_t)$, for example, 
degrades to zero under $\Delta_t$ displacement jumps at or
before reaching the
limit $\delta_t$ defined for a pure shear loading. This reduced limit, 
denoted by $\bardelta_t$ and
termed the ``conjugate" limiting displacement in [\ref{R:PPR2009}],
with $\bardelta_t \le \Delta_t$ is computed by finding the value of $\Delta_t$
that makes the (nonlinear) function $f_t\equiv0$ in Eq.\ (\ref{E:ppr_ft}). 
Consequently, the normal traction vanishes outside the shaded region in 
Fig.\ \ref{fig:ppr_tract_interact}a.

The presence of a normal traction has a similar impact on the 
shear response as indicated in 
Fig.\ \ref{fig:ppr_tract_interact}b. Here, the value of $\bardelta_n$ makes 
the function $f_n\equiv 0$
in Eq.\ (\ref{E:ppr_fn}). Consequently, the
shear traction vanishes outside the shaded region in Fig.\ \ref{fig:ppr_tract_interact}b.
%
\begin{figure}[htb]
\begin{center}
\includegraphics[scale=0.85,angle=-90]{fig_ppr_traction_interaction.eps} 
\caption{{\small Fig. \thefigure: Description of each cohesive mode-interaction ($T_n$,
$T_t$) regions defined by the final crack displacement jumps ($\delta_n$, $\delta_t$)
and the ``conjugate" final crack opening jumps ($\bardelta_n$, $\bardelta_t$);
(a) $T_n$ versus  ($\delta_n$, $\bardelta_t$) space; 
(b) $T_t$ versus  ($\bardelta_n$, $\delta_t$) space.
\normalsize}
\label{fig:ppr_tract_interact}}
%
\end{center}
\end{figure}

%
%   PPR User Properties
%
%\begin{offsetpar}{0.0in}\noindent{\bf{User-Defined PPR Properties}}\end{offsetpar}
%
\noindent \bf{User-Defined PPR Properties}\rm

\noindent The user-definable properties of the \ti{PPR} model are:
%
\squishlist
%
\item The fracture energies ($\Gamma_n$, $\Gamma_t$) which may have the same or
different values
%
\item The peak normal and shear stress levels ($T_{n-p}$, $T_{t-p}$) on the 
traction-separation curves which may have the same or
different values
%
\item The characteristic shapes ($\alpha$, $\beta$) of the post-peak regions of the
traction-separation curves. For values of $2$, the tractions follow a near linear 
decrease with additional
displacement jump. For values $>2$, the post-peak curves have a convex 
shape, while values $<2$ lead to an increasingly 
concave, plateau-like post-peak behavior.
%
\item The indicators ($\lambda_n$, $\lambda_t$) that define the relative stiffness of the
initial linear-elastic response
%
\item Under compression loading, the cohesive material behaves as a 
spatially distributed, uniform linear spring. To limit the amount of interpenetration,
the stiffness of the compressive spring is taken as a factor times the slope of the 
traction-separation curve at the origin as determined from the specified value of
%
$\lambda_n$.  The multiplication factor is specified 
by the user through the property \ti{compression\_multiplier}. 
The default value of the multiplier is 2. 
%
\item Specify the property \ti{killable} to have the interface-cohesive element 
associated with the \ti{PPR} cohesive material added to the list of elements to
be eliminated from the model once the tractions degrade to zero.
%
\item \ti{Note}: For simple normal (Mode I) loadings, 
the traction-separation curve of the \ti{PPR} model can be made 
to match closely the traction-separation curve of the 
\ti{exp1\_intf} model to facilitiate possible comparisons. Set the \ti{PPR}
peak traction and fracture energy to match values for the \ti{exp1\_intf}
model. Set $\alpha=7$, $\beta=7$ and $\lambda_n=0.1$, $\lambda_t=0.1$
for the \ti{PPR} to make the early loading and post-peak traction
shapes agree for the two models.
%
\squishend
%
%=================================================
\begin{table}[htb]	
\centering
{
\setlength{\extrarowheight}{2.5pt}
\begin{tabular}{ | l | c |  c | c | }
\hline
Model Property & Keyword & Mode & Default Value \\
\hline \hline
Request \ti{ppr} option &ppr &\  &\  \\ \hline
Peak normal traction ($T_{n-p}$)	& sig\_peak& number	& 0.0 \\ \hline
Peak shear traction ($T_{t-p}$)	& tau\_peak& number	& 0.0 \\ \hline
Fracture energy: normal mode ($\Gamma_n$)	& G\_normal & number	& 0.0 \\ \hline
Fracture energy: shear mode ($\Gamma_t$)	& G\_shear	 & number	& 0.0 \\ \hline
Separation curve shape: normal mode ($\alpha$)	& shape\_normal & 	number	& 0.0 \\ \hline
Separation curve shape: shear mode ($\beta$) & shape\_shear & 	number	& 0.0 \\ \hline
Initial slope indicator: normal mode ($\lambda_n$) & ratio\_normal	& number & 0.0 \\ \hline
Initial slope indicator: shear mode ($\lambda_t$) & ratio\_shear& number	& 0.0 \\ \hline
Stiffness compression multiplication factor	& compression\_multiplier& number	& 2 \\ \hline
%Mode I (normal) only behavior	& only\_normal\_mode & logical	&  false \\ \hline
%Shear mode only behavior	& only\_shear\_mode & logical	&  false \\ \hline
Put interface-cohesive element in \ti{killable} list	& killable	& logical	& false \\ \hline
\end{tabular}
}	
%
\caption{\small Table \thesection.2 
Properties for \textit{ppr} option of the material model:\ \ti{cohesive}.
\normalsize}
\label{table:ppr}
\end{table}

\noindent A complete example of input for the \ti{PPR} option is:
\small
\begin{verbatim}
   structure cct
           .
           .
    material thin_layer
      properties cohesive type ppr   sig_peak 100,
      tau_peak 100 G_normal 0.154 G_shear 0.154,
      shape_normal 3 shape_shear 3 ratio_normal 0.1,
      ratio_shear 0.1 killable only_normal_mode,
      compression_multiplier 15
            .
            .
    elements 50-85 type inter_8 material thin_layer,
           order 2x2gs surface top
           .
           .

\end{verbatim}
\normalsize


%*****************************************************
\subsection{Exponential Option (exp1\_intf)}
%*****************************************************

This option in the cohesive model combines the
normal and shear tractions into a single ``effective" traction ($\bar T$). The model thus
has a single traction-separation curve (not one for each mode) and a single
energy of separation ($\bar \Gamma$).
Following Camacho and Ortiz ([\ref{R:CO1996}]), the introduction of a user-defined, scalar 
parameter ($\beta$) assigns 
different weights to the (tangential) sliding and (normal) opening displacements.
This simplifies the formulation to have a single 
displacement jump across the cohesive surface. The �effective� opening 
displacement ($\barDelta$) becomes
%
%
\begin{equation}\label{E:bardelta_define}
\barDelta = \sqrt{ \beta^2 \Delta^2_t + \Delta^2_n}~.
\end{equation}
%
This form reflects equal weights ($\beta$) assigned to each of the two (orthogonal)
sliding vectors in the tangent plane, where 
$\scriptstyle{\Delta^2_t = \sqrt{\Delta^2_{t1} +\Delta^2_{t2}}}$. 
Using Eq.\ (\ref{E:traction_potential_a}, \ref{E:traction_potential_b}),
and the definitions above for normal and sliding displacements, the traction vector
may be written in the form
%
\begin{equation}
\bmf{T}= \frac {\partial \psi }{\partial \Delta_n} \bmf{n} + 
\frac {\partial \psi }{\partial \Delta_t} \frac {\bmf{\Delta}_t}{\Delta_t} =
\bmf{T}_n + \bmf{T}_t
\end{equation}
%
Given the effective displacement ($\barDelta$), the work-conjugate, �effective� 
traction ($\bar T$)  may 
be considered to derive from
%
\begin{equation}\label{E:effective_traction}
\bar T = \frac {\partial \psi }{\partial \barDelta}  \left ( \barDelta , \bmf{q} \right )~.
\end{equation}
%
The normal and shear traction vectors can then be written as
%
\begin{equation}
\bmf{T}_n= \frac {\partial \psi }{\partial \barDelta} 
\frac {\partial \barDelta }{\partial \Delta_n} \bmf{n} ~,
\end{equation}
\begin{equation}
\bmf{T}_t= 
\frac {\partial \psi }{\partial \barDelta} \frac {\partial \barDelta }
{\partial \Delta_t}\frac {\bmf{\Delta}_t}{\Delta_t} ~.
\end{equation}
%
Using Eq.\ (\ref{E:effective_traction}), and with derivatives of $\bardelta$
from Eq.\ (\ref{E:bardelta_define}), we can write
%
\begin{equation}
\bmf{T}= \frac {\bar T}{\barDelta} \left ( \beta^2 \bmf{\Delta}_t + \Delta_n \bmf{n} \right )
\end{equation}
%
or equivalently in the form 
%
\begin{equation}
\bar T= \sqrt{ \beta^{-2} \Vert \bmf{T}_t \Vert ^2 + T^2_n }~.
\end{equation}
%
%
\begin{figure}[htb]
\begin{center}
\includegraphics[scale=0.8,angle=0]{fig_exp1_traction.eps} 
\caption{{\small Fig. \thefigure: Exponential,
effective traction-separation curve for the 
\ti{exp1\_intf} cohesive formulation. (a) loading curve, 
(b) illustration of unloading from the nonlinear curve then
reloading.\normalsize}
\label{fig:exp1_intf_tract_sep}}
%
\end{center}
\end{figure}
%
With the above simplification to accommodate mixed-mode loading, a simple
exponential relationship is adopted between the effective traction ($\bar T$) and 
the effective displacement ($\barDelta$) to define a phenomenological, decohesion process. 
The $\bar T-\barDelta$ response follows an irreversible path with unloading always 
directed to the origin. This model represents all the features of the separation process 
by: (1) the shape of the cohesive traction-separation curve, (2) the local material 
strength defined by the peak traction ($\bar T_p$), and the local material 
toughness defined by the work of separation ($\bar \Gamma$) given by 
the area under the $\bar T-\barDelta$ curve.

The constitutive relation for the cohesive surface is derived 
from an exponential form of a free energy potential 
%
\begin{equation}
\psi = \exp(1) \, \bar T_p \, \barDelta_p 
\left [ 1 - \left ( 1 + \frac{\barDelta}{\barDelta_p}\right)
\exp\left(-\frac{\barDelta}{\barDelta_p}\right) \right]~.
\end{equation}
%
Conditions that characterize ``loading'' along the nonlinear
traction-separation curve are: $\barDelta$ currently has the maximum value 
attained ($\barDelta = \barDelta_{max}$) and $\dot \barDelta \ge 0$. Under such 
loading conditions, the relationship $\bar T - \barDelta$ 
follows from 
%
\begin{equation}
\bar T = \frac {\partial \psi }{\partial \barDelta} =
\exp(1) \, \bar T_p \, \frac{\barDelta}{\barDelta_p}\,
\exp\left(-\frac{\barDelta}{\barDelta_p}\right)~.
\end{equation}
%
\noindent For unloading, the response follows
%
\begin{equation}
\bar T = \left ( \frac {\bar T_{max}} {\bardelta_{max}} \right ) \barDelta ~,~~
\rm{if}~ \barDelta < \barDelta_{max}~ \rm{or} ~\dot \barDelta < 0~.
\end{equation}
%
Figure \ref{fig:exp1_intf_tract_sep} illustrates both the loading and unloading responses 
predicted by this exponential form of a cohesive traction-separation law.
 
Using standard procedures to compute the $J$-integral for tractions 
across crack faces, the work of separation (cohesive fracture energy) 
per unit area of cohesive surface is given by 
%
\begin{equation}
\bar \Gamma = \int_0^{\infty} 
\bar T \, d \barDelta
\end{equation}
%
where for this exponential traction-separation curve,
%
\begin{equation}
\bar \Gamma = \exp(1) \, \bar T_p \, \barDelta_p~. 
\end{equation}
%

Under compression loading, the cohesive material behaves as a 
spatially distributed, uniform linear spring. 
The stiffness of the compressive spring is taken as a factor times the slope of the 
exponential, traction-separation curve at the origin,
\begin{equation}
K_L = \frac {\partial \bar T }{\partial \barDelta} \left ( \barDelta = 0 \right ) =
\exp(1) \, \bar T_p \, / \barDelta_p~.
\end{equation}
%
The multiplication factor can be specified 
by the user through the property \ti{compression\_multiplier}. Figure 3.18 
shows schematically the behavior of the cohesive model under 
normal compression for different values of \ti{compression\_multiplier}. 
The default value of the multiplier is 10. 

When the cohesive surfaces undergo a combination of 
compression and shear deformation, 
the effective separation $\barDelta$ is computed only from the shear components
($\Delta_n=0$ in Eq.\ \ref{E:bardelta_define}). The shear tractions then evolve 
according to the constitutive behavior shown in Figure 3.17. 

%
\begin{figure}[htb]
\begin{center}
\includegraphics[scale=1.0,angle=0]{fig_exp1_compression.eps} 
\caption{{\small Fig. \thefigure: Behavior of the
\ti{exp1\_intf} cohesive formulation under compressive loading and 
impact of the \ti{compression\_multiplier} property.\normalsize}
\label{fig:exp1_intf_tract_sep}}
%
\end{center}
\end{figure}
\noindent Summary for the  \textit{exp1\_intf} option:
\squishlist
\item The model is best suited for pure Mode I (normal) or pure shear loading
\item The shape of the $\bar T - \barDelta$ curve [including the initial, linear
stiffness and the cohesive energy, $\bar \Gamma$] is completely determined by the
the the user-specified values of $\bar T_p$ and $\barDelta_p$.
\item The initial, linear stiffness may become too
small (large opening of the interfaces and artificial model anisotropy)
for certain choices of ($\bar T_p$, $\barDelta_p$) needed to match 
the desired fracture behavior
\item Special considerations are needed for interface-cohesive 
elements on symmetry planes (see subsequent section here)
\squishend

\noindent A complete example of input for an \textit{exp1\_intf} cohesive
material and association with interface elements is:
\small
\begin{verbatim}
   structure cct
  c
    material thin_layer
      properties cohesive type exp1_intf  sig_max 100,
      delta_peak 0.000285  beta 0.3,
      killable  compression_multiplier 15
                  .
            .
    elements 50-85 type inter_8 material thin_layer,
           order 2x2gs surface top
           .
           .

\end{verbatim}
\normalsize


%=================================================
\begin{table}[htb]	
\centering
{
\setlength{\extrarowheight}{2.5pt}
\begin{tabular}{ | l | c |  c | c | }
\hline
Model Property & Keyword & Mode & Default Value \\
\hline \hline
Request \ti{exp1\_intf} option &exp1\_intf &\  &\  \\ \hline
Peak effective traction ($\bar T_{p}$)	& sig\_peak*& number	& 0 \\ \hline
Peak effective displacement ($\barDelta_{p}$)	& delta\_peak*	 & real	& 0.0 \\ \hline
Mode mixity weight factor ($\beta$)	& beta* & 	number	& 0.0 \\ \hline
Stiffness compression multiplication factor	& compression\_multiplier& number	& 10 \\ \hline
%Mode I only behavior	& only\_normal\_mode** & logical	&  false \\ \hline
Put interface-cohesive element in \ti{killable} list	& killable	& logical	& false \\ \hline
\end{tabular}
}
* termed \ti{sig\_max}, \ti{delta\_crit}, \ti{beta\_coh} in older WARP3D 
versions (both keywords accepted)
%
\caption{\small Table \thesection.2 
Properties for \textit{exp1\_intf} option of the material model:\ \ti{cohesive}.
\normalsize}
\label{table:exp1_intf}
\end{table}
%

%*****************************************************
\subsection{Model Output}
%*****************************************************

By default, the cohesive material models print no messages during computations. 
If requested during crack growth analysis, the crack growth processor 
prints various messages
about the interface-cohesive elements, the traction values, relative displacements and
status indicators, \ti{e.g.}\ if the tractions exceed the peak values
on the defined traction-separation curve. This output option is requested 
with the crack growth parameter \ti{print status on} (refer to Section 5.4.1). 

The model makes available the work density values at Gauss 
points, $U_0$, to the interface-cohesive routines for subsequent output. 
For the nonlinear cohesive zone model, 
$U_0$ at step $n+1$ is evaluated using the trapezoidal rule 
\begin{equation}
U_0^{n+1}= U_0^n + \frac{1}{2} 
\left ( \bmf{T}^{n+1} + \bmf{T}^n \right ) \bmf{:} 
\left ( \bmf{\Delta^{n+1}-\Delta^n}\right )\ .
\end{equation}

\noindent For the linear elastic interface,  $U_0$ at step $n+1$ becomes simply:
\begin{equation}
U_0^{n+1}= \frac{1}{2} \,
\bmf{T}^{n+1} \bmf{:} \bmf{\Delta}^{n+1}\ .
\end{equation}

\noindent The ``energy'' output file produced by WARP3D 
includes these work contributions for interface-cohesive elements.


%*****************************************************
\subsection{Special Procedures - Models with Symmetry Planes}
%*****************************************************
%
Simulations of mode I crack extension in 3-D models usually adopt one-fourth or
one-eight symmetric finite element meshes. In these models, 
interface-cohesive elements are placed over the initial uncracked 
ligament thereby constraining crack growth
in a self-similar manner. Consequently, the computed opening displacements
(normal to the symmetry plane) have only one-half of the value 
during computations to update cohesive tractions that would be present in
a ``full" model not taking advantage of the symmetry plane. 

%
\noindent To accomplish mode I only or sliding mode only behavior, follow these
guidelines:
%
\begin{offsetpar}{0.0in}
\noindent{\bf{Linear-Elastic option (\emph{linear\_intf})}}.\ This option requires
the input of linear stiffness values that link the normal and shear tractions to the
displacement jumps. For interface-cohesive elements
connected to a symmetry plane, the user should specify 
stiffness values for \ti{stiffn}, \ti{stifft} and \ti{stiffs} that are 
\ti{twice} the values that would be input in a full model
that does not take advantage of the symmetry conditions. 
The specification of displacement boundary conditions also require the 
usual care to not over-constrain or under-constrain
displacements of interface element nodes incident on the symmetry plane.
For large
displacement analyses, Section 3.3 describes the process to set which 
surface of an interface element connects to the adjacent solid element. 

\noindent{\bf{Park-Paulino-Roesler option (\emph{ppr})}}.\ This model requires
the input of peak values for normal and shear tractions
($T_{n-p}$, $T_{t-p}$), and the 
cohesive fracture energies ($\Gamma_n$, $\Gamma_t$). From these input values,
the model routines
compute the opening displacement ($\delta_n$, $\bardelta_n$)
and sliding displacement ($\delta_t$, $\bardelta_t$) values when the 
corresponding normal
and sliding tractions degrade to zero. 

For interface-cohesive elements connected to a symmetry plane: 
(1) specify the unadjusted peak normal and shear traction values; (2) specify 
\ti{one-half} the actual values for $\Gamma_n$ and $\Gamma_t$. 
The specification of displacement boundary conditions also require the 
usual care to not over-constrain or under-constrain
displacements of interface element nodes incident on the symmetry plane.
For large
displacement analyses, Section 3.3 describes the process to set which 
surface of the interface element connects to the adjacent solid element. 


\noindent{\bf{Exponential option (\ti{exp1\_intf})}}. This model requires
input of the peak value for the effective traction ($\bar T_p$) and the value
of the effective displacement jump at peak traction ($\bar \Delta_p$). 
For interface-cohesive elements connected to a symmetry plane: 
(1) specify the unadjusted value for the peak effective traction; (2) specify 
\ti{one-half} the actual value for the effective displacement jump at the
peak traction, and (3) specify a $\beta >0$ to provide a non-zero
resistance to sliding displacement jumps (needed to prevent
singularity in the equilibrium equations). 
The specification of displacement boundary conditions also require the 
usual care to not over-constrain or under-constrain
displacements of interface element nodes incident on the symmetry plane.
For large
displacement analyses, Section 3.3 describes the process to set which 
surface of the interface element connects to the adjacent solid element. 
%
\end{offsetpar}
 

%*****************************************************
\subsection{Adaptive Load Step Sizes}
%*****************************************************
Large load increments imposed on a model
containing interface-cohesive elements generally lead to satisfactory 
convergence of the global
Newton iterations. However, the computed response in the
interface-cohesive elements may miss key details
and features of the
decohesion behavior, and lead to erroneous results during 
crack formation-extension. For example, the 
peak stress attained
in the cohesive zone governs strongly the development 
of plasticity in the
background material and consequently the overall energy 
dissipation of the background material. 

With large load steps, the response state computed within 
interface elements may pass from the pre-peak
to the post-peak side of the traction-separation curves without enforcing
the peak stress level on adjacent background material. To eliminate these
effects, WARP3D provides an option in the global nonlinear solution
procedures that adaptively controls the size of global displacement (or load)
increments based on the response of interface-cohesive elements. When the
background elements have a linear-elastic behavior and only the interface-cohesive
elements are nonlinear, the load-step size dependency of the solution 
is much reduced, thereby minimizing the need for an adaptive loading strategy.

Following each
load increment, the adaptive code locates the interface-cohesive element in the
model that experiences the largest change of an effective displacement jump
across the interface (denoted here by 
$\tilde\Delta$). The code continually adjusts the sizes of subsequent load increments to
maintain the normalized value ($\tilde\Delta/\tilde\Delta_{p}$)
within a user-specified tolerance. Here, $\tilde\Delta_{p}$ denotes the
value of an effective displacement jump when the effective cohesive traction
reaches a maximum value before degrading at further increases of
the displacement jump. Our experience indicates that analyses
using values of 
($\tilde\Delta/\tilde\Delta_{p}$)
up to 0.3 show only small differences in the solutions. 

The definition of an
effective displacement ($\tilde\Delta$) and effective displacement value at the peak
traction ($\tilde\Delta_{p}$) are key aspects of the \ti{exp1\_intf}
option, \ti{i.e.} $\tilde\Delta=\barDelta$. For the \ti{ppr} option, an effective displacement
for use only in these adaptive solution procedures is given by
$\scriptstyle{\tilde\Delta= \sqrt {\Delta_t^2 + \Delta_n^2} }$. Chapter 5
provides additional details necessary for the adaptive load-step
process to be used with the \ti{ppr} option. 

This adaptive approach also maintains
excellent stability and convergence rates of the global Newton 
iterations. Simulations with
large amounts of crack extension may often proceed automatically. Chapter 5 provides
details of the input procedure to invoke adaptive step-size control. 

%*****************************************************
\subsection{Element Extinction}
%*****************************************************
The traction-separation models described here reduce normal and shear tractions
towards zero with increasing relative displacements across the interface
elements. Once the tractions decrease to (relatively) small values, element 
extinction procedures can be invoked to eliminate the associated 
interface element from the model and to initiate the process 
that relaxes the remaining (small) nodal forces to zero over a 
fixed number of subsequent load increments. 
Section 5.4 in Chapter 5 describes this extinction procedure for interface-cohesive elements.
The extinction procedures apply 
only to interface elements having \ti{cohesive} materials
with the material logical property \ti{killable} specified (see tables of 
\ti{cohesive} properties in this section). 


%*****************************************************
\subsection{Computational Efficiency}
%*****************************************************
The key computational routines for this material model reflect a very efficient
organization and data flow. The routines are passed incremental values 
of the displacement jumps, tractions
from the start of the current step, history data at the start of the current step
and user-specified property values.

The cohesive routines are provided all
data at integration point 3, for example, for all elements in a ``block'' in
a single invocation, where blocks are defined by the user 
in the input data. Maximum block sizes are
typically 64, 128 or 256 elements. This strategy leads to 
exceptionally fast, vector-style coding of
computational loops in the cohesive routines. Modern compilers readily optimize 
execution of these loops with their long iteration counts (\ti{i.e.}\ the block size)
to utilize local parallelism at the core level. Entire blocks of elements are processed
concurrently via shared-memory, threaded parallel execution (see Chapter 7
for details).


%*****************************************************
\subsection {References}
%*****************************************************
\small
[\refstepcounter{sectrefs}\label {R:PPR2009}\thesectrefs]~K. Park, G.H. Paulino, and 
J.R. Roesler. A unified potential-based cohesive model of mixed-mode fracture.
\textit{Journal of the Mechanics \& Physics of Solids}, 57:891-908, 2009.


\medskip
\noindent[\refstepcounter{sectrefs}\label {R:CO1996}\thesectrefs]~G.T. Camacho and 
M. Ortiz. Computational modeling of impact damage in brittle materials.
\textit{International Journal of Solids and Structures}, 33:2899-2938, 1996.


\medskip
\noindent[\refstepcounter{sectrefs}\label {R:AN1990a}\thesectrefs]~A. Needleman.
A continuum model for void nucleation by inclusion debonding.
\ti{Journal of Applied Mechanics}, 54:523-531, 1990

\medskip
\noindent[\refstepcounter{sectrefs}\label {R:AN1990b}\thesectrefs]~A. Needleman.
An analysis of tensile decohesion along an interface, 
\ti{Journal of Mechanics \& Physics of Solids}, 38:289-324, 1990.

\medskip
\noindent[\refstepcounter{sectrefs}\label {R:APO1999}\thesectrefs]~A. de Andres, J.L. Perez,
and M. Ortiz. Elasto-plastic finite element analysis of three-dimensional fatigue crack 
growth in aluminum shafts subjected to axial loading.
\ti{International Journal of Solids and Structures}, 36:2231-2258, 1999.

\medskip
\noindent[\refstepcounter{sectrefs}\label {R:OP1999}\thesectrefs]~M. 
Ortiz A. Pandolfi. A finite deformation irreversible cohesive 
elements for three-dimensional crack propagation.
\ti{International Journal for Numerical Methods in Engineering},
44:1267-1282, 1999.

\end{document}
