
\documentclass[11pt]{report}
\usepackage{geometry}
\geometry{letterpaper}

%-----------------------------------

%---------------------------------------------
\setlength{\textheight}{630pt}
\setlength{\textwidth}{450pt}
\setlength{\oddsidemargin}{14pt}
\setlength{\parskip}{1ex plus 0.5ex minus 0.2ex}


%----------------------------------------
\usepackage{amsmath}
\usepackage{layout}
\usepackage{color}
\usepackage{array}
\usepackage{bm}
\usepackage{graphicx}
\usepackage{longtable}
%
\usepackage[comma,numbers,sort,compress]{natbib} % numbered, sequential refs 
\renewcommand{\bibsection}{\subsection{\bibname}} %  makes bib a numbered subsection 
\renewcommand{\bibname}{References} % use References not Bibliography for section name
%----------------------------------------------

\usepackage[us,12hr]{datetime}
\usepackage{fancyhdr} \pagestyle{fancy}
\setlength\headheight{15pt}
\lhead{\small{User's Guide - \textit{WARP3D}}}
\rhead{\small{\textit{Numerical Procedures}}}
\fancyfoot[L] {\small{\  Updated:  \today\ at \currenttime}}

%\fancyfoot[L] {\small{\textit{Chapter {\thechapter}}\ \   (Updated: 8-10-2015. 9:00 AM)}}
\fancyfoot[C] {\small{\thesection-\thepage}}
\fancyfoot[R] {\small{\textit{Contact Procedures}}}

%---------------------------------------------------
\usepackage{graphicx}
% \usepackage[labelformat=empty]{caption}
\numberwithin{equation}{section}

%---------------------------------------------
%     --- make section headers in helvetica ---
%
\usepackage{sectsty} 
\usepackage{xspace}
\allsectionsfont{\sffamily} 
\sectionfont{\large}
\usepackage[small,compact]{titlesec} % reduce white space around sections
%---------------------------------------------->
%
%
%   which fonts system for text and equations. with all commented,
%   the default LaTex CM fonts are used
%
%
\frenchspacing
%\usepackage{pxfonts}  % Palatino text 
%\usepackage{mathpazo} % Palatino text
%\usepackage{txfonts}
\usepackage[stable]{footmisc}

%---------  local commands ---------------------

\newcommand{\ttt} {\texttt}  %typewriter text
\newcommand{\tb} {\textbf}
\newcommand{\nf} {\normalfont}
\newcommand{\df} {\dotfill}
\newcommand{\nin} {\noindent}
\newcommand{\bmf } {\boldsymbol }  %bold math symbol
\newcommand{\bsf } [1]{\textrm{\textit{#1}}\xspace}
\newcommand{\ul} {\underline}
\newcommand{\hv} {\mathsf}   %helvetica text inside an equation
\newcommand{\eg}{\emph{e.g.},\xspace}
\newcommand{\ie}{\emph{i.e.},\xspace}
\newcommand{\ti}{\emph}
\newcommand{\bardelta}{\bar \delta}
\newcommand{\barDelta}{\bar \Delta}
\newcommand{\veps}{\varepsilon}
\newcommand{\noi}{\noindent}

\newenvironment{offsetpar}[1]%
{\begin{list}{}%
         {\setlength{\leftmargin}{#1}}%
         \item[]%
}
{\end{list}}

%
%
%        optional definition for bullet lists which
%        reduces white space.
%
\newcommand{\squishlist}{
 \begin{list}{$\bullet$}
  { \setlength{\itemsep}{0pt}
     \setlength{\parsep}{3pt}
     \setlength{\topsep}{3pt}
     \setlength{\partopsep}{0pt}
     \setlength{\leftmargin}{1.5em}
     \setlength{\labelwidth}{1em}
     \setlength{\labelsep}{0.5em} } }

\newcommand{\squishlisttwo}{
 \begin{list}{$\bullet$}
  { \setlength{\itemsep}{0pt}
     \setlength{\parsep}{0pt}
    \setlength{\topsep}{0pt}
    \setlength{\partopsep}{0pt}
    \setlength{\leftmargin}{2em}
    \setlength{\labelwidth}{1.5em}
    \setlength{\labelsep}{0.5em} } }

\newcommand{\squishend}{
  \end{list}  }
%


%-------------------------------------
\newcounter{sectrefs}
\setcounter{sectrefs}{0}
\setcounter{chapter}{6}
\setcounter{section}{1}
\setcounter{figure}{0}
\renewcommand{\thefigure}{\small{\thesection.\arabic{figure}}} 
%
%--------------------------------------
%
%
%		Editing
\newcommand{\medit}{\color{blue}}
\newcommand{\eedit}{\color{black}}
\definecolor{gold}{rgb}{0.83, 0.69, 0.22}
%
%
%              start document 
%              ==========
%
%
\begin{document}

\section{Numerical Procedures}
\subsection {Overview of Penalty Method}
The penalty method defines a simple approach to enforce displacement constraints in
 the solution of a finite element model. It has a variety of applications, including the 
 imposition of multi-point constraints, incompressible material models, mesh 
 locking problems, and contact enforcement. The penalty method follows 
 from the minimum potential energy formulation of the finite element method. 
 The potential energy  is given by
 
\begin{equation} \label{eq:pe}
  \Pi    = \frac{1}{2} \,\bmf{u}^T \mathbf{K} \bmf{ u} - \bmf{u}^T \bmf{f}
 \end{equation}
 
\noi where $\bmf{u }$ is the vector of nodal displacements, $\mathbf{K}$ 
is the global stiffness matrix, 
 $\bmf{f}$ is the corresponding nodal force vector, and $\Pi$  is the potential energy. An equilibrium 
 configuration (deformed shape) for the structure makes the potential energy take 
 on a local minimum value. The minimum potential energy occurs when 
 $\partial \Pi / \partial \bmf{u} =\bmf{0}$, thus:

 \begin{equation} \label{eq:partialpe}
\frac{\partial  \Pi}{\partial \bmf{u}}    = \mathbf{K} \bmf{u} - \bmf{f} = \bmf{0}
 \end{equation}
 
\noi which results in the standard equilibrium equations $\mathbf{K}\,\bmf{u} = \bmf{f}$.
 The penalty method as applied to contact adds an additional term to Eq. \ref{eq:pe}:
 
 \begin{equation} \label{eq:penaltterm}
  \Pi    = \frac{1}{2} \,\bmf{u}^T \mathbf{K} \bmf{u} - \bmf{u}^T \bmf{f} + 
   \frac{1}{2} \,\bmf{p}^T \mathbf{\Lambda} \bmf{p}
 \end{equation}
 
\noi where $\bmf{p}$ corresponds to the penetration displacement of nodes into contact surfaces, 
 and $\mathbf{\Lambda}$ corresponds to the ``penalty parameters" -- constants which determine the 
 relative importance of forcing subsequent to zero. An increase in the magnitude 
 of the penalty parameter causes the penetration to have a stronger effect on the total 
 potential energy, thus enforcement of $\bmf{p}=\bmf{0}$ becomes proportionally more strict. 
 The terms in $\mathbf{\Lambda}$ have units of stiffness so that $ \bmf{p}^T \mathbf{\Lambda} \bmf{ p}$ 
 has units for energy. Solution of $\partial \Pi / \partial \bmf{u} =\bmf{0}$ in 
 Eq. \ref{eq:penaltterm} transforms the equilibrium equations to 
 $\overline {\mathbf{K} }\bmf{u} = \overline{\bmf{f}}$, where  $\overline {\mathbf{K} }$ is the effective 
 structural stiffness matrix and $\overline{\bmf{f}}$ is the effective force vector including the effects of contact. 

 An equivalent approach to implement the penalty method for contacting bodies creates 
 springs at the contact points (see Figure 6.2.1). The springs, placed between each 
 penetrating node and the closest point on the penetrated surface, have a very high 
 stiffness which reduces the penetration nearly to zero. The spring stiffness 
 corresponds to the penalty parameter, while the amount of remaining penetration 
 corresponds to the error in the enforcement of the constraint. A larger spring stiffness 
 decreases the magnitude of penetration after introduction of the spring. However, 
 too large a spring stiffness can cause numerical difficulties. Addition of such a spring 
 affects two parts of the finite element calculations: inclusion of the contact force into the
  residual force vector, and addition of the spring stiffness in to the global stiffness matrix. 
  Experience indicates that increasing the stiffness of the spring slightly when including it in the 
  stiffness matrix eliminates oscillation problems caused by the over-compensation of penetration. 
  Consequently, WARP3D uses a penalty stiffness 0.1\% higher than the user-specified value for the 
  stiffness matrix calculations. Furthermore, WARP3D adds each spring stiffness into the corresponding 
  element stiffness matrices instead of directly into the global stiffness matrix. Thus, for example, 
  if 6 elements connect to a contacting node, then each element stiffness receives 1/6 of the total spring 
  stiffness introduced at that node. This approach maintains full use of the element-by-element (blocking)
  architecture inside WARP3D. While contact between two deformable bodies requires application of the contact 
  force to the penetrating node and the penetrated element, rigid body contact eliminates 
  the need to compute forces on penetrated elements; the contact springs only affect penetrating nodes. 
  This greatly simplifies the calculation of contact forces and the additions to element stiffness matrices.
  
  %
\begin{figure}
\begin{center}
\includegraphics[trim=0.0in 5.6in 8.5in 1.5in, clip=true,scale=1.5,angle=0]{fig_6_2_1.pdf} 
\caption{{\small Illustration of penalty method.}
\label{fig:penaltymethods}}
%
\end{center}
\end{figure}
%

  Addition of the spring stiffness into the element stiffness matrix can seriously degrade the convergence 
  of iterative solvers for the linear equations. 
  A large spring stiffness increases the spread of eigenvalues for the system, 
  thereby increasing the number of iterations required for convergence, if convergence remains 
  possible at all. Section 6.4 provides guidelines for choosing the contact stiffness. 
  
 Diagonal dominance of the stiffness matrix is also crucial for efficiency of
  iterative solvers. 
  If the contact spring is orthogonal to one of the global coordinate directions, then the spring stiffness 
  adds only to a corresponding diagonal term in the element stiffness matrices. However, if it lies 
  skewed to the global axes, then part of the spring stiffness adds to off-diagonal terms, reducing 
  diagonal dominance of the element stiffness matrix.
   
  To alleviate this problem, the contact processors 
  construct a local coordinate system at the penetrating node which is parallel and
  orthogonal to the spring force. 
  All global data values at the node are rotated into the new coordinate system. As a result, the 
  spring stiffness adds directly into the diagonal of the element stiffness matrix. If the node has 
  no explicit constraints, then formation of the nodal coordinate transformation is straightforward. 
  If the node has one constraint, then formation of the transformation is only possible if the spring 
  force and the direction of the constraint are orthogonal. In cases where formulation 
  of a nodal coordinate transformation is not compatible with specified constraints, the 
  penalty stiffness terms add to the element stiffness in global coordinates. If a node 
  undergoing contact utilizes a (local) nodal coordinate transformation defined 
  previously through user input, then calculation of contact becomes difficult; 
  currently, the contact processors print an error message and stop 
  execution of the code when this occurs. 
  
\subsection{Contact Detection/Calculation}
WARP3D determines contact between nodes of the finite element mesh and a set of 
rigid contact surfaces at the beginning of each global Newton iteration, $i$, to solve the 
nonlinear equilibrium equations that advance the solution
over step $n\rightarrow n+1$. During the contact detection phase, contact processors 
compare all nodes in the model with all defined contact surfaces. The implementation 
currently provides three geometries of rigid contact surfaces; finite-sized rectangular 
planes, cylinders, and spheres. When a node penetrates one or more of the contact 
surfaces, the contact algorithms compute the amount and direction of the 
penetration. This section describes the contact and penetration algorithms 
for each of the contact surface geometries. Please see Section 6.3 for 
additional description of the contact surfaces and the corresponding input. 

%
\begin{figure}[htb]
\begin{center}
\includegraphics[trim=0.0in 3.6in 5.5in 1.5in, clip=true,scale=1.3,angle=0]{fig_6_2_2.pdf} 
\caption{{\small Contact detection for rectangular plane.}
\label{fig:z}}
%
\end{center}
\end{figure}
%

\noindent \bf{Finite-Sized Rectangular Planes}\rm

\noi The geometric description of the rectangular plane includes a base point corresponding 
to one of the rectanglecorners, two vectors which extend along the edges of the 
rectangle from the base point to the two adjoining corners, and a normal vector. 
Figure 6.2.2 shows the geometric description and 
outlines the contact detection algorithm.
%
\begin{figure}
\begin{center}
\includegraphics[trim=0.5in 3.5in 7.0in 1.5in, clip=true,scale=1.6,angle=0]{fig_6_2_3.pdf} 
\caption{{\small Contact detection for cylinder.}
\label{fig:contactcyl}}
%
\end{center}
\end{figure}
%

\noindent \bf{Cylinder}\rm

\noi The spatial orientation of the contact cylinder uses a base point, a vector pointing 
in the direction of the center line, the length of the cylinder, and the radius. 
Figure 6.2.3 outlines the contact detection algorithm.  

\noindent \bf{Sphere}\rm

\noi The sphere requires only a base point and a radius to completely 
describe its orientation in space, making detection of contact very simple. 
The processors compute the vector between the center point of the sphere 
and the node. If the length of the vector is less than the radius of the sphere, 
contact occurs. The spring force acts in the direction of the calculated vector.

\subsection{Penetration of Multiple Contact Surfaces}
If a node penetrates several contact surfaces, the contact algorithms must return the 
node to the correct location. However, the choice of which set of surfaces should be 
considered is not always clear. For instance, Figure 6.2.4 shows an element 
with three of its nodes penetrating a set of three contact planes. Node \ti{a} 
penetrates surface 3, so a single spring returns it to the correct location. 
Node \ti{b} penetrates both surfaces 1 and 2, but the node should return only to surface 2. 
All three planes influence node \ti{c}, but the correct return point 
is to the intersection of surfaces 2 and 3. 

%
\begin{figure}
\begin{center}
\includegraphics[trim=0.5in 5.4in 8.0in 1.50in, clip=true,scale=1.6,angle=0]{fig_6_2_4.pdf} 
\caption{{\small Nodes penetrating multiple contact surfaces.}
\label{fig:contactcorners}}
%
\end{center}
\end{figure}
%

To handle these conditions, the contact processors in WARP3D compare each of the 
penetrated contact surfaces by temporarily returning the node to a contact surface, 
and evaluating if the other shapes are still penetrated given the new location.  
By looping over the contact planes, this process eliminates all the superfluous 
contact surfaces, leaving only the set which must be satisfied simultaneously.  
The processors also calculate the location to which the node returns following 
the imposition of each of the valid contact shapes. Figure 6.2.5 provides 
additional details on the algorithm.

%
\begin{figure}
\begin{center}
\includegraphics[trim=0.1in 2.80in 6.0in 1.00in, clip=true,scale=1.6,angle=0]{fig_6_2_5.pdf} 
\caption{{\small Algorithm for treating nodes which penetrate multiple contact surfaces.}
\label{fig:contactcorners}}
%
\end{center}
\end{figure}
%

A separate algorithm constructs the new return location given a new contact 
surface, as shown in steps 3c and 3d of Figure 6.2.5. The processors 
compute the nearest point on the intersection of the previous return location 
and the new contact surface. The algorithm assumes during this step that all 
contact surfaces are planes. This causes some error for curved surfaces (cylinders, 
spheres), but if the load steps are sufficiently small, this error is negligible. Also, 
the algorithm may require additional Newton iterations for global convergence 
in problems with intersecting curved contact surfaces. 

This algorithm appears to handle correctly cases where nodes penetrate multiple 
contact surfaces. Highly complicated constructions of contact surfaces or large 
load steps may cause this algorithm to fail; the use of relatively few intersecting 
contact surfaces and small load steps is advised.

\subsection{Parallel Implementation}
The shared-memory (threaded) version of the code presents no difficulties in 
achieving efficient parallel execution since contact conditions remain globally available.

For MPI (+ threaded), WARP3D uses a framework having a single �manager� process (rank 0)
and (possibly) many �worker� processes (ranks 1, 2, 3, $\dots$). 
The manager process sends all 
worker processes the data for every defined contact surface. During 
the contact detection phase, processes assess contact for all nodes 
connected to elements which they own. Processes also compute the 
contact force for the nodes which they own. After all processes complete 
evaluation of contact for the appropriate nodes, the worker processes send 
their contact force contributions to the manager process, which reduces the 
contributions into the global contact force vector. The worker processes also 
send the computed nodal coordinate transformation matrices for all 
owned nodes; these are used in subsequent force calculations and result output. 

\end{document}
