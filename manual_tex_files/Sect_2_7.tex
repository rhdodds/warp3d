%
\documentclass[11pt]{report}
\usepackage{geometry} 
\geometry{letterpaper}

%---------------------------------------------
\setlength{\textheight}{630pt}
\setlength{\textwidth}{450pt}
\setlength{\oddsidemargin}{14pt}
\setlength{\parskip}{1ex plus 0.5ex minus 0.2ex}


%----------------------------------------
\usepackage{amsmath}
\usepackage{layout}
\usepackage{color}
\usepackage{hyphenat}

%----------------------------------------------
\usepackage{fancyhdr} \pagestyle{fancy}
\setlength\headheight{15pt}
\lhead{User's Guide - \textit{WARP3D}}
\rhead{\textit{Nodal Constraints and Nodal Releases}}
\fancyfoot[L] {\small{\textit{Chapter {\thechapter}}\ \   (Updated: 4-22-2015)}}
\fancyfoot[C] {\thesection-\thepage}
\fancyfoot[R] {\textit{Model Definition}}

%---------------------------------------------------
\usepackage{graphicx}
\usepackage[labelformat=empty]{caption}
\numberwithin{equation}{section}

%---------------------------------------------
%     --- make section headers in helvetica ---
%
\frenchspacing
\usepackage{sectsty} 
\usepackage{xspace}
\allsectionsfont{\sffamily} 
\sectionfont{\large}
\usepackage[small,compact]{titlesec} % reduce white space around sections
%
%----------------------------------------------

%---------  local commands ---------------------

\newcommand{\tb} {\textbf}
\newcommand{\df} {\dotfill}
\newcommand{\nin} {\noindent}
\newcommand{\bmf } {\boldsymbol }  %bold math symbol
\newcommand{\bsf } [1]{\textrm{\textit{#1}}\xspace}
\newcommand{\ul} {\underline}
\newcommand{\hv} {\mathsf}   %helvetica text inside an equation
\newcommand{\eg}{\emph{e.g.},\xspace}
\newcommand{\ti}{\emph}
\newcommand{\noi}{\noindent}
\newcommand{\ie}{\emph{i.e.},\xspace}
\newcommand{\vs}{\emph{vs.}\xspace}


%
%
%        optional definition for bullet lists which
%        reduces white space.
%
\newcommand{\squishlist}{
 \begin{list}{$\bullet$}
  { \setlength{\itemsep}{0pt}
     \setlength{\parsep}{3pt}
     \setlength{\topsep}{3pt}
     \setlength{\partopsep}{0pt}
     \setlength{\leftmargin}{1.5em}
     \setlength{\labelwidth}{1em}
     \setlength{\labelsep}{0.5em} } }

\newcommand{\squishlisttwo}{
 \begin{list}{$\bullet$}
  { \setlength{\itemsep}{0pt}
     \setlength{\parsep}{0pt}
    \setlength{\topsep}{0pt}
    \setlength{\partopsep}{0pt}
    \setlength{\leftmargin}{2em}
    \setlength{\labelwidth}{1.5em}
    \setlength{\labelsep}{0.5em} } }

\newcommand{\squishend}{
  \end{list}  }
%

%-------------------------------------
\newcounter{sectrefs}
\setcounter{sectrefs}{0}
\setcounter{chapter}{2}
\setcounter{section}{6}
\setcounter{figure}{0}
\renewcommand{\thefigure}{\thesection.\arabic{figure}}

%--------------------------------------
%--------------------------------------
%---------------------------------------

\begin{document}

\section{Nodal Constraints and Nodal Releases}

\noindent \bf{\ti{Constraints}}\rm

\noi WARP3D currently supports three types of displacement constraints 
applied to the model nodes:

\squishlist
\item \ti{\ul{Absolute}}:  zero or non-zero value imposed in a specified direction at a node. 
The direction may be in the global, Cartesian system or in a user-specified (local) 
coordinate system at a node

\item \ti{\ul{Multi-point}}:	a specified linear relationship between two or more displacement 
components at the same or multiple nodes expressed in the global Cartesian system. 
These are often termed relative constraints or multi-point constraints (MPCs).

\item \ti{\ul{Tie-mesh}}: used to ``tie" together multiple pairs of surfaces throughout the 
analysis. The two surfaces must be geometrically identical (to within a 
tolerance) but may be (and most often are) topologically dissimilar. 
This capability enables very convenient mesh transitions and the 
connection of a mesh region made of hex elements, for example, to a 
mesh region made of tet elements. The tie-mesh capability is implemented through multi-point
constraints.
\squishend

 The input sequence to initiate the definition \ti{absolute}
and \ti{multi-point} nodal constraint 
data is the \ti{constraints} command. The input translator destroys \ti{all} previously 
defined absolute and multi-point constraint data upon encountering this 
command. Thus to modify constraints between load (time) steps, 
all the constraints must be specified -- not just the constraints that have changed. 

Two different types of commands are used to define the \ti{tie-mesh} constraints 
as described in a subsection below. Commands for tie-mesh constraints are 
processed independently of the \ti{constraints} command -- we include 
the description here as the end result of the tie-mesh processing is 
a set of automatically generated MPCs. Tie-mesh input is unaffected by
model input entered under the \ti{constraints} command.



To simplify input, we recommend that all absolute constraints be 
specified first, followed by the multi-point constraint equations. 
The keyword \ti{multipoint} separates the two types of constraints. 
The input translators delete previously defined MPCs upon 
encountering the keyword \ti{multipoint}.
\begin{align*}
& \hv{\ul{constraint}s}\\
& \quad \hv{<absolute\ constraint>}\\
& \quad \hv{<absolute\ constraint>}\\
& \quad \hv{<absolute\ constraint>}\\
& \qquad  \cdot \\
& \qquad \cdot \\
& \hv{\ul{multi}point}\\
& \quad \hv{<multipoint\ constraint\ equation>}\\
& \quad \hv{<multipoint\ constraint\ equation>}\\
& \qquad  \cdot \\
& \qquad \cdot
\end{align*}

\noindent \bf{\ti{Release Nodes}}\rm

\noi Once \ti{absolute} constraints have been imposed, they maybe subsequently 
removed from
the model through the \ti{release nodes} command. The release process 
is implemented to allow 
the reaction forces present at the time of the release to be decreased slowly to zero over
some specified number of future load (time) steps. Reactions may be released
fully in the next load (time) step if desired. This capability allows much additional flexibility
in describing the boundary condition history imposed on a model. No such feature is 
implemented at present for multi-point and tie-mesh constraints.

To provide data requesting the removal of constraint components at nodes, the
input sequence begins with the \ti{release nodes} command Section \ref{sec:releasends}.


%--------------------------------------
\subsection{Absolute Constraints in Global Coordinates}
%--------------------------------------
\noi The simplest command to define absolute constraints in 
the global Cartesian system has the syntax
\begin{align*}
& \hv{\ul{constraint}s} \\
&\hv{<node\ list:list>}\ \left [
\begin{Bmatrix}
\hv{\ul{u}} \\ \hv{\ul{v}}\\ \hv{\ul{w}}
\end{Bmatrix}
\hv{(=)<constraint\ value:numr>}(,) \right ]
\end{align*}
\noi where the node list may also be one of the user-defined, named
lists of nodes (see Section 2.16). Examples of global constraints using this command include:

\small
\begin{verbatim}
      constraints
         1-100 by 3 w 4.3 v 0 u 0
         24 u = -1.3 w 0.0
         'end_nodes' u 0 v 0
\end{verbatim}
\normalsize

\noi To simplify the specification of constraints for all nodes that 
lie on a certain coordinate plane of the model, the list of nodes followed 
by a list of directions and constraint values in the 
above command may be replaced by the following:
\begin{align*}
  &\hv{\ul{plane}}\ 
    \begin{Bmatrix}
          \hv{\ul{x}} \\ \hv{\ul{y}}\\ \hv{\ul{z}}
    \end{Bmatrix}
  \hv{=\ <coord\ value:numr>}\ \hv{<constraint\ value(s)>}\ (\hv{<proximity>})\ \hv{(\ul{verify})}
\end{align*}
\noi where the $\hv{<constraint\ value(s)>}$ can have one of several forms
\begin{align*}
    \begin{Bmatrix}
          \hv{\ul{fixed}} \\ \hv{\ul{symmetry}}
    \end{Bmatrix}
  \qquad \hv{\ti{or}}\qquad   \left [
    \begin{Bmatrix}
          \hv{\ul{u}} \\ \hv{\ul{v}}\\ \hv{\ul{w}}
    \end{Bmatrix}
   \hv{ (=)\ <displacement\ value:numr>} \right ]
\end{align*}
\noi The \ti{fixed} option sets $u=v=w=0$ for all nodes located on the specified plane.
The \ti{symmetry} option imposes $u=0$ for all nodes located on the specified $x=constant$ plane,
$v=0$ for all nodes located on the specified $y=constant$ plane, or $w=0$ for all nodes located 
on the specified $z=constant$ plane. Displacement components may be assigned 
prescribed values for all nodes located on a plane normal to one of the
global coordinate axes. The \ti{verify} option 
provides a table of constraints imposed as a result of the \ti{plane} command. 

The tolerance employed in the test to determine if a node resides on the specified
plane is set in one of two approaches: 
\squishlist
\item \ti{user-specified proximity distance}:  The ($<$proximity$>$) option shown above is
\begin{align*}
  &\hv{\ul{prox}imity}\  \hv{<absolute\ distance>}
\end{align*}
\noi which sets the comparison distance directly.

\item \ti{default relative}: the code computes the maximum extent of the model in each
of the coordinate directions $X, Y, Z$ multiplied by $10^{-5}$ to set the 
proximity test distance separately for each of the three planes.
\squishend

\noi The number of nodes affected by the \ti{plane} command is always 
printed for reference as are the proximity test distances. 
Examples of the \ti{plane} command include

\small
\begin{verbatim}
     plane x = -1.3 fixed verify
     plane y = 0 symmetry  proximity 0.001
     plane z = 200 u = -0.0001 w 0.0 verify
\end{verbatim}
\normalsize

%--------------------------------------
\subsection{Nonglobal, Absolute Constraints}
\label{sec:nonglobal}
%--------------------------------------
The capability to specify constraints in non-global coordinates 
enables the analysis of skew supports, for example, that arise 
naturally in structural systems or in 3-D models of axisymmetric structures. 
To define constraints in a non-global Cartesian system, consider the simple 
problem shown in the Fig. \ref{fig:transform}. Here the global and local $z$ axes are 
aligned but the local and global $x, y$ axes are not aligned. The user 
defines a $3 \times 3$ rotation matrix of direction cosines which transforms 
global vector quantities into the local coordinate system. 
The boundary condition shown becomes simply $u=0$ in the local 
coordinate system. Transformation matrices are specified with 
the command sequence: 
\begin{align*}
& \hv{\ul{constraint}s} \\
& \hv{\ul{trans}formation\ \ul{matrix}\ <node\ list:list>} \\
&\quad \left [
\begin{Bmatrix}
\hv{{row\_1}} \\ \hv{{row\_2}}\\ \hv{{row\_3}}
\end{Bmatrix}
\  \left [ \hv{<direction\ cosines:numr>}\right ] \right ]
\end{align*}
\noi where any number of nodes may be associated with the specified 
transformation matrix; the \ti{transformation matrix} command may be repeated 
as necessary within the \ti{constraints} definition. In this example, the constraint 
is specified immediately following definition of the transformation matrix 
although this is not required.
%
\begin{figure}[tb]
\begin{center}
\includegraphics[trim = 1in 6.5in 1in 1.3in, clip, scale=0.9,angle=0]{fig_transform_matrix.pdf} 
\caption{{\small Fig. \thefigure\ Example of local coordinate system for constraint specification.}
\label{fig:transform}}
%
\end{center}
\end{figure}
%

The input system verifies that the rotation matrix specified is 
orthogonal and that the matrix pre-multiplied by its transpose 
becomes an identity matrix to within a tight tolerance. 

As in the above example, users specify constraints in the \ti{local} 
coordinate system defined for the nodes. The \ti{plane} option for specification 
of constraints may be combined with the transformation matrix option. 
The specified $x, y,$ or $z$ plane remains a global plane 
but the specified constraints for nodes on the plane take on 
the meaning of any local coordinate system defined at the node
through the transformation matrix.

These local coordinate systems apply only during the specification of 
constraints. Nodal loads and 
element loads must always be specified in global coordinates. 
All nodal output quantities produced by WARP3D are in global coordinates.

%--------------------------------------
\subsection{Multi-Point Constraints (MPCs)}
%--------------------------------------
These constraints provide a convenient approach to enforce a linear 
relationship between displacement components at a single node or 
across multiple nodes. At present, the MPC capability supports: 
\squishlist
\item homogeneous equations -- the right-hand side
must be zero, 
\item none of the terms may also appear
in an absolute constraint (which effectively makes the 
equation non-homogeneous)\footnote{The current 
homogeneous requirement limits considerably the ability to impose all but the simplest
Periodic Boundary conditions (PBCs), for example, in analyses of representative
volume elements (RVEs) having periodicity properties. Alternative 
strategies to impose general PBCs in 3D as described in  [\ref{R:KGB2006}], for example,
are under development. }, and 
\item  the constraints refer to displacement
components in the global coordinate system.
\squishend

Each multi-point constraint ``equation" 
is specified with input having the form
\begin{align*}
& \hv{\ul{constraint}s} \\
& \hv{<absolute\ constraint>} \\
&\qquad \hv{\cdot} \\
&\qquad  \hv{\cdot} \\
& \hv{\ul{multi}point} \\
&\quad \left [ 
    \begin{Bmatrix}
         \hv{{+}} \\ \hv{{-}}
    \end{Bmatrix}
\hv{<node\ id:integer>\ <coefficient:real>}
    \begin{Bmatrix}
         \hv{{u}} \\ \hv{{v}}\\ \hv{{w}}
    \end{Bmatrix}
\  (,) 
\right ] = 0.0
\end{align*}
\noi Examples of multi-point constraints include

\small
\begin{verbatim}
   - 9954 1.0 u - 9954 1.0 v = 0.0
    175 -1.0 w + 533 4.3 w + 542 -5.7 u = 0.0
 - 300 1.4 u + 927 1.0 v,
           - 9340 1.0 w + 10821 1.4 u = 0
\end{verbatim}
\normalsize

\noi Note that only one node number may precede each $\hv{<coefficient>}$ 
and that the coefficient must be type $\hv{<real>}$. There is no 
default value of 1.0 for the 
$\hv{<coefficient>}$; the $\hv{<coefficient>}$ must be specified as in the first 
example equation above. A space is required on each side of the $+,-$ 
sign for the pre-multiplier terms as indicated in these examples -- prevents the
immediately following node number from 
being interpreted as a negative integer. To continue long equations 
over multiple input lines, insert the comma and line break after 
a direction specifier as in the above example.

In the enforcement of multipoint constraints, WARP3D adopts a modified version of the 
approach described in [\ref{R:AS1979}] . Displacements appearing in multi-point (and tie-mesh) constraints
are separated into dependent and independent terms. The incremental equilibrium equations
are modified to eliminate the dependent displacements, the equations solved and the dependent
terms recovered immediately from the constraint equations. WARP3D does not use the 
formal method of Lagrange multipliers which increases the number of equations by the
number of multi-point/tie-mesh constraints. The processors examine constraint equations and
sort out which displacements should be treated as dependent and independent. In simple two-term equations,
the last displacement listed in
each equation is taken as the independent  term.
When the multi-point constraint equations contain many terms, the selection of dependent/independent
terms becomes non-unique. The equation solving process may fail, requiring the user to try
a different ordering of terms in the multi-point constraints. This does not happen with tie-mesh constraints,
only the user-specified constraints.

User-specified multi-point constraints are frequently employed to enforce all nodes on a surface of the mesh
to have the same displacement component. For example, consider a mesh
with an external surface on $Y=1.0$. The user desires that all nodes on this surface have
the same $v$ displacement.  Pick one node on the surface, say node 75, to have the independent 
displacement. Other nodes on the surface, for example, 2, 30, 120, 152 etc, then have 
dependent $v$ displacements.
Input to define these constraints is then
\small
\begin{verbatim}
       2 1.0 v - 75 1.0 v = 0
       30 1.0 v - 75 1.0 v = 0
      120 1.0 v - 75 1.0 v = 0
      152 1.0 v - 75 1.0 v = 0
\end{verbatim}
\normalsize




In dynamic analyses, the computed displacements satisfy 
the MPCs exactly as in static analyses. The Newmark time-history 
integration procedure generates the nodal velocities and accelerations. 
The computed velocities also satisfy the linear relationships specified 
in the MPCs for geometrically linear solutions 
but the accelerations approximately satisfy the relationships.

In the coming releases, new commands will be added 
here to generate automatically the most common types of 
MPC equations, \eg to make the displacements identical at 
two nodes, to make the displacements at a node 
the average of displacements at two adjacent nodes.

%--------------------------------------
\subsection{Constraints in Nonlinear Analyses}
%--------------------------------------
In a nonlinear analysis, the current constraints define 
the \ti{incremental} changes to be imposed over the next load step
(and time step). Non-zero absolute constraints are enforced during the first 
iterative cycle for the load step. In subsequent iterations, no 
further displacement change occurs on the constrained 
displacements to maintain the value of the specified 
increment or the linear, MPC relationship.

By default, the current set of absolute constraints with a multiplier of 
1.0 are imposed during each nonlinear load step. Alternatively, 
the multiplier for constraint increments over load step may be
included in the definition of the loading step (see Section 2.8.5).
  
%--------------------------------------
\subsection{	Display of Current Constraint Data}
%--------------------------------------
Within the \ti{constraints} command sequence, the \ti{dump}
command may be specified to request a display (listing) 
of the current constraints information taken from internal tables.


%--------------------------------------
\subsection{Tie-Mesh Constraints}
%--------------------------------------
Quite often in large models, there exists the need to:
\squishlist
\item make very coarse transitions in mesh density without 
using special transition elements
\item transition from a mesh region of hex elements, for 
example, to an adjacent region meshed with tet elements
\squishend

\noi In these cases, the faces of elements on each side 
of the transition define a geometrically identical surface 
(within a tolerance) even though no topological congruency 
exists between element faces on each side of the transition 
surface. Moreover, the element types 
may be different on each side of the transition surface.
%
\begin{figure}[tb]
\begin{center}
\includegraphics[trim = 1in 6.5in 1in 1.5in, clip, scale=0.80,angle=0]{fig_tie_mesh_one.pdf} 
\caption{{\small Fig. \thefigure\ Simple 2-D illustration of tie-mesh constraints.}
\label{fig:tie_mesh_1}}
%
\end{center}
\end{figure}
%


Consider the simple 2-D version in Fig. \ref{fig:tie_mesh_1} for purposes of discussion. 
Here the topmost row of elements (labeled \ti{I, II}) are 8-node (quadratic)
isoparametrics with linear triangle elements below. The dotted line defines 
the interface ``edge" for the 2-D case which becomes a ``surface" in the 3-D case. 
Nodes \ti{A, B, C, D, E} are defined with the top region mesh while nodes 
\ti{a-e} are defined in the bottom region. The two mesh regions could 
have been defined by separate working groups, using different meshing programs, etc. 
Edges \ti{A-B-C} and \ti{C-D-E} are shown here as straight but they can be curved in 
general for the quadratic elements. The key requirement is that the mesh developers create 
nodes \ti{a-e} to lie (geometrically) on the interface edge 
represented by the top elements. 
The \ti{tie-mesh} processors determine, for example, that 
node \ti{b} lies at a specific location between nodes \ti{A-B-C} on a 
certain edge of element \ti{I}. A set of multi-point constraints is 
constructed to ``tie" node \ti{b} to have the same displacements as 
that location on edge \ti{A-B-C} of element \ti{I}. The processors
solve for the parametric coordinate on edge \ti{A-B-C} 
corresponding to the geometric location of node \ti{b}. 
The generated multi-point constraints then have the form:
\begin{align*}
&u_b = \psi_A u_A + \psi_B u_B + \psi_C u_C \\
&v_b = \psi_A v_A + \psi_B v_B + \psi_C v_C 
\end{align*}
\noi were $\psi_A,\, \psi_B,\, \psi_C$ are numerical constants derived 
from the edge shape functions for element \ti{I} using the initial 
geometry of the edge. In 3-D, the $w$ displacement at \ti{b} has a similar constraint.

From the above two MPCs, it is clear that node \ti{b} is the dependent 
node and that nodes \ti{A-B-C} are the independent nodes.  The $(u,v)$ 
displacements for node \ti{b} are eliminated prior to solution of the equilibrium equations. 
It is common terminology to define the edge containing nodes  \ti{A-B-C-D-E}, etc. 
as a ``master" edge (a master surface in 3-D) and edge \ti{a-b-c-d-e}, etc. 
as a ``slave" edge (a slave surface in 3-D). Degrees of freedom associated 
with all nodes on the slave edges (2-D) and slave surfaces (3-D) 
are taken as dependent and eliminated. Consequently, absolute 
constraints cannot be imposed on slave nodes.
%
\begin{figure}[tb]
\begin{center}
\includegraphics[trim = 1in 5.70in 1in 1.3in, clip, scale=0.9,angle=0]{fig_tie_mesh_two.pdf} 
\caption{{\small Fig. \thefigure\ Simple 3-D illustration of the tie-mesh constraints to 
couple a quadratic hex mesh to a quadratic tet mesh.
\label{fig:tie_mesh_2}}}
%
\end{center}
\end{figure}
%

The above example also illustrates additional features and consequences of 
the tie-mesh process. The processing routines for tie-mesh readily find 
that nodes \ti{C} and \ti{c} have identical coordinates and thus the 
simpler multi-point constraints are generated as $u_c=u_C$ and $v_c=v_C$ 
(in 3-D, the $w$ displacements are included as well). Now suppose the 
coordinates specified for node \ti{b} are defined by the user such that it does not 
lie exactly along the edge \ti{A-B-C}. The tie-mesh processors support a user-defined 
\ti{tolerance} with an \ti{adjust} option for such situations which allow 
the processors to re-position node \ti{b} to lie exactly on edge \ti{A-B-C} 
prior to construction of the multi-point constraints. Finally, this example shows a 
tie-mesh transition from quadratic elements to linear elements. 
Such a mesh allows the formation of gaps and overlaps between elements 
in the deformed configuration since only slave nodes are constrained. 
Moreover, in this example, there exists a clear gap in the undeformed 
mesh at node \ti{A} since the triangle edge \ti{a-b} must be straight. This represents 
poor practice and should be avoided. When the element types on both 
adjacent surfaces have the same displacement interpolation order, the 
deformed configuration maintains inter-element displacement compatibility. 
If a transition from quadratic to linear elements must be used 
in the mesh, it is recommended that the interface surface 
be a flat plane to avoid initial (geometric) gaps between the master and slave surfaces.

Figure \ref{fig:tie_mesh_2} shows a typical application of the 
tie-mesh capability to connect 
a mesh of 20-node hex elements to a mesh of 10-node tet elements. 
The more refined mesh of tet elements serves as the slave surface. 
In this example, the tie-mesh processors generate MPCs of various 
complexity. The $(u,v,w)$ displacements of node \ti{d} are tied to displacements 
of nodes $A\rightarrow H$, with the coefficients derived from the 
quadratic shape functions of the 20-node hex element. 
Node \ti{a} is simply pinned to node \ti{A}; node \ti{b} is simply 
pinned to node \ti{B}, etc.

The more refined mesh surface should be chosen as the dependent 
(slave) surface. If the more coarse mesh surface is chosen as the 
dependent surface, there could be small elements on the more 
refined surface not connected to the coarse dependent 
surface, thereby allowing a gap in the tied mesh connection.

In geometrically nonlinear analyses, the MPCs generated by tie-mesh 
processors use the undeformed geometry and thus remain
unchanged during the solution. 
In the example above, node \ti{d} remains fixed to the same 
parametric point (\ie material point) on face  $A\rightarrow H$ throughout the solution and 
thus the MPC coefficients do not change.

The tie-mesh specification involves two sets of commands 
not connected to the \ti{constraints} command described in the previous sections. 
First, commands are given to construct \ti{surfaces} composed of faces of 
elements, where the face numbers correspond to those used to define 
element loads. Multiple surfaces are usually defined and
are given names as part of the definition. Second, the \ti(tie) command 
is specified followed by a listing of pairs of \ti{master-slave} surfaces. 
The \ti{tie} command has options to set a tolerance for adjusting nodes 
on the slave surface to match exactly the geometry represented by 
element faces on the corresponding master surface. Multiple \ti{tie}
commands may be used, for example, if it becomes necessary 
to use different values of the tolerance for different surfaces.

\noindent \bf{\ti{Surface Command}}\rm

\noi The \ti{surface} command has the form

\begin{align*}
& \hv{\ul{surface}\ <surface\ id:label>}\\
& \quad \hv{<list\ of\ elements:list>\ \ul{face}\ (\ul{abaqus})\ <element\ face\ no:integer>}\\
& \quad \hv{<list\ of\ elements:list>\ \ul{face}\ (\ul{abaqus})\ <element\ face\ no:integer>}\\
& \qquad  \cdot \\
& \qquad \cdot \\
\end{align*}

\noi Surface definitions are required for each pair of \ti{master} and \ti{slave} sides 
of the interfaces. The face numbers for hex and tet elements are shown in 
Fig. \ref{fig:tie_mesh_3}. These numbers correspond to the face numbers used 
to apply loadings on the elements. The \ti{abaqus} option proves convenient to import
models available as Abaqus input files. The faces of tet elements are numbered
identically in Abaqus and WARP3D. Faces 1-3 and 6 are identical for the hex elements.
Faces (4,5) in WARP3D are numbered (5,4) in Abaqus. Simply append the \ti{abaqus}
option as indicated; the input translator converts specified Abaqus face numbers to
WARP3D face numbers.
An example of this command is 

\small
\begin{verbatim}
     surface piece_a
       47 face 4 abaqus
       165-320 by 5 1000-1300 by 3 822 844 face 1
     surface piece_b
       8000-10000 by 2 12000-15000 by 3 face 3
       200 face 2
       300-800 face 4
\end{verbatim}
\normalsize

\noi When the specified $\hv{<surface\ id>}$ matches a previously defined surface,
the existing surface is destroyed and re-defined 
by the new input commands. As many surfaces as are required may be defined.

\noindent \bf{\ti{Tie Mesh Command}}\rm

\noi The \ti{tie mesh} command has the form
\begin{align*}
& \hv{\ul{tie}\ {mesh}\ <tied\ set\ id:label>}\\
&\quad \hv{\ul{tol}erance\ <tol\ value:numer>\ \ul{adjust}}\  
   \begin{Bmatrix}
         \hv{{on}} \\ \hv{{off}}
    \end{Bmatrix} \\
&\quad \begin{Bmatrix}
         \hv{\ul{master}} \\ \hv{\ul{slave}}
    \end{Bmatrix} 
 \hv{<surface\ label:id>}
   \begin{Bmatrix}
         \hv{\ul{master}} \\ \hv{\ul{slave}}
    \end{Bmatrix} 
\hv{<surface\ label:id>}\\
& \qquad \qquad  \cdot \\
& \qquad \qquad  \cdot \\
\end{align*}
\noi The \ti{tolerance} command is optional. The defaults are \ti{tolerance 0.05 adjust on}. 
The adjust option corrects small geometric mismatches in node locations on 
the specified slave surfaces to match the geometry of the master 
surfaces (more details below). The keywords \ti{master} or 
\ti{slave} are required. Use as many lines as needed to define 
all the master-slave pairs.

Multiple \ti{tie mesh} commands may be used for clarity in 
model definition or to change the tolerance value for 
different pairs of surfaces. Reference to an existing 
\ti{tie mesh set} causes the existing set to be destroyed and re-defined 
by the new input.

Examples of this command include

\small
\begin{verbatim}
     tie mesh crack_insert_a
         tolerance 0.01 adjust on
         master seg_left slave seg_right
         master seg_bottom slave seg_top

     tie mesh crack_insert_b
         slave left_a master left_b
         master left_c slave left_d
         slave left_e master left_f 
\end{verbatim}
\normalsize

The tolerance value supports checking of nodes on the slave
mesh surface to determine if they are sufficiently near the master 
mesh surface to be considered ``in contact" with the master surface 
in the initial, undeformed configuration. Slave nodes that satisfy 
the specified tolerance are included in the automatically generated MPC 
equations produced by the tie-mesh processors. If the absolute (``gap") distance 
from a slave node to the master surface exceeds the specified tolerance, 
the slave node is not included in the MPCs -- a warning message is issued 
that lists the current node and surface data for the omitted slave 
node. When the user knows the gap distance between the slave 
node(s) and master surface(s), the tolerance should be set to 
a slightly larger value to insure that all slave nodes are 
included in the tied mesh MPC equations. 
%
\begin{figure}[tb]
\begin{center}
\includegraphics[trim = 0in 3in 0in 19mm, clip, scale=0.9,angle=0]{fig_face_nos.pdf} 
\caption{{\small Fig. \thefigure\ Face numbers for hex and tet elements 
to define surfaces for tie-mesh constraints. 
\label{fig:tie_mesh_3}}}
%
\end{center}
\end{figure}
%

If the initial gap distance between the slave nodes and master 
surface(s) is not well known, we suggest setting a tolerance 
value to the typical element face size on the dependent surface 
(the more refined mesh surface) as a starting point. The typical 
element size should be a reasonable starting point to set a suitable 
tolerance value. The user should check the output file for any tied 
mesh warning messages about unconstrained slave nodes to obtain 
information about the computed gap distances. This information 
supports changes to increase or decrease the tolerance value  
to either include all slave nodes or to exclude some slave nodes 
from the tied mesh.

The \ti{adjust} feature updates the initial coordinates of slave nodes 
to lie exactly on the master mesh surface. The tie-mesh  
processors perform this location adjustment of the slave nodes 
before the analysis and thus causes no initial strain.

\noindent \bf{\ti{Use of Meshing Software to Generate Surface Lists}}\rm

\noi The generation of element lists and face numbers to define the 
\ti{surfaces} for mesh tie modeling can require considerable effort. 
While most meshing packages do not support this feature directly, the element 
lists-face numbers may be generated readily with the following 
procedure. 

Create a load case in the meshing software for the model and apply 
pressure loads to the faces of elements that define a \ti{surface} for 
mesh tie modeling. Repeat for each additional mesh tie \ti{surface}. 
Have the meshing software output a Patran neutral file (for example).
When translated into WARP3D input file by the \ti{patwarp} program, 
each separate load case will have the list of elements and corresponding 
faces included in the WARP3D input file produced by \ti{patwarp}. 
Simply remove the keyword �pressure� and the loading intensity 
using a text editor and add the surface command to replace the 
load case command.


%--------------------------------------
\subsection{Release Nodes Command}
\label{sec:releasends}
%--------------------------------------
\noi The command has the form
\begin{align*}
& \hv{ \ul{release}\ \ul{const}raints\ ( \ul{step}s\  (=)\ <number\ of\ release\ steps:integer>) }  \\
&\hv{<node\ list:list>}\ \left [
\begin{Bmatrix}
\hv{\ul{u}} \\ \hv{\ul{v}}\\ \hv{\ul{w}}
\end{Bmatrix}(,) \right ]
\end{align*}
\noi Corresponding reaction forces in the specified direction are
reduced to zero uniformly over the indicated number of load (time)
steps. The default number of release steps is one.

 An example of this command
\small
\begin{verbatim}
      release constraints steps = 5
         1-100 by 3 w  v
         3000-3020 u
         10232-12000 by 2, 22500-22560 by 3 w v u
\end{verbatim}
\normalsize

\noi Situations that may arise with the release reaction forces on nodes
include:

\squishlist
\item If the analysis is terminated before the release process finishes,
an analysis restart will complete the reduction of reactions to zero. 

\item In the event a normal,  adaptive subdivision of the steps is required for 
global Newton convergence, the amount of reaction to be reduced over the
step is subdivided accordingly.

\item If the constraints are defined using a non-global coordinate
system (\ref{sec:nonglobal}), the release components \ti{u, v, w} refer to the
non-global system.
\squishend

%*****************************************************
\subsection {References}
%*****************************************************
\small

\noindent[\refstepcounter{sectrefs}\label {R:KGB2006}\thesectrefs]~V.G. Kouznetsova,~M.G.D. Geers,
~W.A.M. Brekelmans. Computational Homogenisation For Non-Linear Heterogeneous Solids.
Multiscale Modeling In Solid Mechanics - Computational Approaches � Imperial College Press
http://www.worldscibooks.com/engineering/p604.html. ISBN: 978-1-84816-307-2. 2009.

\noindent[\refstepcounter{sectrefs}\label {R:AS1979}\thesectrefs]~J.F. Abel and M.S~Shephard. 
An Algorithm for Multipoint Constraints in Finite Element Analysis. \ti{International Journal for Numerical
Methods in Engineering}, Vol.  14, pp. 464-467, 1979.
% ----------------- end of real text ----------------------------------
\end{document}
